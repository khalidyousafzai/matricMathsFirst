
\باب{الکراجی کا مسئلہ ثنائی}\شناخت{باب_الکراجی_مسئلہ_ثنائی}
%page#01
 \حصہ{ثنائی مسئلہ}

یہ سبق بنیادی طور پر
\((x+y)^{n}\)
کے توسیع  سے متعلق ہے جہاں n ایک مثبت ناطق عدد ہے یا صفر ہے۔  جب آپ اس سبق کو مکمل کرلیں گے تب آپ درج ذیل باتوں کے اہل ہوجائیں گے۔

 اگر n چھوٹا ہو تو پاسکل کے مثلث کو استعمال کرکے 
\((x+y)^{n}\)
کا  توسیع معلوم کرنا۔ 

اگر n بڑا ہو تو 
\((x+y)^{n}\)
کے  توسیع میں  حاصل ہونے والے ضربیوں (Coefficients)  کو محسوب کرنا۔

ثنائی مسئلہ کی مناسبت سے 
\(\begin{pmatrix}
n\\
r\\
\end{pmatrix}\) 
علامتی اظہار کے استعمال کے قابل ہونا

9.1
کی  توسیع:۔

ثنائی مسئلہ کو استعمال کرکے  
\((x+y)^{n}\) 
 کو فوری طور پر اور آسانی سے حل کیا جاسکتا ہے۔ یہ قابل استعمال ہوتا ہے اگر 
\((x+y)^{n}\) 
کی اس طرح رجوع کیا جائے  کہ n کی قیمت  2، 3 اور 4 ہو۔

یہ توسیع  کچھ اس طرح ہونگے۔ 
\begin{align*}
   (x+y)^{2}&=x(x+y)+y(x+y)=x^{2}+2xy+y^{2}\\
   (x+y)^{3}&=(x+y)(x+y)^{2}=(x+y)(x^{2}+2xy+y{2}\\
      &=x(x^{2}+2xy+y^{2})+y(x^{2}+2xy+y^{2})\\
  &=x^{3}+2x^{2}y+xy^{2}\\
  x^{2}y+2xy^{2}+y^{3}\\
  &=x^{3}+3x^{2}y+3y^{2}x+y^{3}\\
  (x+y)^{4}&=(x+y)(x+y)^{3}=(x+y)(x^{3}+3x^{2}y+3xy^{2}+y^{3})\\
   &=x(x^{3}+3x^{2}y+3xy^{2}+y^{3})+y(x^{3}+3x^{2}y+3xy^{2}+t^{3})\\
   &=x^{4}+3x^{3}y+3x^{2}y^{2}+xy6{3}+x^{3}y+3x^{2}y^{2}+3xy^{3}+y^{4}\\
   &=x^{4}+4x^{3}y+6x^{2}y^{2}+4xy^{3}+y^{4}\\
   \end{align*}
\\
آپ ان تمام نتائج کا خلاصہ ، 
   \((x+y)^{1}\)
کو شامل کرتے ہوئے ،    درج ذیل انداز میں بیان کرسکتے ہیں۔  تمام ضربیوں کو  جلی  انداز میں  لکھا گیا ہے۔ 
  \begin{align*}
   (x+y)^{1}&=1x+1y\\
   (x+y)^{2}&=1x^{2}+2xy+1y^{2}\\
   (x+y)^{3}&=1x^{3}+3x^{2}y+3xy^{2}+1y^{3}\
   (x+y)^{4}&=1x^{4}+4x^{3}y+6x^{2}y^{2}+4xy^{3}+1y^{4}\\
   \end{align*}
   
ان تمام توسیع کا احتیاط کے ساتھ مطالعہ کیجئے۔ دیکھئے کہ کس طرح سے قوت نما 
   \(x^{n}\)
کے ساتھ  بائیں جانب سے شروع ہوتے ہیں ۔ پھر x کے قوت نما مسلسل 1 سے کم ہوتے جاتے ہیں اور  y کے قوت نما 1 سے بڑھتے جاتے ہیں اور
 \(y^{n}\)
 تک پہونچ جاتے ہیں۔  
%page#02
یہاں یہ بھی نوٹ کیجئے کہ تمام ضربیوں کے ذریعئے پاسکل کا مثلث نما ترتیب  تیار ہوتا ہے جسے آپ نے پہلے دفع نمبر 8.4میں دیکھا ہے اور ایک بار پھر اُسے یہاں درج ذیل خاکہ9.1 میں دکھایا گیا ہے۔ 

پاسکل کے مثلثوں کو تیار کرنے کا ایک آسان طریقہ کچھ اس طرح ہے: 1 سے شروع کیجئے۔  پھر اُوپری قطار کے ارکان کی جوڑیوں کی جمع کیجئے اور اُس مجموعی عدد کو اُن دونوں کے درمیان نیچے کے مقام پر بیچوں بیچ   (درج بالا خاکہ میں قطاروں کے مطابق)  لکھئے؛   اُس قطار کو 1   کے ساتھ مکمل کیجئے۔   یہ بالکل اُسی طرح  سےہے جیسا   کہ ہم نے پچھلی مثال کے وقت
\( (x+y)^{3}\)
اور
\( (x+y)^{4}\)
کے مجموعے کے پھیلاؤ میں دیکھا تھا۔ 

اب آپ اس قابل ہو گئے ہونگے کہ اندازہ لگا سکیں کہ پانچویں قطار میں  حاصل ہونے والے ضربیہ   درج ذیل ہونگے۔ 
 \[1\quad 5\quad 10\quad 10 \quad 5 \quad 1\]
اور یہ بھی کہ 
 \[(x+y)^{5}=x^{5}+5x^{4}y+10x^{3}y^{2}+10x^{2}y^{3}+5xy^{4}+y^{5}\]
 9.1.1   مثال 
 \((1+y)^{6}\)
کے  توسیع  کو  لکھئے۔ 

	پاسکل کے مثلث کے دوسری قطار کو استعمال کیجئے ، قوت نما کی ترتیب کو جاری رکھئے اور x کی جگہ 1 لیجئے: 
	 \begin{align*}
 (1+y)^{6}&=(1)^{6}+6(1)^{5}y+15(1)^{4}y^{2}+20(1)^{3}y^{3}+15(1)^{2}y^{4}+6(1)y^{5}+y^{6}\\
 &=1+6y+15y^{2}+20y^{3}+15y^{4}+6y^{5}+y^{6}
 \end{align*}
9.1.2   مثال
%page#03

فقرہ
 \((2x+3)^{4}\)
میں قوسین  کی حاصل ضرب کی شکل میں لکھئے۔ 
\( (x+y)^{4}\)
کی توسیع  کو استعمال کیجئے اور اس میں xکی جگہ(2x) اورy   کی جگہ 3 کو استعمال کیجئے۔ 
 \begin{align*}
 (2x+3)^{4}&=(2x)^{4}+4\times(2x)^{3}\times3+6\times(2x)^{2}\times 3^{2}+4\times (2x)\times 3^{3}+3^{4}\\
 &=16x^{4}+96x^{3}+216x^{2}+216x+81\\
 \end{align*}
 \\
9.1.3   مثال

توسیع کیجئے  ۔
 \((x^{2}+2)^{3}\)\\
  \[(x^{2}+2)^{3}=(x^{2})^{3}+3\times(x^{2})^{2}\times+3\times x^{2}\times 2^{2}+2^{3}=x^{6}+6x^{4}+12x^{2}+8\]
9.1.4   مثال
 \((3x-4)^{5}\)
کی توسیع میں حاصل ہونے والے
 \(x^{3}\)
کا ضریب معلوم کیجئے۔ 
قطار میں
 \(x^{3}\)
کی اصطلاح  تیسرے مقام پر حاصل ہوگی جس کا ضربیہ 1، 5 ، 10 ، ۔۔۔ ہوگا ۔ اس طرح سے وہ اصطلاح  درج ذیل ہوگی۔ 

 \[10\times(3x)^{3}\times(-4)^{2}=10\times 27\times 16x6{3}=4320x^{3}\]
اسی لئے مطلوبہ ضربیہ  4320  ہے ۔ 

9.1.5   مثال
  \[(1+2x+3x^2)^{3}\]
کی  توسیع  کیجئے  ۔  

ثنائی توسیع کو اسیعمال کرنے کے لئے ، آپ کو
   \[ 1+2x+3x^{2}\]
اس اصطلاح  کو تین اجزا ءکی بجائے دو اجزاء میں لکھنا پڑے گا۔   ایسا کرنے کا ایک آسان طریقہ یہ ہے کہ آپ دی گئی اصطلاح  کو  
   \[ (1+(2x+3)^2)^{3}\]
اصطلاح کو اس طرح لکھیں۔    تب 
    \[(1+(2x+3^2))^{3}=1^{3}+3\times1^{2}\times(2x+3x^{2})+3\times1\times(2x+3x^{2})^{2}+(2x+3x^{2})^{3}\]
اب آپ ثنائی توسیع کے مسئلہ کو استعمال کرکے دی گئی اصطلاح کی توسیع کرسکتے ہیں۔  
   \begin{align*}
    (1+2x+3^{2})^{3}&=1+3(2x+3x^{2})+3((2x)^{2}+2\times(2x)\times(3x^{2})+(3x^{2})^{2})+((2x)^{3}+3\times(2x)^{2}\times(3x^2)+3\times(2x)\times(3x^{2})^{2}+(3x^{2})^{3})\\
    &=1+(6x+9x^{2})+(12x^{2}+36x^{3}+27x^{4})+(8x^{3}+36x^{4}+54x^{5}+27x^{6})\\
    &=1+6x+21x^{2}+44x^{3}+63x^{4}+54x^{5}+27x^{6}\\
    \end{align*}
%page 4-6
%Page#4
اس قسم کے تفصیلی کام میں ، آپ اسے اپنے جوابات کو آسانی سے جانچنے کے لئے استعمال کرسکتے ہیں۔ آپ ایسا کرنے کے لئے 
    \((1+2x+3x^2)^3\)
کی توسیع کیجئے اور بعد میں 
    \(((1+2x)+3x^2)^3\)
کی توسیع کرکے دیکھئے کہ آپ کا جواب ایک جیسا ملتا ہے یا نہیں۔ ایسا کرنے کا اور زیادہ آسان طریقہ یہ بھی ہے کہ آپ x کی قیمت 1 رکھ کر اسے حل کیجئے۔ پھر آپ کو بائیں جانب 
   \( (1+2+3)^3=6^3=216\)
اسی طرح سے دائیں جانب آپ کو 1+6+21+44+63+54+27=216 کچھ اس طرح سے حاصل ہوگا۔ یہاں ایک بات یہ بھی قابل غور ہے کہ اگر آپ کو دائیں جانب اور بائیں جانب ایک جیسے اعداد مل جاتے ہیں تو اس سے یہ ضمانت نہیں دی جاسکتی کہ دی گئی اصطلاح کی توسیع سو فیصد صحیح کی گئی ہے۔ لیکن اگر دونوں جانب حاصل ہونے والے اعداد مختلف ہوں تو یہ بات یقینی طور پر ثابت کرتی ہے کہ دی گئی اصطلاح کی توسیع کرنے میں کوئی غلطی ہوئی ہے۔ 

مشق 9A 

1۔ درج ذیل میں سے ہر ایک کی توسیع کیجئے۔ 
\begin{multicols}{3}
\begin{enumerate}
\item
 \[ (2x+y)^{2}\]
 \item
    \[(5x+3y)^{2}\]
    \item
    \[(4+7p)^{2}\]
    \item
    \[(1-8t)^{2}\]
    \item
    \[(1-5x^{2})^{2}\]
    \item
    \[(2+x^{3})^{2}\]
    \item
   \[ (x^{2}+y^{3})^{3}\]
   \item
    \[(3x^{2}+2y^{3})^{3}\]
    \end{enumerate}
   \end{multicols}
2۔ درج ذیل میں سے ہر ایک کی توسیع کیجئے۔
\begin{multicols}{2}
  \begin{enumerate}
\item
   \[ (x+2)^{3}\]
   \item
   \[ (2p+3q)^{3}\]
   \item
    \[(1-4x)^{3}\]
    \item
    \[(1-x^3)^{3}\]
        \end{enumerate}
    \end{multicols}
3۔ درج ذیل کی توسیع میں حاصل ہونے والے x کے ضریبہ معلوم کیجئے۔ 
\begin{multicols}{2}
  \begin{enumerate}
  \item
       \[(3x+7)^{2}\]
       \item
   \[ (2x+5)^{3}\]
           \end{enumerate}
   \end{multicols}
4۔ درج ذیل کی توسیع میں حاصل ہونے والے
 \(x^{2}\)
  کے ضریبہ معلوم کیجئے۔ 
  \begin{multicols}{2}
   \begin{enumerate}
     \item
    \[(4x+5)^{3}\]
      \item
    \[(1-3x)^{4}\]
               \end{enumerate}
    \end{multicols}
5۔ درج ذیل تمام اصطلاحات کی توسیع کیجئے۔ 
\begin{multicols}{2}
 \begin{enumerate}
   \item
    \[(1+2x)^{5}\]
      \item
    \[(p+2q)^{6}\]
      \item
    \[(2m-3n)^{4}\]
      \item
    \[(1+\frac{1}{2}x)^{4}\]
      \end{enumerate}
   \end{multicols}
6۔ درج ذیل کی توسیع میں حاصل ہونے والے
 \(x^{3}\)
  کے ضریبہ معلوم کیجئے۔ 
  \begin{multicols}{2}
   \begin{enumerate}
      \item
 \[(1+3x)^5\]
    \item
    \[(2-5x)^4\]
          \end{enumerate}
    \end{multicols}

%Page#5
7۔ 
   \((1+x+2x^{2})^{2}\)
کی توسیع کیجئے۔ اپنے جواب کی تصدیق کرنے کے لئے عددی متبادلی طریقہ استعمال کیجئے۔ 
8۔ 
    \((x+4)^{3}\)
کی توسیع کیجئے ۔ اسی طرح سے
 \((x+1)(x+4)^{3}\)
کی بھی توسیع کیجئے۔ 

9۔ 
 \[ (3x+2)^{2}(2x+3)^{3}\]
کی توسیع کیجئے۔ 
10۔ 
  \[(1+ax)^{4}\]
کے توسیع میں ،
\(x^{3}\)
کا ضربیہ 
1372
حاصل ہوتا ہے۔ مستقل aکی قیمت معلوم کیجئے۔ 
11۔ 
 \((x+y)^{11}\)
کی توسیع کیجئے۔ 
12۔
  \((2x+y)^{12}\)
کی توسیع میں 
 \( x^{6}y^{6}\)
کا ضربیہ معلوم کیجئے۔ 

9.2 ثنائی مسئلہ

درج بالا دفع 9.1 میں جو طریقہ کار دیا گیا ہے وہ 
  \((x+y)^n\)
کی توسیع معلوم کرنے کے لئے نہایت موزوں ہے اگر n کی قیمت بہت چھوٹی ہو۔ لیکن اس طریقہ کو استعمال کرکے ہم 
  \((x+y)^{15}\)
کی توسیع میں 
 \( x^{11} y^4\)
کا ضربیہ نہیں معلوم کرسکتے۔ آپ صرف پاسکل کے مثلث کی تمام قطاروں کے متعلق سوچئے جو کہ آپ لکھ سکتے ہیں! یہاں آپ کو
  \((x+y)^{n}\)
کی توسیع میں 
  \(x^{n-r}y^r\)
کے ضربیہ کی قیمت معلوم کرنے کے لئے n اور r میں ایک ضابطے کی ضرورت محسوس ہوگی ۔ 
خوش قسمتی سے ، پاسکل کے مثلث کی nth قطار، دراصل پاسکل کی nth ترتیب ہے جو کہ دفع8.4 میں دی گئی ہے۔ وہاں دکھایا گیا تھا کہ
 \[\begin{pmatrix}
  n\\
  0
  \end{pmatrix}
   \begin{pmatrix}
     n\\r+1
     \end{pmatrix}
    =\frac{n-r}{r+1}
     \begin{pmatrix}
    n\\r
     \end{pmatrix}
     r=1,2,3...\]
حقیقت میں، آپ پاسکل کے مثلث کو درج ذیل انداز میں لکھ سکتے ہیں۔ 

%Page#6
وغیرہ وغیرہ 

اس طرح سے آپ 
 \((x+y)^n\)
کی توسیع کا ایک نہایت خوبصورت اورصاف ستھرا ضابطہ اس طرح بیان کرسکتے ہو،

ثنائی مسئلہ کو اس طرح بیان کیا جاسکتا ہے کہ اگر n ایک فطری عدد ہو تو 
  \[(x+y)^{n}=
     \begin{pmatrix}
     n\\0
     \end{pmatrix}
     x^{n}+
      \begin{pmatrix}
    n\\1
     \end{pmatrix}
     x^{n-1}y+
      \begin{pmatrix}
    n\\2
     \end{pmatrix}
     x^{n-2}y^{2}+...+
      \begin{pmatrix}
    n\\n
     \end{pmatrix}
     y^{n}\]
ضربیوں کی تحسیب کے لئے ، آپ اس دفع کے ابتدائی حصے میں دیئے گئے ماخوذی ضابطے کو استعمال کرکے
\(\begin{pmatrix}
    n\\r
    \end{pmatrix}\)
کا ضابطہ بنا سکتے ہیں۔ مثال کے طور پر، 
\( \begin{pmatrix}
    4\\2
    \end{pmatrix}\)
کی قیمت معلوم کرنے کے لئے آپn = 4 سے ابتدا کرسکتے ہیں ۔ پھر 
  \[  \begin{pmatrix}
    4\\0
     \end{pmatrix}
     =1,
     \begin{pmatrix}
    4\\1
     \end{pmatrix}
     =\frac{4-0}{0+1}
     \begin{pmatrix}
    4\\0
     \end{pmatrix}
     =\frac{4}{1}\times1=\frac{4}{1}
     \begin{pmatrix}
     4\\2
     \end{pmatrix}
     =\frac{4-1}{1+1}
     \begin{pmatrix}
    4\\1
     \end{pmatrix}
     =\frac{3}{2}\times\frac{4}{1}=\frac{4\times3}{1\times2}\]
عام انداز میں ، 
         \[\begin{pmatrix}
     n\\0
     \end{pmatrix}
     =1,
     \begin{pmatrix}
    n\\1
     \end{pmatrix}
     =\frac{n-0}{0+1}\times1=\frac{n}{1},
     \begin{pmatrix}
     n\\2
     \end{pmatrix}
     =\frac{n-1}{1+1}
     \begin{pmatrix}
     n\\1
     \end{pmatrix}
     =\frac{n-1}{2}\times\frac{n}{1}=\frac{n(n-1)}{1\times2},...\]
     %132
%page 7-9

اسی طرح سے تسلسل کو جاری رکھتے ہوئے آپ معلوم کرسکتے ہیں کہ 
      \(\begin{pmatrix}
     n\\r
     \end{pmatrix}
     =\frac{n(n-1)...(n-(r-1))}{1\times2\times...\times r}\)\\
آپ 
      \(\begin{pmatrix}
     n\\r
     \end{pmatrix}\)
  کو اس طرح بھی لکھ سکتے ہیں کہ 
   \[\begin{pmatrix}
     n\\r
     \end{pmatrix}
     =\frac{n(n-1)...(n-(r-1))}{1\times2\times...\times r}\times \frac{(n-r)\times(n-r-1)\times ...\times 2\times1}{(n-r)\times(n-r-1)\times...\times2\times1}=\frac{n!}{r!(n-r)!}\]
نوٹ کیجئے کہ یہ ضابطہ
r=0
  اور
r=n
  اور ان کے درمیان تمام قیمتوں کے لئے صحیح ثابت ہوتا ہے،
  کیونکہ  (دفع  
18.3
) کے مطابق  
\(0!=1\)
  ہوتا ہے۔

اسی طرح سے ، ثنائی  ضربیوں  کو درج ذیل انداز میں لکھا جاسکتا ہے۔ 
  \[ \begin{pmatrix}
     n\\r
     \end{pmatrix}
    =\frac{n(n-1)...(n-(r-1))}{1\times2\times...\times r},
     \begin{pmatrix}
    n\\r
     \end{pmatrix}
     =\frac{n!}{r!(n-r)!}\]
جب آپ
    \(\begin{pmatrix}
     n\\r
     \end{pmatrix}\)
کی کوئی مخصوص قیمت معلوم کرنےکے لئے پہلا ضابطہ  استعمال کریں گے ، مثلاً
\(  \begin{pmatrix}
     10\\4
     \end{pmatrix}\)
 یا   
    \( \begin{pmatrix}
     12\\7
     \end{pmatrix}\)
، تب یہ یاد رکھنا کافی مددگار ثابت ہوگا کہ سب سے اُوپری خط میں بہت زیادہ عوامل موجود ہیں اور اُتنے ہی زیادہ عوامل سب سے نچلے خط میں موجود ہوتے ہیں۔  اسی لئے آپ نسب نما میں اعداد رکھتے ہوئے شروع ہوسکتے ہیں اور پھر 
10
  اور 
12
بالترتیب سے گن سکتے ہیں ، لیکن یہاں اس بات کا یقین رکھنا ہوگا کہ نسب نما اور شمار کنندہ دونوں میں عوامل کی تعداد برابر ہونی چاہیئے۔  
 \[ \begin{pmatrix}
     10\\4
     \end{pmatrix}
     =\frac{10\times9\times8\times7}{1\times2\times3\times4}=210,
     \begin{pmatrix}
     12\\7
     \end{pmatrix}
     \frac{12\times11\times10\times9\times8\times7\times6}{1\times2\times3\times4\times5\times6\times7}=792\]
کئی  کیلکولیٹرس    کے ذریعئے آپ  
    \(\begin{pmatrix}
     n\\r
     \end{pmatrix}\)
 کی قیمت  معلوم کرسکتے ہیں  جو کہ اسے 
 \(\begin{pmatrix}
 _{n}C_{r}
 \end{pmatrix}\)
کی شکل میں ظاہرکرتے ہیں۔   
\( \begin{pmatrix}
    10\\4
     \end{pmatrix}\)
  کی قیمت معلوم کرنے کے لئے آپ کو عام طور پر
\(\begin{pmatrix}
10,_{n}C_{r},4
\end{pmatrix}\)
ترتیبی کنجی استعمال کرنی ہوگی، لیکن پہلے آپ نے کیلکولیٹر کے    دست نامہ    میں تفصیلات دیکھ لینا چاہیئے۔ 

9.2.1
 مثال 
 \( (x+y)^{15}\)
کی توسیع میں
    \(x^{11}y^{4}\)
کا ضربیہ معلوم کیجئے۔ 

مطلوبہ ضربیہ درج ذیل ہوگا۔
  \[\begin{pmatrix}
    15\\4
     \end{pmatrix}
    =\frac{15\times14\times13\times12}{1\times2\times3\times4}=1365\]  
آپ
    \(\begin{pmatrix}
     n\\r
     \end{pmatrix}\)
  کی جن قیمتوں کو ثنائی مسئلہ کے لئے استعمال کرنا چاہتے ہیں، ان کو یقینی بنانے کے لئے ،    مزید ایک اور مرحلہ طے کرنا ہوگا۔   خاکہ  
  9.1
  میں ،  آپ دیکھیں کہ پاسکل کے مثلث کے  پہلے اور ہر قطار کے آخری رکن کو چھوڑکر،   باقی تمام ارکان کو اُس کےفوراً    اُوپر موجود دونوں ارکان کا مجموعہ کرکے حاصل کیا جاسکتا ہے۔ اسی لئے یہ بات بالکل صحیح  ثابت  ہوسکتی ہے کہ 
   \[    \begin{pmatrix}
     n+1\\r+1
     \end{pmatrix}
     =
     \begin{pmatrix}
     n\\r
     \end{pmatrix}
     +
     \begin{pmatrix}
    n\\r+1
     \end{pmatrix}\]
  مثال کے طور پر،
\begin{align*}
\begin{pmatrix}6\\3\end{pmatrix}+\begin{pmatrix}6\\4\end{pmatrix}&=\frac{6\times5\times4}{1\times2\times3}+\frac{6\times5\times4\times3}{1\times2\times3\times4}=\frac{6\times5\times4\times4+6\times5\times4\times3}{1\times2\times3\times4}\\
&=\frac{6\times5\times4}{1\times2\times3\times4}\times(4+3)\\
&=\frac{7\times6\times5\times4}{1\times2\times3\times4}\\
&=\begin{pmatrix}7\\4\end{pmatrix}\\
\end{align*}
  \\
  \(  \begin{pmatrix}
     n+1\\r+1
     \end{pmatrix}
     =
     \begin{pmatrix}
    n\\r
     \end{pmatrix}
     +
     \begin{pmatrix}
     n\\r+1
     \end{pmatrix}\)
کو  ثابت کرنا آسان نہیں ہے۔  آپ اس نتیجہ کو براہ راست قبول کرکے اور اُسکے ثبوت کو ترک کرنا چاہیں گے ،  اور سیدھے مثال 
9.2.2
  پر پہونچنا چاہیں گے۔  

اس نتیجے کو ثابت کرنے کے لئے ، دائیں جانب سے شروع کیجئے۔ 
\begin{align*}
  \begin{pmatrix}
     n\\r
     \end{pmatrix}
     +
    \begin{pmatrix}
     n\\r+1
     \end{pmatrix} &=\frac{ n(n-l) ... (n-(r-l)) }{1\times2\times...\times r}+\frac{n(n-l) ... (n-r) }{1\times2\times...\times r\times(r+1)}\\
    &= \frac{ n(n-l) ... (n-(r-l)) \times (r+ 1) + n(n-l) ... (n-r) }{1\times2\times...\times r\times(r+1)}\\
      &= \frac{n(n-l) ... (n-(r-l)) }{1\times2\times...\times r\times(r+1)}\times((r+1)+(n-r))\\
   &=\frac{_n(n-l) ... (n-(r-l)) }{1\times2\times...\times r\times(r+1)}\times(n+1)\\  
    &=\frac{ ( n + 1 )n( n -1) ... ((  n + 1) -r) }{1\times2\times...\times r\times(r+1)}\\
  &=  \begin{pmatrix}
     n+1\\r+1
     \end{pmatrix}\\
     \end{align*}
اس طرح سے ، استدلال کی وہ    زنجیر مکمل ہوجاتی ہےجو کہ پاسکل کے مثلث کو ثنائی ضربیوں کے ساتھ جوڑتی ہے۔ 

درج ذیل مثال ایسی ہے جس میں  
x
کی قیمت کو چھوٹا یا کم فرض کیا گیا ہے۔   جب مثال کے طور پر  
\(x=0.1\)
ہوتو   
x
کی ہر اگلی قوت نما  
10
  گنا سے کم ہوتی جائے گی اور حقیقت میں اتنی چھوٹی ہوجائے گی کہ تمام بڑی قوت نماؤں کو نظر  انداز  کر دینا  بہتر  ثابت  ہوگا۔  
مثال 
9.2.2
  میں ،   آپ کو  ثنائی  توسیع  کرتے وقت  
x
کی قوت نماؤں کو  چڑھتی ترتیب میں رکھنے کے لئے کہا گیا تھا۔  اس کا مطلب یہ ہے کہ آپ
x
   کی سب سے چھوٹی قوت نما سے شروع کریں اور پھر اگلی سب سے چھوٹی قوت نما پر جائیں اور اس طرح سے تسلسل کو مکمل کریں۔ 
مثال  
9.2.2
x
کے قوت نماؤں کی چڑھتی ترتیب میں،  
\((2-3{x}^{10}\)
کی توسیع کے لئے ابتدائی چار رکن معلوم کیجئے۔  
 \(x=\frac{1}{100}\)
  رکھ کر  ،  
\(1.97^{10}\)
 کی نزدیکی مکمل عدد کی شکل میں تقربی قیمت  معلوم کیجئے۔ 
 \[(2-3{x)}^{10}= {2}^{10}+
\begin{pmatrix}
10\\
1\\
\end{pmatrix}
\times{2}^{9}\times{(-3{x)}}+
\begin{pmatrix}
10\\
2\\
\end{pmatrix}
\times{2}^{8}\times{(-3{x})^{2}}+ \begin{pmatrix}
10\\
3\\
\end{pmatrix}\times{2}^{7}\times{(-3{x)}}^{3}+... \]
\begin{align*}
=1024-10\times512\times3{x}+{\frac{10\times9}{1\times2}}\times256\times9{x}^{2}-{\frac{10\times9\times8}{1\times2\times3}}\times128\times27{x}^{3}+...\\
 1024-15360x+103\ 680{x}^{2}-414\ 720{x}^{3}+...
 \end{align*}


%page 10-12
اسی لئے پہلے چار ارکان
\(1024-15 360{x}+103 680{x}^{2}-414720{x}^{3}\)
ہونگے۔
\(x=\frac{1}{100}\)
رکھنے پر،
\[1.97^{10}=1024-15360{x}\times\frac{1}{100}+103 680\times(\frac{1}{100})^{2}-414720(\frac{1}{100})^{3}\]
\[=880.35328\]
اسی لئے

\({1.97}^{10}\approx 880\)
حاصل ہوا۔  اگلا  رکن  درج ذیل ہوگا،
\(\begin{pmatrix}
10\\
4\\
\end{pmatrix}
\times{2}^{6}\times{3({x})}^{4}= 1088640{x}^{4}=0.0108864\)\\
اس کے بعد والے ارکان مزید چھوٹے ہونگے اور حقیقتاً  اُن سب کو نظر انداز کرنا بہتر ہوگا۔ 
مشق
9B
1۔ درج ذیل میں سے ہر ایک  کی  قیمت معلوم  کیجئے۔
\begin{multicols}{2}
 \begin{enumerate}
 \item
 \[\begin{pmatrix}
7\\
3\\
    \end{pmatrix}\]
    \item
\[\begin{pmatrix}
8\\
6\\
\end{pmatrix}\]
    \item
\[\begin{pmatrix}
9\\
5\\
\end{pmatrix}\]
    \item
\[\begin{pmatrix}
13\\
4\\
\end{pmatrix}\]
    \item
\[\begin{pmatrix}
6\\
4\\
\end{pmatrix}\]
    \item
\[\begin{pmatrix}
10\\
2\\
\end{pmatrix}\]
    \item
\[\begin{pmatrix}
11\\
10\\
\end{pmatrix}\]
    \item
\[\begin{pmatrix}
50\\
2\\
\end{pmatrix}\]
\end{enumerate}
\end{multicols}
2۔ درج ذیل میں سے ہر ایک کی توسیع کے دوران 
\(x^{3}\)
کا ضربیہ معلوم کیجئے۔
\[ {(1+x)}^{5}\quad (b)\quad {(1-x)}^{8}\quad (c)\quad {(1+x)}^{11}\quad (d) {(1-x)}^{16}\]
3۔ درج ذیل میں سے ہر ایک کی توسیع کے دوران
\(x^{5}\)
کا ضربیہ معلوم کیجئے۔
\[ {(2+x)}^{7}\quad (b) {(3-x)}^{8}\quad (c) {(1+2x)}^{9}\quad (d) ({1-\frac{1}{2x}})^{12}\]
4۔ درج ذیل میں سے ہر ایک کی توسیع کے دوران
\( {x}^{6}{y}^{8}\)
کا ضربیہ معلوم کیجئے۔
\begin{multicols}{3}
\begin{enumerate}
\item
\[ {(x+y)}^{14}\quad (b) {(2x+y)}^{14}\] 
\item
\[ c){(3x-2y)}^{14}\]
\item
\[ d)(4x+\frac{1}{2}{y})^{14}\]
\end{enumerate}
\end{multicols}

۔  درج ذیل کی x  کی قوت نماؤں کی چڑھتی ترتیب میں توسیع کے ابتدائی چار  ارکان معلوم کیجئے۔ 
\begin{multicols}{2}
\begin{enumerate}
\item
\[ {(1+x)}^{13}\ (b)\ {(1-x)}^{15}\]
\item
\[ {(1+3x)}^{10}\ (d)\ {(2-5x)}^{7}\] 
\end{enumerate}
\end{multicols}
6۔   درج ذیل کی x  کی قوت نماؤں کی چڑھتی ترتیب میں توسیع کے ابتدائی تین  ارکان معلوم کیجئے۔ 
\begin{multicols}{2}
\begin{enumerate}
\item
\[ {(1+x)}^{22}\]
\item
 \[ {(1-x)}^{30}\]
 \item
 \[ {(1-4x)}^{18}\]
 \item
 \[ {(1+6x)}^{19}\]
 \end{enumerate}
\end{multicols}
7۔  
\( {(1+2x)^{8}}\)
کے لئے ،   x  کی قوت نماؤں کی چڑھتی ترتیب میں توسیع کے ابتدائی تین  ارکان معلوم کیجئے۔ 
 \({x}={0.01}\)
منتخب کرکے
 \( {1.208}^{8}\)
کی تقربی  قیمت معلوم کیجئے۔ 
8۔  
\( {(2+5x)^{12}}\)
کے لئے ،   x  کی قوت نماؤں کی چڑھتی ترتیب میں توسیع کے ابتدائی تین  ارکان معلوم کیجئے۔ 
x 
 کے لئے کوئی مناسب قیمت منتخب کرکے دو عشری مقامات تک   
 \( {2.005}^{12}\)
کی تقربی  قیمت معلوم کیجئے۔ 

9۔
 \({(1+2x)^{16}}\)
کی
 \( {x}^{3}\)
والے رکن تک اور اُس کو شامل کرتے ہوئے  توسیع کیجئے۔  اِسی طرح سے 
 \( {(1+3x)}{(1+2x)^{16}}\)
کی توسیع میں  

 \( {x}^{3}\)
کا ضربیہ معلوم کیجئے۔ 
10۔
\( {(1-3x)^{10}}\)
کی
 \({x}^{2}\)
والے رکن تک اور اُس کو شامل کرتے ہوئے  توسیع کیجئے۔
  اِسی طرح سے 
 \({(1+3x)^{2}}{(1-3x)^{10}}\)
کی توسیع میں  
 \( {x}^{2}\)
کا ضربیہ معلوم کیجئے۔

11۔
\({(1+ax)}{(1+5x)^{40}}\)
کے توسیع میں x  کے ضربیہ کی قیمت 207 دی گئی ہے۔  a کی قیمت معلوم کیجئے۔

12۔
\( {(1-x)^{8}}+{(1+x)^{8}}\)
کو حل کیجئے۔  x  کی کوئی مناسب قیمت منتخب کرکے
\( {0.99}^{8}+{1.01}^{8}\)
 کی بالکل صحیح قیمت معلوم کیجئے۔ 
13۔  
\( {(1+ax)^{n}}\)
کی  توسیع کی ابتداء  
 \(1+36{x}+576{x}^{2}\)
سے ہوتی ہے۔ a   اور n   کی قیمتیں معلوم کیجئے۔ 
متفرق مشق  
9

1۔  
\({(3+4x)}^{3}\)  
کی توسیع کیجئے۔ 

2۔  x کی قوت نماؤں کی چڑھتی ترتیب کے لئے ، درج ذیل کی پہلے تین ارکان تک توسیع کیجئے۔ 
\begin{enumerate}
\item
\({(1+4x)}^{10} \)
\item
\( {(1-2x)}^{16}\)
\end{enumerate}
3۔  درج ذیل کی توسیع میں
\( {a}^{3}{b}^{5}\) 
کا ضربیہ معلوم کیجئے۔ 
\begin{enumerate}
\item
  \[{(3a-2b)}^{8}\]
  \item
  \[ (5a+\frac{1}{2}b)^{8}\]
  \end{enumerate}
4۔   x کی قوت نماؤں کی چڑھتی ترتیب کے لئے ،
\({(3+5x)}^{7}\)
کی توسیع کیجئے جس  میں 
 \( x^{2}\)
کے رکن کو شامل رکھا گیا ہو۔    
\(x=0.01\)
رکھ کر
\({3.05}^{7}\)  
کی تقربی قیمت نزدیکی کامل عدد کی شکل میں معلوم کیجئے۔ 

5۔ x  کی چڑھتی ترتیب میں
\({(2}+\frac{1}{4}{x)}^{8}\)
کی توسیع ابتدائی چار ارکان تک حاصل کیجئے۔ حاصل ہونے والی اس توسیع میں x  کی کوئی مناسب قیمت رکھ کر 
 \({2.0025}^{8}\)
کی قیمت تین عشری مقامات تک صحت کے ساتھ معلوم کیجئے۔ 

(OCR)\\
6۔ x  کی چڑھتی ترتیب میں
  \({(2-3x)}^{8}\)
کی توسیع ابتدائی تین ارکان تک حاصل کیجئے۔ حاصل ہونے والی اس توسیع کو استعمال کرکے 
  \( {1.997}^{8}\)
کی قیمت  قریب ترین مکمل عددکی شکل میں حاصل کیجئے۔  (OCR)

7۔ 
\( ({x}^{2}+\frac{1}{x})^{3}\)
کی توسیع کیجئے اور ہر ایک رکن کو حل کیجئے۔ 


%page 13-15
  8.
  \(({2x}-\frac{3}{x^{2}})^{4}\)
کی توسیع کیجئے۔

9۔
\( ({x}+\frac{1}{2x})^{6}+({x}-\frac{1}{2x})^{6}\)
کی توسیع کیجئے اور حل کیجئے۔

10۔
\( ({x}^{4}+\frac{4}{x})^{3}\)
کی توسیع کیجئے اور
\( {x}^{2}\)
کا ضربیہ معلوم کیجئے۔ 
11۔
\(({2x}+\frac{5}{x})^{6}\)
کی توسیع کیجئے اور
   x
سے آزاد  رکن  معلوم کیجئے۔ 
12۔   
\( {(1+y)}^{12}\)
کی توسیع کیجئے اور
\( {y}^{4}\)
کا ضربیہ معلوم کیجئے۔   اسی طرح سے درج ذیل حل کیجئے۔
	(a)
  \({(1+3y)}^{12}\)
کی توسیع کیجئے اور
   \( {y}^{4}\)
کا ضربیہ معلوم کیجئے۔
	(b) 
  \({(1-2y}^{2})^{12}\)
کی توسیع کیجئے اور
   \( {y}^{8}\)
کا ضربیہ معلوم کیجئے۔
	(c)
 \( ({x}+\frac{1}{2}{y})^{12}\)
کی توسیع کیجئے اور
  \( {x}^{8}{y}^{4}\)
کا ضربیہ معلوم کیجئے۔


13۔ 
\({(2p-q)}{(p+q)}^{10}\)
کی توسیع کیجئے اور
\({p}^{4}{q}^{7}\)
کا ضربیہ معلوم کیجئے
14۔ 
\({(1+2s)}^{20}\ {x}\)
کی توسیع   ابتدائی تین ارکان تک حاصل کیجئے ۔ 
  x
کی کوئی مناسب قیمت منتخب کرکے درج ذیل کی قریب ترین قیمتیں معلوم کیجئے۔
 
	(a)
	 \( {1.002}^{20}\)
اور
 (b)
   \({0.996}^{20}\)
 
15۔
\(({2}-\frac{1}{2x^{2}})^{10}\ {x}\ {1.995}^{10}\) 
کی توسیع
 x
کی چڑھتی قوت نماؤں کی شکل میں ،  ابتدائی تین ارکان تک حاصل کیجئے ۔   اسی طرح سے 
\(1.995^{10}\)
  کی قیمت تین بامعنی اعداد تک صحت کے ساتھ معلوم کیجئے۔
16۔ درج ذیل میں سے کوئی دو توسیع صحیح ہیں اور باقی دو روسیع غلط ہیں۔  جو توسیع غلط ہیں اُنہیں معلوم کیجئے۔ 
\begin{enumerate}
\item
\(({2}-\frac{1}{2x^{2}})^{10}\ {x}\ {1.995}^{10}\) \\
\item
\( {(3+4x)}^{2}=243+1620{x}+4320{x}^{2}+5670{x}^{3}+3840^{4}+1024{x}^{5}\)\\
\item
\( {(1-2x+3{x}^{2})}^{3}=1+6{x}-3{x}^{2}+28{x}^{3}-9{x}^{4}+54{x}^{5}-27{x}^{6}\)\\
\item
\({(1-x)}{(1+4{x})^{4}}=1+ 15{x}+80{x}^{2}+160{x}^{5}-256{x}^{5}\)\\
\item
\({(2x+y)}^{2}{(3x+y)}^{3}=108{x}^{5}+216{x}^{4}{y}+171{x}^{3}{y}^{2}+67{x}^{2}{y}^{3}+13{x}{y}^{4}+{y}^{6}\)\\
\end{enumerate}
17۔
  \(({\frac{1}{2x}+{x}^{3}})^{8}\)
کی توسیع میں 
 x
سے آزاد رکن معلوم کیجئے۔

18۔
   \(({2x}+\frac{1}{x}^{2})^{9}\)
کی توسیع میں
x
 سے آزاد رکن معلوم کیجئے۔

19۔
    \(({x}^{2}-\frac{1}{2x})^{16}\)
کی توسیع میں
   x
سے آزاد رکن معلوم کیجئے۔


20۔
\(({x}^{3}-\frac{1}{x})^{24}\)
کی توسیع میں 
\( {x}^{-12}\)  
کا ضربیہ معلوم کیجئے۔

21۔
\( {(1+3{x}+4{x}^{2}})^{4}\)
کی توسیع xکی قوت نماؤں کی چڑھتی ترتیب میں
\({x}^{2}\)
کے رکن تک حاصل کیجئے۔  x  کی مناسب قیمت منتخب کرکے
 \(1.0304^{4}\) 
کی تقریباً قیمت معلوم کیجئے۔

22۔
   \( {(3x+5)}^{3}-{(3x-5)}^{3}\)
کی توسیع کیجئے اور حل کیجئے۔  اسی طرح سے
   \({(3x+5)}^{3}-{(3x-5)}^{3}=730\)
اس مساوات کو حل کیجئے۔

23۔
\( {(7-6x)}^{3}+{(7+6x)}^{3}=1736\)
اس مساوات کو حل کیجئے۔

24۔ درج ذیل میں، t  کی قوت نماؤں میں،  پہلے تین ارکان معلوم کیجئے ۔ 
	(a)	 \({(1+\alpha{t}})^{5} \)
	(b) \({(1-\beta{t}})^{8}\)
اسی طرح سے،
\( {(1+\alpha{t}})^{5}{(1-\beta{t}})^{8}\)
کی توسیع میں،   الفا   اور    بیِٹا   کی شکل میں
\(t^{2}\)
کا ضربیہ معلوم کیجئے۔ 

25۔  (a) دکھائیے کہ
\begin{enumerate}
\item
\(\begin{pmatrix}
6\\ 
4\\ 
\end{pmatrix}
=
\begin{pmatrix}
6\\
2\\
\end{pmatrix}\)\\
\item
\(\begin{pmatrix}
10\\
3\\
\end{pmatrix}
=
\begin{pmatrix}
10\\
7\\
\end{pmatrix}\)\\
\item
\( \begin{pmatrix}
15\\
12\\
\end{pmatrix}
=
\begin{pmatrix}
15\\
3\\
\end{pmatrix}\)\\
\item
\(\begin{pmatrix}
13\\
6\\
\end{pmatrix}
=
\begin{pmatrix}
13\\
7\\
\end{pmatrix}\)
\end{enumerate}
(b)
 درج ذیل میں ،   x کی ممکنہ قیمتیں لکھئے۔ 
 \begin{enumerate}
 \item
 \( \begin{pmatrix}
11\\
4\\
\end{pmatrix} 
=
\begin{pmatrix}
11\\
x\\
\end{pmatrix}\)\\
\item
\(\begin{pmatrix}
16\\
3\\
\end{pmatrix}
=
\begin{pmatrix}
16\\
x\\
\end{pmatrix}\)\\
\item
\(\begin{pmatrix}
20\\
7\\
\end{pmatrix}
=
\begin{pmatrix}
20\\
x\\
\end{pmatrix}\)\\
\item
\( \begin{pmatrix}
45\\
17\\
\end{pmatrix}
=
\begin{pmatrix}
45\\
x\\
\end{pmatrix}\)\\
\end{enumerate}
(c)
  تعریف 
   \(\begin{pmatrix}
n\\
r\\
\end{pmatrix}
=
\frac{n!}{r!(n-r)!}\)
  کو استعمال کرکے  ،  ثابت کیجئے کہ 

\[  \begin{pmatrix}
n\\
r\\
\end{pmatrix}
=
\begin{pmatrix}
n\\
n-r\\
\end{pmatrix}\]
\\
26۔  دفع 8.4  میں امالی خاصیت
\( \begin{pmatrix}
n\\
r+1\\
\end{pmatrix}
=
\frac{n-r}{r+1}
\begin{pmatrix}
n\\
r\\
\end{pmatrix}\)
دی گئی ہے۔ اُسے استعمال کرکے پاسکل کے مثلث کی خاصیت کو ثابت کیجئے کہ

\[
\begin{pmatrix}
n\\
r\\
\end{pmatrix}
+
\begin{pmatrix}
n\\
r+1\\
\end{pmatrix}
=
\begin{pmatrix}
n+1\\
r+1\\
\end{pmatrix}\]
\\
27۔ (a) دکھائیے کہ 
\begin{enumerate}
\item
\(4\times \begin{pmatrix}
6\\
2\\
\end{pmatrix}
=
3\times \begin{pmatrix}
6\\
3\\
\end{pmatrix}
=
6\times \begin{pmatrix}
5\\
2\\
\end{pmatrix}\)\\
\item
\( 3\times \begin{pmatrix}
7\\
4\\
\end{pmatrix}
=
5\times \begin{pmatrix}
7\\
5\\
\end{pmatrix}
=
7\times\begin{pmatrix}
6\\
4\\
\end{pmatrix}\)\\
\end{enumerate}
(b)
 c اور  b    , a  اعداد لکھئے جو کہ 
 \begin{enumerate}
 \item
\( a\times\begin{pmatrix}
8\\
5\\
\end{pmatrix}
=
b\times \begin{pmatrix}
8\\
6\\
\end{pmatrix}
=
c\times \begin{pmatrix}
7\\
5\\
\end{pmatrix}\)\\
\item
\( a\times \begin{pmatrix}
9\\
3\\
\end{pmatrix}
=
b\times\begin{pmatrix}
9\\
4\\
\end{pmatrix}
=
c\times \begin{pmatrix}
8\\
3\\
\end{pmatrix}\)\\
\end{enumerate}
(c).
 ثابت کیجئے کہ
 \[
 (n-r)\times
\begin{pmatrix}
n\\
r\\
\end{pmatrix}
=
(r+1)\times
\begin{pmatrix}
n\\
r+1\\
\end{pmatrix}
=
n\times
\begin{pmatrix}
n-1\\
r\\
\end{pmatrix}
\]\\
 28.ثابت کیجئے کہ
\(\begin{pmatrix}
n\\
r-1\\
\end{pmatrix}
+
2\begin{pmatrix}
n\\
r\\
\end{pmatrix}
+
\begin{pmatrix}
n\\
r+1\\
\end{pmatrix}
=
\begin{pmatrix}
n+2\\
r+1\\
\end{pmatrix}\)\\
29۔
\( {1.0003}^{18}\)
کی قیمت 15 عشری مقامات تک صحت کے ساتھ معلوم کیجئے۔ 

30۔  (a)
 \( {(2\sqrt{2}+\sqrt{3}})^{4}\)
کو
 \( a+b\sqrt{6}\)
کی شکل میں توسیع کیجئے۔ یہاںa اور bصحیح اعداد ہیں۔
 
(b)
  \( {(2\sqrt{2}+\sqrt{3}})^{5}\)
کی بالکل صحیح قیمت معلو م کیجئے۔ 

31.
(a)
   \({(\sqrt{7}+\sqrt{5}})^{4}+ {(\sqrt{7}-\sqrt{5}})^{4}\)
کو حل کیجئے اور اُسکی توسیع کیجئے
   \( 0<\sqrt{7}-\sqrt{5}<1\)
اس حقیقت کو استعمال کرکے ، ایسےمتصل صحیح اعداد معلوم کیجئے جن کے درمیان 
    \({(\sqrt{7}+\sqrt{5}})^{4}\)
پایا جاتا ہو۔ 
(b)
کیلکولیٹر کو استعمال کئے بغیر،   ایسے متصل  صحیح اعداد معلوم کیجئے جن کے درمیان 
\( {(\sqrt{3}+\sqrt{2}})^{6}\)
کی قیمت آتی ہو

32.
درج ذیل کی توسیع میں، n  کی شکل میں      x  کے ضربیہ معلوم کیجئے
\({(1+4x)}+{(1+4x)^{2}}+{(1+4x)^{3}}+...+{(1+4x)^{n}}\)
33.
دیا گیا ہے کہ 

\( {a}+{b(1+x)^{3}}+{c(1+2x)^{3}}+{d(1+3x)^{3}}={x}^{3}\), 
x
 کی تمام قیمتوں کے لئے ،  مستقل a   b c            اور       d  کی قیمتیں معلوم کیجئے۔
