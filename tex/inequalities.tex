\باب{عدم مساوات}\شناخت{باب_عدم_مساوات}
یہ باب عدم مساوات کا تعلق اور عدم مساوات کے حل کے بارے میں ہے۔ اس باب کے مکمل ہوتے ہی آپ یہ چیزیں سیکھ جائیں گۓ۔ 
\begin{itemize}
\item
عدم مساوات کی علامتوں کے ساتھ کام کرنے کے اصول سیکھ جائیں گے۔
\item
لکیری عدم مساوات کو حل کرنے کے قابل ہو جائیے گے۔
\item
چوکور عدم مساوات کو حل کرنے کے قابل ہو جائیے گے۔
\end{itemize}
\حصہ{عدم مساوات کے اشارے}
آپ اکثر ایک نمبر کا دوسرے سے موازنہ کرنا چاہتے ہیں اور کہتے ہیں کے کون سا بڑا ہے۔ یہ عدم مساوات کی \عددی{>} ، \عددی{<} علامتوں کے لیے استعمال ہوتے ہیں۔ اور آپ پہلے ہی عدم مساوات کو اباب \حوالہ{باب_تفاعل_اور_ترسیمات} اور \حوالہ{باب_دو_درجی_مساوات} میں ہڑھ چکے ہیں۔

علامت \عددی{>} کا مطلب یہ ہے کہ \(a\) بڑا ہے \(b\) سے۔ آپ اسکی جغرافیائی طور پر تصویر بنائیں۔جیسا کہ تصویر \حوالہء{1.5} ظاہر کرتی ہے کہ تین عدم لکیریں جو \(a\) اور \(b\) کی طرف ظاہر کرتی ہے۔


نوٹ کریں کے اس سے کوئی فرق نہیں پڑتا کہ \(a\) اور \(b\) مثبت ہیں یا منفی۔ \(O\) کی پوزیشن \(a\) اور \(b\) کے سلسلے میں لکیر پر غیر متعلقہ ہے۔ تینوں لکیر پر ---کے طور 6 نیچے اور لکیر پر 7-<4
----------------------
اسی طرح علامت --- کا مطلب ہے کہ \(ab\) سے کم ہے۔ آپ اس کا تصور کر سکتے ہیں کہ جغرافیائی طور پر \(b\) بائیں طرف ہے \(a\) کہ۔
--------------------
یہ چار تاثرات برابر ہیں۔
\(a\)
 بڑا ہے \(b\) سے
\(b\)  
کم ہے \(a\) سے
علامت --- کا مطلب ہے کہ -- بڑا ہے --- سے پر پھر --- برابر ہے --- کے لیکن ---سے کم نہیں ہے۔

یہ تاثرات برابر ہیں۔
\begin{enumerate}[a.]
\item \(a\)
بڑا ہے \(b\) سے یا \(a\) برابر ہے \(b\) کے
\item \(b\)
کم ہے \(a\) سے یا \(b\) برابر ہے \(a\) کے
\end{enumerate}
علامتوں < اور > کو  سخت  عدم مساوات علامتیں کہا جاتا ہے۔ اور اسی طرح --- اور --- کو کمزور عدم مساوات علامتیں کیا جاتا ہے۔
\حصہ{لکیری عدم مساوات کا حل کرنا}
جب آپ عدم مساوات کا حل کرتے ہیں جیسے ---تو آپ کو آسان تر لکھنا پڑتا ہے۔ بلکل اسی معنی کے ساتھ اس معاملے میں آسان بیان نکلا ہے---۔ لیکن آپ پیچیدہ بیان سے سادہ بیان تک کیسے  پہنچے گے۔
\حصہ{دونوں اطراف میں ایک ہیتعداد میں  اضافہ یا گھٹانا}
آپ عدم مساوات کے دونوں اطراف کو ایک ہی تعداد سے جوڑ یا گھٹا سکتے ہیں۔ مثال پر آپ نمبر 11 دونوں اطراف میں شامل کر سکتے ہیں۔ مثال کے طور پر آپ کو مل جاۓ گا
----------------------------------------------
اسی طرح کےاقدام کے جواز میں یہ ظاہر کرنا مشکل ہے کہ کوئی بھی نمبر --- یا پھر -------مثبت ہے۔ یہ کہاں جاسکتا ہے۔ \(O\) کی لکیر پر \(a\) دائیں طرف ہے \(b\) کہ تصویر 2.5 سے ظاہر ہوتا ہے اور یہ سچ ہے چاہیے \(C\) مثبت ہو یا منفی 

چونکہ \(C\)  جمع کرنا۔ \(C\) شامل کرنی کے مترادف ہے، لہزا آپ یہ بھی کر سکتے ہیں کہ دونوں اطراف سے ایک ہی تعداد کو گھٹائیں۔
---------------------
اس مثال میں --- کو دونوں طرف سے گھٹائیں
------
\حصہ{ایک مثبت تعداد کے ذریعے دونوں اطراف سے ضرب کرنا}
آپ عدم مساوات کے دونوں اطراف کو مثبت تعداد کے ذریعہ ضرب (یا تقسیم) کر سکتے ہیں

مثال کے طور پر آپ مثبت نمبر 7 (دونوں کے ضرب) کے ذریعے دونوں اطراف تقسیم کر سکتے ہیں۔ 
---------
یہاں قدم کا جوازپیش کیا گیا ہے اگر \(O\)، <> اور --- پر --- لکیر پر ہے۔

بطور ---،--- دائیں طرف ہے --- کہ لکیر پر

جیسا کہ ---، --- اور --- کے عہدوں کی توسیع --- اور --- کے مطابق ---ہے۔
تصویر 5.3 ظاہر کرتا ہے کہ چاہیے --- اور --- مثبت ہوں یا منفی --- دائیں طرف ہے --- کہ تو---
\حصہ{دونوں اطاف کو منفی تعداد سے ضرب کرنا}
اگر --- اور آپ دونوں اطراف سے \(a+b\) کو منفی کریں۔  اور --- کو حاصل کریں۔ جو --- جیسا ہے۔ یہ ایسا ظاہر کرتا ہے کہ 1۔ عدم مساوات کو دونوں اطراف ظاہر کرتا ہے۔ اور آپ عدم مساوات کی سمت تبدیل کر دیں۔ اور فرض کریں آپ --- کو 2۔ عدم مساوات سے ضرب دینا چایتے ہیں تو یہ ایک جیسا ہے --- کو 2۔ سے ضرب دیں تو ---۔
آپ 2۔ سے ضرب لگانے کے بارے میں سوچ سکتے ہیں۔ جیسے \(a\) اور \(b\) اصل میں ہیں پھر ایک توسیع پزیر کے طور پر 2 سے ضرب کریں۔
آپ  یہ کہہ کر حلاصہ کر سکتے ہیں کہ اگر آپ ضرب (یا تقسیم) عدم مساوات کو دونوں اطراف سے منفی تعداد سے کریں تو آپ کو عدم مساوات کی سمت تبدیل کرنی ہو گئی۔ اگر --- اور --- تو 

\حصہ{عدم مساوات پر آپریشن کا خلاصہ}
\begin{itemize}
\item
آپ عدم مساوات کی دونوں جانب کسی ہندسے کو جمع یا تفریق کر سکتے ہیں.
\item
آپ عدم مساوات کو کسی مثبت ہندسے سے ضرب یا تقسیم کر سکتے ہیں. 
\item
آپ عدم مساوات کو کسی منفی ہندسے سے ضرب یا تقسیم کر سکتے ہیں مگر آپ کو عدم مساوات کی سمت تبدیل کرنا ہوگی. 
\end{itemize}
عدم مساوات کو حل کرنا محض ان تین قاعدوں کا درست استعمال ہے.

مثال

 عدم مساوات کو حل کریں۔
 اس مثال میں آپ کو دونوں اطراف کو تقسیم کرنے کی ضرورت ہے۔ تبدیل کرنے کیلٰے یاد رکھنا عدم مساوات کی سمت ۔۔۔۔۔۔۔۔۔ بن جاتی ہے۔
 
 مثال 
 
 ۔۔۔۔۔۔۔۔۔۔۔۔ عدم مساوات کو حل کریں۔
 

ترتیب میں دونوں اطراف سے ضرب کرنے کے لیے ایک مثبت تعداد سے ضرب لگانے کے بارے میں قاعدہ کا استعمال کریں۔حل کرتے ہوئے کسور کو صاف کریں۔ایک وجہ ہے کہ ایک ہی آپریشن کیا جا سکتا ہے۔ جو عدم مساوات کو متاثر کرتی ہے۔

۔۔۔۔۔۔۔۔۔۔۔۔۔۔۔۔۔۔۔۔۔۔۔۔۔۔۔۔۔۔۔
۔۔۔۔۔۔۔۔۔۔۔۔۔۔۔۔۔۔۔۔۔۔۔۔۔۔۔۔۔۔۔
۔۔۔۔۔۔۔۔۔۔۔۔۔۔۔۔۔۔۔۔۔۔۔۔۔۔۔۔۔۔۔
۔۔۔۔۔۔۔۔۔۔۔۔۔۔۔۔۔۔۔۔۔۔۔۔۔۔۔۔۔۔۔
اس قسم کی عدم مساوات کو حل کرنے کے مترادف ہے ۔ تاہم آپ ضرب لگاتے ہیں یا کسی عدد کو تقسیم کرتے ہیں۔ اگر تعداد منفی ہو تو عدم مساوات کو ختم کرنا ضروری ہے۔ 
۔۔۔۔۔۔۔۔۔۔۔۔۔۔۔۔
۔۔۔۔۔۔۔۔۔۔۔۔۔۔۔
۔۔۔۔۔۔۔۔۔۔۔۔۔۔۔ 

%page 68
مشق 5A

سوال:- عدم مساوات کو حل کریں۔
۔۔۔۔۔۔۔۔۔۔۔۔۔۔۔۔۔۔۔۔۔۔۔۔۔۔۔۔

چوکور عدم مساوات 5.3
چوتھے باب میں آپ نے دیکھا ہے کہ ایک چوکور دک تقریب میں سے تین سے ایک شکل ہو سکتی ہے۔
معمول کی شکل ۔۔۔۔۔۔۔۔۔۔۔۔۔۔۔
عزصر کی شکل ۔۔۔۔۔۔۔۔۔۔۔۔۔۔۔۔
مکمل مربع فارم ۔۔۔۔۔۔۔۔۔۔۔۔۔۔۔۔

اگر آپ ۔۔۔۔۔۔۔۔۔۔۔۔۔۔۔۔ کو چوکور عدم مساوات کو حل کرنے کی ضرورت ہے۔ اب تک استعمال کرنے میں سب سے آسان فارم عزصر کی شکل ہے۔
یہاں کچھ ایسی مثالیں ہیں جو چوکور عدم مساوات کو حل کرنے کے طریقے دکھاتی ہے۔

%page 69


مثال 5.3.1

۔۔۔۔۔۔۔۔۔۔ عدم مساوات کو حل کریں۔
طریقہ 1:-
۔۔۔۔۔۔۔۔۔۔ ترسیم کا خاکہ بنائیں ۔
ترسیم ۔۔۔۔۔۔۔۔ اور ۔۔۔۔۔۔۔۔ پر کاٹتا ہے ۔
۔۔۔ کی قابلیت مثبت ہے ۔ قطع مکافی اوپر موڑتا ہے۔ جیسا تصویر 5.5 میں دکھایا گیا ہے۔
آپ کو ۔۔۔ کی اقدار تلاش کرنے کی ضرورت ہے جیسے۔۔۔۔ ترسیم سے دیکھ سکتے ہیں کہ 4 اور 2 درمیان میں ہے ۔ تو ۔۔۔۔ اور ۔۔۔۔۔ یہ یاد رکھنا ۔۔۔۔ بھی ۔۔۔۔ جیسا ہے ۔ تو آپ دیکھ سکتے ہیں ۔ اسکا مطلب ہے x بڑا ہے 2 سے اور کم ہے 4 سے۔ جب آپ ۔۔۔۔ اور ۔۔۔۔ قسم کی مساوات لکھتے ہیں ۔۔۔۔۔۔۔۔۔۔۔۔۔ کی شکل میں ہے اور ضروری ہے کہ ۔۔۔۔۔۔۔۔ لکھنا کوئ معنی رکھتا ۔ اسی طرح x دونوں سے زیادہ ہو سکتا ہے ۔۔۔۔۔۔۔۔۔۔۔۔۔۔۔۔۔۔۔۔۔۔۔ کو وقفہ کہا جاتا ہے ۔
طریقہ 2:-
۔۔۔ کی اقدار تلاش کریں ۔ جس کے لۓ ۔۔۔۔۔۔۔۔۔
۔۔۔۔۔ اور ۔۔۔۔۔ اقدار کو عدم مساوات کے لیے اہم اقدار کہا جاتا ہے ۔
موضوع میں عوامل کی علامت کو ظاہر کرنے کے لیے ایک انظال بنائیں ۔۔۔۔۔۔۔۔۔۔۔۔۔
۔۔۔۔۔۔۔۔۔۔۔۔۔۔
۔۔۔۔۔۔۔۔۔۔۔۔۔۔
۔۔۔۔۔۔۔۔۔۔۔۔۔۔

مثال 5.3.2 :-
۔۔۔۔۔۔۔۔۔۔۔۔۔۔ عدم مساوات کو حل کریں ۔
تصویر ۔۔۔۔۔۔۔۔۔۔۔۔۔۔ کی ترسیم دکھاتا ہے۔ جیساا کہ ۔۔ کی قابلیت منفیی ہے۔ قطع مکافی کی چوٹی اوپر کی طر                   ف ہے۔ تو ۔۔۔۔ جب   کہ ۔۔۔۔۔۔ اور۔۔۔۔۔۔۔۔۔ ہے۔ نوٹ کریں کہ اس معاملے میں عدم مساوات بھی اہم اقدار ۔۔ اور ۔۔ بھی ربط سے مطمعن ہے۔ 

%page no 70
مثال ۔۔۔
۔۔۔۔۔ جہاں ۔۔۔ ہے۔ عدم مساوات کو حل کریں۔ یہ ۔۔۔۔۔۔ اور ۔۔۔۔۔ کے جیسا ہے۔ ۔۔۔۔ اور ۔۔۔۔ کی اہم اقدار ہے۔
انظال
۔۔۔۔۔۔۔۔۔۔
انظال ۔۔۔ سے پتی چلتا ہے اگر ۔۔۔۔ کو ۔۔۔۔ یہ ۔۔۔ اور ۔۔۔۔ جیسا دکھتا ہے۔
مثال ۔۔۔ کا نتیجہ اہم ہے۔ آپ اسے مزید لکھ سکتے ہیں اگر ۔۔۔ ہے یہ بیانات مساوی ہیں۔
۔۔۔۔۔۔۔۔۔۔  ۔۔۔۔۔۔۔۔۔
جغرافیائی یا انظال کے طریقوں سے استعمال کر کے عدم مساوات کو حل کرنا سب سے آسان ہے۔ آپ اس خاکہ کو کامل کرنے کے لئے جغرافیائی احساب و کتاب کے آلہ کو استعمال کر سکتے ہیں۔ اور یہ عمل سب سے آسان ہے۔
مثال ۔۔۔ ظاہر کرتا ہے کہ کس طرح عدم مساوات کے دلائل کو زیادہ الجبری شکل میں ظاہر کیا جا سکتا ہے۔ 
مثال ۔۔۔
الف ۔۔۔۔۔۔۔
ب ۔۔۔۔۔۔۔
عدم مساوات کو حل کریں۔
اگر دو عوامل کی پیداوار منفی ہے تو ان میں سے ایک منفی ہونا ضروری ہے۔ اور دیگر مثبت کو غور کرنے کے لیے دو امکانات ہیں۔
اگر ۔۔۔۔ منفی ہے تو ۔۔۔ مثبت ہے تو ۔۔۔۔۔ اور ۔۔۔۔ ہے۔ جو کہ نا ممکن ہے۔ لیکن اگر ۔۔۔۔ مثبت اور ۔۔۔۔ منفی ہے تو ۔۔۔۔۔ اور ۔۔۔۔ ممکن ہوتا ہے۔
ب اگر دو عوامل کی مصنوع مثبت ہے تو دونوں مثبت ہیں تو یہ دونوں منفی ہیں۔
اگر ۔۔۔۔ اور ۔۔۔ مثبت ہے تو ۔۔۔ اور۔۔۔۔ بنتا ہے۔ ۔۔۔۔۔ اگر دونوں ۔۔۔ اور ۔۔۔ بنتا ہے۔

%page no 71
مثال ۔۔۔۔
الف ۔۔۔۔۔۔۔۔۔
ب ۔۔۔۔۔۔۔۔۔
الگ الگ طور پر عدم مساوات کو حل کریں۔
۔۔۔۔۔۔۔۔۔ مربع مکمل کریں۔
۔۔۔۔۔۔۔۔ کی سب سے چھوٹی اقدار ۔۔ ہے اور یہ اس وقت ہوتا ہے جب ۔۔۔۔  تو ۔۔ کی کوئی اقدام نہیں ہوتی۔
(ب)  ۔۔۔۔۔۔۔۔۔۔۔۔
الف  مندرجہ ذیل عدالت عدم مساوات کو حل  کر نے کے لئے خاکھ ترمیم کا استعمال کریں 
ب مندرجہ ذیل عدم مساوات کو حل کرنے کے لئے اھم اقدار پر مبنی انظال استعمال کریں  
  ج  مندرجہ ذیل عدم مساوات کو دور کرنے کے لئے الجبری طریقہ استعمال کریں-غیر معقول تعداد کواس میں جوڑیں-اضافے کی شرائط میں × کی عدم مساوات کی اقدار درست ھو سکتی ھے۔ 
د  کوئی بھی طریقہ استعمال کریں عدم مساوات کو حل کرنے کے لئے 

%Page no 72
متنوع دوہرائی                    
سوال 
الف   عدم مساوات کو حل کریں 
ب         عدم مساوات کو حل کریں       
ج  عدم مساوات کو حل کریں    
ہ  عدم مساوات کو حل کریں 


عدم مساوات کو حل کریں 
سوالات   سے  کے جوابات دینے میں امتیازی   کا استعمال کریں-آ پ کو جانچ پڑتال کرنے کی ضرورت پڑ سکتی ہے۔ قدر  الگ الگ 
سوال    کی وہ اقدار تلاش کریں جس کے لئے مندرجہ ذیل مساوات کی دو الگ الگ جڑیں ھیں 
الف   ب    ج 
سوال     کی اقدار کی حد معلوم کریں-جس کے لئے       مساوات کی جڑیں ھیں ۔
سوال     کی اقدار کا سیٹ تلاش کریں ۔جس کے لئے       ھے۔
سوال        اور      کی ترسیم کا خاکھ بنائیں ۔اور                              

%PAGE NO 73
کی اقدار کو اس طرح تلاش کریں کہ لکیر مساوات کے ساتھ وکر سے مل سکے ۔۔۔۔۔۔۔صرف ایک بار۔

۔۔۔۔ اور۔۔۔۔ مساوات کے ساتھ کے ساتھ ایک ہی محور پر منفی خطوط ظاہر کریں اور چورہا کے ان نقاط کو ظاہر کریں۔


ایک میل آرڈر والی فوٹو گرافی والی کمپنی اپنی تصویر تیار کرنے کی خدمت پیش کرتی ہے۔گاہکوں کو یہ شیشے ندیچے اور ائتاطار کے سائز پر مبنی ہے ۔ اس ۔۔ فی میٹر وصول کرتا ہے۔ فریم اور فریم کے گلاس کے لیے فی میٹر مربع ۔۔۔ کے تاثرات لکھیں ۔ ایک فریم میں تصویر کو بڑھا کر اور بڑھتے ہوۓ لاگت کے لیے جو ۔۔ میٹر چوڑا ہے اور ۔۔۔ میٹر اوذچائ کی ایک تصویر کو بڑھیا گیا تھا ۔ اور اس کی قیمت میں سائڑز ۔۔۔ میٹر کے مربع فریم میں کیا گیا ہے ۔ ۔۔۔ کے لیے مربع مساوات کو مرتب کریں اور حل کریں۔

۔۔۔۔۔ کے زریعے سیدھی لکیر کی مساوات تلاش کریں جو لکیر پر ہے۔ ۔۔۔۔۔ اور ۔۔۔۔۔ نقاط سے گزرتا ہے۔ مثلث ۔۔۔ کے علاقے کو تلاش کریں۔ آپ کا جواب آسان تر شکل میں ہونا چاہیے۔

عدم مساوات کو حل کریں ۔
الف۔ ب۔ ج

چوکور مساوات ۔۔۔۔۔۔۔۔۔۔۔ کی ایک جڑ دہرائ ہوتی ہے ۔ P کی ایک ممکن اقدار تلاش کریں۔

بیک وقت مساوات کو حل کریں۔
۔۔۔۔۔۔۔۔۔۔۔۔۔۔۔۔۔۔۔

کے عدم مساوات کو حل کریں۔
د ہرایی کی مشق
۱)ایک لکیران نقطوں میں سے گزرتی ہے۔۔۔۔۔۔۔۔۔۔اور۔۔۔۔۔۔۔۔۔۔۔ہے ۔جوایک مساوات ظاہرکرتی ہے۔ لکیر۔۔۔۔کو۔۔۔۔پر کاٹتی ہے۔جس کی مساوات۔۔۔۔۔۔۔۔۔ہے۔۔۔۔پر۔۔۔کے نقا ط تلاش کریں۔
۲)) یہ ظاہر کریں مساوات کی کو ئی یہی جڑہ۔۔۔۔۔۔۔۔کی جڑہےاور۔۔۔۔۔کی جڑ ہے اور ظاہرکریں۔۔۔۔۔۔۔۔کا کوئی حل نہیں
۳)۔۔۔۔۔۔۔۔کو۔۔۔۔۔۔۔۔کو فارم میں لکھیں۔جہاں۔۔۔اور۔۔۔۔کی اقدارتلا ش کریں۔
 الف     ۔۔۔۔۔۔۔۔۔۔۔کی سب سے کم اقدارلکھیں اور ۔۔۔۔ کی تلاش کریں
ب)۔۔۔۔۔۔۔۔۔۔۔۔۔۔۔۔۔کی اقدارتلاش کریں
۴)۔۔۔۔۔۔۔۔۔۔۔۔۔۔۔۔۔اسان کریں
۵)عدم مساوات کو حل کریں۔
الف                         ب                       ج
۶)د کھائیں کہ مساوات کو۔۔۔۔۔۔۔۔۔۔۔۔۔۔کو۔۔۔۔۔۔۔۔۔۔۔۔۔کی شکل میں لکھاجا سکتا ہے۔لہذا۔۔۔۔کی قیمتکی تلاش کر یں جو مساوات کو پورا کریں

%PAGE NO 74

یہ ثابت کریں کہ نقاط۔۔۔۔۔۔اور۔۔۔پر کونے والے مثلث دائیں کونے میں ہیں۔ اور اسکے علاقے کا حساب لگائیں۔
یہ معلوم کریں کہ لکیر۔۔۔منفی خطوط سے ملتا ہے۔۔۔۔مساوات میں۔۔۔کو کسی چیز سے کٹوتی کر سکتے ہیں۔
۔۔۔ اور۔۔۔ رومبس میں مخالف راس ہیں۔ اسکے وتر کی مساوات تلاش کریں۔ دوسری عمودی حصوں میں سے ایک۔۔۔ہے۔ چوتھا محور تلاش کریں۔
نقاط۔۔۔کے وسطہ نقطہ لکھیں۔
۔۔۔کے فاصلہ کا حساب لگائیں۔
نقطہ۔۔۔دائرے پر ہے۔ ۔۔۔کے قطر ہے اور۔۔۔نقاط ہے۔ جہاں۔۔۔مثبت ہے۔ ۔۔۔کے اقدار کا حساب لگائیں۔
عدم مساوات کو حل کریں۔
مثلث کے دونوں اطراف کی لمبائی۔۔۔ سینٹی میٹر اور۔۔۔ سینٹی میٹر ہے اور انکے درمیان۔۔۔کا زاویہ ہے۔ تیسرے پہلو کا حساب لگائیں اور جواب۔۔۔ کی فارم میں ہونا چاہئے۔

.................. ایک مثلث کے راس ہیں۔۔۔ کے عمودی تنصیف کی مساوات جبکہ ۔۔۔کا عمودی تنصیف معلوم کریں. ۔۔ کے محدد معلوم کریں جہاں یہ خطوط ایک دوسرے کو قطع کرتے ہیں اور  ۔۔۔۔ سے ۔۔ تک کا فاصلہ معلوم کریں. 
.......کا رقبہ معلوم کریں. .... سے.....  تک عمود کی لمبائی معلوم کریں اور ..... اخز کریں کہ......  کا زاویہ.....  ہے.
ایک چوکود میں۔۔۔۔۔۔۔۔ کے راس اور لمبائی دو عمود۔۔۔اور۔۔۔کی مساوات لکھیں۔
۔۔۔ قدر معلوم کرنے کے لیئے۔۔۔متبادل۔۔۔استعمال کریں۔ جو مساوات کو پورا کرتی ہے۔۔۔۔
چوکود کی تقریب۔۔۔ہے۔۔۔۔۔۔۔۔۔۔۔۔۔۔۔کی اقدار تلاش کریں۔
ظاہر کریں۔۔۔ اور اس کی وضاحت کے لیئے۔۔۔ کے قریب کریں۔ حساب و کتاب کا آلہ استعمال کیئے بغیر۔۔۔کی جڑ لکھیں۔
۔۔۔کو۔۔۔اور۔۔۔کی شکل میں لکھیں۔

اشاریہ اشارے میں
اضافی اشارے میں

