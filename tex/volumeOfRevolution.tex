\باب{انقلاب کا حجم}

 یہ باب کسی حجم یا ٹھوس حجم کو تلاش کرنے کے لیے انضمام کے استعمال کے بارے میں ہے۔ جس کو ٹھوس ردعمل کہا جاتا ہے۔جب آپ اس باب کو مکمل کرلیں گے تو آپ \(x\) اور \(y\) محور میں سے کسی ایک کے بارے میں انقلاب کا حجم تلاش کرنے کے قابل ہو جائیں گے۔
\حصہ{انقلاب کی جلدیں}
 \(O\) ایک لکیر پر ہے اور\(O\) ایک مہدا ہے۔
\(OA\)کی ایک لکیر بنائیں۔
جیساتصور 17.11 میں دکھایا گیا ہے۔لائن \(OA\)اور \(x\)۔محور کے سایہ دار دکھائے جانے والے خطے پر غور کریں۔اگر آپ اس خطے کو کے گرد \(360^0\) کے ذریعے گھماتے ہیں تو،یہ ایک ٹھوس شنک نکال دیتا ہے۔ 17۔2 تصور میں اس طرح سے تعمیر ہونے والی شکل کو انقلاب کا ٹھوس کہا جاتا ہے۔ٹھوس انقلاب کے حجم کو بعض اوقات انقلاب کا حجم کہا جاتا ہے۔

ایک خط کے منحنی خطوط کے حساب کتاب کرنے کے لئے متعدد طریقوں سے انقلاب کے حجم کا حساب لگانا یکساں ہے ، اور اس کی مثال ایک مثال سے دی جاسکتی ہے۔

فرض کریں \(y=\sqrt{x}\) کے ترسیم اور \(x=1\) سے \(x=4\) کے ترسیم کے درمیان کے علاقےکو تصویر 17۔3 میں دکھا جا سکتا ہے، \(x\)-محور کے گرد انقلاب کا ٹھوس بنانے کے لیے گھمایا جاتا ہے۔%problem
کلیدی طور پر ایک اور عام سوال پوچھ کر شروع کرنا ہے۔ اسکا حجم \(V\) ہے۔ \(x=1\)سے کسی بھی قدر کی قدر کے انقلاب کا ٹھوس ہے۔ یہ ٹھوس تصویر 17 میں دکھایا گیا ہے۔

فرض کریں \(\delta{x}\) کو بڑھایا ہوا ہے۔ چونکہ \(y\) اور \(V\) دونوں ہی \(x\) کے افعال ہے۔ اسی سے \(y\)  اور \(V\)  میں اضافے کو  \(\delta{y}\) اور \(\delta{V}\) لکھا جاسکتاہے۔ تصویر 17.5 میں رنگین حجم  میں اضافہ \(\delta{V}\)کے درمیان ہے۔ فرض نما نلی کی مقدار کی چوڑائی 6 ریڑی \(y+\delta{y}\) ہے ۔ ان دونوں قرض کا مرکز  تصویر 17۔5 کے دائیں میں دکھایا گیا ہے۔ \(\delta{V}\)؛  \(\pi y^{2}\delta{x}\) اور \(\pi(y+\delta{y})^{2}\) کے درمیان ہے۔ جس سے اسکی پیروی ہوتی ہے۔ \(\frac{\delta{V}}{\delta{x}}\)؛ \(\pi y^{2}\) اور \(\pi(y+\delta{y})^{2}\) کے درمیان میں ہے۔

اب \(\delta{V}\)  کی طرف جاتا ہے اور یہ حصہ 4۔7 کی تعریف میں \(\frac{\delta{V}}{\delta{x}}\), \(\frac{\dif{V}}{\dif{x}}\)  کی طرف جاتا ہے۔ تو \(y+\delta{V}\), \(y\) کی طرف جاتا ہے۔ اور اس کے بعد 
\[\frac{\dif{V}}{\dif{x}}=\pi y^{2}\]
تو  \(V\) ایک ایسا فعل ہے ۔ جس کا ماخوز \(\pi y^{2}\) ہے۔ اور \(y=\sqrt{x}\)
 \(\frac{\dif{V}}{\dif{x}}=\pi x\) ہے.

%problem
اسی طرح
\[V=\frac{1}{2}\pi x^{2} - \frac{1}{2}\pi\]
حجم \(x=4\) تلاش کرنے کے لیے \(V\) کے اظہار کے لیے \(x=4\) کی جگہ لیں۔ تو حجم ہے۔
\[\frac{1}{2}\pi\times4^{2}-\frac{1}{2}\pi = \frac{1}{2}\pi(16-1)=\frac{15}{2}\pi\]
آپ حصہ 16۔3 کو استعمال کر کے آخری حصے کع متعارف کریں گے اور اسے مختصر کریں گے۔
\[V=\int_{1}^{4}\pi y^{2}\dif{x}=\int_{1}^{4}\pi x\dif{x} =\left[\frac{1}{2}\pi x^{2}\right]_{1}^{4}=\frac{1}{2}\pi\times16-\frac{1}{2}\pi\times1=\frac{15}{2}\pi\]
نوٹ کریں کے مثال کے شروع میں جو اسندلال استعمال کیا گیا ہے۔ وہ مکمل طور پر عام تھا اور کسی طرح بھی  اصل وکر کی مساوات پر انعصار نہیں کرتا تھا-جب  \( x=a\) اور \(x=b\) کے درمیان \(y=f(x)\) کا ترسیم ہوتا ہے تو تحت خطہ \(a<b\) \(x\)-محور  کے گرد گھمایا جاتا ہے۔ انقلاب کا ٹھوس کا حجم ہوتا ہے۔
\[\int_{a}^{b}\pi(f(x))^2\dif{x}\quad\int_{a}^{b}\pi y^{2}\dif{x}\]
\ابتدا{مثال}
\(x=-1\) اور \( x=1\) کو \(x\)-محور کے گرد چار دائیں زاویہ سے گھمایا  جاتا ہے۔ اور حجم ٓٓٓ\(y=1+x^{2}\)کے ترسیم کے نیچے پیدا ہوتا ہے۔اسکا حجم تلاش کریں۔

چار دائیں زاویوں کا فقرہ بعض اوقات\(360^0\) کی جگہ پر مکمل بیان کرنے کے لیے استعمال ہوتا ہے۔ اور \(x\)-محور کے گرد گردش کرتا ہے۔تو مطلوبہ حجم \(V\) ہے۔ جہاں
\begin{align*}
V&=\int_{-1}^{1}\pi y^{2}\dif{x}=\int_{-1}^{1}\pi\left(1+2x^{2}+x^4\right)\dif{x}\\
&=\left[\pi\left(x+\frac{2}{3}x^{3}+\frac{1}{5}x^{5}\right)\right]_{-1}^{1}\\
&=\pi\left\{\left(1+\frac{2}{3}+\frac{1}{5}\right)-\left((-1)+\frac{2}{3}(-1)^{3}+\frac{1}{5}(-1)^{5}\right)\right\}=\frac{56}{15}\pi
\end{align*}
یہ معمول کی بات ہے نتیجہ \(\pi\) کے عین مطابق متعدد کے طور پر دیا ہے۔اہم اعداد و شمار یا اشاری جگہوں کی دی گئی تعدادکا صیح جواب دیں۔ اور ثابت کریں کہ بنیاد کے ساتھ ایک شنک کا حجم \(V\) دراس \(r\) اور اوچائی \(V=\frac{1}{3}\pi r^{2}h\) یے۔
شنک دینےکے لیے گھومنے والا مثلث تصویر 17.6 میں دکھایا گیا ہے۔ جسکی اوچائی پورے صفے پر تیار کی گئی ہے۔ اور اسکا میلان---پر ہے جو کہ \(\frac{r}{h}\) ہے اور مساوات \(y=\frac{r}{h}x\) بنتی ہے۔

لہزا یاد رکھے کے \(n\)، \(r\) اور \(h\) ثابت قدم ہیں- اور \(x\) پر انعصار نہیں کرتے ہیں۔
\begin{align*}
V&=\int_{0}^{h}\pi y^{2}\dif{x}=\int_{0}^{h}\pi\left(\frac{r}{h}x\right)^{2}\dif{x}\\
&=\pi\frac{r^{2}}{h^{2}}\int_{0}^{h}x^{2}\dif{x}=\pi\frac{r^{2}}{h^{2}}\left[\frac{1}{3}x^{3}\right]_{0}^{h}=\pi\frac{r^{2}}{h^{2}}\times\frac{1}{3}h^{3}=\frac{1}{3}\pi r^{2}h
\end{align*}
\انتہا{مثال}
\حصہ{\(y\)-محور کے گرد انقلاب کی جلدیں}
ٓتصویر 17.7 مع \(y=f(x)\) کے ترسیم میں درمیان کا علاقہ \(y=c\) اور \(y=d\) ہے۔ اور اسے\(x\)-محور کے گرد گھمایا جاتا ہے۔جو تصویر 17۔8 میں ٹھوس دکھایا گیا ہے۔ \(y\)-محور کے گردٹھوس انقلاب کو تلاش کرنے کے لیے کردار کو تبدیل کریں۔ جوکہ حصہ 17.1 میں \(x\) اور \(y\) کی گفتگو کی گئی ہے۔
خطہ \(y=f(x)\) کے ترسیم سے جڑا ہوا ہے۔ تو لکیر \(y=c\)  اور \(y=d\) , \(y\)-محور کے گرد گھمایا جاتا ہے،تشکیل شدہ ٹھوس حجم ہوتا ہے۔
\[\int_{c}^{d} \pi \ x^{2} \ dy.\]
\ابتدا{مثال}
خطہ \(y=x^{3}\) اور اس کے درمیان \(y\)-محور سے جڑا ہوا ہے۔ تو پیدا شدہ حجم تلاش کریں۔ اور \(y\)-محور کے درمیان  \(y=1\) اور \(y=8\) کو   \(360^0\)  \(y\)-محور کے گرد گھمایا جاتا ہے۔
\begin{align*}
V&=\int_{1}^{8} \pi \ y^\frac {2}{3} \ dy =\pi [\frac {3}{5} y^\frac {5}{3}]_{1}^{8} =\pi(\frac{3}{5}\times 8^\frac{5}{3}) - \pi(\frac{3}{5}\times 1^\frac{5}{3})\\
&= \pi (\frac {3}{5}\times 32)-\pi(\frac{3}{5}\times 1)=\frac{93}{5}\pi\\
\end{align*}

\انتہا{مثال}
\ابتدا{مشق}
اس مشق کے تمام سوالات کو اپنے جوابات میں \(\pi\) کی ضرب کے طور پر لکھیں۔
\begin{enumerate}[a.]
\item
جب خط  \(x=a\)کے درمیان \(y=f(x)\) کے ترسیم کے پیدا ہوتا ہے۔ تب حجم تلاش کرے \(x=b\)کو \(360^0\) کے ذریعے  \(x\)-محور کے گرد گھمایا جاتا ہے؟
\begin{multicols}{2.}
\begin{enumerate}[a.]
\item 
\(f(x)=x; \quad a=3, \quad b=5\)
\item
 \(f(x)=x^{2}; \quad a=2,\quad b=5 \)
\item 
\(xf(x)=x^{3};  \quad a=2, \quad  b=6 \)
\item
\(f(x)= \frac {1}{x} ; \quad a=1,\quad b=4\)
\end{enumerate}
\end{multicols}
\item
جب حجم  \(x=a\) اور \(y=f(x)\) کے درمیان ترسیم کے نیچے بناۓ گئے۔ حجم کا پتہ لگائیں۔ \(x=b\) کو \(360^0\) \(x\)-محور کے گرد گھمایا جاتا ہے۔
\begin{multicols}{2.}
\begin{enumerate}[a.]
\item
\(f(x)=x+3;\quad a=3, \quad b=9 \)
\item
 \(f(x)=x^{2}+1;  \quad a=2, \quad b=5 \)
\item
\(f(x)=\sqrt {x+1} ;  \quad a=0, \quad b=3 \)
\item
\(f(x)= x(x-2) ; \quad a=0, \quad b=2\)
\end{enumerate}
\end{multicols}
\item
جب خطہ \(y\)-محور اور \(y=f(x)\)  کے ترسیم کے ساتھ جڑا ہوا ہو۔  تب پیدا شدہ حجم تلاش کریں۔ اور \(y=c\)  اور \(y=d\)  کی لکیر کو \(y\)-محور کے گرد گھمایا جاتا ہے۔ تا کہ ٹھوس رستہ نکالا جا سکے۔ 
\begin{multicols}{2.}
\begin{enumerate}[a.]
\item
\(f(x)=x^{2};  \quad c=1, d=3  \)
\item
\(f(x)=x+1;  \quad c=1, d=4 \)
\item
\(f(x)=\sqrt {x} ; \quad c=2, d=7  \)
\item
\(f(x)= \frac {1}{x} ; \quad c=2, d=5\)
\item
\(f(x)= \sqrt {9-x};  \quad c=0, d=3\)
\item
\( f(x)=x^{2}+1;  \quad c=1, d=4 \)
\item
\(f(x)=x^{\frac {2}{3}} ;  \quad c=1, d=5\)
\item
\(f(x)= \frac {1}{x} +2 ; \quad c=3, d=5\)
\end{enumerate}
\end{multicols}
\item
ہر معاملے مین خطا مندرجہ ذیل منحنی خطوط  اور \(x\)-مھور کے درمیانمنسلک ہوتا ہے۔ \(x\)-محور کے گرد \(360^0\) کے ذریعے پیدا کردہ ٹھوس کا حجم تلاش کریں۔
\begin{multicols}{2.}
\begin{enumerate}[a.]
\item
\(y=(x+1)(x-3)\)
\item
\(y=1-x^{2} \)
\item
\(y=x^{2}-5x+6 \)
\item
\(y= x^{2}-3 \)
\end{enumerate}
\end{multicols}
\item
\(y=x \) اور \(y=x^{2}\) کے ترسیموں کے درمیان منسلک خطے \(R\) کے ذریعے گھمایا جاتا ہے تو جو حجم ہوتا ہے، اسے تلاش کریں۔
\begin{multicols}{2.}
\begin{enumerate}[a.]
\item
\(x\)-محور کے گرد
\item
\(y\)-محور کے گرد
\end{enumerate}
\end{multicols}
\item
\(y=x^{2}\) اور \(y=4x \) کے ترسیموں کے درمیان منسلک خطے \(R\) کے ذریعے گھمایا جاتا ہے تو جو حجم ہوتا ہے، اسے تلاش کریں۔
\begin{multicols}{2.}
\begin{enumerate}[a.]
\item
\(x\)-محور کے گرد
\item
\(y\)-محور کے گرد
\end{enumerate}
\end{multicols}
\item
 \(y=\sqrt{x} \) اور \(y=x^{2}\) کے ترسیموں کے درمیان منسلک خطے \(R\) کے ذریعے گھمایا جاتا ہے تو جو حجم ہوتا ہے، اسے تلاش کریں۔
\begin{multicols}{2.}
\begin{enumerate}[a.]
\item
\(x\)-محور کے گرد
\item
\(y\)-محور کے گرد
\end{enumerate}
\end{multicols}
\item
گلاس کا پیالہ \(y\)-محور کے ترسیموں کے مابین اس علاقے کے گرد گھماتے ہوۓ تشکیل دیا جاتا ہے۔

\(y=x^{2}\) اور \(y=x^{3}\) پیالے میں شیشے کی مقدار مرلوم کریں۔
\item
یہ خط دونوں محوروں سے منسلک ہے۔ لکیر \(x=2\) اور وکر \(y=\frac{1}{8} x^{2}+2\) کے اردگرد گھمایا گیا ہے۔ایک محور بنانے کے لیے \(y\)-محور ٹھوس کا حجم تلاش کریں۔
\end{enumerate}


\انتہا{مشق}
\ابتدا{مشق}
\begin{enumerate}[a.]
\item
یہ خط وکر \(y=x^{2}+1 \) \(x\)-محور اور لکیر \( x=2\) سے جڑا ہوا ہے۔ \(x\)-محور کے گرد گھمایا گیا ہے۔ \(\pi \) اور \(\pi \) کے رحاظ سے تشکیل شدہ ٹھوس کا حجم تلاش کریں۔
\item
یہ وضاحت کریں کے نقاط \(x,y\) مرکزہ ایک مطمئن دراس کی مساوات \(x^{2}+y^{2}=a^{2} \) کی نشاندہی کریں۔ \(x\)-محور کت نیم کے اوپر دائرہ گھمایا جاتا ہے۔\(360^0\) کے ذریعے \(x\)-محور کو گھمایا جاتا ہے۔دراس کا دائرہ \(a\) کی وضاحت کریں۔اضاحت کریں کے حجم \(V\) کیوں ہے۔ اس دائرہ کا \(V\) مزجانب دیا گیا ہے۔
\[V=2 \pi \int_{0}{a} (a^{2}-x^{2})dx. \]
یہ ثابت کریں \(V=\frac{4}{3} \pi a^{3} \)
\item
مساوات والا بیضوی \(\frac {x^{2}}{a^{2}} + \frac {y^{2}}{b^{2}}=1 \)  تصویر میں دکھایا گیا ہے۔\(a\) اور \(b\) کا محور ایک ہی ہے۔ \(a^{2}\) اور \(b^{2}\) بہضوی شکل بنانے کے لیے \(x\)-محور کے گرد گھمایا جاتا ہے۔اس بیضوی کا حجم تلاش کریں۔ \(b\) بناتے ہوۓ بیضوی کی مقدار کم کریں۔ اور \(y\)-محور کے گرد گھمایا جاے۔

\item
تصویر میں \(y=x^-{\frac{2}{3}}\) عکر دکھایا گیا ہے۔
\begin{enumerate}
\item
دکھائیں کے سایہ دار علاقہ \(A\) لامحدود ہے۔
\item
رنگیں علاقہ \(B\) تلاش کریں۔
\item
\(A\)
 رقبہ کے گرد \(360^0\)  کے ذریعے گھمایا جاتا ہے۔\(x\)-محور حجم تلاش کریں۔
\item
علاقہ \(B\) \(360^0\) کے ذریعے گھمایا جاتا ہے۔ \(y\)-محور حجم تلاش کریں۔
\end{enumerate}
\item
مساوات کاعلاقہ سوال 4 میں دیا گیا ہے۔ ان کی مساوی علاقوں اور جلدوں کی تعقیقات کریں۔
\[(i)\>\>\>\>y=x^-{\frac{3}{5}},  \hspace{2.5cm} (ii)\>\>\>\>y=x^-{\frac{1}{4}}.\]
\item
نقطہ موڑ اور نقاط کے بتاۓ ہوۓ وکر \(y=9-x^{2} \) کا خاکہ بنائیں۔ محور کے ساتھ چو رہا یے۔ محدود خط جس میں منحنی خطوط پر مشتمل ہوتا ہے۔ اور \(x\)-محور \(R\) کے زریعے ظاہر ہوتا ہے۔
\begin{enumerate}
\item
 \(R\)
کا رقبہ تلاش کریں اور اسی وجہ سے دوسری صورت میں --- تلاش کریں۔
\item
جب \(R\) کو \(360^0\) کے ذریعے گھمایا جاتا یے، تو حاصل کی جانے والی ٹھوس  انقلاب کا حجم \(x\)-محور کے گرد تلاش کریں۔
\item
جب \(R\) کو \(360^0\) کے ذریعے گھمایا جاتا یے،تو حاصل کی جانے والی ٹھوس  انقلاب کا حجم \(y\)-محور کے گرد تلاش کریں۔
\end{enumerate}
\item
خطے کو منحنی خطوط وکر \(y=(x-2)^{\frac{3}{2}}\) ہے۔ جس کے لیے \(2 \le x \le 4\) ہے۔ جو \(x\)-محور کے ساتھ ہے۔ \(x=4\) تلاش کریں۔ \(\pi\) کے لہاظ سے حاصل کردہ ٹھوس کا حجم جب \(R\) ہوتا ہے۔ \(x\)-محور کے گرد چار زاویوں سے گھمایا جاتا ہے۔
\end{enumerate}
\انتہا{مشق}
