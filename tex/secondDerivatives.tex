\باب{دہرا تفرقات}\شناخت{باب_دہرا_تفرقات}
سبق مشتق کے اگلے تصور کو پیش کرتا ہے۔ اِس سبق کو مکمل کرنے کے بعد ،    آپ اِن باتوں کے اہل ہوجائینگے۔\\
ترسیمات کی ساخت اور اُن کے حقیقی دُنیا میں اطلاق کے لئے ،  دو درجی   مشتق کی افادیت کو سمجھنا۔\\
نقطہ  ِعظیمت  اور نقطہ ِ اقلیت  کے درمیان امتیازی فرق کو سمجھنے کے لئے دو درجی مشتق کو استعمال کرنا۔\\
نقطہ  ِ موڑ   پر دودرجی مشتق کےصفر ہوجانے کے تصور کو سمجھنا۔ \\
 15.1ترسیمات کی تیاری اور اُنکے مفہوم\\
سبق نمبر 7  میں حاصل ہونے والے نتائج ،  کسی تفاعل کی خصوصیات اور مشتق کی قیمتوں کے درمیانی تعلق،  صرف اُن تفاعل تک ہی محدود تھے جو کہ اپنے اپنے دائرہ کار   میں مسلسل   ہوتے تھے۔   اُن تمام نتائج میں اِس بات کو استعمال کیا گیا تھا کہ ترسیم کے کسی خاص نقطے پر مشتق کی قیمت،   صرف اُس نقطے پر تفاوت    کی پیمائش ہی نہیں کرتا ہے بلکہ وہ خود ایک تفاعل کے طور پر تصور کیا جاتا ہے۔ \\
اِس سبق میں ہمیں مزید ایک پابندی لگانی پڑے گی ۔    اُن تفاعل پر جنکے ترسیم میں اچانک تبدیلی نہیں ہوتی ہے،    اُن تفاعل کو ہموار تفاعل    کہا  جاتا ہے۔   یعنی مثال کے طور پر ،   ایک تفاعل   
\(x^{\frac{3}{2}}\left(1-x\right) \)
کے لئے ،  آپ کو اُس کے دائرہ کار میں سے  نقطہ ِعجب       کو باہر نکال دینا ہوگا،  جوکہ اِس مثال میں مبدا ہے ۔  (مثال   7.2.3 سے )\\
ہموار ہونے کی شرط سے ظاہر ہوتا ہے کہ مشتق ،   جوکہ  خود ایک تفاعل ہے،  مسلسل ہے   اور اُس کا تفرق لیا جاسکتا ہے۔ اُس کے نتیجے کو  دو  درجی مشتق  کہا جاتا ہے۔    اُسے عام طور پر 
\(f^{\prime\prime}\left(x\right) \)
  سے ظاہر کیا جاتا ہے۔  اِسی طرح سے اُسے  
 \(\frac{d^{2}y}{dx^{2}} \)
   سے بھی ظاہر کرتے ہیں۔ \\
مثال   
\(y=f\left(x\right)=x^{3}-3x^{2} \)
   15.1.1     کے ترسیم میں،  اُن وقفوں کی شناخت کیجئے جہاں 
\(f^{\prime}\left(x\right) \)
   
\(f\left(x\right) \)
    اور 
\(f^{\prime\prime}\left(x\right) \)
     مثبت ہوتے ہوں،  اُن کا ترسیمی مفہوم بیان کیجئے۔ \\
\[\frac{dy}{dx}=f^{\prime}\left(x\right)=3x^{2}-6x \]

\[\frac{d^{2}y}{dx^{2}}=f^{\prime\prime}\left(x\right)=6x-6 \]
خاکہ  15.1  میں اِس تفاعل ،  اُسکے پہلےمشتق  اور دوسرے مشتق کی ترسیمات دکھائی گئی ہیں۔ \\
اس خاکہ سے ظاہر ہوتا ہے کہ  تفاعل
\(f(x)=x^2(x-3)\)
  میں ،  
  جب 
\(,x>3\)
     ہو تب 
\(,f(x)>0\)
     ہوتا ہے۔  xکی اِن تمام قیمتوں کیلئے کی ترسیم،   - xمحور کے اُوپر حاصل ہوتی ہے۔ \\
اِسی طرح سے تفاعل 
\(f'(x)=3x(x-2)\)
میں ،  جب
\(x>2\)
 یا  
 \(x<0\)
  ہو تب ہوتا ہے۔
   f(x)
 کی ترسیم میں،  اِس وقفے میں تفاوت کی قیمت مثبت ہوتی ہے، تاکہ f(x) کی قیمت بڑھتی جائے۔ \\
آخر میں،  تفاعل 
\(f''(x)=6(x-1)\)
میں،  
جب
,\(x>1\)
 ہو تب
  \(f''(x)>0\)
  ہوتا ہے۔ ایسا ظاہر ہوتا ہے کہ اِس وقفے میں f(x)  کی ترسیم اُوپر کی جانب منحرف ہوتی ہوئی دکھائی دیتی ہے۔ \\
"اُوپر کی جانب منحرف ہونے"   کے اِس تصور کو آسانی سے سمجھنے کیلئے ، تفاوت کیلئے حرفg  کو استعمال کرتے ہیں\\
 یعنی
\(g=f'(x)\)
 ہوتا ہے۔  اِسی طرح سے 
\(f''(x)=\frac{\dif g}{\dif x}\)
 ہوتا ہے،   جو کہx  کی مناسبت سے تفاوت کی تبدیلی کی شرح کو ظاہر کرتا ہے۔ جس وقفے میں
  \(f''(x)>0\)
 ہوتا ہے،  وہاں تفاوت کی قیمت بڑھتی جاتی ہے،  جیسے جیسے x   کی قیمت بڑھنے لگتی ہے۔ \\
درج  ِ بالا خاکہ  15.1   میں درمیانی ترسیم میں اِسے دیکھا جاسکتا ہے،  جو کہ ایک مربعی ترسیم ہے جس کا نقطہ راس ((1, -3   ہے۔\\
اسی لئے اس ترسیم کے بائیں جانب تفاوت کی قیمت نقطہ (1,-2)  پر بڑھتے ہوئے 3-  ہوجاتی ہے۔  نقطہ اقلیت (2,-4)  سے گزرتے ہوئے صفر ہوجا تی ہے اور پھر بڑھ کر مثبت ہوجاتی ہے۔ اس کے بعد جب x > 2  ہوتا ہے تو لگاتار بڑھنے لگتی ہے۔ \\
درج  ِ بالا خاکہ  15.2   میں تین منحنی دکھائے گئے ہیں۔  اگر 
  \(f''(x)>0\)
ہوتو  اُوپر کی جانب انحراف ہوتا ہے  اور اگر
  \(f''(x)<0\) 
  ہوتو  نیچے کی جانب انحراف ہوتا ہے۔   یہاں یہ بات نہایت اہمیت کی حامل ہے کہ یہ خاصیت ہمیشہ تفاوت کی علامت پر منحصر نہیں ہوتی ہے۔ ایک منحنی اُوپر کی جانب منحرف ہوسکتی ہے  اگر  اُس کا تفاوت مثبت  ہو یا  منفی  ہو  یا  صفر ہو۔ \\
مثال نمبر  15.1.2  \\
  اگر  y = f(x)     جہاں  
  \(f(x)=\frac{1}{x}- \frac{1}{x^{2}},x>0\)
   ہو اور اُس کا دائرہ کار   
   \(x>0\)
    ہو۔ تفاعل  f (x) کی تفتیش کیجئے۔\\
دیئے گئے تفاعل 
\(f(x)\)
   کو آپ یا تو 
   \( \frac{x-1}{x^{2}}  \)

       اس طرح  سے  یا   
    \( x^{-1}-x^{-2} \)

        اس طرح سے  لکھ سکتے ہیں۔\\
اسی لئے ،
\[f'(x)=-x^{-2}+2x^{-3}=- \frac{1}{x^2}+\frac{2}{x^3}=\frac{-x+2}{x^3}\]
اور  
\[f''(x)=2x^{-3}-6x^{-4} =\frac{2}{x^3}-\frac{6}{x^4} =\frac{2(x-3)}{x^4}\]
  ہوتے ہیں۔ 
دیئے گئے دائرہ کار میں، ایسا لگتا ہے کہ                      
 \[
\begin{array}{lcl}
f(x)<0,x<1 & \text{and} & f(x)>0,x>1; \\
 f'(x)>0,x<2  & \text{and} & f'(x)<0,x>2; \\
 f''(x)<0,x<3  & \text{and} &  f''(x)>0,x>3 
\end{array}
\]    
اس لئے اِس کی ترسیم   محور کے نیچے ہوتی ہے اگر  
\( 0<x<1\)
 ہو اور    محور کے اُوپر ہوتی ہے اگر   
 \(x>1\)
  ہو۔ اور یہ ترسیم   محور کو نقطہ   پر قطع کرتی ہے۔ اِس کی تفاوت مثبت ہوتی ہے اگر   
  \(0<x<2\)
  ہو اور منفی ہوتی ہے اگر 
  \(x>2\)
     ہو۔ اس دوران اُس کا نقطہ عظیمت  
     \((2,\frac{1}{4})\)
       ہوتا ہے۔  اور یہ ترسیم   
       \(0<x<3\)
       کیلئے نیچے کی جانب منحرف ہوتی ہے اور 
       \(x>3\)
         کیلئے اُوپر کی جانب منحرف ہوتی ہے۔ \\
	یہ تمام معلومات کافی ہیں جن کے ذریعئے،  کی دیئے گئے وقفے میں،   فاضل قیمتوں  کیلئے  ترسیم کی ساخت کا تصور سمجھا جاسکتا ہے۔ لیکن تفتیش مکمل کرنے کیلئے یہ ضروری ہوجاتا ہے کہ   کی بہت چھوٹی اور بہت بڑی قیمتوں کیلئے ترسیم کیسی ہوگی۔ اس کیلئے درج  ِ ذیل تحسیب کرنا چاہیئے مثلاً \\
	  \(f(0.01)=100-10000=-9900\)

   اور   
     \(f(100)=0.01-0.0001=0.0099.\)\\
اس سے ظاہر ہوتا ہے کہ جب   x کی قیمت چھوٹی ہوتی ہے تب    y بہت بڑی قدر  کیساتھ منفی ہوتی ہے۔ اور جب  y کی قیمت بڑی ہوتی ہے تب  x بہت چھوٹا لیکن مثبت عدد ہوتا ہے۔ \\
نوٹ:۔  اِس مثال میں دی گئی معلومات کا استعمال کرکے آپ خود اِس کی ترسیم بنانے کی کوشش کیجئے۔ اگر آپ کے پاس ترسیمی تحسیب کار  ہوتو اُسے استعمال کرکے اپنے بنائے ہوئے ترسیم کی جانچ کیجئے۔ \\
ترسیم تیار کرنے کی یہ صلاحیت دراصل اُن نقاط کی مشق کرنا ہے جن کے محدد کچھ معنی رکھتے ہوں۔ مثال 15.1.2   میں    نقطہ ہماری توجہ کا مرکز ہوتا ہے جہاں ترسیم   محور کو قطع کرتی ہے اور نقطہ   
\((2, \frac{1}{4})\)
  جوکہ اس ترسیم کا نقطہ عظیمت ہوتا ہے۔  ایک اور دلچسپ نقطہ  
  \((3,\frac{2}{9}\)
   بھی ہے جہاں ترسیم نیچے کی جانب انحراف سے تبدیل ہوکر اُوپر کی جانب انحراف میں تبدیل ہوجاتا ہے۔ یہاں نوٹ کیجئے کہ اِسی نقطہ پر  
   \(f''(x)\)
     کی قیمت بھی منفی سے مثبت ہورہی ہے،  اور 
     \(f''(3)=0\)
       ہوتا ہے۔ \\
کسی بھی ترسیم کا ایسا نقطہ ، جہاں ترسیم ایک جانب انحراف سے تبدیل ہوکر دوسری جانب انحراف دکھاتا ہے اُسے اُس ترسیم کا نقطہ  موڑ  کہتے ہیں۔
اگر کسی ترسیم میں نقطہ 
\(p.f(q)\)
ہوتا ہے۔  
،  نقطہ موڑ کے طور پر موجود ہو تو  اُس نقطہ پر  ہوتا ہے۔   \\
 .215دو درجی مشتق کا عملی استعمال\\
	حقیقی دُنیا میں کئی حالتوں میں دو درجی مشتق کافی اہم ہوتے ہیں ، کیونکہ اِن کے ذریعئے ہم پہلے سے ہی مستقبل کی راہیں متعین کرسکتے ہیں۔ 
		مثال کے طور پر ،  پچھلے کئی وقتوں سے کمپیوٹروں کو گھریلو استعمال کافی بڑھ رہا ہے۔ کمپیوٹر تیار کرنے والے کارخانہ داروں نے    t سالوں میں H   کمپیوٹرس تیار کرنے کا تخمینہ کیا ۔  ایسی حالت میں  وقت اور کمپیوٹرس کی تعداد کے درمیان تیار ہونے والے ترسیم کی تفاوت
		\(\frac{\dif{H}}{\dif{t}}\)
		مثبت ہوگی۔  لیکن کمپیوٹرس تیار کرنے کی یہ شرح آگے بھی بڑھ رہی ہے یا کم ہورہی ہے اِسے معلوم کرنے کیلئے کارخانہ داروں کو    
		\(\frac{d^{2}H}{dt^{2}}\)
		کی قیمت معلوم کرنا پڑے گا۔  (اگر کمپیوٹرس کی کھپت کی شرح منفی حاصل ہوتو کارخانہ داروں نے اپنے کمہیوٹرس کی کوالیٹی پر غور کرنا ہوگا۔) اِس طرح کے حالات میں   کی قیمت کا کافی اثر پڑتا ہے۔ اِسی طرح سے اگر محکمہ موسمیات والے وقتt  میں ہوا کے دباو  pکی قیمت کے ذریعئے زیادہ یقین کے ساتھ معلومات نہیں دے سکتے اگر  منفی ہو۔  لیکن اگر اُنہیں   
		\(\frac{d^{p}}{dt^{2}}\)
		کی قیمت بھی منفی مل جائے تو وہ یقین کے ساتھ کہہ سکتے ہیں کہ موسم میں زبردست تبدیلیاں رونما ہونے والی ہیں۔ \\
اس مشق میں ،  پہلی اور دوسری مشتق کی معلومات کو استعمال کرکے ترسیم تیار کیجئے۔ اگر آپ ترسیم تیار کرلیتے ہیں تو ترسیمی عداد  کو استعمال کرکے اپنی ترسیم کی جانچ کیجئے۔ \\
1۔  y = f(x) جہاں
\( f(x)=x^3-x\)
 ہو کے ترسیم پر غور کیجئے۔\\
  اِس حقیقت کو استعمال کرکے معلوم کیجئے کہ -xمحور کو ترسیم کس نقطہ پر قطع کرتا ہے؟  اُس کا ترسیم بھی بنائیے۔ \\
 	  \(f(x)=x(x^2-1)=x(x-1)(x+1)\)
	(b)   \(y=f'(x)\)  معلوم کیجئے اور  \(y=f'(x)\) کی ترسیم بنائیے۔ \\
	(c)  \(y=f''(x)\) معلوم کیجئے اور  \(y=f''(x)\)  کی ترسیم بنائیے۔
	(d) اپنے تیار کئے گئے ترسیمات کی مستقل مزاجی  معلوم کیجئے۔ مثال کے طور پر،y = f(x) کے ترسیم کی جانچ کیجئے کہ اگر    ہوتو ترسیم اُوپر کی جانب منحرف ہوتی ہے۔ \\
2۔ 
\[y=x^{3}+x\]
 کی ترسیم کے لئے،\\
	(a) اجزائے ٖضربی  کو استعمال کرکے ثابت کیجئے کہ ترسیم -
	محور کو صرف ایک بار قطع کرتا ہے۔ \\
	(b)   \(\frac{dy}{dx}\)
	 اور  
	 \(\frac{d^{2}y}{dx^{2}}\)
	  کی قیمتیں معلوم کیجئے۔ \\
	(c) وہ وقفہ معلوم کیجئے جہاں ترسیم اُوپر کی جانب منحرف ہورہی ہے۔\\
	(d) \[y=x^{3}+x\]
	 کی ترسیم سے حاصل ہونے والی معلومات کو استعمال کیجئے۔ \\
3۔ y = f(x)  کی ترسیم تیار کرنے کے لئے  اور   کی معلومات استعمال کیجئے جہاں\\
\(f(x)=x^{3}-3x^{2}+3x-9\)
(نوٹ کیجئے کہ
  \(x^3-3x^2+3x-9=(x-3)(x^2+3)\)\\
4۔ مندرجہ ذیل کی ترسیمات بنائیے اور اُن نقاط کے محدد معلوم کیجئے جہاں
 \(\frac{dy}{dx}=0\)
  اور 
  \(\frac{d^2y}{d^2x}=0\)
   ہوں۔\\
    
 \begin{multicols}{3}
  \begin{enumerate}[.a]
  \item \( y=x^4-4x^2 \) 
  \item \( y=x^3+x^2 \)
  \item \( y=x+\frac{1}{x} \)
  \item \( y=x-\frac{1}{x} \) 
  \item \( y=x+\frac{4}{x^2} \)
  \item \( y=x-\frac{4}{x^2} \) 
  \end{enumerate}
  \end{multicols}
5۔  (a)  مندرجہ ذیل ترسیم قیمت (P) اور وقت (t) کے درمیان تیار کی گئی ہے۔ افراط  ِ زر کی شرح   
\(\frac{dp}{dt}\)
بڑھ رہی ہے۔ اس ترسیم میں  
\(\frac{d^{2}p}{dt^{2}}\)
 کیا ظاہر کرتا ہے اور اُس کی قیمت کے متعلق کیا کہا جاسکتا ہے؟\\
(b) ترسیم بنائیے جس میں دکھایا گیا ہو کہ قیمتیں بڑھ رہی ہیں۔  لیکن افراط  ِ زر کی شرح کم ہوتی جارہی ہے جس کا مکمل اضافہ 20   کی طرف جارہا ہے۔\\ 
۔  y = f(x) کی ترسیمات کے لئے f `(x) اور f ``(x)  کی مثبت یا منفی علامتیں لکھئے۔ (e)  اور   (f)میں آپ کو متعلقہ وقفے کی حالت کی بھی ضرورت پڑے گی۔ \\
۔ درج  ِ ذیل ترسیم ایک کمپنی کے شیئرس کی قیمتیں S دکھاتے ہیں۔\\
(a) اس ترسیم کے ہر مرحلے کےلئے 
\(\frac{dS}{dt}\)
اور
\(\frac{d^{2}s}{dt^{2}}\)
 کے متعلق اظہار ِ خیال کیجئے۔ \\
(b) \\غیر تکنیکی الفاظ میں وضاحت کیجئے کہ اس ترسیم میں کیا واقع ہورہا  ہے؟
8۔ کولین اپنی اسکول کے لئے نکل چکا ہے،  جو کہ اُس کے گھر سے 800میٹر فاصلے پر واقع ہے۔  اُس کی رفتار،  باقی بچے ہوئے فاصلے کے ساتھ راست تناسب میں ہوتی ہے۔ فرض کریں کہx  میٹرس کا فاصلہ اُس نے طے کرلیا ہے اورy  میٹرس کا فاصلہ ابھی باقی ہے۔ \\
	(a)X
	  بالمقابل t
	اور
	y
	بالمقابل 
	t
	کے ترسیمات بنائیے\\
	(b)
	  \(\frac{dx}{dt}\)
	  \(\frac{d^{2}x}{dt^{2}}\)
	  \(\frac{dy}{dt}\)
	 اور
    \(\frac{d^{2}y}{dt{^2}}\)
	  کی علامتیں کیا ہونگی؟\\
9۔ ایک تابکار عنصر کے انحطاط کی شرح ، دیئے گئے وقت t پر ،  اُس میں موجود جوہروں کی تعداد  Nکے ساتھ راست تناسب میں ہوتی ہے۔\\
	(a)	\\اِس معلومات کو ظاہر کرنے کے لئے ایک مساوات لکھئے۔
(b)	 Nبالمقابل  t کے لئے ترسیم بنائیے۔
	(c)	 \(\frac{d^{2}N}{dt^{2}}\)
	  کی علامت کیا ہوتی ہے؟\\
	
10۔ درج  ِ ذیل تمام معاملات کے لئے y = f(x)کی ترسیمات کے مختلف حصوں کے خاکے تیار کیجئے۔ (مثال کے طور پر، (a) میں،  آپ صرف -yمحور کے قریب والے حصے کی ترسیم بنا سکتے ہیں کیونکہx  کی دگر قیمتیں نہیں دی گئی ہیں۔ )\\
  \begin{multicols}{2}
  \begin{enumerate}[.a]
  \item \( f(0)=3, \,  f'(0)=2, \,f''(0)=1 \)
  \item \( f(5)=-2, \, f'(5)=-2, \,f''(5)=-2 \)
  \item \( f(0)=-3, \, f'(0)=0, \,f''(0)=3\)
  \end{enumerate}
  \end{multicols}
3 اقلیتی  اور اعظم  قیمتوں پر نظر ِ ثانی \\
	پچھلی مشق میں ، آپ نے کچھ مقامات پر دیکھا ہوگا کہ معلومات کے مختلف ٹکڑے آپس میں منضبط ہوتے ہیں۔ یہ بات خاص طور پر اُن نقاط پر بالکل صحیح ثابت ہوتی ہے جہاں ترسیم کی قیمت یا تو اعظم ہو یا اقل ترین۔ اگر آپ نے نشاندہی کی ہوگی کہ اقلیتی نقطے  پر f `(x) کی علامت تبدیل ہوتی ہے، تب آپ نے یہ بھی دیکھا ہوگا کہf ``(x)  سے ترسیم اُوپر کی جانب منحرف ہوتی ہے۔ \\
	خاکہ .215 میں ایک عام نتیجہ سکھایا گیا ہے:\\
اگر   
   \(f'(q)=0\)
اور
    \(f''(q)>0\)
   ہوں تب
     \(x=q\)
       پر اقل ترین نقطہ حاصل ہوگا۔ \\
اگر 
\(f'(q)=0\)
  اور
  \(f''(q)<0\)
     ہوں تب  
        \(x=q\)
       پر اعظم ترین نقطہ حاصل ہوگا۔ \\
اِسے اکثر اوقات نہایت آسانی سے استعمال کیا جاسکتا ہے بجائے اِس کے کہ یہ دیکھنا کہ جس نقطہ پر   کی علامت تبدیل ہوتی ہے وہاں ترسیم کا اعظم یا اقل ترین نقطہ ہوتا ہے۔  \\
دفع 7.3  میں دکھائے گئے طریقہ کار کو ، درج ذیل انداز میں ترمیم کیا جاسکتا ہے۔ \\
 \(y=f(x)\)
 کی ترسیم کے لئے اعظم نقطہ یا اقل ترین نقطہ معلوم کرنا۔\\
مرحلہ نمبر (1):۔ اُس دائرہ کار کو متعین کیجئے جس میں آپ دلچسپی رکھتے ہوں۔\\
مرحلہ نمبر (2):۔ f `(x) کے لئے ایک فقرہ (Expression) معلوم کیجئے۔ \\
مرحلہ نمبر (3):۔ اُس دائرہ کار میںx  کی قیمتوں کی فہرست بنائیے جن کے لئے f `(x)  کی قیمت صفر ہو۔  (اگر وہاں حاصل ہونے والی قیمتوں کے لئے f `(x) غیر معروف  ہو، تب دفع 7.3  میں دکھائے گئے طریقہ کار کو استعمال کریں۔ )\\
مرحلہ نمبر (4):۔ f ``(x) کے لئے ایک فقرہ (Expression) معلوم کیجئے۔\\
مرحلہ نمبر (5):۔ مرحلہ نمبر (3) میں،  xکی ہر قیمت کے لئے f ``(x)کی علامت معلوم کیجئے۔ اگر علامت مثبت ہو تو ترسیم کا اقل ترین نقطہ ہوگا اور اگر علامت منفی ہو تو ترسیم کا اعظم نقطہ ہوگا۔  (اگر f ``(x)کی قیمت صفر حاصل ہوجائے تو پُرانا طریقہ استعمال کیا جائے گا۔ )\\
مرحلہ نمبر (6):۔  xکی ہر قیمت کے لئے، جوکہ اعظم یا اقل ترین نقطہ دیتی ہے، f(x)محسوب کریں۔ \\

نوٹ کیجئے کہ یہ طریقہ کار ، دو حصوں میں منقسم ہوتا ہے۔\\
اول یہ کہ ،  یہ طریقہ صرف ہموار تفاعل کی ترسیمات کے لئے کارآمد ثابت ہوتا ہے۔ اِسی لئے جن نقاط پر f ` (x) غیر معروف ہو وہاں اِسے استعمال نہیں کیا جاسکتا۔ \\
دوم یہ کہ، اگر f ` (q) = 0  اور   f `` (q) = 0 ہوں تو  x = q پر f(x) کی قیمت یا تو اعظم ہوگی یا اقل ترین ہوگی یا  دونوں نہیں۔ اِسے x = 0 پر
 \(f(x) = x^{3}\)
  اور
   \(g(x) = x^{4}\)
    کا موازنہ کرکے دکھایا جاسکتا ہے۔\\

 آپ آسانی کے ساتھ دیکھ سکتے ہیں کہ f `(0) = f ``(0) = 0 اور g` (0) = g ``(0) = 0 ۔\\
 لیکنx = 0   پر g(x)  کی قیمت اقل ترین ہوتی ہے جبکہ وہاں f(x) ناتو اعظم ہوتا ہے اور نا ہی اقل ترین۔\\
(درحقیقت  y = f(x)  کی ترسیم میں مبدے پر نقطہ موڑ حاصل ہوتا ہے کیونکہ   
\(f''(x)=6x\)
ہوتا ہے، جوکہx < 0  کیلئے منفی ہوتا ہے اور x > 0 کیلئے مثبت۔ )\\
آگے چل کر آپ دیکھیں گے کہ کچھ تفاعل کیلئے دودرجی مشتق معلوم کرنے کیلئے بہت محنت درکار ہوتی ہے۔ ایسے معاملات میں،  پُرانا طریقہ کار اپنانا ہی زیادہ موثر ہوتا ہے۔ \\

15.3.1 مثال:\\
\( \text{f}(x) = x^{4} + x^{5} \)
 کی ترسیم کیلئے اعظم ترین اور اقل ترین نقاط معلوم کیجئے۔ \\
مرحلہ نمبر (1):۔ دیا گیا تفاعل تمام حقیقی اعداد کے لئے معروف  ہے۔ \\
مرحلہ نمبر (2):۔ 

\( \text{f}'(x) = 4x^{3} + 5x^{4} = x^{3}(4 + 5x) \)\\
مرحلہ نمبر (3):۔ اگر  x = 0یا x = -0.8 ہو توf `(x) = 0  ہوتا ہے۔ \\
مرحلہ نمبر (4):۔ 
\( \text{f}''(x) = 12x^{2} + 20x^{3} = 4x^{2}(3 + 5x) \)\\
مرحلہ نمبر (5):۔
\( \text{f}''(-0.8) = 4(-0.8)^2(3 - 4) < 0 \)
 اِس لئےx = - 0.8  اعظم نقطہ دیتا ہے۔اِسی طرح سے  f ``(0) = 0  اِسی لئے پُرانا طریقہ کار استعمال کرنا ہوگا۔\\
 \(0.8<x<0\)
  کے لئے
\(x^{3}<0\)
اور
\(4+5x>0\)
اِسی لئے
\(f'(x)<0\)
;
\(x>0\)
  کے لئے
  \(f'(x)>0\)
اِسی لئے
\(x=0\)
  ایک اقل ترین نقطہ ہوتا ہے\\
مرحلہ نمبر (6):۔ (- 0.8, 0.08192)  نقطہ اعظم  ہے  اور (0, 0)  نقطہ اقلیت \\
15.3.2 مثال\\
\( y = \frac{(x+1)^2}{x} \),
  کی ترسیم کیلئے اعظم نقطہ اور اقل ترین نقطہ معلوم کیجئے۔\\
دیا گیا تفاعل صفر "0" چھاڑ کر باقی تمام حقیقی اعداد کیلئے معروف  ہے۔ 
اِس تفاعل کا مشتق لینے کے لئے اِسے درج  ِ ذیل انداز میں لکھا جاتا ہے۔\\
 \( y = \frac{(x+1)^2}{x} = \frac{x^{2} + 2x +1}{x} = x + 2 + x^{-1}\),\\
اب اس کا مشتق لیتے ہیں۔\\
 \( \frac{\text{d}y}{\text{d}x} = 1 - x^{-2} = 1 - \frac{1}{x^2} = \frac{x^2 - 1}{x^2} \),
اِسی لئے اگر
\( \frac{\text{d}y}{\text{d}x} = 0 \)
  ہو تو \\
\[ x^2 - 1 = 0\] 
\[x = \pm1 \], 
اس کا دوسرے درجہ کا مشتق درج  ذیل ہوگا۔\\
\[\frac{\text{d}^2y}{\text{d}x^2} = 2x^{-3} = \frac{2}{x^3} \],
اس کی قیمت 2- حاصل ہوتی ہے اگر x = -1	\\
اور اس کی قیمت 2 ہوتی ہے اگر x = 1 ۔ اس لئے (0 ،1-) ایک نقطہ عظمہ ہوگا  اور (4، 1) ایک اقلیتی نقطہ۔ 
یہاں اقل ترین قیمت ، اعظم قیمت سے بڑی حاصل ہوئی۔ یہ کیسے ممکن ہوا؟\\
درج ذیل تفاعل اور مساواتوں کی ترسیمات پر موجود ساکن نقاط  کو پلاٹ کرنے اور وضاحت کرنے کیلئے پہلے اور دوسرے درجہ کی مشتق کا استعمال کیجئے۔ اگر یہ طریقہ کار ناکام ثابت ہو تو
\(\frac{\dif{y}}{\dif{x}}\)
 کی علامت کے تبدیل ہونے کو استعمال کرکے اعظم نقطہ، اقل ترین نقطہ اور نقطہ موڑ معلوم کیجئے۔ \\
 \( \text{f}(x) = 3x - x^3 \)\\
\( \text{f}(x) = x^3 - 3x^2 \)\\
\( \text{f}(x) = 3x^4 + 1 \)\\
\( \text{f}(x) = 2x^3 - 3x^2 - 12x + 4 \)\\
\( \text{f}(x) = \frac{2}{x^4} - \frac{1}{x} \)\\
\( \text{f}(x) = x^2 + \frac{1}{x^2} \)\\
\( \text{f}(x) = \frac{1}{x} - \frac{1}{x^2} \)\\
\( \text{f}(x) = 2x^3 - 12x^2 + 24x + 6 \)\\
\( y = 3x^4 - 4x^3 - 12x^2 - 3 \)\\
\( y = x^3 - 3x^2 + 3x + 5 \)\\
\( y = 16x - 3x^3 \)\\
\( y = \frac{4}{x^2} - x \)\\
\( y = \frac{4 + x^2}{x} \)\\
\( y = \frac{x - 3}{x^2} \)\\
\( y = 2x^5 - 7 \)\\
\( y = 3x^4 - 8x^3 + 6x^2 + 1 \)\\
			
4 منطقی امتیازات\\
آپ نے دیکھا ہے کہ ہموار تفاعل کی ترسیمات کے لئے، یہ صحیح ہوتا ہے کہ
اگر (q,f(q)) ایک نقطہ عظمہ یا اقلیتی نقطہ ہو تب f'(q)=0ہوتا ہے۔\\
لیکن اس کا معکوس بیان ، کہ
اگر f'(q)=0 ہو تب (q,f(q))   ایک نقطہ عظمہ یا اقلیتی نقطہ ہوگا ،   یہ بیان غلط ہوتا ہے۔ \\
آپ اُسے غلط ثابت کرسکتے ہیں ایک متضاد مثال کو استعمال کرکے، مثلاً ایک تفاعل جس کے لئے "اگر ۔۔۔۔۔۔" والا حصہ تو موجود ہو لیکن "تب ۔۔۔۔" والا حصہ موجود نا ہو۔ \\
	ایسا ایک تفاعل
\( \text{f}(x) = x^{3} \)
	  ہے جس میں q = 0 ہے۔ چونکہ
\( \text{f}'(x) = 3x^{2 }\)
	    ہے\\
اور f `(x) = 0 ہے، لیکن (0،0) اس تفاعل کیلئے نا تو اعظم نقطہ ہے اور نا ہی اقل ترین نقطہ۔ \\
ایسی ہی صورت حال نقطہ موڑ کے ساتھ بھی آتی ہے۔ ہموار تفاعل کے لئے یہ صحیح ہوتا ہے کہ
	اگر (p, f(p)) ایک نقطہ موڑ ہو تب f ``(p) = 0ہوتا ہے۔ لیکن اس کے معکوس کے مطابق،
اگر f ``(p) = 0 ہو تب نقطہ (p, f(p))  ایک نقطہ موڑ ہوتا ہے، یہ بات غلط ہوتی ہے۔ \\
	اس معاملے میں ایک متضاد مثال، x = 0کے لئے تفاعل
	\( \text{f}(x) = x^{4} \)
	 کی ہوسکتی ہے۔ چونکہ 

\( \text{f}''(x) = 12x^{2} \)
	  ہوتا ہے جس کیلئے f ``(0) = 0  ہوگا۔ لیکن (0،0) ایک نقطہ اقلیت ہے
\( \text{f}(x) = x^{4 }\)
	  کے ترسیم میں، نا کہ نقطہ موڑ۔\\
اعلیٰ ریاضیات  میں عام مسائل  کو مخصوص تفاعل کے لئے استعمال کیا جاتا ہے۔ بہت سےمسئلے ایسے ہوتے ہیں جن کے معکوس بھی صحیح ثابت ہوتے ہیں، مثلاً فیثاغورث کا مسئلہ۔ لیکن، جیسا کہ اُوپر مثال میں تھا، اگر کسی مسئلہ کا معکوس غلط ہو، تب یہ بہت اہم ہوجا تا ہے کہ آپ (صحیح) مسئلہ کو استعمال کررہے ہیں ناکہ (غلط) معکوس کو۔ \\
15.5 \( \text{f}(x) = x^4 \)
 سنکیتن  کی توسیع\\
	حالانکہ 
\( \text{f}(x) = x^4 \)
	 بذات خود ایک علامت ہے، اِسی لئے اِسے اجزا میں تقسیم نہیں کرنا چاہیئے لیکن کئی مرتبہ اِسے y کو الگ کرکے لکھنے کے کئی فائدے ہوتے ہیں۔ یعنی اِسے 
\( \frac{\text{d}}{\text{d}x}y \)
	  اس طرح لکھا جاتا ہے۔ اسی لئے اگر  y = f(x) ہوتو آپ اِسے اس طرح لکھ سکتے ہیں،\\
\[ \text{f}'(x) = \frac{\text{d}}{\text{d}x}\text{f}(x) \]
ایک نہایت قابل استعمال مخففی انداز ہے۔ مثال کے طور پر اگر 
\( y = x^{4 }\)
  ہو تب
\( \frac{\text{d}y}{\text{d}x} = 4x^3 \)
    ہوگا۔ \\
اِسے مخففی انداز میں اس طرح لکھ سکتے ہیں، 
\[ \frac{\text{d}}{\text{d}x}x^4 = 4x^3 \]
آپ
\(\frac{d}{\dif{x}}\)
کو علامتی ھدایت  سمجھ سکتے ہیں جس کے عمل کے بعد مشتق حاصل ہوجاتا ہے۔ آپ نے ایسے تحسیب کار دیکھے ہونگے جو تحسیبی عمل کے علاوہ الجبرا بھی کرتے ہیں۔ اِن میں، اگر آپ ایک تفاعل مثلاً
\( x^{4} \)
لیں اور اُسے 'مشتق' کا حکم  دیں تب وہ آپ کو ماحصل کے طور پر 
\( 4x^{3} \)
پیش کرےگا۔ علامت
\( \frac{\text{d}}{\text{d}x} \)
 کو کبھی کبھی مشتقی عامل  بھی کہا جاتا ہے۔ اس طرح یہ علامت 'مشتق کے حکم' لگانے جیسا ہی عمل کرتی ہے۔ \\
	اِسی انداز میں دوسرے درجہ کی مشتق میں بھی یہی سنکیتن کو استعمال کیا جاسکتا ہے۔ \\
دوسرے درجہ کی مشتق یعنی
\( \frac{\text{d}y}{\text{d}x} \)
 کا مشتق لینا جسے عام طور پر ہم
\( \frac{\text{d}}{\text{d}x}\frac{\text{d}y}{\text{d}x} \)
 کے طور پر لکھتے ہیں۔ اگر آپ اس اصطلاح کو سمیٹ کر ایک اصطلابنائیں تو اُوپری حصہ میں
\( \text{d}^2y \)
   ہوگا اور نچلے حصہ میں 
\( (\text{d}x)^2 \)
   ہوگا۔  یہاں وحدانی خطوط کو ہٹا کر لکھیں تو یہ 
\( \frac{\text{d}^2y}{\text{d}x^2} \)
    بن جاتا ہے۔ \\
15.6 اعلیٰ درجی مشتق\\
	دو درجی مشتق پر اکتفا کرنے یا رُک جانے کی کوئی خاص وجہ نہیں ہے۔ چونکہ 
\( \frac{\text{d}^2y}{\text{d}x^2} \)
	 بذات خود بھی ایک تفاعل ہے، اگر وہ ایک ہموار تفاعل ہوتو اُسکا مزید مشتق لیا جاسکتا ہے جو کہ سہ درجی مشتق ہوگا۔ اِس عمل کو مسلسل جاری رکھنے پر اعلیٰ مشتقوں کا ایک سلسلہ مل جاتا ہے۔\\
	 \( \frac{\text{d}^{3}y}{\text{d}x^{3}},\frac{\text{d}^{4}y}{\text{d}x^{4}},\frac{\text{d}^{5}y}{\text{d}x^{5}}\)
	   وغیرہ وغیرہ۔ تفاعلی انداز میں اِسے کچھ اس طرح لکھا جاتا ہے

\[ \text{f}'''(x),\text{f}^{(4)}(x),\text{f}^{(5)}(x),\]	   
	یہاں آپ نوٹ کیجئے کہ تیسرے درجہ تک مشتق کو ظاہر کرنے کیلئے “ dashes  ” کو استعمال کیا گیا لیکن چوتھے مشتق سے آگے کیلئے وحدانی خطوط میں عدد لکھ کر اُس مشتق کے درجے کا اظہار کیا گیا ہے۔ \\
	یہ تمام اعلیٰ درجی مشتقیں، حقیقی دُنیا میں یا ترسیمات کی تیاری میں کوئی خاص  تفہیمی کردار نہیں ادا کرتے ہیں۔ لیکن کچھ معاملات میں یہ اہم بھی ہوتے ہیں۔ مثلاً تقربی تحسیب  میں اور سلسلہ وار تفاعل کے اظہار کے لئے اِن کا اہم استعمال ہوتا ہے۔ \\
۔ مندرجہ ذیل کیلئے 
\(\frac{\text{d}y}{\text{d}x}, \frac{\text{d}^2y}{\text{d}x^2}, , \frac{\text{d}^3y}{\text{d}x^3}\)
  اور 
\(\frac{\text{d}^4y}{\text{d}x^4}\)
   معلوم کیجئے۔ \\
\[y = x^2 + 3x - 7\]

\[y = 2x^3 + x + \frac{1}{x}\]

\[y = x^4 - 2\]

\[y = \sqrt{x}\]

\[y = \frac{1}{\sqrt{x}}\]

\[y = x^{\frac{1}{4}}\]

2۔ مندرجہ ذیل کیلئے  f`(x) ،  f``(x) ،  f```(x) اور f(4)(x)  معلوم کیجئے۔ \\
\[y = x^2 - 5x + 2\]

\[y = 2x^5 - 3x^2\]

\[y = \frac{1}{x^4}\]

\[y = x^2(3 - x^4)\]

\[y = x^{\frac{3}{4}}\]

\[y = x^{\frac{3}{8}}\]
3۔ اگر
\(y = x^n\)
 ہو تو
\(\frac{\text{d}^ny}{\text{d}x^n}\) 
  معلوم کیجئے جہاں n  ایک مثبت عدد ہے۔ \\
4۔ اگر
\(y = x^{n+2}\)
 ہو تو 
\(\frac{\text{d}^ny}{\text{d}x^n}\) 
 معلوم کیجئے جہاںn   ایک مثبت عدد ہے۔\\
5۔ اگر
\(y = x^m\)
ہو تو
\(\frac{\text{d}^ny}{\text{d}x^n}\) 
 معلوم کیجئے جہاں  m  ایک مثبت عدد ہے  اور n > m  \\
متفرق مشق \\15
1۔
\(x^3 - 6x^2 + 9x + 6\)
کی اعظم قیمت اور اقل قیمت معلوم کیجئے، ساتھ ہی ساتھ یہ بھی بتائیے کہ آپ نے اِنہیں کیسے معلوم کیا؟\\
2۔ تفاعل 
\(\text{f}(x) = 16x + \frac{1}{x^2}\)
کیلئے اعظم قیمت اور اقل قیمت معلوم کیجئے، ساتھ ہی ساتھ یہ بھی بتائیے کہ آپ نے اعظم اور اقل نقطہ کیسے متعین کیا۔ \\
3۔ تفاعل
\(\text{f}(x) = \sqrt{x} + \sqrt{30-5x}\)
میں اعظم قیمت اور اقل قیمت معلوم کیجئے اور xکی متعلقہ قیمتیں بھی دیجئے۔ \\
4۔ تفاعل
\(y = \frac{1}{x} + \frac{1}{1-4x}\)
کی ترسیم میں اعظم نقطہ اور اقلیتی نقطہ کے محدد لکھئے۔ \\
5۔ نسرین کی کافی کے سرد ہونے کی شرح،  کافی کے درجہ حرار
ت
\( \theta\)
 اور ماحول کے درجہ حرارت
  \( \alpha\)
 کے فرق کے ساتھ راست تناسب میں ہے۔\\
 
\( \theta\)
اور
t
کے درمیان ترسیم بنائیے. 
اگر
t=0
  پر
  \( \alpha = 20\)
  ہو اور 

\( \theta = 95\)
ہو.
اگر
\( t > 0\)
ہو تو 
\( \theta\)
 کی 
 \( \theta, \frac{\text{d}\theta}{\text{d}t}\)
 اور 
 
\( \frac{\text{d}^2\theta}{\text{d}t^2}\)
کی علامتیں بتائیے۔ \\
۔ اُڑان کے دوران، ہوائی جہازوں میں ایک مزاحمت محسوس کی جاتی ہے جسے ہوائی رگڑ  کہا جاتا ہے۔ ایک مخصوص جہاز کیلئے، کم رفتاروں کے لئے، ہوائی رگڑ کی قیمت 

\( kS^2\)
کے برابر ہے، جہاں k  ایک مستقل ہے جسے ہوائی رگڑ کا ضریب  ہے اورS   اُس جہاز کی رفتار ہے۔ \\
اگر رفتاروں کو بڑھایا جائے تو رفتار کے ساتھ ساتھ k کی قیمت بھی بڑھتی جاتی ہے۔ اور k  بالمقابل S تیار ہونے والی ترسیم درج ذیل ہے۔ (آواز کی رفتار کے قریبی قیمتوںوالے علاقے  کو عام طور پر سمعی رکاوٹ  کہا جاتا ہے۔)\\
(a) ترسیم میں، تینوں علاقوں میں
\( \frac{\text{d}k}{\text{d}S}\)
اور
\( \frac{\text{d}^2k}{\text{d}S^2}\)
کی علامتیں بتائیے۔\\
(b) کس علاقے میں k  \\کی قیمت نہایت تیزی سے تبدیل ہورہی ہے؟
(c) بہت زیادہ تیز رفتاروں کیلئے  k \\کی قیمتوں سے کیا نتیجہ اخذ کیا جاسکتا ہے؟
7۔ ایک کھڑکی کا نچلا حصہ مستطیل نما ہے اور اُوپری حصہ نیم دائرہ نما ہے۔ نچلے مستطیل نما حصےکو ABCD سے دکھایا گیا ہے جس کی چوڑائی 2x  ہے اور اُونچائی  yہے۔ اُوپری نیم دائرہ نما حصہ کا قطر AB ہے، یعنی نیم دائرے کا نصف قطر  xہے۔ \\
	کھڑکی کا مجموعی محیط 10 میٹرس ہے۔x  اور
\(\pi\)
  کی شکل میں کھڑکی کے مجموعی رقبے کے لئے فقرہ حاصل کیجئے۔ ساتھ ہی ساتھ  xکی وہ قیمت معلوم کیجئے جس کے لئے رقبہ کی قیمت اعظم ہوگی۔  xکی اُس مخصوص قیمت کو معلوم کرنے کیلئے
 \( \frac{\text{d}^2y}{\text{d}x^2}\)
   کی قیمت کا استعمال کیجئے۔ \\
۔ اگر   a>0   ہو تو ، درج ذیل تفاعل کیلئے اعظم اور اقلیت کی تفتیش کیجئے \\

\[x^2(x-a)\]

\[x^3(x-a)\]

\[x^2(x-a)^2\]

\[x^3(x-a)^2\]


تفاعل 
\[x^n(x-a)^m\]
 کیلئے ایک اٹکل  بنائیے۔ \\
 \(f^{n}(x)\)
 کے لئے ایک فقرہ تیار کیجئے جہاںf(x)  درج ذیل ہو۔ \\
\[\text{f}(x) = \frac{1}{x^3}\]
\[\text{f}(x) = \sqrt{x}\]
10*۔ درج  ذیل مساواتوں کی منحنیوں کیلئے نقطہ موڑ کے محدد  معلوم کیجئے۔ \\

\[y = x^4 - 8x^3 + 18x^2 + 4\]

\[y = x^2 - \frac{1}{x} + 2\]

