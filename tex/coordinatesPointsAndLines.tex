\باب{محدد، نقطے اور خط}\شناخت{محدد-نقطے-اور-خط}
% Page 01
 اس سبق میں ہم محدد کی مدد سے نقطوں اور خط کی   دو ابعادی میں تعریف کریں گے۔ یہ سبق پڑھ لینے کے بعد آپ اس قابل ہوں گے کہ ؛
\begin{itemize}
\item   
دو نقطوں کے بیچ کا فاصلہ معلوم کریں۔
\item
کسی خط کے انتہائ نقطوں کے محدد معلوم ہوں تو اس خط کا درمیانی نقطہ معلوم کر سکیں۔
\item
کسی خط کے انتہائ نقطوں کے محدد معلوم ہوں تو اس خط کی ڈھلوان معلوم کریں۔
\item
ایک خط کی ڈھلوان سے اسکی مساوات معلوم کریں۔
\item
دو نقطوں کو ملانے والی لکیر کی مساوات معلوم کریں۔
\item
لکیروں میں تفریق کریں مختلف طرح کی مساوات سے۔
\item
دو لکیریں کے مشترک نقاط  معلوم کریں۔
\item
ڈھلوان سے معلوم کریں کہ لکیریں عمودی ہیں یا متوازی ہیں۔ 
\end{itemize}
\حصہ{دو نقطوں کے بیچ کا فاصلہ}\شناخت{دو-نقطوں-کے-بیچ-کا-فاصلہ}
جب آپ مبدا چن لیتے ہیں تو صفحے پے افقی سمت میں x محدد بنایئں۔ اور عمودی خط میں y محدد بنائیں، اور اسطرح آپ ایک نظام محدد بنا رہے ہیں۔ اور اس نظام محدد کو کارتیسی نظام محدد کہیں گے، اور یہ نام سترھویں صدی کے  ایک فرانسیسی ریاضی دان رین ڈیسکارٹس\حاشیہد{Rene Descarts}  کے نام پر رکھا گیا
 شکل   \حوالہء{1.1} میں دو نقطے ہیں  \عددی{A} اور \عددی{B}۔  \عددی{A} کے محدد \عددی{(4,3)} ہیں اور  \عددی{B} کے \عددی{(10,7)}   محدد ہیں۔  خط کا وہ حصہ جو  \عددی{A}   اور  \عددی{B}     کے درمیان واقع ہو اسے لکیری قطع کہیں گے۔ لکیری قطع کی لمبائی  دو نقتوں کے بیچ کا فاصلہ ہے۔
شکل \حوالہء{1.1}  میں ایک نقطہ  \عددی{C}    بھی شامل کر لیا گیا ہے اور اسطرح ایک قائمہ الزاویہ مثلث وجود میں آتی ہے۔ آپ دیکھ سکتے ہیں کہ   \عددی{C}  کا  x-محدد    \عددی{B} جیسا ہے جبکہ  \عددی{A}   اور     \عددی{C} کا y-محدد ایک ہی ہے۔  اور یوں  کے  \عددی{C} محدد  (10,3) ہیں۔
یہ بہت واضع ہے کہ   کی AC لمبائ     \عددی{10-4=6} ہے اور    CB  کی لمبائ   \عددی{7-3=4} ہے۔ فیثاغورث کے کلیے کو استعمال کرتے ہوۓ  مثلث ABC    سے یہ واضع ہے کہ قطع خط  AB کی لمبائ \[ \sqrt{(10-4)^2 + (7-3)^2} = \sqrt{6^2 + 4^2} = \sqrt{36+16} = \sqrt{52}
\] 
 آپ اعداد کی مدد لے سکتے ہیں اور یوں آپ کے پاس نتیجہ   \عددی{7.21}   آۓ گا لیکن بہتر یہی ہے کہ آپ اسے جذر کی صورت میں ہی رہنے دیں۔
 محدد جیومیٹری کی تجویز اس لیے پیش کی گئ کہ حساب کتاب کے لیے الجبرا کا استعمال کیا جا سکے، جیسے اگر   \عددی{A}   اور   \عددی{B}   کوئ بھی دو نقطے ہوں اور شکل\حوالہء{1.1}والے نہ ہوں تو بھی ہمارے لیے کافی مدد گار ہوتا ہے کہ صرف محدد دیکھ کہ یہ پتہ چل جاۓ کہ کس نقطے کی بات ہو رہی ہے۔ اسکا ایک طریقہ یہ ہے کہ علامات استعمال کی جائیں جیسے پہلے نقطے کے محدد   \((x_{1},y_{1})\)
  % Page 02
اور دوسرے نقطے کے محدد  \((x_{2},y_{2})\)ہوں گے۔جبکہ\( x_{1} \) دراصل پہلے نقطے کا  محدد x  ہے۔
شکل \حوالہء{1.2} میں ایک عام مثلث بنائ گئ ہے اور آپ دیکھ سکتے ہیں کہ   نقطہ  \عددی{C}  کے محدد اب  \((x_{2},y_{1})\)ہیں اور یہ کہ اب \(AC = x_2 - x_1 \) اور \(CB=y_2 -y_1 \) ۔
فیثاغورث کے کلیے کے مطابق؛
\[ AB = \sqrt{(x_2 - x_1)^2 + (y_2 - y_1 )^2} \]
ایک اور فائدہ الجبرا استعمال کرنے کا کہ مثلث کی جیسی بھی شکل ہو اور وہ جس بھی جگہ ہو یہ کلیہ کام کرتا ہے  شکل \حوالہء{1.3} میں \عددی{A} کے  محدد منفی ہیں اور شکل \حوالہء{1.4}  میں لکیر کی ڈھلوان نیچے کی طرف ہے بجاۓ اوپر کی طرف ہونے کے جیسے جیسے آُپ بائیں سے دائیں جانب چلتے ہیں۔ شکل \حوالہء{1.3} اورشکل \حوالہء{1.4} میں اپنے طور پر \عددی{AB}  کی لمبائ معلوم کرنے کی کوشش کریں۔ اور پھر آپ کلیہ استعمال کر سکتے ہیں اپنے جواب کی پڑتال کرنے کے لیے۔
شکل \حوالہء{1.3} کے لیے
\( x_2 - x_1 = 3-(-2) = 3+2 = 5\)\, اور \, \(y_2 - y_1 = 5-(-1) = 5+1 = 6\)
\[ AB = \sqrt{(3-(-2))^2 +(5-(-1))^2}= \sqrt{5^2 + 6^2} = \sqrt{25+36} =\sqrt{61}\]
اور  شکل \حوالہء{1.4}  میں 
\( x_2 - x_1 = 6-1 = 5\) \, اور \, \(y_2 -y_1 = 2-5 = -3\)
 \[ AB = \sqrt{(6-1)^2 + (2-5)^2 }= \sqrt{5^2 + (-3)^2} = \sqrt{25 + 9} = \sqrt{34}\]
ایک اور بات اس سے فرق نہیں پڑتا کہ آپ نقطوں کو کس ترتیب میں رکھتے ہیں، اگر آپ \عددی{B} کو پہلا نقطہ تصور کریں یوں کہ   \((x_{1},y_{1})\)  اور \عددی{A}  کو  دوسرا نقطہ      \((x_{2},y_{2})\)  تو کلیے پر اسکا کوئ اثر نہیں ہوگا ۔شکل\حوالہء{1.1} کے لیے یہ 
\[ BA = \sqrt{(4-10)^2 + (3-7)^2} = \sqrt{(-6)^2 + (-4)^2 } = \sqrt{36+16} = \sqrt{52}\]
% Page 03
%make box here
دو نقطوں  \((x_{1},y_{1})\) اور  \((x_{2},y_{2})\) کا درمیانی فاصلہ (یا اس قطع لکیر کی لمبائ جو ان دونوں کو جوڑ رہا ہے ) ؛
\[ \sqrt{(x_2-x_1)^2 + (y_2 - y_1)2}\]
%end box here
\حصہ {قطع لکیر کا وسط}
آپ محدد کی مدد سے بھی ایک قطع لکیر کا درمیانی قطع معلوم کر سکتے ہیں۔
شکل\حوالہء{1.5} میں ایک قطع لکیر دکھایا گیا ہے جیسا کہ شکل \حوالہء{1.1} میں تھا لیکن اب اس میں درمیانی نقطہ   \عددی{M} بھی شامل کیا گیا ہے۔ \عددی{M}  سے گزرتی ہوئ   y-محدد کے مساوی خط    \عددی{AC} کو چھوۓ گا اور اس نقطے کو ہم نام دیں گے  \عددی{D}  کا ، اور یوں مثلث  \عددی{ADM } کے اطراف کی لمبائ \عددی{ACB}   کے اطراف کی لمبائ سے   آدھی ہیں، اور اسی لیے ؛   \[ AD = \frac{1}{2} AC = \frac{1}{2} (10-4) = \frac{1}{2} (6)= 3,\]
\[ DM = \frac{1}{2} CB= \frac{1}{2} (7-3) = \frac{1}{2} (4)=2\]
نقطے  \عددی{M} اور  \عددی{D} کے x-محدد ایک ہی ہیں جو کہ؛
\[ 4+AD=4+ \frac{1}{2} (10-4) = 4+3 =7\] 
نقطے   \عددی{M} کا y-محدد جو کہ؛
\[ 3+MD = 3+ \frac{1}{2} (7-3) = 3+2 = 5\]
لہٰذہ درمیانی  نقطے    \عددی{M} کے محدد    \( (7،5 )  \) ہیں 
شکل \حوالہء{1.6} میں شکل \حوالہء{1.2} ہی ہے لیکن اب اسمیں دو نقطے  \عددی{M} اور   \عددی{D}شامل کیے گۓ ہیں 
\[ AD = \frac{1}{2} AC = \frac{1}{2}(x_2 -x_1) , \quad \quad DM =\frac{1}{2} CB = \frac{1}{2} (y_2 - y_1)\]
لہٰذہ نقطے   \عددی{M} کا  x-محدد ہے؛
\begin{align*}
x_1 +AD &= x_1 + \frac{1}{2} (x_2 - x_1) = x_1 + \frac{1}{2} x_2 - \frac{1}{2} x_1 \\ &= \frac{1}{2} x_1 + \frac{1}{2} x_2 = \frac{1}{2} (x_1 + x_2).
\end{align*}
اور اسی طرح نقطے \عددی{M} کا   y-محدد ہے؛
\begin{align*}
y_1 +DM &= y_1 + \frac{1}{2} (y_2 - y_1) = y_1 + \frac{1}{2} y_2 - \frac{1}{2} y_1 \\ &= \frac{1}{2} y_1 + \frac{1}{2} y_2 = \frac{1}{2} (y_1 + y_2).
\end{align*}
%begin box
دو نقطوں   \((x_{1},y_{1})\)  اور  \((x_{2},y{2})\)  کو ملانے والے قطع لکیر کے درمیانی حصے کے محدد ہیں ؛  \[\big( \frac{1}{2} (x_1 +x_2)  , \, \frac{1}{2} (y_1 +y_2) \big)\]
%end box
% Page 04
اور اب چونکہ آپ کے پاس وسطی نقطہ   \عددی{M} کے محدد کے لیے الجبرائ کلیہ موجود ہے، آپ اسے کسی بھی دو نقطوں کے لیے استعمال کر سکتے ہیں، مثال کے طور پر شکل \حوالہء{1.3} کے لیے  \عددی{AB} کا درمیانی نقطہ؛
\[ \big( \frac{1}{2}   ((-2) +3),\frac{1}{2} ((-1)+5) \big)=\big( \frac{1}{2} (1) , \frac{1}{2} (4) \big) = \big( \frac{1}{2},2 \big). \]
اور شکل \حوالہء{1.4} کے لیے 
\( \big( \frac{1}{2} (1+6), \frac{1}{2}(5+2)  \big) = \big( \frac{1}{2}(7) , \frac{1}{2}(7) \big) = \big( 3\frac{1}{2} , 3\frac{1}{2} \big) \)
یہاں بھی اس بات سے کوئی مسئلہ نہیں ہوگا کہ آپ کس نقطے کو پہلا نقطہ کہتے ہیں اور کسے دوسرا، شکل\حوالہء{1.5} میں اگر آپ    (10,7)     کو  \((x_{1},y_{1})\) جبکہ  (4,3) کو  \((x_{2},y{2})\) تصور کر لیں تو درمیانی نقطہ \( \big( \frac{1}{2} (10+4) , \frac{1}{2}(7+3) \big) = (7,5)  \) جوکہ پہلے والا جواب ہی ہے۔
  \حصہ{قطع خط کا ڈھلاؤ} \شناخت{قطع-خط-کا-ڈھلاؤ}
کسی لکیر کا ڈھلاؤ دراصل بتاتا ہے کہ کوئ لکیر کتنی ترچھی ہے، لکیر جتنی ذیادہ ترچھی ہو گی اتنا ذیادہ ڈھلاؤ ہوگا۔
فاصلے اور درمیانی نقطے کے برعکس ڈھلاؤ پوری لکیر کی خصوصیت ہےنہ کہ صرف ایک قطع لکیر کی ۔ اگر آپ لکیر کے کوئ سے بھی دو نقطے چنتے ہیں اور آپ محسوس کرتے ہیں کہ  x-محدد اور    y-محدد کی قیمتیں بڑھ رہی ہیں جیسے جیسے آپ ایک نقطے سے دوسرے کی طرف جاتے ہیں ، جیسا کہ شکل \حوالہء{1.7} میں دکھایا گیا ہے تو کسر کچھ ایسا بنتا ہے،
\[ \frac{\text{\RL{قدم \(y\)}}}{\text{\RL{قدم \(x\)}}}\]
اور یہ بدلتا نہیں ہے آپ جو بھی نقطے چنیں۔ اور یہی ایک لکیر کا ڈھلاؤ ہے۔
کلیے پر کوئ اثر نہیں پڑتا محدد مثبت ہوں یا منفی ، شکل 1.3 میں مثال کے طور پر   AB کا ڈھلاؤ 
\( \frac{5-(-1)}{3-(-2)} = \frac{5+1}{3+2} = \frac{6}{5} \) 
ہے
لیکن اس بات کا خیال رکھیں کہ شکل\حوالہء{1.4} میں ڈھلاؤ \( \frac{2-5}{6-1} = \frac{-3}{5} = - \frac{3}{5} \)، منفی ڈھلاؤ کا مطلب ہے کہ جب آپ بائیں سے دائیں جانب چل رہے ہوں تو ترچھاؤ نیچے کی طرف ہو۔
باقی کلیوں کی طرح یہاں بھی اس بات سے فرق نہیں پڑتا کے کس محدد کو ایک کہیں گے اور کسے دو،  شکل 1.1 میں آپ ڈھلاؤ کہہ سکتے ہیں کہ \( \frac{7-3}{10-4} = \frac{4}{6} = \frac{2}{3} \) یا ہم ایسے بھی کہہ سکتے ہیں   \( \frac{3-7}{4-10} = \frac{-4}{-6} = \frac{2}{3} \)
اگر دو لکیروں کا ڈھلاؤ برابر ہے تو ہم کہہ سکتے ہیں کہ یہ دونوں لکیریں متوازی یا مساوی ہیں۔
\ابتدا{مثال}
ایک لکیر کے انتہائ نقطے \( (p-q, p+q) \) اور  \((p+q, p-q) \) ہیں 
اس لکیر کی لمبائ ، ڈھلاؤ اور درمیانی نقطے کے محدد معلوم کریں۔
لمبائ اور ڈھلاؤ معلوم کرنے کے لیے  آپکو حساب لگانا ہوگا۔
\begin{align*}
x_2 - x_1 &= (p+q) - (p-q) = p+q-p+q= 2q \\
y_2 - y_1& = (p-q) - (p+q) = p-q-p-q = -2q&&\text{اور}
\end{align*}
لمبائ 
\( \sqrt{(x_2 - x_1 )^2 + (y_2 - y_1 )^2 }  = \sqrt{(2q)^2 + (-2q)^2}  = \sqrt{4q^2 + 4q^2 } = \sqrt{8q^2}.  \) 
 ہے جبکہ ڈھلاؤ \( \frac{y_2 - y_1}{x_2 - x_1} = \frac{-2q}{2q} = -1 \)
 درمیانئ نقطے کے لیے 
\begin{align*}
x_1 + x_2 &= (p-q) + (p+q) = p-q+p+q= 2p \\
y_1 - y_2 &= (p+q) + (p-q) = p+q+p-q = 2p&&\text{اور} 
\end{align*}
لہٰذہ درمیانی نقطہ \( \big( \frac{1}{2} (x_1 + x_2 ) , \frac{1}{2} (y_1 + y_2) \big) = \big( \frac{1}{2} (2p), \frac{1}{2} (2p)= (p,p). \)
ہے۔
کوشش کریں کہ آپ خود سے شکل بنائیں مثال کے نتیجے کو ظاہر کرنے کے لیے۔
\انتہا{مثال}
\ابتدا{مثال}
ثابت کریں کے ان نقطوں \(  A(1,1) , \, B(5,3), \, C(3,0) \, \text{اور} \, D(-1,2) \) سے ایک متوازی الاضلاع شکل بنتی ہے۔
آپ اس مثال کو کئ طریقوں سے حل کر سکتے ہیں لیکن جو بھی طریقہ چنیں گے اس میں شکل بنانا لازمی ہے ، جوکہ شکل \حوالہء{1.8} میں دکھائ گئ ہے۔
پہلی ترکیب \موٹا{(لمبائی کا استعمال کرتے ہوۓ)} 
  
 اس طریقے میں مخالف سمتوں کی لمبائ معلوم کریں ، اگر مخالف سمتوں کی لمبائ برابر ہے تو دیے گۓ نقطے ایک متوازی الاضلاع شکل بنائیں گے۔
\begin{align*}
AB &= \sqrt{(5-1)^2 + (3-1)^2} = \sqrt{20} \\ DC &= \sqrt{(3-(-1))^2 + (0-(-2))^2} = \sqrt{20} \\ CB &= \sqrt{(5-3)^{2} + (3+0)^{2}} = \sqrt{13} \\ DA &= \sqrt{(1-(-1))^2 + (1-(-2))^2} = \sqrt{13} 
\end{align*}
اسی لیے \( AB = DC \)  اور   \( CB = DA \) اور ثابت ہو گیا کہ دیے گۓ نقطے ایک متوازی الاضلاع شکل بناتے ہیں
طریقہ 2\موٹا{(درمیانی نقطوں کی مدد سے )} 
اس طریقے میں , اخترن    AC  اور     BD   کے درمیانی نقطے معلوم کریں۔ اگر یہ نقطے ایک ہی ہیں تو اسکا مطلب اخترن ایک دوسرے کو کاٹتے ہیں  لہٰذہ یہ بند شکل ایک متوازی الاضلاع شکل ہوگی۔
اخترن  AC کا درمیانی نقطہ \( \big( \frac{1}{2}(1+3),\frac{1}{2}(1+0)\big)  \) جوکہ \(  \big( 2, \frac{1}{2}\big)\) ہے، اخترن  BD کا  درمیانی نقطہ \(    \big( \frac{1}{2}(5+(-1)) , \frac{1}{2}(3+(-2))     \big) \) اور یہ بھی\(  \big( 2, \frac{1}{2}\big)\) لہٰذہ ہم کہہ سکتے ہیں کہ دیے گۓ نقطے ایک متوازی الاضلاع شکل بناتے ہیں۔
% Page 06 
طریقہ 3 \موٹا{(ڈھلاؤ کی مدد سے)}
اس طریقہ کار میں مخالف سمتوں کے ڈھلاؤ معلوم کریں، اگر آمنے سامنے کے دونوں خط متوازی ہوں تو ہم کہہ سکتے ہیں کہ    \عددی{ABCD}ایک متوازی الاضلاع شکل ہوگی۔ خط AB  اور  DC  کے ڈھلاؤ \(  \frac{3-1}{5-1}=\frac{2}{4}=\frac{1}{2}        \) اور \( \frac{0-(-2)}{3-(-1)}=\frac{2}{4}=\frac{1}{2}  \)  ہیں ، لہٰذہ  \عددی{AB} اور \عددی{DC} متوازی لکیریں ہیں، لکیروں     \عددی{DA} اور    \عددی{CB}  دونوں کا ڈھلاؤ  برابر  \(  \frac{3}{2}\)  ہے ، اسی لیے ہم کہہ سکتے ہیں کہ    \عددی{DA} اور    \عددی{CB} متوازی ہیں اور یوں یہ ثابت ہوتا ہے کہ \عددی{ABCD }   ایک متوازی الاضلاع شکل ہے۔
\انتہا{مثال}
%  exercise 1A%done editting till here
اعداد کا استعمال نہ کریں اور جہاں مناسب ہو اپنے جواب کو جذر کی صورت میں لکھیں۔
\ابتدا{سوال}
درج ذیل نقطوں کو جوڑنے والے قطع لکیر کی لمبائ معلوم کریں. جز  \عددی{(e)  } اور \عددی{(h)  } میں فرض کریں کہ    \(a> 0\)  جبکہ جز   (i)    اور(J) میں      \( p>q > 0   \)   ہے۔
\begin{multicols}{2}
\begin{enumerate}[.a]
\item \( (2,5), \,  (7,1)  \) 
\item \( (-3,2), \,  (1,-1) \)
\item \( (4,-5), \,  (-1,0) \)
\item \( (-3,-3), \,  (-7,3) \)
\item \( (2a,a), \,  (10a, -14a) \)
\item \( (a+1, 2a+3), \,  (a-1, 2a-1) \)
\item \( (2,9), \,  (2,-14) \)
\item \( (12a,5b), \,  (3a,5b) \)
\item \( (p.q), \,  (q,p) \) 
\item \( (p+4q, p-q), \,  (p-3q,p) \)
\end{enumerate}
\end{multicols}
\انتہا{سوال}
\ابتدا{سوال}
ثابت کریں کہ نقطے \( (1,-2), \, (6,-1), \, (9,3), \, (4,2) \) ایک متوازی اللاضلاع شکل بناتے ہیں۔
\انتہا{سوال}
\ابتدا{سوال}
ثابت کریں کہ نقطوں \( (-3,-2), \, (2,-7), \, (-2,5) \) سے بننے والی مثلث ایک مساوی الساقین مثلث ہے۔
\انتہا{سوال}
\ابتدا{سوال}
ثابت کریں کہ نقطے \( (7,12), \, (-3,-12), \, (14,-5)\)  ایک دائرے کا حصہ ہیں جسکا رداس\عددی{(2,0)}   ہے۔
\انتہا{سوال}
\ابتدا{سوال}
درج ذیل نقطوں کو جوڑنے والے قطع لکیر کا وسطی نقطی معلوم کریں۔
\begin{multicols}{2}
\begin{enumerate}[.a]
\item \( (2,11), \, (6,15) \)
\item \( (5,7), \, (-3,9) \)
\item \( (-2,-3), \, (1,6) \)
\item \( (-3,4) , \, (-8,5) \)
\item \( (p+2, 3p-1), \, (3p+4,p-5) \)
\item \( (p+3,q-7), \, (p+5,3-1) \)
\item \( (p+2q.2p+13q), \, (5p-2q,-2p-7q) \)
\item \( (a+3,b-5), \, (a+3,b+7) \)
\end{enumerate}
\end{multicols}
\انتہا{سوال}
% Page 05
\ابتدا{سوال}
نقطے \( A(-2,1) , \, (6,5) \) ایک دائرے کے قطر کے دو انتہائ نقطے ہیں۔ قطر کے درمیانی نقطے کے محدد معلوم کریں۔ 
\انتہا{سوال}
\ابتدا{سوال}
ایک نقطے \(A(3,4)\) اور \عددی{B } کو جوڑنے والے قطع لکیر کا درمیانی نقطہ  \(M(5,7)\)       ہے ۔ نقطہ    \عددی{B }  کے محدد معلوم کریں
\انتہا{سوال}
\ابتدا{سوال}
نقطے \( A(1,-2), \, B(6,-1), \, C(9,3), \, D(4,2) \) ایک متوازی الاضلاع شکل کے کونے ہیں ۔ ثابت کریں کے وتر \عددی{AC}اور     \عددی{BD}ایک ہی نقطے پر ٹکراتے ہیں۔
\انتہا{سوال}
\ابتدا{سوال}
درض ذیل محدد \( A(5,2), \, B(6,-3), \, C(4,7) \) میں سے ایک باقی دو کا وسطی نقطہ ہے اسے تلاش کریں۔ دو فاصلوں کو معلوم کر کے آپ اپنا جواب ثابت کر سکتے ہیں ۔
\انتہا{سوال}
%
\ابتدا{سوال}
درج ذیل نقاط کا ڈھلاؤ معلوم کریں۔
\begin{multicols}{2}
\begin{enumerate}[.a]
\item \( (3,8), \, (5,12) \)
\item \( (1,-3), \, (-2,6) \)
\item \( (-4,-3), \, (0,-1) \)
\item \( (-5,-3), \, (3,-9) \)
\item \( (p+3, p-3), \, (2p+4,-p-5) \)
\item \( (p+3,q-5), \, (q-5, p+3) \)
\item \( (p+q-1, q+p-3), \, (p-q+1, q-p+3) \)
\item \( (7,p), \, ( 11,p) \)
\end{enumerate}
\end{multicols}
\انتہا{سوال}
%Page 07
\ابتدا{سوال}
لکیروں \عددی{AB}   اور \عددی{BC}   کا ڈھلاؤ معلوم کریں جبکہ  \( A (3,4), \, B (7,6), \, C (-3,1). \)۔ ان تینوں نقطوں کے بارے میں اپنی راۓ کا بھی اظہار کریں۔
\انتہا{سوال}
\ابتدا{سوال}
نقطہ  \( P(x,y) \) ایک سیدھی لکیر کا حصہ ہے جس کے انتہائ نقطے  \(A(3,0), \, (5,6) \) ہیں ۔ لکیر   \عددی{AP} اور    \عددی{PB} کے ڈھلاؤ کے لیے ریاضیاتی بیانیہ معلوم کریں ۔ اور یہ مساوات    \( y=3x-8 \) بنا کے دکھائیں۔
\انتہا{سوال}
\ابتدا{سوال}
ایک لکیر جوکہ مثلث کے ایک کونے کو مخالف طرف کے درمیان سے ملاتی ہے اسے اوسط کہتے ہیں۔ اسی اوسط \عددی{AM}   کی لمبائ معلوم کریں جب مثلث کے کونے \( A(-1,1), \, B(0,3), \, C(4,7). \) ہوں۔
\انتہا{سوال}
\ابتدا{سوال}
ایک مثلث  کے کونے \( A(-2,1), \, B(3,-4), \, C( 5,7). \) ہیں ۔
\begin{enumerate}[a.]
\item
لکیر\عددی{AB }  کا وسطی نقطہ\عددی{N  }   اور لکیر \عددی{AC}     کا وسطی نقطہ   \عددی{N  }  معلوم کریں
\item
ثابت کریں کہ \عددی{MN}      کے  \عددی{BC }    متوازی ہے 
\end{enumerate}
\انتہا{سوال}
\ابتدا{سوال}
نقطے  \( A(2,1), \, B(2,7), \, C(-4,-1) \) ایک مثلث بناتے ہیں۔ 
\begin{multicols}{2}
\begin{enumerate}[a.]
\item
لکیروں  \عددی{MN }   اور  \عددی{ BC  }کی لمبائ معلوم کریں 
\item
ثابت کریں کہ \عددی{BC=2MN} 
\end{enumerate}
\end{multicols}
\انتہا{سوال}
\ابتدا{سوال}
ایک چوکور شکل ABCD  کے کونے \( A(1,1), \, B(7,3), \, C(9,-7), \, D(-3,-3) \) ہیں۔ نقطے   \عددی{P}،   \عددی{Q}،    \عددی{R} اور    \عددی{S}  بالترتیب \عددی{  BC}  ،\عددی{ AB  } ،   \عددی{CD  } اور \عددی{DA  } کے وسطی نقطے ہیں۔
\begin{multicols}{2}
\begin{enumerate}[a.]
\item
شکل \عددی{PQRS } کی تمام اطراف کا ڈھلاؤ معلوم کریں۔ 
\item
 یہ چوکور شکل  \عددی{PQRS } دراصل کیسی شکل ہے؟
\end{enumerate}
\end{multicols}
\انتہا{سوال}
\ابتدا{سوال}
مبدا  \عددی{ O }اور     نقطے \( P(4,1), \, Q(5,5), \, R(1,4) \) ایک چوکور شکل بناتے ہیں۔ 
\begin{multicols}{2}
\begin{enumerate}[a.]
\item
ثابت کریں کے \عددی{OR }      اور  \عددی{ PQ   }  متوازی ہیں۔
\item
ثابت کریں کے \عددی{ OP    }اور   \عددی{ RQ  } متوازی ہیں۔
\item
ثابت کریں کہ\عددی{OP = OR} 
\item
چھارطرفہ\عددی{ OPQR  }  کی اصل شکل کیا ہے؟
\end{enumerate}
\end{multicols}
\انتہا{سوال}
\ابتدا{سوال}
مبدا  \عددی{O}  اور نقطے  \( L(-2,3), \, M(4,7), \, N(6,4) \) مل کے ایک چھارطرفہ بناتے ہیں۔
\begin{multicols}{2}
\begin{enumerate}[a.]
\item
ثابت کریں \عددی{ON=LM}
\item
ثابت کریں کہ \عددی{ON}    اور\عددی{ LM}متوازی ہیں 
\item
ثابت کریں کہ\عددی{ OM =LN }
\item
چھار طرفہ  \عددی{OLMN}   کس شکل کا ہے؟
\end{enumerate}
\end{multicols}
\انتہا{سوال}
\ابتدا{سوال}
ایک چھار طرفہ کے کونے \( P(1,2), \, Q(7,0), \, R(6,-4), \, S(-3,-1) \) ہیں 
\begin{multicols}{2}
\begin{enumerate}[a.]
\item
ایک چھار طرفہ کے چاروں طرف کا ڈھلاؤ معلوم کریں۔
\item
ایک چھار طرفہ  \عددی{ PQRS } کی شکل کیا ہوگی؟
\end{enumerate}
\end{multicols}
\انتہا{سوال}
\ابتدا{سوال}
ایک چھار طرفہ کے کونے \( T(3,2), \, U(2,5), \, V(8,7), \, W(6,1). \) ہیں ۔ لکیروں  \عددی{ UV }  اور     \عددی{VW}   کے وسطی نقطے بالترتیب  \عددی{M}    اور  \عددی{ N}    ہیں۔ثابت کریں کہ مثلث    \عددی{TMN}    ایک مساوی الساقین مثلث ہے۔      
\انتہا{سوال}
%
\ابتدا{سوال}
ایک چھار طرفہ کے کونے \( D(3,-2), \, E(0,-3), \, F(-2,3), \, G(4,1). \) ہیں۔
\begin{multicols}{2}
\begin{enumerate}[a.]
\item
چھار طرفہ کی تمام اطراف کی لمبائ معلوم کریں
\item
چھار طرفہ\عددی{ DEFG } کس طرح کی شکل ہے؟
\end{enumerate}
\end{multicols}
\انتہا{سوال}
\ابتدا{سوال}
نقطے  \( A(2,1), \, B(6,10), \, C(10,1) \) ایک مساوی الساقین مثلث ہے اور اس میں  \عددی{AB }   اور\عددی{BC}       کی لمبائ برابر ہے۔ نقطہ\عددی{ G}  کے  محدد  (6,4)   ہیں
\begin{multicols}{2}
\begin{enumerate}[a.]
\item
لکیر \عددی{ AC} کے  وسطی نقطے\عددی{M}    کے محدد لکھیں
\item
ثابت کریں کہ \عددی{BG=2GM}      اور یہ کہ \عددی{BGM}       ایک سیدھی لکیر ہے۔
\item
لکیر \عددی{ BC}    کے وسطی نقطے  \عددی{ N}   کے محدد لکھیں۔
\item
ثابت کریں کہ \عددی{AG=2GN} اور یہ کہ \عددی{ AGN } ایک سیدھی لکیر ہے۔
\end{enumerate}
\end{multicols}
\انتہا{سوال}
%Page 08
\حصہ{ایک سیدھی لکیر یا خط کی مساوات سے کیا مراد ہے ؟}\شناخت{ایک-سیدھی-لکیر-یا-خط-کی-مساوات-سے-کیا-مراد-ہے }
اگر آپکو فیصلہ کرنا ہو تو آپ یہ کیسے اندازہ لگائیں گے کہ نقطے \( (3,7)   \)  اور  \(    (1,5)   \) خم \( y=3x^2 +27 \) پے موجود ہیں ؟ اسکا جواب ہے آپ ان محدد کو مساوات میں ڈالیں اور دیکھیں کہ کیا مساوات درست ثابت ہوتی ہے؟
اگر ہم محدد \( (3,7)   \)  کو مساوات میں ڈالنا چاہیں تو مساوات کی دائیں جانب \( 3 \times 3^2 +2 = 29 \) جبکہ بائیں جانب 7 ہوگی، لہٰذہ مساوات درست ثابت نہیں ہوتی اور یوں یہ بات ثابت ہوتی ہے کہ نقطہ \( (3,7)   \)  بتاۓ گۓ خم کا حصہ نہیں ہے۔
اگر محدد \(    (1,5)   \)  پر غور کیا جاۓ تو مساوات  کے دونوں اطراف کا جواب 5 آۓ گا اور یوں یہ مساوات درست ثابت ہوتی ہے اور یہ بھی ثابت ہوتا ہے کہ نقطہ \(    (1,5)   \)  خم کا حصہ ہے۔ 
ایک سیدھی لکیر یا خم کی مساوات دراصل ایک اصول ہے جو اس بات کا تعین کرتا ہے کہ دیے گۓ محدد بتائ گئ لکیر یا خم کا حصہ ہوں گے یا نہیں۔
لکیر یا خم کی مساوات کو دیکھنے کا یہ نظریہ بہت اہمیت کا حامل ہے۔
 \حصہ{لکیر کی مساوات}\شناخت{لکیر-کی-مساوات}
\ابتدا{مثال}
ایک لکیر جسکا ڈھلاؤ \عددی{2}  ہے اور جو محدد \(   (2,1)  \) سے گزرتی ہے ایسی لکیر کی مساوات تلاش کریں۔
شکل \حوالہء{1.9} میں ایک لکیر دکھائ گئ ہے جسکا ڈھلاؤ  \عددی{2} ہے  اور یہ محدد \( A (2,1)  \)  سے بھی گزر رہی ہے۔ جبکہ ایک اور نقطہ \(   P (x,y)\)  بھی اس لکیر پر موجود ہے۔ نقطہ  \عددی{P}   اس لکیر پر موجود ہوگا صرف اور صرف اس صؤرت میں اگر لکیر \عددی{AP}    کا  ڈھلاؤ \عددی{2}  ہوگا۔ 
لکیر   \عددی{AP} کا کا ڈھلاؤ  \( \frac{y-1}{x-2} \) ۔ یہ ترکیب چونکہ \عددی{2} کے برابر ہے   \( \frac{y-1}{x-2} =2 \)جسکا نتیجہ یہ نکلتا ہے کہ \( y-1=2x-4 \) اور \(y=2x-3\)۔
عام طور پر لکیر کی مساوات معلوم کرتے ہیں جسکا ڈھلاؤ  \عددی{ m} ہو اور جو نقطے  \عددی{A } سے گزرتی ہو جبکہ \عددی{A } کے محدد  \(   (x_{1},y_{1})  \) ہوں 
شکل \حوالہء{1.10} میں یہ لکیر اور ایک نقطہ \عددی{P}   دکھاۓ گۓ ہیں جسکے محدد  \(     (x,y)\) ہیں۔ لکیر  \عددی{AP}     کا ڈھلاؤ \( \frac{y-y_1}{x-x_1} \)ہے اور چونکہ ڈھلاؤ \عددی{ m} کے برابر ہوتا ہے \( \frac{y-y_1}{x-x_1} =m, \, یا \, y-y_1 = m(x-x_1) \)
\انتہا{مثال}
 ایک لکیر جو \(   (x_{1},y_{1})  \) سے گزرے اور جسکا ڈھلاؤ m ہو اسکی مساوات \( y-y_1 = m(x-x_1) \) ہوگی۔
یہ بات ذہن نشین کر لیں کہ نقطہ \عددی{ A } کے محدد \(   (x_{1},y_{1})  \) کی قیمت سے یہ مساوات درست ظابت ہوتی ہے۔
 % page 9
\ابتدا{مثال}
ایک لکیر کی مساوات معلوم کریں جسکا ڈھلاؤ -1 ہو جو نقطہ \( (-2,3) \) سے گزرتی ہو۔
مساوات \( y-y_1 = m(x-x_1) \) کو استعمال کرتے ہوۓ ہم کہہ سکتے ہیں کہ \( y-3=-1(x-(-2)), \) جوکہ \( y-3=-x-2 \) یا \(y=-x+1\) ہے۔  مساوات کی درستگی کا تعین کرنے کے لیے محدد \( (-2,3) \) کو مساوات کے دونوں اطراف استعمال کریں اگر مساوات کے دونوں اطراف کا جواب برابر ہے تو یہ نقطہ دراصل اسی لکیر پر ہوگا جسکی ہم نے  مساوات معلوم کی ہے۔
\انتہا {مثال}
\ابتدا{مثال}
ایک لکیر کی مساوات معلوم کریں جو کہ دو نقطوں کو جوڑنے سے بنی ہے، نقطوں کے محدد بالترتیب \( (3,4), \) اور  \(  (-1,2) \) ہیں۔
مساوات معلوم کرنے کے لیے ، پہلے آپ اس لکیر کا ڈھلاؤ معلوم کریں اور پھر آپ کلیہ \( y-y_1 = m(x-x_1), \) کا استعمال کر سکتے ہیں۔
لکیر جوکہ نقطہ \( (3,4), \) کو \(  (-1,2) \) سے جوڑتی ہے اسکا ڈھلاؤ  ہوگا \( \frac{2-4}{(-1)-3} = \frac{-2}{-4} = \frac{1}{2} \)
لہٰذہ نقطہ \( (3,4), \)  سے گزرنے والی لکیر جسکا ڈھلاؤ \( \frac{1}{2} \)  ہے اسکی مساوات \( y-4= \frac{1}{2} (x-3). \) ہوگی۔ اس مساوات کو سادہ شکل میں دیکھا جاۓ تو یہ کچھ ایسی دکھے گی۔\( 2y-8=x-3 \)  یا\( 2y=x+5 \)۔
اس مساوات کی درستگی کو دیکھنے کے لیے اس میں دیگر فرضی نقطوں کے محدد بھی ڈال کے دیکھیں ۔
\انتہا{مثال}
\حصہ{لکیر کی مساوات کی پہچان}
مثالوں \حوالہء{1.5.1} سے \حوالہء{1.5.3} تک سب کے جوابات مساوات \( y=mx+c, \) کی صورت میں لکھے جا سکتے ہیں جبکہ \عددی{m}   اور \عددی{ c }اعداد ہیں۔
ایسی کسی بھی مساوات کو سیدھی لکیر کی مساوات ثابت کرنا نہایت ہی آسان ہے۔ اگر \( y=mx+c, \) تو ہم کہہ سکتے ہیں کہ \( y-c = m(x+0) \) اور 
\[ \frac{y-c}{x-0} \quad (x \neq 0) \]
یہ مساوات ہمیں بتاتی ہے کہ تمام نقطوں کے لیے کہ جنکے محدد \( (x,y) \) ہوں گے، وہ لکیر جو نقطہ \( (0,c) \) کو جوڑے گی \( (x,y) \) سے، اسکا ڈھلاؤ \عددی{m} ہوگا۔ لب لباب یہ کہ \( (x,y) \) اس لکیر کا حصہ ہوگا جسکا ڈھلاؤ \عددی{m} ہوگا اور جو نقطہ \( (0,c) \) سے گزرتی ہوگی۔
نقطہ \( (0,c) \) y-محور پر موجود ہے۔ اس ہندسے   \عددی{c} کو قطع واۓ کہیں گے۔
قطع ایکس معلوم کرنے کے لیے مساوات میں \( y=0 \) یہ ڈالیں، اور یوں آپکو ملے گا \( x=-\frac{c}{m} \)، لیکن یہ بات ذہن نشین کر لیں کہ آپ یہ تقسیم نہیں کر سکتے اگر \( m=0 \) ہو۔ ایسی صورت حال میں یہ لکیر   x-محور  کے متوازی ہو جاتی ہے اور اسکا کوئ قطع ایکس نہیں ہوتا۔ 
جب ایسی صورت حال ہو کہ ڈھلاؤ کی قیمت صفر ہو جاۓ تو ایسی لکیر پر موجود تمام نقاط کے محدد \( (\text{\RL{کچھ بھی}} ,c) \) کچھ ایسے ہوں گے۔لہٰذہ نقاط \( (1,2), \, (-1,2), \, (5,2), \cdots \) سب ایک ہی سیدھی لکیر پر موجود ہیں جو کہ \( (y=2) \) ہے اور (شکل \حوالہء{1.11})  میں دکھائ بھی گئ ہے۔ ایک خاص صورت اسمیں یہ بھی ہے کہ x-محور کی مساوات  \( y=0 \) ہے۔
% page 10
ایسے ہی ایک سیدھی لکیر جو کہ  y-محور کے متوازی ہے ، اسکی مساوات \( x=k \) ایسی ہوگی۔ اس لکیر پر موجود تمام نقاط کے محدد  \( k، کچھ بھی \) کچھ ایسے ہوں گے۔ لہٰذہ یہ تمام نقاط \( (3,0), \, (3,2), \, (3,4), \cdots \) ایک لکیر پر موجود ہیں اور وہ لکیر \( x=3 \) ہے اور (شکل \حوالہء{1.12}) میں یہی لکیر دکھائ گئ ہے۔ یہاں   y محور کی اپنی مساوات \( x=0 \) ہے۔
لکیر  \( x=k \) کا کائ ڈھلاؤ نہیں ہے، دراصل اسکا ڈھلاؤ متعین نہیں کیا جا سکتا۔ اور اسکی مساوات \( y=mx+c \) ایسے نہیں لکھی جا سکتی۔
\حصہ{مساوات \( ax+by+c=0 \)} 
فرض کریں کہ آپکے پاس ایک مساوات ہے \( y= \frac{2}{3} x + \frac{4}{3}. \)۔ یہ آسان ہے کہ اس مساوات کو 3 سے ضرب دیں اور یوں مساوات \( 3y=2x+4 \) سادہ ہو جاۓ گی۔ اور اسکی ترتیب تھوڑی بدلیں تو مساوات کچھ ایسا روپ دھار لے گی، \( 2x-3y+4=0 \)۔ مساوات عام طور پر \( ax +by +c =0 \) ایسی ہوتی ہے جسمیں a ،  b اور  c مستقل ہیں۔ 
اس بات کا خیال رکھیں کہ مساوات  \( y=mx+c \) اور \( ax+by+c=0 \) دونوں میں عدد  \عددی{c} موجود ہے لیکن اس کا مطلب دونوں مساوات میں مختلف ہے۔ مساوات \( y=mx+c \) میں  \عددی{c} قطع واۓ ہے لیکن مساوات \( ax+by+c=0 \) میں ایسا کوئ معاملہ نہیں ہے۔
مساوات \( ax+by+c=0 \) کا ڈھلاؤ معلوم کرنے کا ایک طریقہ یہ بھی ہے کہ مساوات کو  \( y= \cdots \) کی شکل میں لکھا جاۓ ، آگے چل کے ہم اسکی کچھ مثالیں حل کریں گے۔
\ابتدا{مثال}
مساوات \( 2x+3y-4=0 \) کا ڈھلاؤ معلوم کریں،
مساوات کو اس \( y= \cdots \) شکل میں لکھیں اور پھر اس اصول کو استعمال کریں کہ مساوات \( y=mx+c) \) میں \عددی{m} ڈھلاؤ کو ظاہر کرتا ہے۔ 
مساوات \(  2x+3y-4-0 \) میں آپ دیکھیں گے کہ \( 3y = -2x +4 \)اور\( y= - \frac{2}{3} x + \frac{4}{3}. \) لہٰذہ اس مساوات کا اگر  اس مساوات \( y=mx+c ,\) سے تقابل کیا جاۓ تو ہم اس نتیجے پر پہنچیں گے کہ ڈھلاؤ  \(- \frac{2}{3} \) ہے
  \انتہا{مثال} 
\ابتدا{مثال}
متوازی الاضلاع کی ایک طرف ایک سیدھی لکیر \( 3x -4y -7 = 0 \) کے ساتھ موجود ہے ، نقطہ \( (2,3) \) متوازی الاضلاع کا ایک کونہ ہے، دوسری طرف کی مساوات معلوم کریں۔
لکیر \( 3x-4y-7=0 \) اور \( y= \frac{3}{4}x - \frac{7}{4} , \) ایک ہی ہیں لہٰذہ ہم کہہ سکتے ہیں کہ ڈھلاؤ \( \frac{3}{4}\) ہے۔ لکیر جو کہ نقطہ \( (2,3) \) سے گزر رہی ہے اور جسکا ڈھلاؤ \( \frac{3}{4}\) ہے ، اسکی مساوات \( y-3=\frac{3}{4} (x-2) , \) یا \( 3x-4y+6=0 \) ہے
 \انتہا{مثال}
  \حصہ{دو لکیروں کا مشترک نقطہ} 
فرض کریں کہ آپکے سامنے دو لکیریں ہیں جنکی مساوات \( 2x-y=4  \) اور \(  3x+2y=-1\) ہیں، آپ ان دونوں لکیروں کے مشترک نقطے کے محدد کیسے معلوم کریں گے؟
دراصل آپ کو ایک نقطے \(  (x,y)    \) کی تلاش ہے جو کہ دونوں لکیروں پر موجود ہو، لہٰذہ اس نقطے کے محدد ایسے ہونے چاہئیں کہ دونوں مساوات درست  ثابت ہوں، اسی لیے آپکو ان دونوں مساوات کو ایک ساتھ حل کرنا ہوگا۔
% page 11
ان دو مساوات سے ، اپ معلوم کر سکیں گے کہ \( x=1\) اور \( y=-2, \) ، لہٰذہ مشترک نقطہ \( (1,-2) \) ہے۔ 
یہ طریقہ ہر سیدھی لکیر پر لاگو ہو سکتا ہے بشرطیکہ یہ متوازی نہ ہوں، مشترک نقطہ معلوم کرنے کے لیے لکیروں کی مساوات حل کریں، یہ طریقہ خموں میں مشترک نقطے معلوم کرنے کے لیے بھی استعمال کیا جا سکتا ہے۔ 
% exercise 1B
\ابتدا{سوال}
دیکھیں کہ کیا دیے گۓ نقطے ، بتائ گئ مساوات کی لکیر پر موجود ہیں یا نہیں؟
\begin{multicols}{2}
   \begin{enumerate}[.a]
\item \( (1,2),\,y=5x-3 \)
\item \( (3,-2),\,y=3x-7 \)
\item \( (3,-4),\,x^2 + y^2 = 25 \)
\item \( (2,2),\,3x^2 + y^2 = 40 \)
\item \( \big(1,1\frac{1}{2} \big),\,y=\frac{x+2}{3x-1} \)
\item \( \big( 5p, \frac{5}{p},\,y=\frac{5}{x} \)
\item \( \big(p,(p-a)^2 +1 \big),\,y=x^2 -2x +2 \)
\item \( \big( t^2, 2t \big),\,y^2 = 4x \)
\end{enumerate}
\end{multicols}
\انتہا{سوال} 
\ابتدا{سوال}
 بتاۓ گۓ نقطوں سے بنی اور درج ذیل ڈھلاؤ والی   سیدہی لکیر کی مساوات معلوم کریں۔ آپکے جواب کسر کی صورت میں نہیں ہونا چاہئیے۔
\begin{multicols}{4}
\begin{enumerate}[.a]
\item \((2,3), 5\)
\item \((1,2), -3\)
\item \((0,4),\frac{1}{2}\)
\item \((-2,1),-\frac{3}{8}\)
\item \((0,0),-3\)
\item \((3,8), 0\)
\item \((-5,-1),-\frac{3}{4}\)
\item \((-3,0), \frac{1}{2}\)
\item \((-3, -1),\frac{3}{8}\)
\item \((3,4),-\frac{1}{2}\)
\item \((2,-1),-2\)
\item \((-2,-5), 3\)
\item \((0,-4), 7\)
\item \((0,2),-1\)
\item \((3,-2), -\frac{5}{8}\)
\item \((3,0), -\frac{3}{5}\)
\item \((d,0),7\)
\item \((0,4),m\)
\item \((0,c),3\)
\item \((c,0),\)
\end{enumerate}
\end{multicols}
\انتہا{سوال}  
\ابتدا{سوال}
درج ذیل نقاط کو جوڑ کر بننے والی لکیروں کی مساوات معلوم کریں۔ آپکے جواب میں کسر موجود نا ہوں اور آپکا جواب \(y=mx+c  \)    یا   \(ax+by+c=0\) کی صورت میں ہونا چایئیے۔
\begin{multicols}{3}
\begin{enumerate}[a.]
\item \((1,4),(3,10)\)
\item \((4,5),(-2,-7)\)
\item \((3,2),(0,4)\)
\item \((3,7),(3,12)\)
\item \((10,-3),(-5,-12)\)
\item \((3,-1),(3,-4,20)\)
\item \((2,-3),(11,-3)\)
\item \((2,0),(5,-1)\)
\item \((-4,2),(-1,-3)\)
\item \((-2,-1),(5,-3)\)
\item \((-3,4),(-3,9)\)
\item \((-1,0),(0,-1)\)
\item \((2,7),(3,10)\)
\item \((-5,4),(-2,-1)\)
\item \((0,0),(5,-3)\)
\item \((0,0),(p,q)\)
\item \((p,q),(p+3,q-1)\)
\item \((p,-q),(p,q)\)
\item \((p,q),(p+2,q+2)\)
\item \((p,0),(0,q)\)
\end{enumerate}
\end{multicols}
\انتہا{سوال}  
%  page 12 
\ابتدا{سوال}
درج ذیل لکیروں کا ڈھلاؤ معلوم کریں؛
\begin{multicols}{4}
\begin{enumerate}[a.]
\item \(2x+y=7 \)
\item \(3x-4y=8 \)
\item \(5x+2y=-3\)
\item \(y=5 \)
\item \(3x-2y =-4 \)
\item \(5x=7 \)
\item \(x+y=-3 \)
\item \(y=3(x+4) \)
\item \(7-x=2y \)
\item \(3(y-4) =7x \)
\item \(y=m(x-d) \)
\item \(px+qy=pq \)
\end{enumerate}
\end{multicols}
\انتہا{سوال}  
\ابتدا{سوال}
ایک لکیر، جو کہ نقطہ \( (-2,1) \) سے گزرتی ہے اور \( y=\frac{1}{2}x-3 \) کے متوازی ہے، کی مساوات معلوم کریں۔
\انتہا{سوال} 
\ابتدا{سوال}
ایک لکیر کی مساوات معلوم کریں جو کہ \((4,-3)\) سے گزرتی ہے اور ایک دوسری لکیر \( y+2x=7 \) کے مساوی ہے۔
\انتہا{سوال} 
\ابتدا{سوال}
ایک لکیر جہ کہ نقطہ \((1,2)\) سے گزر رہی ہے ، یہ لکیر ایک دوسری لکیر کے متوازی ہے جو کہ نقاط \( (3,-1)\)اور \( (-5,2) \) سے مل کر بنی ہے۔
\انتہا{سوال}  
 \ابتدا{سوال}
ایک لکیر کی مساوات معلوم کریں جہ کہ نقطہ \(  (3,9) \) سے گزر رہی ہے اور مساوی ہے ایک لکیر کے جو نقاط \( (-3,2) \) اور \( (2, -3) \) سے مل کر بنی ہے۔
\انتہا{سوال}  
\ابتدا{سوال}
لکیر کی مساوات معلوم کریں جو کہ \( (1,7) \) سے گزرتی ہے اور  x -محور کے متوازی ہے
\انتہا{سوال}  
 \ابتدا{سوال}
ایک لکر کی مساوات معلوم کریں جو کہ \( (d,0) \) سے گزرتی ہے اور ایک دوسری لکیر \( y=mx+c \) کے متوازی ہے۔
\انتہا{سوال}  
\ابتدا{سوال}
درج ذیل سیدھی لکیروں کی مساوات معلوم کریں۔
\begin{multicols}{2}
\begin{enumerate}[a.]
\item \(3x+4y=33, \, 2y=x-2 \)
\item \(y=3x+1 , \, y=4x-1 \)
\item \(2y=7x, \, 3x-2y=1 \)
\item \(y=3x+8 , \, y=-2x-7 \)
\item \(x+5y=22 , \, 3x+2y=14 \)
\item \(2x+7y=47, \, 5x+4y=50 \)
\item \(2x+3y=7, \, 6x+9y =11 \)
\item \(3x+y=5, \, x+3y=-1 \)
\item \(y=2x+3, \, 4x-2y =-6 \)
\item \(ax+by=c, \, y=2ax \)
\item \(y=mx+c, \, y=-mx+d \)
\item \(ax-by=1, \, y=x \)
\end{enumerate}
 \end{multicols}
\انتہا{سوال} 
\ابتدا{سوال}
فرض کریں کہ \عددی{p} جسک محدد \( (p,q) \)  ہیں اور یہ خم \(y=mx+c) \)  کا ایک مستقل نقطہ ہے اور ایسے ہی ایک نقطہ \عددی{Q}ہے جسکے محدد \( (r,s) \)  ہیں اور یہ بھی مساوات \( y=mx+c \) کے خم کا ایک مستقل نقطہ ہے ۔ یہ بات ثابت شدہ ہے کہ نقطوں \عددی{p} اور \عددی{Q} کے محدد سے مساوات \(y=mx+c) \)  درست ٹھرتی ہے، ثابت کریں کہ خط \عددی{ PQ}کا ڈھلاؤ \عددی{m} ہوگا نقطہ \عددی{Q}کی تمام حالتوں کے لیے۔
\انتہا{سوال}  
\ابتدا{سوال}
نقاط \عددی{a},\عددی{b} اور  \عددی{c}کی چند ایک قیمتوں کے لیے مساوات  \(ax+by+c=0 \) ایک سیدھی لکیر کی نہیں رہتی۔ ایسی چند قیمتیں معلوم کریں۔           
\انتہا{سوال}
\حصہ{عمودی لکیروں کا ڈھلاؤ} 
 
(حصہ\حوالہء{1.3} ) میں یہ بتایا گیا ہے کہ دو لکیریں متوازی ہوتی ہیں اگر انکے ڈھلاؤ برابر ہوں۔ لیکن اگر دو لکیریں عمودی ہوں تو انکے ڈھلاؤ کیسے ہوں گے۔
اگر ایک لکیر جسکا ڈھلاؤ مثبت ہو تو عمودی لکیر کا ڈھلاؤ منفی ہوگا، اور اسکا الٹ بھی درست ہوگا، لیکن آپ سے ذیادہ بہتر اندازہ لگا سکتے ہیں
(شکل \حوالہء{1.3} ) میں یہ دکھایا گیا ہے کہ خط \عددی{PB}      کا ڈھلاؤ \عددی{m}      ہو تو ایک ڈھلاؤ مثلث \عددی{ PAB} بنائ جا سکتی ہے جسمیں \عددی{PA} کی لمبائ ایک اکائ ہے اور خط \عددی{AB}کی لمبائ \عددی{m} اکائیاں ہے۔                    
%Page 13
 (شکل \حوالہء{1.14}    ) میں ڈھلاؤ مثلث \عددی{PAB} کو گھمایا گیا ہے ایک قائمہ زاویہ سے اور اب مثلث \( P^{\prime}A^{\prime}B^{\prime}\) ہے کچھ یوں کہ خط \( P^{\prime}B^{\prime} \) عمودی ہے خط   \عددی{ PB} پر۔ اس مثلث کا x-محدد \عددی{ -m}ہے جبکہ  x-محدد \عددی{1} ہے، اور یوں ؛
\[PB' \text{ڈھلوان}= \frac{\text{\RL{قدم \(y\)}}}{\text{\RL{قدم \(x\)}}} = \frac{1}{-m} = -\frac{1}{m} \]
اور اسی لیے خط \عددی{ PB} کے عمودی لکیر کا ڈھلاؤ  \( -\frac{1}{m} \) ہے۔
اور پس اگر دو عمودی لکیروں کا ڈھلاؤ بالترتیب \( m_{1}\) اور  \( m_{2}\) ہو اور پھر \( m_{1}m_{2}=-1\) بھی ہو تو یہ سچ ہے کہ دونوں لکیروں کے ڈھلاؤ بالترتیب \( m_{1}\) اور  \( m_{2}\) ہوں گے اور اگر \( m_{1}m_{2}=-1\) بھی ہو تو یہ دونوں لکیریں عمودی ہیں۔اس بات کے ثبوت کے لیے آخر میں موجود مشق کا سوال \عددی{22}  دیکھیں۔
دو لکیریں جن کا ڈھلاؤ بالترتیب \( m_{1}\) اور  \( m_{2}\) ہو ، یہ دونوں لکیریں عمودی ہوں گی اگر \[ m_1m_2=-1, \quad m_1 = -\frac{1}{m_2} \quad m_2= - \frac{1}{m_1} \]
یہ بات ذہن نشین کر لیں کہ یہ خصوصیت بے کار ہو گی اگر لکیریں محور کے متوازی ہوں گی۔ لیکن آپ آسانی سے دیکھ سکتے ہیں کہ ایک
 لکیر \(  x = \text{مستقل}\) ایک دوسری لکیر \( y= \text{مستقل}\) کے عمودی ہی ہوگی۔
\ابتدا{مثال}
ثابت کریں کہ نقاط \( (0,5), \, (-1,2), \, (4,7). \, (5,0) \) مجموعی طور پر ایک رومبس بناتے ہیں۔
آپ اس مسئلے کو کئ طریقوں سے حل کر سکتے ہیں، اس حل میں ہم نے ثابت کیا کہ یہ نقاط ایک متوازی الاضلاع چکل بنا رہے ہیں اور یہ کہ اس کے وتر عمودی ہیں تو یہ ایک رومبس کہلاۓ گی۔
وتر کے درمیانی نقاط \( \big( \frac{1}{2} (0+4), \frac{1}{2} (-5+7) \big) \) اور \( (2,1) \) ہیں اور چونکہ یہ دونوں ایک ہی نقطہ ہیں اور بتائ گئ شکل ایک متوازی الاضلاع شکل ہے۔
اب اگر ڈھلاؤ کو دیکھا جاۓ تو \( \frac{7-(-5)}{4-0} = \frac{12}{4} = 3 \)  اور \( \frac{0-2}{5-(-1)} = \frac{-2}{6}= - \frac{1}{3} \) ہیں اور چونکہ ڈھلاؤ کا مضرب \عددی{-1} ہے اسی لیے وتر عمودی ہیں اور یوں ثابت ہوا کہ یہ نقاط مل کر ایک رومبس کو جنم دیتے ہیں۔ 
\انتہا{مثال}
\ابتدا{مثال}
عمودی لکیر کی بنیاد کے محدد معلوم کریں جبکہ \( A(-2,-4)\)  جڑا ہوا  ہے نقاط \(B(0,2) \)  اور \( C(-1,4) \) کے ساتھ۔ لکیر کی مدد سے۔
سب سے پہلے ایک شکل بنائیں جیسے کہ (شکل \حوالہء{1.15} ) ہے اس پر پیمانے کی ضرورت نہیں ہے۔ عمودی لکیر کی بنیاد دراصل وہ مشترک نقطہ\عددی{P}  ہے جہ کہ لکیر \عددی{BC}پر موجود ہےاور ساتھ ہی ساتھ \عددی{A} سے گزرنی والی عمودی لکیر \عددی{ BC} پر بھی موجود ہے۔ سب سے پہلے خط \عددی{ BC} کا ڈھلاؤ اور اسکی مساوات معلوم کریں۔
 %page 14
\انتہا{مثال}
 خط \عددی{BC} کا ڈھلاؤ  \(\frac{4-2}{-1-0} = \frac{2}{-1} =-2 \) ہے ۔ خط \عددی{BC} کی مساوات  \(y-2=-2(x-0), \) ہے جوکہ سادہ ہو کر \( 2x+y=2. \) ایسی صورت اختیار کر لے گی۔
لکیر جوکہ \عددی{A} سے گزرتی ہے اور خط \عددی{BC} کے عمودی ہے اسکا ڈھلاؤ   \( - \frac{1}{-2} = \frac{1}{2} \) ہے۔ 
اس لکیر کی مساوات 
\[ y-(-4)= \frac{1}{2} (x-(-2)), \quad x-2y=6 \]
یا \( x-2y=6. \) ہے۔
یہ لکیریں نقطہ \عددی{P}پر ملتی ہیں جن کے محدد مساوات \(2x+y=2 \)  اور \( x-2y=6\) کو درست ثابت کرتے ہیں ۔ اس نقطے کے محدد \( (2,-2) \) ہیں
\ابتدا{سوال}       
ہر حصے میں خط کا ڈھلاؤ معلوم کریں جو کہ ایک دوسری لکیر کے عمودی ہے جسکا ڈھلاؤ دیا گیا ہے۔
\begin{multicols}{6}
\begin{enumerate}[a.]
\item \(2\)
\item \(-3\)
\item \(\frac{3}{4} \)
\item \(-\frac{5}{6} \)
\item \(-1 \)
\item \(1\frac{3}{4} \)
\item \(-\frac{1}{m} \)
\item \(m \)
\item \(\frac{p}{q} \)
\item \(0 \)
\item \(-m \)
\item \(\frac{a}{b-c} \)
\end{enumerate}
\end{multicols}
\انتہا{سوال}
\ابتدا{سوال}     
ہر حصے میں خط کی مساوات معلوم کریں جو کہ بتائ گئ لکیروں کے عمودی ہیں۔ آپکا جواب کسر کی صورت میں نہیں ہونا چاہئیے۔
\begin{multicols}{3}
\begin{enumerate}[a.]
\item \((2,3),  y=4x+3 \)
\item \((-3,-1), y=1\frac{1}{2}x+3 \)
\item \((2,-5),   y=-5x-2 \)
\item \((7,-4), y=2\frac{1}{2} \)
\item \((-1,4),  2x+3y=8 \)
\item \((4,3),  3x-5y=8 \)
\item \((5,-3),   2x=3 \)
\item \((0,3),   y=2x-1 \)
\item \((0,0),  y=mx+c \)
\item \((a,b),  y=mx+c \)
\item \((c,d),  ny-x=p \)
\item \((-1,-2), ax+by=c \)
\end{enumerate}
\end{multicols}
\انتہا{سوال}
\ابتدا{سوال}            
ایک خط کی مساوات معلوم کریں جو کہ نقطہ \( (-2,5) \)  سے گزرتی ہے اور لکیر  \( y=3x+1 \)  کے عمودی ہے، ان دونوں لکیروں کا مشترک نقطہ بھی معلوم کریں۔
\انتہا{سوال}
\ابتدا{سوال}
ایک خط کی مساوات معلوم کریں جو کہ نقطہ \( (1,1) \)  سے گزرتی ہے اور یہ خط \( 2x-3y=12 \) کے عمودی ہے، ان دونوں لکیروں کا مشترک نقطہ بھی معلوم کریں۔             
\انتہا{سوال}
\ابتدا{سوال}            
ایک لکیر جو مثلث کے ایک کونے سے گزرے اور مخالف سمت کے عمودی ہو، اس لکیر کو اونچائ کا نام دیتے ہیں۔  اس لکیر کی مساوات معلوم کریں جو کہ مثلث \عددی{ABC} کے  کونے \عددی{A} سے گزرتی ہے نقاط کے محدد بالترتیب \( A(2,3), \, B(1,-7), \, C(4,-1). \) ہوں گے۔
\انتہا{سوال}
%Page1
\ابتدا{سوال}            
نقاط \( P(2,5), \, Q(12,5), \, R(8,-7) \) مل کے ایک مثلث بناتے ہیں
\begin{multicols}{2}
\begin{enumerate}[a.]
\item
اونچائ کی مساوات تلاش کریں جو کہ نقطہ \عددی{R} اور پھر نقطہ \عددی{Q} سے گزرے۔  
\item
ان دونوں اونچائیوں کا مشترک نقطہ معلوم کریں
\item
ثابت کریں کہ نقطہ \عددی{P} سے گزرنے والی اونچائ اس مشترک نقطے سے بھی گزرتی ہے۔ 
\end{enumerate}
\end{multicols}
\انتہا{سوال}
%Miscellaneous Exercise 1
\ابتدا{سوال}            
ثابت کریں کہ نقاط \( (-2,5) , \, (1,3), \, (5,9) \) سے بننے والی ایک مثلث قائمہ زاویہ مثلث ہے۔
\انتہا{سوال}
\ابتدا{سوال}            
لکیروں \( 2x+y = 3 \)  اور  \( 3x+5y -1 =0 \) کا مشترک نقطہ معلوم کریں
\انتہا{سوال}
\ابتدا{سوال}            
نقاط \( A(-1,3), \, B(5,7), \, C(0,8). \) کو ملانے سے ایک مثلث بنتی ہے۔
\begin{enumerate}
\item
ثابت کریں کہ زاویہ \عددی{ACB} ایک قائمہ زاویہ ہے۔ 
\item
اس نقطے کے محدد معلوم کریں جہاں \عددی{B}سے آنے والی  خط   \عددی{AC} کے متوازی لکیر x-محور کو کاٹتی ہے۔
\end{enumerate}
\انتہا{سوال}
\ابتدا{سوال}            
ایک مربع شکل ہے جسکے دو کونے \( A(7,2), \, C(1,4) \)  ہیں 
\begin{multicols}{2}
\begin{enumerate}[a.]
\item
وتر \عددی{BD} کی لمبائ معلوم کریں
\item
نقاط \عددی{B} اور \عددی{D}کے محدد معلوم کریں
\end{enumerate}
\end{multicols}
\انتہا{سوال}
\ابتدا{سوال}            
نقاط \( A(-3,2), \, B(4,3), \, C(9,-2), \, D(2,-3). \) کو ملانے سے ایک چوکور شکل بنتی ہے۔
\begin{multicols}{2}
\begin{enumerate}[a.]
\item
ثابت کریں کہ چاروں سمتوں کی لمبائ برابر ہے۔
\item
ثابت کریں کہ شکل \عددی{ ABCD}  ایک مربع نہیں ہے۔
\end{enumerate}
\end{multicols}
\انتہا{سوال}
\ابتدا{سوال}            
\عددی{P} ایک نقطہ ہے جبکہ  \( I_{1} \) ایک لکیر ہے جسکی مساوات \( 3x+4y=16 \) ہے۔
\begin{multicols}{2}
\begin{enumerate}[a.]
\item
ایک لکیر     \( I_{2} \)  کی مساوات معلوم کریں جو کہ نقطہ \عددی{P} سے گزرتی ہے اور لکیر  \( I_{1} \) کے عمودی ہو۔
\item
دونوں لکیروں کا مشترک نقطہ معلوم کریں
\item
نقطے \عددی{ P} سے خط \( I_{1} \) کا عمودی فاصلہ معلوم کریں
\end{enumerate}
\end{multicols}
\انتہا{سوال}
\ابتدا{سوال}            
ثابت کریں کہ مثلث جس کے کونے \( (-2,8), \, (3,20), \, (11,8) \) ہیں ایک مساوی الثاقین مثلث ہے۔ اسکا حدود اربعہ معلوم کریں
\انتہا{سوال}
\ابتدا{سوال}            
تین سیدھی لکیریں \( y=x, \, 7y=2x, \, 4x+y =60 \) ایک مثلث بناتی ہیں۔ اسکے کونوں کے محدد معلوم کریں۔ 
\انتہا{سوال}
\ابتدا{سوال}            
ایک لکیر کی مساوات معلوم کریں جو کہ نقطہ \( (1,3) \) سے گزرتی ہے اور یہ لکیر متوازی ہے ایک دوسری لکیر کے جس کی مساوات \(2x+7y=5. \) ہے ۔ یاد رکھیں آُپکا جواب کچھ اس \(ax+by=c \) صورت میں ہونا چاہئیے۔
\انتہا{سوال}
\ابتدا{سوال}            
 نقاط \( (2,-5), \, (-4,3) \) کو ملانے سے بننے والی لکیر کی عمودی دوئزک کی مساوات معلوم کریں۔
\انتہا{سوال}
\ابتدا{سوال}            
نقاط جن کے محدد \( A(1,2), \, B(3,5), \, C(6,6), \) ہیں اور نقطہ \عددی{D} مل کر ایک متوازی الاضلاع شکل بناتے ہیں۔ خط \عددی{AC} کے درمیانی نقطے کے محدد معلوم کریں، اور اس جواب کو استعمال کرتے ہوۓ نقطہ \عددی{D} کے محدد معلوم کریں۔  
\انتہا{سوال}
\ابتدا{سوال}            
ایک خط \(y=3x\) پے ایک نقطہ \(A(0,3) \) سے ایک عمودی لکیر پر نقطہ  \عددی{P} عمودی خط کا بنیادی خط ہے۔
\begin{multicols}{2}
 \begin{enumerate}[a.]
\item
خط \عددی{AP}کی مساوات معلوم کریں۔ 
\item
نقطہ \عددی{P}کے محدد معلوم کریں 
\item
نقطہ \عددی{A} کا خط  \(y=3x\) سے عمودی فاصلہ معلوم کریں۔
\end{enumerate}
\end{multicols}
\انتہا{سوال}
%PAGE 16
\ابتدا{سوال}            
وہ نقاط جو ایک ہی لکیر پر موجود ہوں انہیں ہم ہم پلہ نقاط کہتے ہیں، ثابت کریں کہ نقاط \( (-1,3), \, (4,7), \, (-11,-5) \) ہم پلہ ہیں۔
\انتہا{سوال}
\ابتدا{سوال}            
سیدھی لکیر کی مساوات معلوم کریں جہ کہ نقاط \( (3,-1), \, (-2,2), \) سے گزرتی ہے ، اور اپنا جواب \(ax+by+c=0 \) کی صورت میں لکھیں۔ x-محور اور اس لکیر کا مشترک نقطہ معلوم کریں۔
\انتہا{سوال}
\ابتدا{سوال}            
نقاط \عددی{A} اور\عددی{B} کے محدد بالترتیب \( (3,2)\)  اور  \((4,-5) \) ہیں، خط \عددی{AB} کے درمیانی نقطے کے محدد معلوم کریں نیز خط \عددی{AB} کا ڈھلاو بھی معلوم کریں۔
اور خط \عددی{ AB} کے عمودی دوئزی کی مساوات بھی معلوم کریں، آپکا جواب \(ax+by+c=0 \) کی صورت میں ہونا چاہئیےجسمیں  \عددی{a}\عددی{b} اور \عددی{c} اعداد صحیح ہیں۔  
\انتہا{سوال}
\ابتدا{سوال}            
خم \( y= 1 + \frac{1}{2+x} \) x-محور کو نقطہ \عددی{A} پے کاٹتا ہے جبکہ y-محور کو نقطہ \عددی{ B}پے کاٹتا ہے۔
\begin{multicols}{2}
 \begin{enumerate}[a.]
\item  
نقاط \عددی{A} اور \عددی{B} کے محدد معلوم کریں
\item
خط \عددی{AB} کی مساوات معلوم کریں 
\item
خط \عددی{AB} اور مساوات \( 3y=4x \) کی لکیر کا مشترک نقطہ معلوم کریں۔
\end{enumerate}
\end{multicols}
\انتہا{سوال}
\ابتدا{سوال}            
ایک سیدھی لکیر  \عددی{P} ایک نقطے \( (10,1) \)  سے گزرتی ہے اور یہ لکیر عمودی ہے ایک دوسری لکیر \عددی{r} کے جسکی مساوات \( 2x+y=1 \) ہے۔ آپ لکیر \عددی{P} کی مساوات معلوم کریں۔
دونوں لکیروں کا مشترک نقطہ بھی معلوم کریں جبکہ نقطے \( (10,1) \)  کا لکیر \عددی{r} سے عمودی فاصلہ بھی معلوم کریں۔
\انتہا{سوال}
\ابتدا{سوال}            
حساب کتاب سے ثابت کریں کہ نقاط \( P(0,7), \, Q(6,5),\, R(5,2), \, S(-1,4) \) ایک مستطیل بناتے ہیں
\انتہا{سوال}
\ابتدا{سوال}            
لکیر \( 3x-4y=8 \) x-محور کو نقطہ \عددی{A} پے کاٹتی ہے، نقطہ \عددی{ C}کے محدد    \( (-2,9). \) ہیں، نقطہ \عددی{ C} سے گزرنے والی لکیر ایک دوسری لکیر  \( 3x-4y=8 \) پر عمودی ہے۔ مثلث \عددی{ABC} کا حدود اربعہ معلوم کریں۔
\انتہا{سوال}
\ابتدا{سوال}            
نقاط \( A(-3,-4) , \, C(5,4) \) ایک رومبس \عددی{ABCD} کے وتر کے انتیائ نقاط ہیں
\begin{multicols}{2}
\begin{enumerate}[a.]
\item
وتر \عددی{BD} کی لمبائ معلوم کریں 
\item
اگر یہ مان لیا جاۓ کہ خط \عددی{BC} کا ڈھلاؤ  \( \frac{5}{3} \) ہے تو آپ نقاط \عددی{B} اور \عددی{D} کے محدد معلوم کریں
\end{enumerate}
\end{multicols}
\انتہا{سوال}
\ابتدا{سوال}            
وسطانیہ کی مساوات معلوم کریں اگر مثلث کے کونے \( (0,2), \, (6,0), \, (4,4) \) ہیں یہ بھی ثابت کریں کہ تمام وسطانیے ایک ہی نقطے سے گزرتے ہیں۔
\انتہا{سوال}
\ابتدا{سوال}            
دو لکیروں کی مساوات بالترتیب \( y=m_1x +c_1  \) اور  \(y=m_2x + C    \) ہیں جبکہ \(m_{1}m_{2}=-1. \). ثابت کریں کہ لکیریں عمودی ہیں۔ 
\انتہا{سوال}

