\باب{دو درجی مساوات}\شناخت{باب_دو_درجی_مساوات}
\section*{دو درجی الجبرا}
 یہ باب \(ax^{2}+bx+c\) کی طرز دودرجی الجبرائی عبارت اور ترسیمات سے متعلق ہے، اِسکے اختتام پر آپ مندرجہ ذیل معلومات حاصل کر چکے ہوں گے کہ\\
 1) دودرجی الجبرائی عبارت کا مربع کیسے لیا جاتا ہے\\
   2) دودرجی الجبرائی ترمیم \(y=ax^{2}+bx+c\) کے راس اور محور تشاکل کو کیسے معلوم کیا جاتا ہے\\
 3) دودرجی مساوات کو کیسے حل کیا جاتا ہے\\
 4) ہمزاد مساوات کا حل جس میں ایک دودرجی مساوات اور  دوسری خطی مساوات ہو\\
 5) اٌں مساوات کی شناخت اور حل جنکی دودرجی مساوات میں تحفیف ترکیب بدل کرکی جاسکتی ہو\\
 4.1۔دودرجی عبارات\\
 آپ جانتے ہیں \(y=bx+c\) کہ کا ترسیم خطِ مستقیم ہے \(y=bx+c\) خطی مساوات کہلاتی ہیں۔باب سوم میں آپ نے سیکھا کہ اگر اسمیں \(ax^{2}\) جمع کریں تو مساوات \(y=ax^{2}+bx+c\) حاصل ہوگی جسکا ترسیم قطع مکافی ہے یہ عبارت  \(ax^{2}+bx+c\) کہ جسمیں b,a اور c مستقل ہیں۔دودرجی عبارت کہلاتی ہے۔مثلاً \(x^{2}\) ،\(x^{2}-6x+8\)،\(2x^{2}-3x+4\)  اور  \(-3x^{2}-5\) دودرجی مساوات ہیں آپ کسی بھی دودرجی کیلۓ \(ax^{2}+bx+c\) کی طرح لکھ سکتے ہیں جسمیں a,b اور c مستقل ہیں۔ b اور c آپکی پسند کے کوئی بھی اعدد ہو سکتے ہیں مشمول '0' ،لیکں a قطعاً '0' نہیں ہوسکتا ہے۔(عبارت دودرجی نہیں رہے گی) اعداد a,b اور c عددی سر کیلاتے ہیں: \(x^{2}\) کا عددی سرx, a  کا عددی سر b اور c عمومی طور پر مستقل جزو کہلاتا ہے \(2x^{2}-x+4\) میں x اور \(x^{2}\) کے عددی سر بالترتیب 1 اور 2 ہیں جبکہ جزو 4 ہے۔\\
\section*{ 2.4-کامل مربعی صورت}
 آپ ایک دودرجی الجبرائی عبارت, \(x^{2}-6x+8\) کو بہت سے طریقوں سے لکھ سکتے ہیں جنمیں جزوی صورت \((x-4)(x-2)\) شامل ہے جوکہ اٌفقی محور پر قطع مکانی \(y=x^{2}-6x+8\) کا مقامِ انقطاع معلوم کرنے کیلئے استعمال کی جاسکتی ہے۔ جیسا کہ تصویر۔۔۔۔۔۔۔۔۔۔۔۔ میں دیکھایا گیا ہے۔جبکہ صورت، قطع مکافی کے راس کی نساندہی کیلیۓ اور تفاعل \(f(x)=x^{2}-6x+8\) کی حدود معلوم کرنے کیلیۓ استعمال کی جاسکتی ہے۔ جیسا کہ مثال میں دی گئی تصویر میں دکھایا گیا ہے۔ یاد رہے کہ آپ ہمیشہ دودرجی عبارت کو جزوی صورت میں نہیں لکھ سکتے ہیں۔مثال کے طور پر \(x^{2}+2x+3\)\\
 \newpage
 % صفحہ نمبر 52
 اگر آپ ترسیمی مساوات \(y=x^{2}-6x+8\) کو \(y=(x-3)^{2}-1\) کی صورت میں لکھیں تو آپ باآسانی مجورِ تشاکل اور اس کی نشاندہی کر سکتے ہیں۔ کیونکہ \((x-3)^{2}\) ایک کامل مربع ہے۔لہزا اِسکی قیمت ہمیسہ 0 یا اس سے زیادہ ہوگی اور 0 صرف تب جب \(x=3\) ہو یعنی  \((x-3)^{2}\ge 0\)  ہو اور چونکہ \(y=(x-3)^{2}-1\) ہے تو  \(y\ge – 1\) ہوگا۔جیسے کہ \((x-3)^{2}=0\)  جب \(x=3\) ہو لہزا نقطہ راس  \((3,-1\) ہے اور محورِ تشاکل خط  \(x=3\) ہے\\ \((x-3)^{2}-1\) کو کامل مربع صورت کہتے ہیں۔ذیل میں اِسکے استعمال کی کچھ مزید مثالیں دی  گئی ہیں۔\\
\section*{مثال نمبر               4 . 2 .1}
 دودرجی ترسیم  \(y=3-2(x+2)^{2}\)  کے راس اور تشاکل کی نساندہی کریں۔چونکہ  \(2(x+2)^{2}\ge 0\) اور  \(2(x+2)^{2}=3-y\) ہے۔تو اِسکی پیروی کرتے ہوۓ  \(3-y\ge 0\) لہزا \(y\le 3\) جیسے \((x+2)^{2}=0\)جب \(x=-2\)،ترسیم کا راس وہ نقتہہے جسکے محدد \((-2,3)\) ہیں،y کی سب سے بڑی قیمت 3 ہے۔اور محورِ تشاکل x=-2 ہے۔\\
\section*{ مثال نمبر 4 . 2 . 2}
 مساوات کو حل کریں۔\\
 جیسے  \(3(x-2)^{2}-2=0\) , \(3(x-2)^{2}=2\) اور  \((x-2)^{2}=\frac{2}{3}\) چنانچہ \((x-2)=\pm \sqrt{\frac{2}{3}}\), \(x=2\pm \sqrt{\frac{2}{3}}\) \\ 
 \section*{ 4.3 مربع مکمل کرنا} 
  جب دودرجی عبارت کو کامل مربع کی صورت میں لکھنے کی کوشش کرتے ہیں۔اِس نقطہ پر توجہ کریں کہ جب آپ  \(x+\frac{1}{2}b\) کا مربع ہیں تو آپ کو  \(\big(x+\frac{1}{2}b\big)^{2}=x^{2}+bx+\frac{1}{4}b^{2}=x^{2}+bx=\big(x+\frac{1}{2}\big)-\frac{1}{4}b^{2}\) حاصل ہوگا لہٰذا۔  \\اب c کو طرفیں میں جمع کریں                     
\section*{ مثال نمبر  4.3.1}
        ۔\(x^{2}+10x+32\) کو کامل مربع صورت میں لکھیں۔ 
        \[x^{2}+10x+32=(x^{2}+10x)+32={(x+5)^{2}-25}+32=(x+5)^{2}+7\] 
        ۔\(ax^{2}+bx=\big(x+\frac{1}{2}b\big)^{2}-\frac{1}{4}b^{2}\)  کو ذہن نشین کرنے کی کوشش نہ کریں۔ یہ سیکھ لیں کہ آپ x کے عددی سر کا نصف کریں اور لکھیں  \(ax^{2}+bx=\big(x+\frac{1}{2}b\big)^{2}-\frac{1}{4}b^{2}\)  پھر اِس میں مساوات کے دونوں جانب c جمع کریں۔اگر آپ نے  \(ax^{2}+bx+c\) کو کامل مربع صورت میں لکھنا ہو لیکن \(x^{2}\) کا عددی سر a کی قیمت 1 نہ ہو تو کے پہلے دوجزو میں سے جزو ضربی a کو باہر نکال کر لکھ سکتے ہیں:
\[ax^{2}+bx+c=a\big(x^{2}+\frac{b}{a}x\big)+xc\] 
 تب دودرجی عبارت کے مربع کو قوسین میں مکمل کریں۔\\
 \newpage 
%  صفحہ نمبر 53
\section*{مثال نمبر 4.3.2}
۔ \(2x^{2}+10x+7\) کو کامل مربع صورت میں لکھیں \\ جن جزو میں x موجود ہے ان میں سے جزو ضربی کو ابتداءً باہر نکال لیں\\
\[2x^{2}+10x+7=2\big(x^{2}+5x)+7.\]
 قوسین میں موجود جزو کو حل کرتے ہوئے۔\\
\[x^{2}+5x=\big(x+\frac{5}{2}\big)^{2}-\frac{25}{4},\]
\[2x^{2}+10x+7=2\big(x^{2}+5x\big)+7=2\big\{\big(x+\frac{5}{2}\big)^{2}-\frac{25}{4}\big\}+7\]
\[=2\big(x+\frac{5}{2}\big)^{2}-\frac{25}{2}+7=2\big(x+\frac{5}{2}\big)^{2}-\frac{11}{2}.\]
  اِس مقام پر ذہنی طور پر نتیجہ کو پرکھنا قابِل قدر ہے۔
 اگر  \(x^{2}\) کا عددی سر منفی ہو تو بھی بنیادی طریقہ کار یہی ہے۔جیسا مثال نمبر 4.3.3 میں دکھایا گیاہے۔
\section*{ مثال نمبر 4.3.3}
۔ \(3-4x-2x^{2}\) کو کامل مربع صورت میں ظاہر کریں۔\\
 جن جزو میں x موجود ہے ابتداءً ان میں سے جزو ضربی 2- کو باہر نکال لیں۔ قوسین میں موجود جزو کو حل کرتے ہوۓ۔
\section*{ مثال نمبر 4.3.4}
۔\(12x^{2}-7x-12\) کو کامل مربع صورت میں ظاہر کریں اور نتائج کو استعمال کرتے ہوئے۔اسکا جزو ضربی معلوم کریں۔
\[12x^{2}-7x-12=12\big(x^{2}-\frac{7}{12}x\big)-12=12\big\{\big(x-\frac{7}{24}\big)^{2}-\frac{49}{576}\big\}-12\]
\[12\big\{\big(x-\frac{7}{24}\big)^{2}-\frac{625}{576}\big\}=12\big\{\big(x-\frac{7}{24}\big)^{2}-(\frac{25}{24})^{2}\big\}\]
 اب آپ کلیہ، \(a^{2}-b^{2}=(a-b)(a+b)\) کو استعمال کرسکتے ہیں، قوسین میں موجود مساوات کی تجزی کیلئے a کو بطور \(x=\frac{7}{24}\) اور بطور \(\frac{-25}{24}\) لیں۔\\
% صفحہ نمبر 54
\section*{ مثال نمبر 4.3.5}
۔\(x^2-8x+12\) کو کامل مربع کی صورت میں ظاہر کریں۔ اپنے نتائج کو استعمال کرتے ہوئے تعاعل \(f(x)=x^2-8x+12\) کی حدود معلوم کریں۔جوکہ x کی تمام حقیقی قیمتوں کیلئے تعریف شدہ ہے۔\\
    \[x^2-8x+12=(x-4)^2-4\]

 جیسا کہ x کی تمام قیمتوں کیلیۓ  \(y>-4\) ہے۔\\
 \(x^2-8x+12=(x-4)^2-4 \leq -4\)
 لہزا
    \(f(x) \leq -4\)
   
۔y کو بطورِ f(x)  لکھیں توحد \(y\leq -4\) ہے۔
 \section*{مشق نمبر (A)4}
 1-مندرجہ ذیل ترسیمات کا (i) راس اور (ii) خظِ تساکل کی مساوات معلوم کریں۔\\
 2-مندرجہ ذیل دودرجی عبارت کی (i) کم سے کم (اگر مناسب۔ ہو تو زیادہ سے زیادہ) قیمت اور (ii) x کی وہ قیمت میں کیلیۓ یہ ہے۔\\
 3- مندرجہ ذیل دودرجی عبارت کو حلکریں۔ غیر معقول اعداد جواب کا حصہ رہنے دہں۔\\
 4-مندرجہ ذیل کو کامل مربع صورت میں ظاہر کریں۔\\
 5-کامل مربع صورت کو استعمال کرتے ہوئے ذیل میں دیئے گئی ہر ایک عبارت کی کم سے کم یا زیادہ سے زیادہ مناسب قیمت معلوم کریں اور x کی وہ قیمت میں کیلیۓ یہ ہے۔\\
\newpage
 %صفحہ نمبر 55
 7-ذیل میں دیئے گئے ہر ایک تفاعل، x کی  حقیقی قیمتوں کیلئے تعریف شدہ ہے۔ مربع مکمل کرتے ہوئے f(x) کو
 کے طور پر لکھیں اور انکی حدود معلوم کریں۔\\
 8-مربع مکمل کرتے ہوۓ (i) راس اور (ii) ذیل میں دیئے گئے ہر ایک قطع مکافی کے خطِ تشاکل کی مساوات معلوم کریں۔\\
 9-ذیل میں دیئے گئے ہر ایک تفاعل کا دائرہ کار تمام مثبت حقیقی اعداد پر محیط ہے۔ہر تفاعل کی سعت معلوم کریں۔\\
\section*{ 4.4 دودرجی مساوات کو حل کرنا}
 یقیناً آپ  \(x^2-6x-8\) صورت کی دودرجی مساوات کے بذریعہ تجری \(x^2-6x+8\) سے  \((x-2)(x-4)\) میں حل سے واقف ہیں اور تب بذریعہ استدلال اگر ۔۔۔۔۔۔۔
 تب یا تو۔۔۔۔۔۔۔۔یا۔۔۔۔۔۔۔
 لہزا\( x=2\) یا \( x=4\)
 مساوات \(x^2-6x+8\) کا کاحل\( x=2\)  یا \( x=4\) ہے۔اعداد 2 اور 4 مساوات کے جزر ہیں
اگر آپ دودرجی عبارت کا جزر یا آسانی معلوم کر سکتے ہوں تو یقیناً یہ مساوات کے حل کا تیز تر طریقہ ہے۔
تاہم،ممکن ہے کہ عبارت کے جزر نہ ہوں یا انہیں معلوم کرنا مشکل ہو مثلاً  \(30x^{2}-11x-30\)  کے جزر معلوم کرنے کی کوشش کریں\\
 اگر آپ مساوات کو حل کرنے کیلئے ایک دودرجی عبارت کی تجزی نہیں کر سکتے ہوں تب دودرجی کلیہ استعمال کریں،
۔\(ax^{2}+bx+c=0\)  کا حل جہاں \(a\neq0\)ہے
\[x=\frac{-b\pm\sqrt{b^{2}-4ac}}{2a}\]
 یہ جاننا مفید ہے کہ کیسے کامل مربع صورت،  \(ax^{2}+bx+c\) سے یہ کلیہ اخذ کیا گیا ہے  ابیداءً مساوات کے دونوں اطراف کو a سے تقسیم کریں (a کی قیمت 0  نہیں ہوسکتی ہے۔ ورنہ یہ دودرجی مساوات نہیں رہے گی)
 \[x^{2}+\frac{b}{a}x+\frac{c}{a}=0\]
 \newpage
% صفحہ نمبر 56
 بائیں طرف عبارت کا مربع مکمل کرنے سے آپ کو معلوم ہوگا کہ
\[x^{2}+\frac{b}{a}x+\frac{c}{a}=\left(x+\frac{b}{2a}\right)^{2}-\frac{b^{2}}{4a^{2}}+\frac{c}{a}=\left(x+\frac{b}{2a}\right)^{2}-\frac{b^{2}-4ac}{4a^{2}}\]
 لہزا آپ مساوات کے حل کو جاری رکھ سکتے ہیں۔
\[\left(x+\frac{b}{2a}\right)^{2}-\frac{b^{2}-4ac}{4a^{2}}=0\quad \left(x+\frac{b}{2a}\right)^{2}=\frac{b^{2}-4ac}{4a^{2}}\]
 یہاں دو ممکنات ہیں۔
 $x+\frac{b}{2a}=+\sqrt{\frac{b^{2}-4ac}{4a^{2}}}\quad-\sqrt{\frac{b^{2}-4ac}{4a^{2}}}$
 یا
$x=-\frac{b}{2a}\pm\frac{\sqrt{b^{2}-4ac}}{2a}=\frac{-b\pm\sqrt{b^{2}-4ac}}{2a}$
\\
$x=\frac{-b\pm\sqrt{b^{2}-4ac}}{2a}$
مساوات \(ax^2+bx+c=0\) کے دو جزر ہوں گے۔

%%%%%%%%%%%%%%%%%%%%%%%%%%%%%%OWAIS PAGE6/7%%%%%%%%%%%%%%%%%%%%%%%%%%%%%%%%%%%%%%%%%%%%%%%%%%%%%%%%%%
یہ ظاہر کرتا ہے کہ اگر $ \ ax^{2} + bx+ c = 0 $ اور $a \neq 0 $ تو $\ x= -b \pm \dfrac{\sqrt{b^{2}-4ac}}{2a}$\\
%page 56%
\section*{
مثال نمبر
 4.4.1}
مساوات کے حل کیلے دو درجی کلیہ استعمال کریں
(a)
اسکا  $(a) ax^{2} +bx +c$ کے ساتہہ موازنہ کرتے ہوۓ ، $\ =b=3, a=2 $ اور $ c=4 $ درج کریں 
تو
\[x=\frac{-(-3)\pm\sqrt{(-3)^{2}-4\times2\times(-4)}}{2\times2}=\frac{3\pm\sqrt{9+32}}{4}=\frac{3\pm\sqrt{41}}{4}\]
 آپ سے بعض اوقات ضرر کو غیر معقول صورت میں رہنے دینا متوقع ہوگا۔ بعض دیگر اوقات آپ 
سے صزر $\frac{3-sqrt{41}}{4} = 2.35 $ اور $  \frac{3-sqrt{41}}{4} = -0.85 $کی صورت میں درکار ہوگا۔ مساوات میں ان اعداد کی
ترکیب بدلی کے نتائج ملاحظہ کریں۔\\
(b)
 b=3,a=2 اور c=4
 درح کرنے سے 
\[x=\frac{-(-3)\pm\sqrt{(-3)^{2}-4\times2\times(4)}}{2\times2}=\frac{3\pm\sqrt{9-32}}{4}=\frac{3\pm\sqrt{-23}}{4}\]
لیکن 23- کا جضر اطربع ممکن نیہی ہے اسکا مطلب یہ ہے کہ مساوات  $ 2x^{2}-3x+4=0 $ کا 
کوئ جذر نہیں ہے-\\
 $y=2x^{2}-3x+4$
   کی کامل مربعی صورت میں تحویل کی کوشش کریں- آپ
    $ y=2x^{2} -3x +4$۔
   کے ترسیم سے کیا اخذ کرتے ہیں؟ \\ 
\newpage
%page 57%
$\ b = -11،a=30(c)$ اور $ c=30 $ درج کرنے سے \\
تیسری مثال کی تجزی تو ہوتی ہے لیکن جزر معلوم کرنا مشکل ہے۔ تاہم اگر مساوات کے جذر \\
معلوم ہو جائیں تو آپ اخز کر سکتے ہیں کہ $ 30x^{2} -11x + 300(6x+5)(5x-6)$
$4.5$ ممیز $a^{2} -4ac$
اگر آپ واپس مسال نمبر 4-1-1 پر نظر ڈالیں تو آپ دیکھیں گے کہ جزو $(a)$ کی مساوات کے \\
 جذر میں غیر معقول اعداد بھی وابستہ ہیں جزو $ \ (b) $ میں جزر نہیں تھا اور جزو $(c)$ میں جذر کسور \\
 تھیں۔
 جزو اطرابع کی علامت کے نیچے موجود عبارت $b^{2}-4ac$ کی قیمت کے حساب سے آپ\\
 پیش گوئ کر سکتے ہیں کہ کونسا معاملہ پیش آۓ گا۔ اور دو درجی کلمے $ x= \dfrac{-b \pm \sqrt{b^{2}-4ac}}{2a}$ پر اسکے 
تجویہ سے \\
$\bullet$ اگر $ b^{2}-4ac$ \ ایک کامل مربع ہے تو جذر عدر صحیح پاکسور ہوں گے۔\\
$\bullet$ اگر $ b^{2} - 4ac > 0 $ تو مساوات $ ax^{2} + bx + c = 0 $ کے دو جذر ہوں گے\\
$ \hspace{10pt} $ \\
$\bullet$ اگر $b^{2}-4ac>0$ تو کوئ جزو نہیں ہوگا۔\\
$\bullet$ اگر $b^{2}-4ac=0$ ہو تو جذر $ x\ =\ \frac{-b}{2a} $ سے حاصل ہوں گے۔ دروصل ایک ہی جذر ہوگا\\
بعض اوقات کہا جاتا ہے کہ دو موافق جذر یا ایک دہرا جذر ہے کیونکہ جذر کی قیمتیں \\
$ \frac{-b + 0}{2a}$ اور $ \frac{-b-0}{2a} $ برابر ہیں۔\\ 
$\ b^{2}-4ac$ دو درجی عبارت $\ ax^{2}+bx+c$ کا ممیز کہلاتا ہے کیونکہ اسکی قیمت کی مدد سے مساوات $ax^{2}+bx+c=0$ کے حل کی اقسام میں تمیز کی جاتی ہے۔\\
مثال نمبر 4-5-1
\\
مندرجہ ذیل مساوات کے دو درجی اجزاء کے ممیز کی قیمتوں سے آپ کیا\\
 اخذ کرسکتے ہیں؟\\
$(a)$ جیسے کہ $a=2----$ ممیز شبت ہے لہہذا مساوات $2x^{2}-3x-4=0$ کے دو جذر ہیں \\
مزید جیسا کہ $"41"$ کامل مربع نہیں ہے تو جذر ناطق ہیں۔\\
$(b)$ ممیز مثبت ہے  لہذا مساوات $2x^{2}-3x-5$ کے دو جذر ہیں اور چونکہ $49$ کامل مربع ہے۔\\
لہذا جذر ناطق ہے۔\\
$(c)$ کیونکہ ممیز منفی ہے اسلۓ مساوات $2x^{2}-4x+5=0$ کا کوئ جذر نے۔\\
\newpage
%page 58%
$ \hspace{10pt} $\\
$(d)$ چونکہ ممیز صفر ہے اسلیے مساوات $2x^{2}-4x+2=0$ کا صرف ایک (دہرا) جذر ہے۔\\
مثال نمبر 4-5-2\\
مثاوات $kx^{2}-2x-7=0$ کے دو حقیقی جذر ہیں، آپ مستقل $k$ کی قیمت کے بارے\\
میں کیا اخذ کر سکتے ہیں؟\\
ممیز $(-2)^{2}-4(k)(-7)=\ 4 + 28k$ ہے۔ مساوات کے دو جقیقی جذر ہیں لہذا ممیز کی قیمت\\
مثبت ہوگی۔ بس $4+28k>0$ اور $k>\frac{-1}{7}$۔\\
مثال نمبر 4-5-3\\

%%%%%%%%%%%%%%%%%%%%%%%%%%%%%%%%%%%%%%ARFA KHATOON%%%%%%%%%%%%%%%%%%%%%%%%%%%%%%%%%%%%%%%%%%%%%%%%%%%%
اگر $ b-4ac=0 $ ہو تو ہی مساوات کے دہرے جذر ہوتے ہيں۔يعنی اگر $ 2^2-4٭3٭k=0 $ راس سے k کی قيمت 1/3 حاصل ہوگی۔
مشاہدہ کريں کہ کيسے مندرجہ ذيل بالا ميں دو درجی مساوات کو حل کرنےضرورت ہی پيش نہيں آئی۔آپ کو جو بھی معلوم کرنا ہو کرسکتے ہيں۔
\section{مشق نمبر4B}
\paragraph{1} \quad مندرجہ ذيل مساوات کو حل کرنے کےليے دو درجی کليہ استعمال کريں۔غير ناطق جوابات کو غير معقول صورت ميں چھوڑ ديں۔اگر حل ممکن نہيں تو بھی بتائيے۔
اپنے جوابات کو سوالنمبر 8 ميں استعمال کيلئے محفوظ رکھيں۔
\paragraph{2} \quad مميز  $ b^2 - 4ac=0 $ کی قيمت کو استعمال کرتے ہوئے معلوم کريں کہ مندرجہ ذيل مساوات کے جذر کتنے ہيں(ايک ہے،دو ہيں يا کوئی نہيں)
جزو(i) اور (ii) ميں p اور q کی قيمتيں مثبت ہيں۔
\paragraph{3} \quad مندرجہ ذيل پر مساوات کا ايک دہرہ جذر ہے۔ ہر معاملے ميں k کی قيمت معلوم کريں۔ 
اپنے جوابات کو عدد صحيح، مکمل کسور يا غير معقول صورت ميں رہنے ديں۔
\paragraph{4} \quad مندرجہ  ذيل مساوات کے جذر کی تعداد مي دی گئ ہے۔ جس قدر ممکن ہو k کی قيمت اخذ کريں۔
   \paragraph{5} \quad مميز کی قيمت کو استعال کرتے ہوئے محور x پر مندرجہ ذيل ترميمات کے نقاط انقطاع کی تعداد معلوم کريں۔
\paragraph{6} \quad اگر a اور c دونوں مثبت ہوں تو ترميم $ y=ax^2+bx+c $ سے متعلق کيا بيان کرسکتے ہيں؟ 
\paragraph{7} \quad اگر a منفی اور c مثبت ہو تو ترميم $ y=ax^2+bx+c $ سے آپ کيا  بيان کرسکتے ہيں؟
\paragraph{8} \quad آپ کو سوالنمبر 1 کے جوابات ناطق يا غير معقول صورت ميں درکار ہوں گے نہ کہ اعشاريہ۔
سوالنمبر 1 (a)،(b) اور (c) کيلئے جذر کی (i) جمع اور (ii) ضرب کريں۔ آپ کيا مشاہدہ کرتے ہيں؟
اگر صرف ايک ہی (دہرا) جذر ہو تو کيا ہو گا؟
\paragraph \quad (B)  دودرجی مسااوات $ x^2+bx+c=0 $ کے جذر $ \alpha $ اور $ \beta $ ، $ x^2+bx+c $ کے اجذائے ضرب $ x-\beta $ اور $ x-\alpha $
سے ہی اخذ ہوں گے۔ آپ مساوات $ x^2+bx+c=0 $ کے جذر واضح کريں جنکی جمع b" اور ضرب c ہو۔ 
\paragraph{c} جز B کو طول ديتے ہوۓ جز a,b اور cپر مشتمل مساوات $ x^2+bx+c=0 $ کے جذر کی (i) جمع اور (ii) ضرب کيلۓ عبارات معلوم کريں۔ 
 %%%%%%%%%%%%%%%%%%%%%%%%%%%%%%%%%%%%%%%%%%%%%%%%%%%%%%%%%%%%%%%%%%%%%%%%%%
 %\section*{page no 9}
\section*{4.6 ہمزاد مساوات}
يہ جزو ظاہر کرے گا کہ $y=x^2 $ اور $ 5x+4y=21 $ جيسے ہمزاد مساواتوں کے جوڑوں کو کيسے حل کيا جاتا ہے اس ميں جزو 3.7 کے مقدمات کو مزيد آگے بڑھايا جاۓ 
\section*{مثال نمبر 4.6.1}
ہمزاد مساوات $ x+y=6 $ , $ y=x^2 $  کو حل کريں۔
عمومی طور پر ان مساوات کو حل کرنے کا بہترين طريقہ يہ ہے کہ ايک مساوات سے x يا y کيلۓ عبارت معلوم کرکے دوسری مساوات ميں درج کر دی جاۓ۔ يہاں y کی قيمت کيلۓ پہلی مساوات کو استعمال کرتے ہوۓ دوسری مساوات ميں ترکيب بدلی نسبتا آسان ہے جسکا ماحصل $ x+x^2=6 $ ہے۔ اسے مرتب کرنے سے $ x^2+x-6=0 $ ۔ لہذا $ (x+3)(x-2)=0 $ يعنی $ x=2 $ يا $ x=-3 $ ۔آپ y کی متعلقہ قيمتيں مساوات $ y=x^2 $ سے معلوم کرسکتے ہيں۔ جو کہ باالترتيب $ y=4 $ اور $ y=9 $ ہيں۔ 
لہذا اسکا حل $ y=4 $ , $ x=2 $ يا $ y=9 $ , $ x=3 $ ہے۔ جانچ ليں کہ قيمتوں کے ہر جوڑے کيلۓ $ x+y=6 $ ۔ توجہ رہے کہ جوابات باہم جوڑوں کی شکل ميں ہيں۔ جوابات کو $ x=-3 $ , $ x=2 $ ,يا $ y=4 $ , $ y=9 $ کی صورت ميں لکھنا غلط ہے کيونکہ جوڑے $ y=9 $, $ x=2 $ اور $ x=-3 $ , $ y=4 $ اصل مساوات کو ثابت نہيں کرتے ہيں۔ 
آپ يہ تب ملاحظہ کرسکتے ہيں اگر سوال کی تشريح ترسيمات
$ y=x^2$
اور
$ x+y=6 $
کے نقاط انقطاع معلوم کرنے کيلۓ کريں جيسے کہ شکل 4.2 ميں۔
\section*{ مثال نمبر 4.6.2}
ہمزاد مساوات $ x^2-2xy+3y^2=6 $ اور $ 2x-3y=3 $ کو حل کريں۔
پہلی مساوات x يا y کيلۓ عبارت معلوم کرنا مشکل ہے لہذا دوسری مساوات سے ابتدا کريں۔
اگر آپ کسور سے گريز کريں تو غلطی کے امکانات کم ہوں گے۔ دوسری مساوات سے مساوات $ 2x=3+3y $ حاصل ہوئ لہذا اسکا مربع لينے سے $$ 4x^2=(3+3y)^2=9+18y+9y^2 $$ 
اب آپ کے پاس 
$4x^2$
 اور 2x کيلۓ عبارت موجود ہيں لہذا اب آپ پہلی مساوات ميں ترکيب بدل سکتے ہيں۔ پہلی مساوات کو 4 س ضرب دينا مددگاررہے گا۔ لہذا يہ $ 9y^2 6y-15=0 $ ميں تخفيف ہو جاتا ہے اور اسے 3 سے تقسيم کريں تو $ 3y^2+2y-5=0 $ ۔ اس مساوات کو حل کرنے سے $ (y-1)(3y+5)=0 $ حاصل ہوا لہذا $ y=1 $ يا $ y=-5/3 $ ۔
   
 
%%%%%%%%%%%%%%%%%%%%%%%%%%%%%%%%%%%%%%%%%%%KAINAT_KHIZAR%%%%%%%%%%%%%%%%%%%%%%%%%%%%%%%%%%%%%%%%%%%%%%
%58pageforward \\marked 10 and 11 on Dr.Cheema's translation
دوسرى  مسياوات  میں  ترکیب  بدلنے سے ‌‌‌x  کی قیمت بااترپیب x=3 اور x= -1 حاصل ہوگی۔ لہزا حل  x=-1 ,y=3 اور ,x=3,y=1 ہے۔
\textbf{\section*{4-6.3  مثال نمبر }}
                              کتنے نقاط پر خط  $  x+2y=3 $   منحنی    $ 2x^+y^2=4  $ کو مفقطع کرتا ہے؟  $  x+2y=3 $  لہزا  $ x=3-2y $ کی ترکیب   $2x^2+y=4 $  ميں درج کرنے سے   $     2(3-2y)^2+y^2=4  $  لہزا  $  (9-12y+4y^2)+y^2=4 $     کی تہقيق سے مساوات   $  y^2-24y+14=0   $   حاصلہوئ۔
   اس مساوات کا ممیز    $   24^2-4*9*14=576-504=72  $       ہے۔کيونکہ  مميز مثبت ہے۔ اس ليے مساوات کے دو حل ہوں گے، معلوم ہوا کہ خط منحنی کودو نقاط پر منقطع کرتا ہے۔
   
 \textbf{\section*{ دودرجی مساوات ميں قابل تخفيف مساوات 4.7  }}
  بعض اوقات آپکا سامنا ايسی مساوات سے ہوگا جو دودرجی نہيں ہوں گی۔ درست تريب ميں بدلی کے ذريعے انہی دودرجی مساوات ميں تبديل کرنا ممکن ہے۔
  
   \textbf{\section*{ 4.7.1  مثال نمبر       }}
   مساوات        $       t^4-13t^2+36=0   $  کو حل کريں۔
   حبزو $ t^4 $   کی موجودگی کے بامث یہ ايک دودرجی مساوات ہے ليکن اگر $  x $  کو بطور   $  t^4 $  ليں تو ييہ مساوات  $  x^2-13x+36=0 $  ميں  تبديل ہو جاۓ گی جو کہ $ x$  کی دودرجی مساوات ہے۔
   تو $     (x-4)(x+9)=0  $  لہزا  $ x=4 $ يا $x = 9 $ ۔
   واپس $ x=t^2 $ درج کرنے سے $ t^2=4 $  يا $ t^2=9 $ يعنی نتيجہ $   t=\pm 2 $ يا $  t=+-3 $۔
   
 \textbf{\section*{ 4.7.2   مثال نمبر     }}  
                                                    مساوات  $   sqrt{x}=6-x\  $ کو حل کرين۔\\
                                        ۔$ (a)$ y  کو  $ \sqrt{x} $  ۔کيليۓ استعمال کرتے ہوۓ۔    \\
                                        ۔$( b)$  مساوات کی طرقيبيں کا مربع لينے سے ۔\\
 ۔$(a)$ $ \sqrt{x} $  کی جگہ  y درج کرنے سے مساوات ،  $ y=6-y^2 $   يا   $y^2+y=6$  ميں تحيل ہو جاتی ہے۔ لہزا   $ (y+3)(y-2)=6 $  پس   $ y=2 $   يا   $ y=3 $  ليکن چونکہ  $ y=\sqrt{x} $  جبکہ  $ \sqrt{x} $  قطعاُٰٰٰٰٰٰۃ منقی نہيں ہو سکتا ہے۔تو واحد حل $  y=2 $ ہی ہے جس سے  $   x=4 $ حاصل ہوا۔

\newpage
%{\{صفحہ نمبر (62/61)201/202}}
طرفين کا مربع لينے سے $  (6-x)^2=36-12x+x^2   $  يا $ x^2-13x+16=0 $  ۔ لہزاہ $ (x-4)(x-9)=0  $  تو ماحصل $ x=9 $ يا $  x =4  $ ہے۔ جوابات کو جانچنے سے معلوم ہوتا ہے کہ جب $ x=4 $ ہو تو مساوات $\sqrt[2]{x) = 6-x} $ درست ثابت ہوتی ہے ليکن جب $ x=9 $ ہو تو , $  sqrt[2]{x})=3 $ اور $  6-x=-3 $ يعنی $x=9 $ پس $x=9 $ جضر نہيں ہے لہزاہ $ x=4$ واحد جضر ہے۔ يہ اہم ہے کہ اگر آپ مساوات $  sqrt[2]{x} = 6-x  $   کا مربع ليں تو اس کے جزر سميت وہ جزر جو آپ اصلا معلوم کرنا چاہ رہے تھے معلوم کريں گے۔ قابلغور ہے کہ $ x=4 $ تو اس مساوات کو درست ثابت کرتا ہے۔ليکن $ x=9 $ نہيں کرتا۔نتيجہ يہ ہے کہ جب آپ کسی مساوات کو حل کرتے ہوے اس کا مربع ليں تو ضروری ہے کہ اپنے جوابات کو جانچ ليں۔

\textbf{\section*{مشق نمبر   4C}} 
\begin{enumerate}
\item
                                     مندرجہ ذيل ہمزاد مساوات کے جوڑوں کو حل کريں۔                               
\item
                                     خط مستقيم اور منحنی کے نقاط انقطاغ کے محرد معلوم کريں۔
\item
                                      مندرجہ ذيل سوالات ميں خط مستقيم اور منحنی کے نقاط انقطاع کی تعداد معلوم کريں۔
\item
    مندرجہ ذيل مساوات کو حل کريں۔غير ناطق جوابات ، غير معلوم صورت ميں ديں۔
\item
    مندرجہ ذيل مساوات کو حل کريں۔ (زيادہ تر معملات ميں مناسب عبارت سے ضربے مساوات کو قابل مہم بنادے گی۔
\item
    مندرجہ ذيل مساوات کو حل کريں۔
 \end{enumerate}   
 \textbf{\section*{   متفرق مشق 4  }}
 \begin{enumerate}
\item
                                            ہمزاد مساوات $ X+Y=2 $ اور $  x^2+2y^2=11  $ کو حل کريں۔
\item
                                            دودرجی کثيررکنی عبارت $ x^2-10x+17  $ کو $  f(x)  $ سے ظاہر کرتے ہيں۔طاقی قيمتوں کو واضع کرتے ہوے $  f(x)  $ کو $ (x-a)^2+b $ صورت ميں لکھيں۔لہزاہ $  fx $  کے ليے کم سے کم ممکن قيمت اور اس کے موافق $  x $ کی قيمت معلوم کريں۔
\item
                                            ہمزاد مساوات $ 2x+y=3 $ اور $  2x^2-xy=10$ کو حل کريں۔
\item
                                           ۔k کی کن قيمتوں کے ليے مساوات  $ 2x^2-kx+8=0 $ دہراجذر رکھتی ہيں؟
\item
                                            تغاعل $ fx=(2x+4)(x-4) $ کو کامل مريصی صورت ميں ظاہر کر کے تغاعل $ fx$ کی سعت معلوم کريں۔
\item
                                            مساوات $ $ کو حاصل کرکے جواب ہر ممکن صدتک منفق اور غير معقول صورت ميں ہے۔
 \end{enumerate}    
    
%%%%%%%%%%%%%%%%%%%%%%%%%%%%%%%%%%%%%%%%%%%%%%%%%%%%%%%%%%%%%%%%%%%%%%%%%%%%%%%%%%%%%%%%%%%%%%%%%%%%%
%%%%%%%%%%%%%%%%%%%%%%%%%%%%%%%%%%LIZA SAJID%%%%%%%%%%%%%%%%%%%%%%%%%%%%%%%%%%%%%%%%%%%%%%%%%%%%%%%%
%page 12 and 13 of urdudoc Dr.Cheema translation
%page 62/61 ,page12 
(b)
مساوات 
\(x^{4}-(6\sqrt{3})x^{2}+24=0\)
کے  چاروں  ممکنہ حل کے جوابات دو درجہ اعشاريہ تک ديں۔\\
(7) ظاہر کريں کہ خط 
\(y=3x-3\)
اور منحنی 
\(y=(3x+1)(x+2)\)
ایک دوسرے کو منقطع نہیں کرتے ہیں۔\\
(8) \(9x^{2}-36x+52\)
کو 
\((Ax^{2}+Bx)^{2}+C\)
کی صورت ميں ظاہر کريں جبکہ B,A اور C عدد صحيح ہيں لہزا يا دوسری صورت ميں 
\(9x^{2}-36x+52\)
کيلۓxکی حقيقی قيمتوں کا مجموعہ معلوم کريں۔\\
(9) دو درجہ اعشاريہ تک درست محدد ديتے ہوۓ منحنی 
\(y=6x^{2}+4x-3\)
اور 
\(y=x^{2}-3x-1\)
کے نقاط انقطاع معلوم کريں۔\\
(a)(10)\(9x^{2}+12x+7\)
کو 
\((ax+b)^{2}+c\)
کی صورت ميں ظاہر کريں،يہاں b,a اور c مستقل ہيں جنکی قيمتيں معلوم کرنا مقصود ہيں۔\\
(b)\(\frac{1}{9x^{2}+12x+7}\)
کیلۓ درست، x کی حقیقی قیمتوں کا مجموع معلوم کریں۔\\
(11) مساوات
\(8x^{4}-8x^{2}+1=\frac{1}{2}\sqrt{3}\)
 کے تمام جزر معلوم کریں جو تین معنی خیز اعداد تک درست ہوں۔\\
x(12) 
کی تمام قیمتوں کیلۓ درست مستقبلa,bاورc معلوم کریں۔\\
لہزا 
\(y=3x^{2}-5x+1\)
کیےترمیم پر سب سے کم قیمت نقط کے محدود معلوم کریں۔\\
(یاداشت: سب سے کم اور سب سے زیادہ قیمت والے نقاط راس ہیں۔)\\
(13) قوس 
\(xy=6\)
اور 
\(y=9-x\)
خط کے نقاط انقطاع معلوم کریں۔\\
(14) \(y=ax^{2}-2bx+c\)
قوس کی مساواتa,bاور cہے اور مشتقل ہیں جبکہ
a>0\\
(a)
قوس کے راس کے محدد کوa,bاورcکی صورت میں معلوم کریں۔\\
(b)
ہمیں معلوم ہے کہ قوس کا راس خط،
\(y=x\)
پر ہے۔cکیلۓa اور bکی صورت میں عبارت معلوم کریں۔یہ بھی ظاہر کریں کہ bکی تمام قیمتوں کیلۓ
\(c\leq\frac{-1}{4a}\)\\
(a)(15)\(y=x-1\)
اور 
\(y=kx^{2}\)
کے ترسیمات کو تصویر میں دکھایا گیا ہے جہاں ایک مثبت مستقل ہےترسیمات دو متفرق نقاط Aاور B پر ایک دوسرے کو منقطع کرتے ہیں۔AاورBکیلۓx کیلۓ دو درجی مساوات تحریر کریں اور ظاہر کریں کہ
\(K<\frac{1}{4}\).\\
(b)
مندرجہ ذیل معملات میںترسیمات 
\(y=x-1\)
اور 
\(y=kx^{2}\)
کے باہمی تعلق کو واضح کریں.\\
(1)\(k=\frac{1}{4}\)   (2)\(k>\frac{1}{4}\)   \\ 
(c)
ترمیم یا کسی اور دلیل سے ثابت کریں کہ جب kمنفی مستقل ہو تو مساوات
\(x-1=kx^{2}\)
 کے دو حقیقی جزر ہوتے ہیں،ایک جزر 0اور1 کے درمیان ہوگا۔

\newpage
%page 62/61 ,page13 urdu doc

\(y=3x+5\)
کے عمودکی مساوات معلوم کۓ بغیر درج ذیل طریقہ سے خط 
\(y=3x+5\)
اور نقط 
\((1,2)\)
کے درمیان کم سے کم فاصلہ معلوم کریں۔\\
(a)\((x,y)\)
خط پر ایک عمومی نقطہ ہے،ظاہر کریں کہ نقطہ 
\((1,2)\)
سے اس کا فاصلہ'd',
\(d^{2}=(x-1)^{2}+(y-2)^{2}\)
کے ذريعے حاصل ہو گا۔\\
(b)
خط کی مساوات کو حل کر کے ظاہر کریں کہ
\(d^{2}=(x-1)^{2}+(3x+3)^{2}\)\\
(c)
ظاہر کریں کہ
\(d^{2}=10x^{2}+16x+10\)\\
(d)
مربع کی تکمیل کے ذريعے ظاہر کریں کہ کم سے کم ممکن فاصلہ\\
(17)سوالنمبر 16کی ترکيب کو استعمال کرتے ہوۓ۔\\
(a)\((2,3)\)
کا 
\(y=2x+1\)
سے عمودی فاصلہ معلوم کريں۔\\
(b)\(1,3)\)
کا 
\(y=-2x+5\)
سے عمودی فاصلہ معلوم کريں۔\\
(c)\((2,-1)\)
کا 
\(3x+4y+7=0\)
سے عمودی فاصلہ معلوم کريں۔\\
(18)نوےدرجے پر قائم دو سڑکوں کا نقطہ انقطاع'0'ہے؛ ایک سڑک شمال سے جنوب اور دوسری مشرک سے مغرب کی جانب ہے۔گاڑی(A)نقطہ0 کے100میٹر مغرب سے مشرق کی جانب20m/sکی رفتار سے بڑھ رہی ہے اور گاڑی(B)نقطہ0کے 80 میٹر شمال سے جنوب کی جانب20m/sکی رفتار سے بڑھ رہی ہے۔\\
(a)
ظاہر کریں کہ't'وقت کے بعد انکا باہمی فاصلہ'd'ہوگا۔\\
\(d^{2}=(100-20t^{2})+(80-20t^{2})\)\\
(b)
ظاہر کریں کہ باسکی تحقیق کے نتیجہ میں
\(d^{2}=400(5-t^{2})+(4-t^{2})\)\\
(c)
ظاہر کریں کہ دونوں گاڑیوں کا کم سے کم باہمی فاصلہ 
\(10\sqrt{2}\)
میٹر ہے\\
(19)نوے درجے پر قائم دو سڑکوں کا نقطہ'0'انقطاع ہے؛ ایک سڑک شمال سے جنوب اور دوسری مشرک سے مغرب کی جانب ہے۔دونوں موٹر بائيک AاورB کے درمیان کم سے کم فاصلہ معلوم کریں جو کہ ابتدائی طور پر نقطہ'0'کی جانب مندرجہ ذیل صورتوں میں گامزن ہیں\\
(a)
دونوں موٹربائیک'0'سے 10 میٹر کے فاصلہ پرہیں 
A 20m/s
اور
B 10m/s
سےسفر کر رہا ہے\\
A(b)
,'0'سے 120 میٹر کے فاصلہ پر ہے اور اسکی رفتار 20m/s ہے جبکہ '0',B سے 80 میٹر پر ہےاور اسکی رفتار 10m/s ہے۔\\
A(c)
,'0'سے 120 میٹر کے فاصلہ پر ہے اور اسکی رفتار 20m/s ہے جبکہ '0',B سے 60 میٹر پر ہےاور اسکی رفتار 10m/s ہے۔\\

%%%%%%%%%%%%%%%%%%%%%%%%%%%%%%%%%%%%%%%%%%%%%%%%%%%%%%%%%%%%%%%%%%%%%%%%%%%%%%%%%%%%%%%%%%%%%%%%%%%
%Page 14 
$ 2-4x-x^2 $ (a)
اور
$  24+8x+x^2 $
کو کامل مربع صورت میں ظاہر کریں۔\\(b)
ظاہر کریں کہ مساوات 
$ y=2-4x-x^2 $
اور
$ 24+8x+x^2 $
کے ترسیمات ایک دوسرے کو منقطع نہیں کرتے ہیں۔ \\ (c)
ایک مثال کے ذایعے ظاہر کریں کہ 
$y=A-(x-a)^2$
اور
$y=B-(x-b)^2$
جیسی مساوات کے ذریعے ترسیم معلوم کیا جاسکتا ہے جوکہ ایک دوسرے کو منقطع نہیں کرتے ہیں۔\\(21)
ایک "ریسا ئکلنگا فرم " مختلف مقامات سے دھاتی ڈبے جمع کرتی ہے اور انہیں پیس کر دھات واپس صنعت کار کو بیچ دیتے ہیں۔ہر ہفتہ l ٹں دھاتی ڈبوں سے p منافع ہوتا ہے۔
\(p=100-\frac{1}{2}t^{2}-200\)\\
تکمیل مربع سے معلوم کریں کہ فرم زیادہ سے زیادہ کیناہفتہوار منافع حاصل کرتی ہے اور اتنا منافع حاصل اور ہر ہفتہ منافع حاصل کرنے کے لیَے لتنے ٹں دھاتی ڈبے اکٹھا کرکے بیچنا ہوں گے؟
  
