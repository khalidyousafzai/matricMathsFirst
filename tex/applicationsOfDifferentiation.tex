\باب{تفرق کے استعمال}\شناخت{باب_تفرق_کے_استعمال}

گزشتہ باب میں آپ نے سیکھا کہ تفریق کا معنی کیا ہے۔ اور کئ اقسام کے تفاعلات کی تفریق کیسے کی جاتی ہے۔اس باب میں آپ دیکھیں گے کہ تفریق کو ترسیمات کی خاکہ نگاری اور حقیقی دنیاوی معموں کو ھل کرنے کے لیے کیسے استعمال کیا جاتا ہے۔ جب آپ اس باب کو مکمل کریں تو آپکو چاہئیے۔
\begin{itemize}
\item
اس بات کو سمجھنا کہ کسی تفاعل کا تفرق بھی تفاعل ہوتا ہے
\item
مثبت، منفی اور صفر کے تفرقات کی اہمیت کی قدر دانی کرنا۔
\item
ذیادہ سے ذیادہ اور کم سے کم نقطوں کو ترسیم پر بٹھانے کی قابل ہونا
\item
اس بات کو جاننا کہ آپ تفرق کی تشریح ایک متغیرہ میں دوسرے کے متعلق تبدیلی کی شرح سے کر سکتے ہیں۔
\item
تفرق کے لیے \(\frac{\dif y}{\dif x}\)    کی علامت سے واقف ہونا
\item
ان طریقہ کاروں کو حقیقی دنیاوی معموں کو حل کرنے کے لیے استعمال کرنے کے قابل ہونا۔
\end{itemize}


\حصہ{تفرقات بہ صورت تفاعلات}

باب \حوالہ{باب_تفرقات}میں متعدد جستجوئیں کر کے آپ کو تفریق سے متعارف کروایا گیا تھا۔ مثلاً مشق 6 الف کے سوال 5 میں آپ سے تفاعل \(f(X)=x^2-2\) کی ترسیم کے مختلف نقطوں پر مماسہ کے ڈھلوان کے بارے میں اندازہ لگانے مکے لیے پاچھا گیا تھا۔ جدول 7.1 میں وہ نتائج موجود ہیں جنھیں حاصل کرنے کی آپ سے توقع تھی۔

اس سے ظاہر ہوتا ہے کہ مماسے کا ڈھلوان بھی \عددی{x} کا تفاعل ہے۔ جو کہ قاعدہ \عددی{2x} میں دیا گیا ہے۔ باب \حوالہ{باب_تفرقات} میں اسی قاعدے کو متفرق کہا گیا ہے۔لیکن جب آپ اسکی قیمت کو کسی مخصوص \عددی{x} کے لیے استعمال کرنے کے بجاۓ اسے ایک تفاعل خیال کر رہے ہوتے ہیں، تو بعض اوقات اسے مشتق تفاعل کہا جاتا ہے۔ اسے \(f'(x)\) سے ظاہر کیا جاتا ہے۔ اور اس مثال میں یہ \(f'(x)=2x\) ہے۔

مزید برآں جس طرح آپ تفاعل \(f(x)\) کی ترسیم سازی کر سکتے ہیں۔ اسی طرح مشتق تفاعل \(f'(x)\) کی ترسیم سازی بھی ممکن ہوتی ہے۔ انہی دو ترسیمات کو صفحے پر ایک دوسرے کے اوپر ایک قطار میں دکھانا بہت دلچسپ معلوم ہوتا ہے۔ جس طرح سے سورت 7.2 میں دکھایا گیا ہے۔

ترسیم کے بائیں جانب جہاں \(x<0\) ہے \(f'(x)\) کا ترسیم \عددی{x} کے محور سے نیچے موجود ہے۔ جو ظاہر کرتا ہے کہ \(f(x)\) کا ڈھلاوان منفی ہے۔ دائیں جانب جہاں ڈھلوان \(f(x)\) کا مثبت ہے، وہاں \(f'(x)\) کا ترسیمہ \عددی{x} کے محور کے اوپر موجود ہے۔

جو آپ تفریق کت بارے میں جانتے ہیں وہ آپ مشتق تفاعل کی صورت میں لکھ سکتے ہیں۔؛

اگر \(f(x)=x^n\) ہو، جہاں \عددی{n} ناطق عدد ہو تو اسکا مشتق تفاعل \(f'(x)=nx^{n-1}\) ہوگا\\  \(f(x)+g(x)\) کا مشتق تفاعل \(f'(x)+g'(x)\) ہوگا۔\\ \(cf(x)\) گا مشتق تفاعل جہاں \عددی{c} مستقل ہے،\(cf'(x)\) ہوگا۔

\ابتدا{مثال}\شناخت{مثال_استعمال_تفرقات_الف}
مساوات \(f(x)=x^2-\frac{1}{3}x^3\) کا مشتق تفاعل معلوم کریں۔اوپر بتاۓ گۓ نتائج کے استعمال سے معلوم ہوتا ہے کہ پوچھے گۓ تفاعل کا مشتق تفاعل \(f'(x)=2x-x^2\) ہے۔ مثال \حوالہء{7۔3 ؟؟} میں موجود تفاعل \(f(x)\) اور مشتق تفاعل \(f'(x)\) کی ترسیمات شکل \حوالہء{شکل 7.3} میں بتائ گئ ہیں۔ یہاں بعض نقات غور طلب ہیں، جب \(x<0\)، تفاعل \(f(x)=x^2-\frac{1}{3}x^3\) کی ترسیمات کی ڈھلوان منفی ہے، اور مشتق تفاعل \(f'(x)\)  کی قیمتیں بھی منفی ہیں۔ جب \(x=0\) ، \(f(x)\) کا ڈھلوان \عددی{0} ہے، اور \(f'(0)\) کی قیمت بھی صفر ہے۔ \(x=0\) اور \(x=2\) کے درمیان \(f(x)\) کا ڈھلوان مثبت ہے اور \(f'(x)\) کا بھی مثبت ہے۔ جب \(x=2\) ، \(f(x)\) کا ڈھلوان \عددی{0} ہے، اور \(f'(2)=0\)۔ جب \(x>2\)، \(f(x)\) کا ڈھلوان منفی ہے، اور مشتق تفاعل \(f'(x)\) کی قیمتیں بھی منفی ہیں۔
\انتہا{مثال}
  %Exercise 7A
\ابتدا{سوال}
مندرجہ ذئل صورتوں میں سے ہر ایک کے لیے تفاعل  \(f(x)\)  اور مشتق تفاعل \(f'(x)\) کی ترسیمات بنائیں۔ اور ان ترسیمات کا موازنہ کریں۔
\begin{multicols}{3}
\begin{enumerate}[.a]
\item \(f(x=4x)\)
\item \(f(x)=3-2x\)
\item \(f(x)=x^2\)
\item \(f(x)=5-x^2\)
\item \(f(x)=x^2+4x\)
\item \(f(x)=3x^2-6x\)
\end{enumerate}
\end{multicols}
\انتہا{سوال}
 \ابتدا{سوال}
مندرجی ذیل صورتوں میں سے ہر ایک کے لیے تفاعل  \(f(x)\)  اور مشتق تفاعل \(f'(x)\) کی ترسیمات بنائیں اور ترسیمات کا موازنہ کریں 
\begin{multicols}{3}
\begin{enumerate}[.a]
\item \(f(x)=(2+x)(4-x)\)
\item \(f(x)=(x+3)^2\)
\item \(f(x)=x^4\)
\item \(f(x)=x^2(x-2)\)
\item \(f(x)=\sqrt{x} \quad x \geq 0 \)
\item \(f(x)=\frac{1}{x} \quad x \neq 0\)
\end{enumerate}
\end{multicols}
 \انتہا{سوال}  
\ابتدا{سوال}
سوال کے ہر حصے میں دی گی شکل \(y=f'(x)\) مشتق تفاعل کی ترسیمہ کو ظاہر کرتی ہے۔ \(y=f(x)\) کی ممکنہ ترسیمہ بنائیں۔
\انتہا{سوال}
 \حصہ{بڑھتے ہوۓ اور گھٹتے ہوۓ تفاعلات}
آسانی کے لیے تفاعل کے لفظ سے مراد اس باب میں وہ تفاعلات ہیں جو اپنے دائرہ کار میں استمراری (مسلسل) ہوتے ہیں۔ اس میں وہ تمام  تفاعلات شامل ہیں جو آپ ابھی تک دیکھ چکے ہیں، لیکن اس میں تفاعلات جیسے  \عددی{x} کا کسری حصہ شامل نہیں ہیں جوکہ تمام مثبت حقیقی اعداد کے لیے واضع ہیں لیکن ان کی ترسیمہ میں ، جیسا کہ شکل  \حوالہء{شکل 7.4} میں دکھایا گیا ہے، ہچکولے موجود ہیں۔

کسی ترسیمہ کو اسکی مساوات سے واضع کرنے کے لیے آپ اس تصور کو جسکے مطابق کسی تفاعل کا متفرق بھی تفاعل ہوتا ہے، استعمال کر سکتے ہیں۔
\ابتدا{مثال}\شناخت{مثال7.2.1}
وہ وقفہ معلوم کریں جس میں \(f(x)=x^2-6x+4\) بڑھتا ہوا ہے، اور وہ وقفہ جس میں گھٹتا ہوا ہے۔ \(f(x)\) کا متفرق ہے۔ \(f'(x)=2x-6=2(x-3)\) اس کا مطلب ہے کہ \(x>3\) کے لیے ترسیمہ کا ڈھلوان مثبت ہے، یعنی،\(f(x)\)،\(x<3\) کے لیے بڑھتا ہوا ہے۔

   \(x<3\)  
 کے لیے ڈھلوان منفی ہے اور جونہی  \عددی{x} کی قیمتیں بڑھتی جاتی ہیں،  \عددی{y } کی قیمتیں گھٹتی جاتی ہیں، یعنی  \(f(x)\)  ، \(x<3\) کے لیے گھٹتا ہوا ہے۔ 

نتائج صورت \حوالہء{شکل 7.5} میں ظاہر کیئے گئے ہیں۔
\انتہا{مثال}         

خود \(x=3\) کے بارے میں کیا؟ پہلی نظر میں آپ یہ سوچیں گے کہ اس کو دونوں بڑھتے ہوۓ اور گھٹتے ہوۓ وقفوں سے باہر چھوڑ دینا چاہئیے لیکن ایسا کرنا غلط ہوگا! اگر آپ خط منحنی پر بائیں سے دائیں جانب آگے کو بڑھ رہے ہوں اور جیسے ہی آپ \(x=3\) سے گزر چکیں، ڈھلوان مثبت ہو جاۓ گا اور قوس بلند ہونے لگے گا۔تاہم جتنا آپ \(x=3\) کے قریب ہوں گے،  \عددی{ y} کی قیمت \(f(3)=-5\) سے بڑی ہوگی۔ 

پس آپ کہہ سکتے ہیں کہ تفاعل   \(f(x)\) ،\(x \geq 3\) کے لیے بڑھتا ہوا ہے، اسی طرح \(x \leq 3\)کے لیے گھٹتا ہوا ہے۔

آپ مثال  \حوالہء{مثال7.2.1} میں موجود توجیہ کو کسی بھی تفاعل کے لیے استعمال کر سکتے ہیں۔ صورت  \حوالہء{شکل7.6} تفاعل \(y=f(x)\) کی ترسیمہ کو ظاہر کرتی ہیں۔ جس کا متفرق وقفہ \(p \leq x \leq q\) میں مثبت ہے۔ آپ دیکھ سکتے ہیں کہ \عددی{ y} کی بڑی قیمتیں \عددی{x}کی بڑی قیمتوں کے ساتھ منسلک ہیں۔ بعینہ طور پر اگر \(x_1\) اور \(x_2\) وقفہ \(p \leq x \leq q\) میں \عددی{x} کی دو قیمتیں ہوں اور اگر \(x_2 > x_1\) ہو تو \(f(x_2) > f(x_1)\) ہوگا۔

اس خصوصیت کے حامل تفاعل کو وقفہ \(p \leq x \leq q\) پر سے بڑھتا ہوا کہا جاتا ہے۔ اگر \(f'(x)\) منفی ہو ، وقفہ  \(p\le xle q\) میں جیسا کہ صورت  \حوالہء{شکل7.7} میں ہے، تو تفاعل کی خصوصیت برعکس ہوگی؛ اگر  \(x_2 > x_1\) ہو تو، \(f(x_2) < f(x_1)\) ہوگا۔اس خصوصیت کے حامل تفاعل کو وقفہ \(p \leq x \leq q\) پر گھتٹا ہوا کہا جاتا ہے۔
%Page 99
اگر\(f'(x)>0\)ہو  وقفہ \(p \leq x \leq q\) میں تو   \(f(x)\)   وقفہ \(p \leq x \leq q\)  میں بڑھتا ہوا ہوگا۔
اگر  \(f'(x)<0\) ہو وقفہ \(p<x<q\) میں ، تو \(f(x)\) وقفہ \(p \leq x \leq q\) میں گھٹتا ہوا ہوگا۔

اس بات کو دھیان میں رکھیں کہ تفاعل  \(f(x)\) کو وقفہ \(p \leq x \leq q\) میں بڑھتا ہوا ہونے کے لیے مشتق تفاعل \(f'(x)\) کے ڈھلوان کا وقفہ کے خاتمے پر جہاں \(x=p\)یا \(x=q\)ہے، مثبت ہونا لازمی ہے، ان نقطوں پر یہ صفر یا بالکل غیر واضع ہوگا۔ یہ شاید ایک خفیف تفاوت لگے لیکن اسکے بہت اہم نتائج ہیں۔ یہ صرف استمراری تفاعلات کے ساتھ کام کرنے کے فیصلے کا انجام تھا۔

وقفہ کا لفظ صرف  \عددی{x} کی ان قیمتوں کے لیے استعمال نہیں ہوتا جوکہ محدود انتہاؤں کے درمیان ہوتی ہیں۔ بلکہ  \عددی{x} کی ان قیمتوں کے لیے بھی استعمال ہوتا ہے، جو عدم مساواتوں \(x>p\) یا \(x<q\) کو باور کرواتی ہیں۔
 \ابتدا{مثال}
تفاعل \(f(x)=x^4-4x^3\) کے لیے ، معلوم کریں، وہ وقفہ جس میں  \(f(x)\)  بڑھتا ہوا ہو، اور، وہ وقفہ جس میں گھٹتا ہوا ہو۔

شروع کرتے ہیں \(f'(x)\) کو اجزاۓ ضربی میں بیان کرتے ہوۓ ، جیسے \(f'(x)=4x^3-12x^2=4x^2(x-3)\) جیسا کہ \(x^2\) ہمیشہ مثبت ہے،(صرف\(x=0\) کے علاوہ) یہ معلوم کرنے کے لیے کہ کس جگہ\(f'(x)>0\) ہے۔۔ آپ کو عمد مساوات \(x-3>0\) کا حل \(x>3\) ہے؛ لیکن یہ معلوم کرنے کے لیے کہ کہاں \(f'(x)<0\) ہے، آپ کو \(x=0\) کو خارج کرنا پرے گا تاکہ \(f'(x)<0\) ہو۔ اگر \(x<0\) یا \(0<x<3\) ہو۔ اس لیے \(f(x)\)  وقفہ \(x \leq 3\) اور وقفہ \(0 \leq x \leq 3\) میں گھتتا ہوا ہے۔

تاہم پچھلے دو وقفوں میں \(x=0\) کی قیمت مشترک ہے۔ اس طرح آپ ان کو ایک ہی وقفے \(x \leq 3\) میں یکجا کر سکتے ہیں۔ اس سے یہ نتیجہ نکلتا ہے کہ \(f(x)\) وقفہ \(x \leq 3\) میں گھٹتا ہوا ہے۔
 \انتہا{مثال}  
اس بات کو مد نطر رکھیں  کہ \(f'(x)=0\) ہے جب \(x=0\) اور \(x=3\) ہیں۔ آپ صورت  \حوالہء{شکل7.8} میں دکھائ گئ \(y=f(x)\) کی ترسیمہ سے ان تمام خصوصیات کی پڑتال کر سکتے ہیں۔مثال  \حوالہء{مثال7.2.2} سے ظاہر ہوتا ہے کہ، اوپر دہے گۓ اصول ( جو کہ \(f'(x)\) کی علامت کو    \(f(x)\)  کی خصوصیت سے جوکہ بڑھتا ہوا یا گھٹتا ہوا ہے ہے،  جو --- دیتا ہے۔) کو ذرا کشادہ کیا جا سکتا ہے۔

اگر \(f'(x)>0\) ہو وقفہ \(p<x<q\)  میں سواۓ ان علیحدہ نقطوں پر جہاں \(f'(x)=0\) ہے۔تو     \(f(x)\) وقفہ \(p \leq x \leq q\) میں بڑھتا ہوا ہوگا۔ اگر \(f'(x)<0\) ہو وقفہ \(p<x<q\) میں سواۓ ان علیحدہ نقطوں پر جہاں \(f'(x)=0\) ، تو \(f(x)\) وقفہ \(p \leq x \leq q\) میں گھٹتا ہوا ہوگا۔

اگلی مثال اس تفاعل کے بارے میں ہے جس میں  \عددی{x}  کی طاقت مکسور شامل ہے۔( \(x<0\) کے لیے )۔ مکسور طاقتیں بعض اوقات مشکلات پیدا کرتی ہیں کیونکہ ان میں کچھ ، جب \عددی{x} منفی ہو تو غیر واضع ہوتے ہیں۔ لیکن اس مثال میں صرف جزرالکعب معلوم کرنا کوئ مشکل کام نہیں ہے۔

 \ابتدا{مثال}
ان وقفوں کو معلوم کریں جن میں تفاعل \(f(x)=x^{\frac{2}{3}}(1-x)\) بڑھتا ہوا ہو، اور جن میں گھٹتا ہوا ہو۔ 

تفریق کرنے کے لیے تفاعل   \(f(x)\) کو اس طرح لکھیں۔
\[f(x)=x^{\frac{2}{3}}-x^{\frac{5}{3}}\]
تاکہ \[f'(x)=\frac{2}{3}x^{-\frac{1}{3}}-\frac{5}{3}x^{\frac{2}{3}}\]
جس کو آپ اس طرح بھی لکھ سکتے ہیں۔\[f'(x)=\frac{1}{3}x^{-\frac{1}{3}}(2-5x)\]

اس آکری فقرے میں \(x^{-\frac{1}{3}}\)مثبت ہوگا جب \(x>0\) اور منفی ہوگا جب \(x<0\)۔\(2-5x\)کا جزرضربی مچبت ہوگا جب \(x<0.4\) ہو اور منفی ہوگا جب \(x>0.4\)
 
صورت  \حوالہء{شکل7.9} سے ظاہر ہوتا ہے کہ؛
\(f(x)\)
وقفہ \(0 \leq x \leq 0.4\) میں بڑھتا ہوا ہے۔

\(f(x)\)
وقفہ \(x \geq 0\) اور \(x \geq 0.4\) میں گھٹتا ہوا ہے۔

 \انتہا{مثال}

\حصہ{ذیادہ سے ذیادہ اور کم سے کم نقطے}
مثال \حوالہء{مثال7.2.1}سے ظاہر ہوا کہ تفاعل\(f(x)=x^2-6x+4\) ،\(x \leq 3\) کے لیے گھٹتا ہوا ہے، اور  \(x \geq 3\) کے لیے بڑھتا ہوا ہے۔بڑھتے ہوۓ اور گھتتے ہوۓ تفاعلات کی تعریف سے معلوم ہوتا ہے کہ اگر \(x_1 <3\) ہو تو \(f(x_1) >f(3)\) ہوگا۔ اور یہ کہ اگر \(x_2 >3\) ہو تو \(f(x_2) >f(3)\)۔ یعنی  \عددی{x}  کی  \عددی{ 3} کے علاوہ ہر قیمت کے لیے ،تفاعل \(f(3)=-5\) سے بڑا ہوگا۔

آپ کہہ سکتے ہیں کہ \(f(3)\)، \(f(x)\) کی کم سے کم قیمت ہے اور یہ کہ (3،-5)\(y=f(x)\) کی ترسیمہ کا کم سے کم نقطہ ہے۔

ضروری نہیں کہ کم سے کم نقطہ کل ترسیمہ پر سب سے کمتر نقطہ ہو، بلکہ یہ اپنے قرب و جوار میں سے کمتر نقطہ ہوتا ہے۔

صورت \حوالہء{شکل7.9} میں \( (0,0)\) سے ایک کم سے کم نقطہ ہے؛ یہ بات اس حقیقت سے ظاہر ہوتی ہے کہ   \(f(x)>0\) ہے، \(x<1\) سے ہر عدد کے لیے سواۓ \(x=0\) کے، اگر چہ \(f(x)<0\) جب \(x>1\) ہے۔

یہ ایک وضاحت (تعریف) کی طرف رہنمائ کرتا ہے، جو کہ صورت  \حوالہء{شکل7.10} میں ظاہر کیا گیا ہے۔
 %page101
تفاعل  \(f(x)\)  کا،  \(x = q \) پر کم سے کم نقطہ ہوگا، اگر ایسا وقفہ  \(p <x <r \) ہو جس میں      \عددی{q } موجود ہو، جہاں   \( f(x) > f(q) \) ہو، \عددی{x} کی ہر قیمت کے لیے سواۓ \عددی{q }  کے۔
اس کا ذیادہ سے ذیادہ نقطہ ہوگا، اگر وقفے میں   \عددی{x}      کی ہر قیمت کے لیے سواۓ  \عددی{ q}    کے \( f(x)>f(q) \)ہو۔ نقطہ    \( (q,f(q) \)   کو کم سے کم نقطہ یا ذیادہ سے ذیادہ نقطہ کہا جاتا ہے۔چنانچہ مثال      \حوالہء{7.2.3}میں   \(f(x)\)  کا کم سے کم نقطہ \(x=0\)      پر ہے اور ذیادہ سے ذیادہ  نقطہ  \(x=4\)    پر ہے۔

کم سے کم اور ذیادہ سے ذیادہ نقطوں کو بعض اوقات نقطہ تغیر بھی کہا جاتا ہے۔

آپ دیکھیں گے کہ  شکل     \حوالہء{7.8} کے کم سے کم اور شکل   \حوالہء{7.9}  کے ذیادہ سے ذیادہ نقطوں پر ترسیمہ کا ڈھلوان صفر ہے۔ لیکن شکل    \حوالہء{7.9}   میں کم سے کم نقطہ    پر ترسیمہ کا خط مماس  \عددی{ y}کا محور ہے، اس لیے ڈھلوان غیر واضع ہے۔

یہ مثالیں ایک عمومہ اصول کو ظاہر کرتی ہیں؛
اگر    \( (q,f(q)) \) ترسیمہ   \(y=f(x) \)    کا کم سے کم یا ذیادہ سے ذیادہ نقطہ ہو ، تو یا    \( f'(q) =0 \)  ہوگا      یا \(f'(q) \) بالکل غیر واضع ہوگا۔

دھیان رہے اگرچہ شکل \حوالہء{ 7.8 } میں ایک اور نقطہ بھی ہے جہاں ڈھلوان صفر ہے جو کہ نو تو کم سے نقطہ ہے نا ذیادہ سے ذیادہ۔مثلاً نقطہ       ترسیمہ پر وہ نقطہ جہاں ڈھلوان صفر ہو ساکن نقطہ کہلاتا ہے۔ اسی طرح شکل 7.8 اور 7.9 اس حقیقت کو واضع کرتی ہیں کہ ساکن نقطہ کم سے کم یا ذیادہ سے ذیادہ نقطہ ہوسکتا ہے یا دونوں میں سے کوئ بھی نہیں ہو سکتا۔

کم سے کم اور ذیادہ سے ذیادہ نقطے میں فیصلہ کرنے کا ایک طریقہ ڈھلوان     \(f'(x)\)     کی علامت کو  	 \( x=q \)	 کے  دونوں طرف معلوم کرنا ہے۔ تفصیلات کے لیے شکل \حوالہء{ 7.10 }کی ترسیمات کی طرف رجوع کرنا آپ کے لیے دوبارہ سے مددگار ثابت ہو سکتا ہے۔

%21

اگر      \(f'(x)<0 \) ہو، وقفہ	 \(p < x < q \) میں اور  \( f'(x) >0 \)	 ہو وقفہ  \(q < x < r \) میں تو 	 \( (q,f(q) \)	 ایک کم سے کم نقطہ ہوگا۔

اگر	\(f'(x)>0\)	 ہو،وقفہ	\( p < x < q \)میں ،اور	\( f'(x)>0 \)	ہو وقفہ	\(q < x < r \)		میں ، تو 	\(q,f(q)) \) 	 ایک ذیادہ سے ذیادہ نقطہ ہوگا۔

آپ ترسیمات کی شہادتکی بنا پر اس کو قبول کرتے ہوۓ شائد خوش ہوں گے، لیکن یہ ان بیانات سے بھی ثابت ہو سکتا ہے جن کا سامنا آپ سے پہلے ہی ہو چکا ہے۔
فرض کریں \(x_1\)	   وقفہ  \(p < x < q \) 	میں ایک عدد ہے،تو، (چونکہ \( f'(x)>0 \)	ہےاس وقفے میں)، ضمن \حوالہء{7.2 } سے معلوم ہوتا ہے کہ	\( f(x_{1})>f(q) \)  	ہوگا۔


اب فرض کریں،	\(x_2\)	وقفہ 	\(q < x < r \)	 میں ایک عدد ہےچونکہ اس وقفے میں \(f'(x)>0 \)ہے،\( f(q)<f(x_{2}) \)ہوگا۔

اس سے یہ ظاہر ہوتا ہے کہ اگر 	  \عددی{x} وقفہ \(p < x < r \)		میں 	\عددی{ q}	کے علاوہ کوئ بھی عدد ہو تو 	\(p < x < r \)	ہوگا۔ تعریف کے مطابق اس کا مطلب ہےکہ  \(f(x)\)  	کا کم سے کم نقطہ \(x=q\)		پر ہے۔ 	

%22
ان تمام نتائج کو ایک طریقہ کار کی شکل میں جمع کیا جا سکتا ہے۔

مساوات	\(y=f(x) \)	کی ترسیمہ پر کم سے کم اور ذیادہ سے ذیادہ نقطوں کو معلوم کرنے کے لیے؛
\begin{enumerate}[a.]
\item
وہ دائرہ کار طے کریں جس سے آپ کو سروکار ہو۔
\item
 \(f'(x)\) کے لیے ایک ریاضیاتی بیان معلوم کریں
\item
 دائرہ کار میں موجود ان  \عددی{x} کی قیمتوں کو درج کریں جن کے لیے \(f\prime(x)\)  یا تو صفر ہو یا غیر واضع ہو۔
\item
ان تمام   \عددی{x} کی قیمتوں میں سے ہر ایک کو باری باری لیتے ہوۓ،اسی قیمت کے قریب ترین دائیں اور بائیں وقفوں میں	 \(f\prime(x)\)	کی علامت معلوم کریں۔
\item
گر یہ علامتیں علی الترتیب منفی اور مثبت ہوں تو ترسیمہ کے ہاں کم سے کم نقطہ ہوگا۔ اگر یہ مثبت اور پھر منفی ہوں تو ذیادہ سے ذیادہ نقطہ ہوگا۔اگر علامتیں بدل نہ رہی ہوں اور ایک جیسی ہوں تو دونوں میں سے کوی بھی نہیں ہوگا۔
\item
 \عددی{x} کی ہر قیمت کے لیے جو کہ کم سے کم یا ذیادہ سے ذیادہ ہے  \(f(x)\) 	معلوم کریں۔
\end{enumerate}

%23
 \ابتدا{مثال}
مساوات \( y=\sqrt{x}+\frac{4}{x} \) کی ترسیمہ پر کم سے کم نقطہ معلوم کریں۔
فرض کریں \( y=f(x)=\sqrt{x}+\frac{4}{x} \)		۔۔

درجہ الف-- جیسا کہ $ \sqrt{x}$		،	  $ x\geq0$	کے لیے واضع ہے، لیکن 	$ \frac{1}{x} $،	  $x=0 $	کے لیے غیر واضع ہے۔ اس لیے سب سے بڑا ممکنہ دائرہ کار 	   \(f(x)\)   	کے لیے مثبت حقیقی اعداد ہوگا۔
درجہ ب--متفرق	$ f'(x)=\frac{1}{2} x^{\frac{-1}{2}} -4x^{-2}$کو اس طرح بھی لکھا جا سکتا ہے جیسے  $f'(x)=\frac{x^{\frac{3}{2}}-8}{2x^{2}} $۔۔۔
درجہ ج-- متفرق تمام حقیقی اعداد کے لیے واضع ہے اور جب $ x^{\frac{3}{2}}=8 $ ہو تو صفر ہے دونوں اطراف کو طاقت  $\frac{2}{3} $		 تک اٹھانے اور طاقت در طاقت کے اصول کو استعمال کرنے کے بعد    $ x=(x^{\frac{3}{2}})^{\frac{2}{3}}=8^{\frac{2}{3}}=4 $۔
--
درجہ د-- اگر$ 0<x<4 $ ہو ، تو کسر کے نیچے والا فقرہ 	   $2x^{2} $	مثبت ہوگا،اور    $x^{\frac{3}{2}}-8<4^{\frac{3}{2}}-8=8-8=0 $   ہوگا،نتیجۃً	 $f'(x)<0 $	ہوگا۔ 
اگر 	\( x\g 4\) ہو تو 	$2x^{2} $ برابر مچبت ہی رہے گا، لیکن $x^{\frac{3}{2}}-8>4^{\frac{3}{2}}-8=0 $ ہوگا، اور نتیجۃًٍٍ \( f'(x)>0 \) ہوگا۔
درجہ ہ --  \(f\prime(x)\) کی علامت  \عددی{ 4} کے بائیں جانب منفی ہے اور دائیں جانب مثبت ، اسطرح تفاعل کا کم سے کم نقطہ $ x=4 $  ہوگا۔
درجہ ی-- 		 $ f(4)=\sqrt{4}+\frac{4}{4}=2+1=3 $	حساب کرنے پر کم سے کم نقطہ 	\(  (4,3)\)	 نکل آتا ہے۔
 \انتہا{مثال}   

 %25
اگر آپ کے پاس ترسیمہ شمار کنندہ ہو تو اسے استعمال کر کے \(y=f(x) \) کو  \(y=\sqrt{x} \) اور  \(y=\frac{4}{x} \)  کے ساتھ اکٹھے دکھانا جس سے یہ تفاعل بنا ہے، بہت دلچسپ لگے گا۔ آپ جان لیں گے کہ \(y=f(x) \)  کم سے کم نقطے کے اردگرد بہت ہموار ہے۔ یہ بتانا آنکھوں سے بہت مشکل کام ہوگا کہ کم سے کم نقطہ ٹھیک کہاں واقع ہے۔

اس بات کو دھیان میں رکھیے گا کہ یہ نظریہ آپکو بعض تفاعلات کی سعت معلوم کرنے کے لیے، کوئ اور راستہ فراہم کرتا ہے۔مثال \حوالہء{7.3.1} کے تفاعل کی جسکا دائرہ کار 	\( x\g 0\)	 ہے،سعت 	\( y\ge 3\) ہوگی۔

\ابتدا{سوال}
مندرجہ ذیل ہر تفاعل 	\(f(x)\) 	کا متفرق \( f\prime (x)\)معلوم کریں، اور وہ وقفہ معلوم کریں جس میں 		  \(f(x)\)   برھتا ہوا ہو۔ 
\begin{multicols}{3}
\begin{enumerate}[a.]
\item \(x^{2}-5x+6 \hspace{50pt}\)
\item  \(x^{2}+6x-4 \hspace{50pt}\)
\item   \(7-3x-x^{2} \)
\item    \(3x^{2}-5x-7 \hspace{50pt}\)
\item     \(5x^{2}+3x-2 \hspace{50pt}\)
\item      \((f)7-4x-3x^{2}\)
\end{enumerate}
\end{multicols}
\انتہا{سوال}

\ابتدا{سوال}
مندرجہ ذیل تفاعلات  \(f(x)\) میں سے ہر ایک کا متفرق 		\( f\prime (x)\) معلوم کریں اور وہ وقفہ معلوم کریں جس میں \(f(x)\)  گھٹتا ہوا ہو۔		
\begin{multicols}{3}
\begin{enumerate}[a.]
\item \(  x^{2}+4x-9  \)
\item \(   x^{2}-3x-5 \hspace{50pt}  \)
\item  \( 5-3x+x^{2}     \)
\item  \(  2x^{2}-8x+7 \hspace{50pt}   \)
\item  \( 4+7x-2x^{2} \hspace{50pt}   \)
\item \(  3-5x-7x^{2}  \)
\end{enumerate}
\end{multicols}

\انتہا{سوال}
\ابتدا{سوال}
مندرجہ ذیل تفاعلات  \(f(x)\) میں سے ہر ایک کا متفرق 		\( f\prime (x)\) معلوم کریں، اور کوئ سا وقفہ معلوم کریں، جس میں 	 \(f(x)\) 	گھٹتا ہوا ہو۔جز(و) میں \عددی{ n} عدد صحیح ہے۔
\begin{multicols}{3}
\begin{enumerate}[a.]
\item \(  x^{3}-12x \hspace{50pt}  \)
\item \(   2x^{3}-18x+5 \hspace{50pt} \)
\item  \( 2x^{3}-9x^{2}-24x+7   \)
\item  \(  x^{3}-3x^{2}+3x+4 \hspace{50pt}   \)
\item  \(  x^{4}-2x^{2} \hspace{50pt}    \)
\item \(   x^{4}+4x^{3}   \)
\item \( 3x-x^{3} \hspace{50pt}   \)
\item \(  2x^{5}-5x^{4}+10 \hspace{50pt}  \)
\item  \( 3x+x^{3}    \)
\end{enumerate}
\end{multicols}
\انتہا{سوال}
\ابتدا{سوال}
مندرجہ ذیل ہر تفاعل 	 \(f(x)\) 	کا متفرق \( f\prime (x)\)معلوم کریں اور وہ وقفہ معلوم کریں جس میں   \(f(x)\) بڑھتا ہوا ہو۔
\begin{multicols}{3}
\begin{enumerate}[a.]
\item \(  x^{3}-27x for x\geq 0 \hspace{30pt}  \)
\item \(  2x^{3}18x+5  \)
\item  \(x^{3}-3x^{2}+3x-1 \)
\item  \(  12x-2x^{3} \hspace{50pt}  \)
\item  \(  2x^{3}+3x^{2}-36x-7 \hspace{50pt}  \)
\item \(  3x^{4}-20x^{3}+12   \)
\item \(36x^{2}-2x^{4} \hspace{50pt}     \)
\item \( 2x^{5}-5x \hspace{50pt}   \)
\item \(x^{n}-nx (n>1)   \)
\end{enumerate}
\end{multicols}
\انتہا{سوال}
\ابتدا{سوال}
مندرجہ ذیل تفاعلات 	  \(f(x)\) 	میں سے ہر ایک کا متفرق تفاعل 	\( f\prime (x)\)	معلوم کریں، اور وہ وقفے معلوم کریں جن میں 	  \(f(x)\) 	گھٹتا ہوا ہو۔اور وہ وقفے جن میں 		  \(f(x)\) بڑھتا ہوا ہو۔
\begin{multicols}{3}
\begin{enumerate}[a.]
\item \(  x^{\frac{3}{2}}(x-1) \hspace{50pt}  \)
\item \(  x^{\frac{3}{4}}-2x^{\frac{7}{4}},for x>0 \hspace{50pt}   \)
\item  \(x^{\frac{2}{3}}(x+2)\)
\item  \(x^{\frac{3}{5}}(x^{2}-13) \hspace{50pt}  \)
\item  \( x+\frac{3}{x} for x\neq 0 \hspace{50pt}    \)
\item \( \sqrt{x}+\frac{1}{\sqrt{x}},x>0    \)
\end{enumerate}
\end{multicols}
\انتہا{سوال}
\ابتدا{سوال}
مندرجہ ذیل تفاعلات\(f(x)\) کی ترسیمات میں سے ہر ایک کے لیے ؛
\begin{multicols}{3}
\begin{enumerate}[a.]
\item 
ساکن نقطوں کے محدد معلوم کریں
\item 
دلیل کے ساتھ بتا دیں کہ آیا یہ ذیادہ سے ذیادہ یا کم سے کم نقطہ ہے۔
\item  
راس معلوم کرنے کے لیے ، تکمیل مربع کے قاعدے کو استعمال کر کے اپنے جواب کی پڑتال کریں
\item  
ان قیمتوں کی سعت بیان کریں جن قیمتوں کو تفاعل 		لے سکتا ہے۔
\end{enumerate}
\end{multicols}

\begin{multicols}{3}
\begin{enumerate}[a.]
\item \(  x^{2}-8x+4 \hspace{50pt}   \)
\item \( 3x^{2}+12x+5 \hspace{50pt}   \)
\item  \( 5x^{2}+6x+2   \)
\item  \( 4-6x-x^{2} \hspace{50pt}    \)
\item  \( x^{2}+6x+9 \hspace{50pt}    \)
\item \(  1-4x-4x^{2}  \)
\end{enumerate}
\end{multicols}
\انتہا{سوال}
\ابتدا{سوال}
مندرجہ ذیل تفاعلات کی ترسیمات پر ساکن نقطوں کے ہم پلہ نقطے معلوم کریں، نیز معلوم کریں کہ آیا نقاط ذیادہ سے ذیادہ نقاط ہیں یا کم سے کم نقاط ہیں
\begin{multicols}{3}
\begin{enumerate}[a.]
\item \( 2x^{3}+3x^{2}-72x+5 \hspace{20pt}   \)
\item \(  x^{3}-3x^{2}-45x+7 \hspace{20pt}  \)
\item  \(3x^{4}-8x^{3}+6x^{2}    \)
\item  \( 3x^{5}-20x^{3}+1 \hspace{20pt}    \)
\item  \( 2x+x^{2}-4x^{3} \hspace{20pt}   \)
\item \( x^{3}+3x^{2}+3x+1   \)
\item \( x+\frac{1}{x} \hspace{20pt}   \)
\item \( x^{2}+\frac{54}{x} \hspace{20pt}   \)
\item  \( x-\frac{1}{x}   \)
\item  \(x-\sqrt{x},for x>0 \hspace{20pt}\)
\item  \(\frac{1}{x}-\frac{3}{x^{2}} \hspace{20pt}    \)
\item \(  x^{\frac{1}{3}}(4-x) \hspace{20pt}  \) 
\item  \( x^{\frac{1}{5}}(x+6) \hspace{20pt}   \)
\item  \(x^{4}-8x^{2}     \)
\item \(   x^{2}-\frac{16}{x}+5 \)
\end{enumerate}
\end{multicols}
\انتہا{سوال}
 \ابتدا{سوال}
ان تفاعلات کی سعتیں معلوم کریں، جوکہ سب سے بڑے ممکنہ دائرہکاروں میں واضع ہوں
\begin{multicols}{3}
\begin{enumerate}[a.]
\item \( x^{2}+x+1    \)
\item \( x^{4}-8x^{2}   \)
\item  \( x+\frac{1}{x}   \)
\end{enumerate}
\end{multicols}
\انتہا{سوال}

 \حصہ{متفرقات، تبدیلی کی شرح کے موافق}
تعلق 	 \(y=f(x)\)  	میں موجود		  \عددی{x} اور \عددی{ y} کی مقداروں کو بسا اوقات متغیرات کہا جاتا ہے، کیونکہ 	 \عددی{ x} دائرہکاروں میں موجود کوئ بھی عدد ہوتا ہے اور  \عددی{ y} سعت میں موجود کوئ بھی عدد ہو سکتا ہے۔ جب آپ ترسیمہ بناتے ہیں تو 	\عددی{x}کی قیمتوں کے چناؤ میں آذاد ہوتے ہیں۔ اور پھر  \عددی{ y}کی قیمتوں کو وضع کرتے ہیں۔ اس لیے 	\عددی{x}  	کو آذاد اور 	 \عددی{ y}	کو تابع متغیرہ کہا جاتا ہے۔

یہ تغیرات بسا اوقات طبعی یا معاشی مقداروں کو ظاہر کرتے ہیں اور پھر دیگر حروف کو استعمال کرنا، جوکہ ان مقداروں کے بارے میں بتاتے ہیں، بہت معقول لگتا ہے۔ مثلاً وقت کے لیے 	  \عددی{ t} 	، حجم کے لیے  \عددی{ V}  		،دام کے لیے 	  \عددی{ c} 	،آبادی کے لیے 	  \عددی{p } 	اور وغیرہ وغیرہ۔

یہ بات بہت جلد واضع کی جاۓ گی، کہ حرف 	  \عددی{ d} 	کو گہرائ کے لیے کیوں استعمال نہیں کیا گیا۔ حرف 		  \عددی{ z} عمودی سمت میں فاصلے کے لیے ذیادہ تر استعمال ہوتا ہے۔

تابع متغیر ،دباؤ 	  \عددی{ p} 	ہے، جسے بارس میں ناپا جاتا ہے۔ سطح پر غوطہ خور صرف ہوائ دباؤ محسوس کرتا  ہے، جو کہ بار 1 کے لگ بھگ ہوتا ہے لیکن جوں جوں غوطہ خور نیچے اترتا جاتا ہےدباؤ بڑھتا جاتا ہے۔ ساحلی گہرائیوں پر متغیرات تقریباً اس مساوات کے ذریعے جڑے ہوۓ ہوتے ہیں۔\(p=1+0.1z\)

ترسیمہ کا ہم پلہ نقطہ 		\( (z,p)\)	ایک خط مستقیم ہے، جس طرح شکل \حوالہء{7.11}  میں دکھائ گئ ہے۔مستقل عدد\عددی{  0.1،}   مساوات میں موجود، وہ مقدار ہے جس سے دباؤ ہر اضافی گہرائ کی لمبائ کے لیے بڑھ جاتا ہے۔ یہ دباؤ کی گہرائ کے متعلق تبدیلی کی شرح ہے۔

اگر غوطہ خور 	\( \delta z   \)	میٹر کے فاصلے تک نیچے اترتا ہے تو دباؤ		\( \delta p   \)مقدار تک بڑھتا جاۓ گا۔ یہ تبدیلی کی شرح 		\( \frac{\delta p}{\delta z} \)ہے۔ یہ ترسیمہ کے ڈھلوان سے ظاہر کیا گیا ہے۔
%31
لیکن سمندر کی گہرائیوں میں ترسیمہ \( (z,p)\)مزید خط مستقیم نہیں رہتی، بلکہ اسکی شکل   \حوالہء{شکل7.12}  والی بن جاتی ہے۔مقدار\(\frac{\delta p}{\delta z} \) اب،اضافی گہرائ \( \delta z   \)میں تبدیلی کی متوسط شرح کو ظاہر کرتی ہے۔

صورت  \حوالہء{شکل7.12} میں وتر کا ڈھلوان اسی چیز کو ظاہر کرتا ہے۔ گہرائ کے متعلق دباؤ کی تبدیلی کی شرح،\( \frac{\delta p}{\delta z} \) کی حد ہے،(جیسے ہی \( \delta z   \)صفر کو بڑھتا ہے)

متفرق\( f'()\)کی علامت جیسے اب تک اس حد کے لیے استعمال کیا گیا ہے۔، معیاری نہیں ہے، کیونکہ اس میں  \عددی{  p} کا تزکرہ نہیں ہے، ایک ایسی علامت کا ہونا ضروری ہے، جس میں متغیرات کے لیے استعمال کیے گۓ دونوں حروف موجود ہیں۔ایک متبادل علامت  \( \frac{\delta p}{\delta z} \)وضع کیا جاتا ہے، جسے متوسط شرح میں حرف  \( \delta\)کو حد میں  \عددی{d} سے بدل کر حاصل کیا جا سکتا ہے۔

باقاعدہ طور پر،\[ \frac{dp}{dz}=\lim_{\delta z \to 0}\frac{\delta p}{\delta z} \]

یہاں کوئ نیا تصور نہیں ہے۔ یہ صرف  \حوالہء{باب تفرق } میں دیے گۓ متفرق کی تعریف کو ایک نۓ مختلف انداز میں لکھنے کا طریقہ ہے۔ اک کا فائدہ یہ ہے کہ مختلف حروف کو استعمال کر کے اسے ملایا جا سکتا ہے، جب کبھی دو متغیرات میں تفاعلی تعلق ہو، ان میں تبدیلی کی شرح کو بیان کرنے کے لیے۔

اگر  \عددی{ x }اور  \عددی{ y }علی الترتیب ، آذاد اور تابع متغیرات ہوں، کسی تفاعلی تعلق میں ، تو متفرق،
\[\frac{dp}{dz}=\lim_{\delta x \to 0}\frac{\delta p}{\delta z}   \]
متغیر\عددی{ y }کی \عددی{ x }کے متعلق تبدیلی کی شرح کی ناپتا ہے۔ اگر \( y=f(x)\)ہو،تو\(\frac{dy}{dx}=f'(x) \)ہوگا۔

ہر چند کہ \(\frac{dy}{dx}\)ایک کسر لگتا ہے،فی الحال آپ کو اسے ایک غیر متفق علامت جیسا خیال کرنا چاہئیے جو چار حروف اور ایک افقی لکیر سے بنایا گیا ہو۔
%33
جو علامت \( \dif{x}\)اور  \( \dif{y}\) ہیں، وہ آپ کوئ معنی نہیں رکھتے (بعد میں،گوِ آپ کو معلوم ہوگا کہ بعض صورتوں میں علامت \(\frac{dy}{dx}\) ایک کسر کیطرح پیش آتا ہے۔ یہ\(f'() \)کی علامت کے اوپر اسکا ایک اور فائدہ ہے)

اس علامت کو وسیع معنوں میں استعمال کیا جا سکتا ہے۔ مثال کے طور پر، اگر جلے ہوۓ گھاس کا رقبہ آگ لگنے کے \عددی{  t}  منٹ بعد \عددی{A  }  مربع میٹر \(   m^{2}  \)ہو تو    \( \frac{dA}{dt} \)اس شرح کو ناپتا ہے، جس سے آگ مربع میٹر فہ منٹ کے حساب سے پھیل رہی ہو۔ اگر زمین کی سطح پو موجود کسی نقطے پر، میدان میں    \عددی{x}  میٹر کے فاصلے کو نقشے پر \عددی{ y } میٹر سے ظاہر کیا جاۓ تو   \(\frac{dy}{dx}\)  اس نقطے پر نقشے کے پیمانے (درجے) کو ظاہر کرتا ہے۔ 

 \ابتدا{مثال}
خواتین کی \عددی{ 100 }  میٹر دوڑ ، میں ایک تیز دوڑنے والی   \عددی{  36}    میٹر طے کرنے کے بعد، اپنی بلند ترین رفتار    \عددی{ 12 } میٹر فی سیکنڈ پر پہنچ جاتی ہے، اس فاصلے تک ، اسکی رفتار طے کۓ گۓ فاصلے کی جذر سے متناسب ہے۔

یہ ثابت کریں کہ جب تک وہ آخری رفتار تک نہیں پہنچ جاتی، اس کی رفتار میں فاصلے سے متعلق تبدیلی کی شرح ، اسکی رفتار کے بالعکس متناسب ہے۔

فرض کریں کہ \عددی{x} میٹر دوڑنے کے بعد اسکی رفتار \عددی{ S }میٹر فی سیکنڈ ہوتی ہے۔ آپکو کہا گیا ہے کہ $x=36 $میٹر تک رفتار  $S=k\sqrt{x} $ہوگی، اور یہ بھی کہ جب  $S=12$ہوگا تو  $x=36$ ہوگی۔ تو،
\[12=k\sqrt{36}\]
جوکہ\عددی{ k }کی قیمت دے گا،
\[ k=\frac{12}{6}=2\]
لہٰذہ\(   (x,S)    \)  کا تعلق؛
\(   0<x<36 \)
 کے لیے \( S=2\sqrt{x}  \)ہوگا۔

فاصلے سے متعلق رفتار میں تبدیلی کی شرح، متفرق$\frac{dS}{dx}$ہوگی، اور$\sqrt{x}$ کا متفرق (حصہ  \حوالہء{حصہ6.5} سے)   $\frac{1}{2\sqrt{x}}$ ہے۔ 
  
اس لیۓ،
$$ \frac{dS}{dx}=2 \times \frac{1}{2\sqrt{x}}=\frac{1}{\sqrt{x}} $$
چونکہ  \(   \sqrt{x}=\frac{S}{2}    \)ہے، \(   \frac{dS}{dx}    \) کو \(\frac{2}{S}\)لکھا جا سکتا ہے۔ تبدیلی کی شرح، اس لیۓ، اسکی رفتار کے بالعکس متناسب ہے۔
%35 leaving here 
باقی (بچی ہوئ) دوڑ تک کے لیے اگر وہ اپنی بلند ترین رفتار برقرار رکھتی ہے، تو رفتار برقرار رکھتی ہے، تو رفتار میں فاصلے کی نسبت تبدیلی کی شرح \عددی{  0}  تک گر جاۓ گی،    \(   x\g 36    \)کے لیے صورت  \حوالہء{شکل 7.13 } ظاہر کرتی ہے کہ ڈھلوان (چونکہ تبدیلی کی شرح کو ظاہر کرتا ہے) ، جیسے ہی اسکی رفتار بڑھتی جاۓ گی، چھوٹا ہوتا جاۓ گا،اور پھر صفر ہوگا۔ جونہی وہ بلند ترین رفتار پہ پہنچ جاۓ گی۔
\انتہا{مثال}
%
 \ابتدا{مثال}
کاروں کی ایک قطار ، جس میں ہر کوئ  \عددی{  5}  میٹر لمبی ہے، ایک مستقل رفتار  \عددی{  S}  کلومیٹر فی گھنٹہ کے حساب سے ایک کھلی سڑک پر سفر کر رہی ہے۔ کاروں کی ہر جوڑی کے درمیان ایک تجویز کردہ فاصلہ ہے جو کہ قاعدہ\( (0.18S+0.006S^2)\)میٹر میں دیا گیا ہے، سڑک میں گنجائش کے مطابق کاروں کی تعداد کو بڑھانے کے لیے ، کاروں کو کس رفتار سے سفر کرنا چاہئیے؟

فاصلے کے قاعدے  کو  \((aS+bS^2)\)کی شکل میں لکھنا، ایک اچھا تصور ہے، جہاں \(a=0.18 \)اور  \(b=0.006\)ہیں۔ یہ ایک صاف قاعدہ دیتا ہے، اور عددی سروں کو قاعدے میں تبدیل کرنے سے قاعدے پر پڑنے والے اثر کو بھی کھوجنے کے قابل بناتا ہے۔ لیکن یہ یاد رہے، جب آپ  \عددی{ a }اور \عددی{b} کا تفرق لیتے ہیں تو وہ محض مستقل اعداد ہوتے ہیں۔

ایک باڑ جو کہ کار کی لمبائ کا ہے اور اس کے سامنے علیحدگی کا فاصلہ سڑک کے  $5+aS+cS^2\si{m}$ میٹر  یا 
  $\frac{5+aS+cS^2}{1000}\si{km} $
کو گھیر لیتا ہے، ایک گھنٹے میں ایک نگرانی کرنے والے مقام سے گزرنے والے بلاکس کی سب سے بڑی تعداد کے لیے ، ایک بلاک سے چیک پوسٹ سے گزرنے کا وقت  \عددی{  T}(گھنٹوں میں) جتنا ممکن ہو سکے کم سے کم ہونا چاہئیے۔ چونکہ بلاک  \عددی{  S}کلومیٹر فی گھنٹہ کے حساب سے حرکت کر رہا ہے۔
\begin{align*}
TS&=frac{5+aS+cS^2}{1000}\\
T&=frac{5+aS+cS^2}{1000S}\\
T&=0.001(5S^{-1}+a+bS
\end{align*}
اب  \عددی{  T}  کی کم سے کم  قیمت کو معلوم کرنے کے طریقہ کار کی پیروی کریں۔ 
\begin{enumerate}[a.]
\item
چونکہ رفتار کو مثبت ہونی چاہئیے، اس لیے دائرہ کار \( S \g 0 \) ہوگا۔
\item
تفرق ہوگا،  \( \frac{dT}{dS}= 0.001(5S^{-2}+a+b)\)
\item
یہ متفرق دائرہ کار میں ہر جگہ واضع ہے ، اور جب \( -\frac{5}{S^2}+b=0  \)ہے تو صفر ہے۔ 
جس سے  \(S=\sqrt{\frac{5}{b}}\) آتا ہے
\item
ونہی \عددی{ S }  بڑھتی ہے  \( \frac{5}{\S^2}\)کم ہوتا ہے، اس لیے  \(-\frac{5}{S^2}+ \)بڑھتا ہے، چونکہ\(\frac{dT}{dS}\)صفر ہے، جب \(S=\sqrt{\frac{5}{b}}\)ہے، اور جب\(S<\sqrt{\frac{5}{b}}\)ہے تو  \(\frac{dT}{dS}\)کی علامت منفی ہے اور جب  \(S>\sqrt{\frac{5}{b}}\) ہے تو علامت مثبت ہے۔
\item
ونکہ \(\frac{dT}{dS}\)منفی سے مثبت تک تبدیل ہوتا ہے،  \عددی{T}  کم سے کم ہوگا، جب \(   S\g \sqrt{\frac{5}{b}}  \) ہوگا۔
\item
\( a=0.18  \)اور  \( b=0.006  \)  کو متبادل استعمال کرنے پر   
\( S=\sqrt{\frac{5}{0.006}}\approx 28.87 \)آتی ہے اور     \(  T\approx 0.0005264  \)  کم سے کم نقطے پر آتا ہے۔ 
\end{enumerate}
 
اس سے ظاہر ہوتا ہے کہ کاروں کی حرکت  \عددی{  29}  کلومیٹر فی گھنٹہ کی رفتار پہ سب سے بہتر ہوگی۔(ہر بلاک پھر تقریباً  \عددی{  0.000526} گھنٹہ یا  \عددی{ 1.89 }  سیکنڈز لے گا، (چیک پوائنٹ سے گزرنے کے لیے) نتیجۃً ایک گھنٹے میں گزرنے والی کاروں کی تعداد تقریباً      \( \frac{1}{0.000526} \approx 1900 \) ہوگی۔)
\انتہا{مثال}
%
  \ابتدا{مثال}
ایک خالی مخروطہ ، جس کی تہ کا رداس \عددی{ a }  سینٹی میٹر اور اونچائ\عددی{  b}  سینٹی میٹر ہیں، ایک میز پر پرا ہوا ہے۔ اس سب سے بڑے بیلن کا حجم کیا ہوگا، جسے اسکے اندر چھپایا جا سکتا ہو؟

رداس \عددی{ r }  سینٹی میٹر اور اونچائ \عددی{ h }  کے بیلن کا حجم \عددی{V  }   ہے، جوکہ   \[V=\pi r^2h\]

آپ برملا اپنی مرضی سے \عددی{ r } اور \عددی{ h } کو بڑا سے بڑا رکھ کے ، اسے بڑا بنا سکتے ہیں،۔ لیکن اس سوال میں متغیرات اس اقنضا کے پابند ہیں کہ بیلن کو مخروطے کے سانچے میں پورا آنا چاہئیے ہوگا۔

ذیادہ سے ذیادہ قیمت معلوم کرنے کا طریقہ کار کی پیروی کرنے سے پہلے آپ کو اس چیز کو معلوم کرنے کی ضروررت ہے کہ یہ حد بندی \عددی{ r }  اور \عددی{ h } کی قیمتوں کو کیسے اثر انداز کرتی ہے۔

صورت  \حوالہء{شکل7.14}ظاہر کرتی ہے ایک تین سمتی ڈھانچے کو اور صورت \حوالہء{شکل7.15}ایک عمودی حصہ  ہے، جوکہ مخروطہ کے سب سے اوپر ہے۔ صورت  \حوالہء{شکل7.15}میں بھاری لکیروں سے منتخب کی ہوئ مثلثیں (جوکہ مماثل ہیں)یہ ظاہر کرتی ہیں کہ \عددی{ r } اور \عددی{ h } مندرجہ ذیل مساوات کے ذریعے جڑے ہوۓ ہیں۔؛
 \[ \frac{h}{a-r}=\frac{b}{a} \]
لہٰذا۔
 \[ h= \frac{b(a-r)}{a} \]
ہوگا۔

 \عددی{ h }کے اس فقرے کو  \عددی{  V}کے کلیے میں متبادل استعمال کرنے پر ملتا ہے۔
 \[ V=\frac{\pi r^2b(a-r)}{a}=\frac{\pi b}{a}(ar^2-r^3) \]   
یہ بات دھیان میں رہے کہ\عددی{  V}کے ابتدائ ریا ضیاتی فقرے میں دو آذاد متغیرات\عددی{ r }  اور\عددی{ h } موجود ہیں۔
متبادل استعمال کرنے کے نتیجے میں آذاد متغیرات کی تعداد کم ہو کر ایک رہ جائیگی، \عددی{ h } غائب ہو جاتا ہے اور صرف \عددی{ r }  باقی رہتا ہے۔ یہ طریقہ کار کو استعمال کر کے ذیادہ سے ذیادہ قیمت معلوم کرنے کو ممکن بناتا ہے۔ اس طبعی مسئلے کا کوئ حقیقی مطلب تب ہوگاجب\( 0<r<a \)ہو، لہٰذہ اس وقفے کو تفاعل کے دائرہ کار کے طور پر لے لیں۔عمومی اصول کے مطابق تفریق کر کے (یاد رہے کہ   $\pi $ \عددی{a}   اور \عددی{ b }  مستقل اعداد ہیں) معلوم ہوتا ہے،
\begin{align*}
\frac{dV}{dr}&=\left(\frac{\pi b}{a}\right)(2ar-3r^2)\\
&=\left(\frac{\pi b}{a}\right)r(2a-3r)
\end{align*}
دائرہ کار  میں موجود \عددی{ r } کی صرف وہ قیمت جس کے لیے \(\frac{dV}{dr}=0 \) ہے ،$\frac{2}{3}a$ہے۔ اس بات کی جانچ پڑتال کرنا کہ \(\frac{dV}{dr}\)کی علامت\(0<r<\frac{2}{3}a\)کے لیے مثبت ہے۔اور$\frac{2}{3}a<r<a $ کے لیے منفی ہے، بہت آسان ہے۔
لہٰذہ ، ذیادہ سے ذیادہ حجم کے بیلن کا رداس\( \frac{2}{3}a \)ہوگا، اونچائ $\frac{1}{3}b $ ہوگی اور حجم  $\frac{4}{27}\pi a^2b $ ہوگا۔  
\انتہا{مثال}
%109 
%109
\ابتدا{سوال}
سوال کے ہر حصے میں ہر تفرق کو ظاہر کریں فلاں کی نسبت سے فلاں میں تبدیلی کی شرح میں اور اس کی طبعی اہمیت بیان کریں۔
\begin{enumerate}[a.]
\item
معلوم کریں \(    \frac{dh}{dx}    \) جبکہ \عددی{h} سطح سمندر سے بلندی، اور \عددی{x}  ، سیدھی سڑک پر طے کیا گیا  افقی فاصلہ ہے۔
\item
معلوم کریں\(      \frac{dN}{dt}   \) جبکہ \عددی{N} وقت\عددی{t} پر اسٹیڈیم کا گیٹ کھلنے کے بعد لوگوں کی تعداد ہے۔  
\item
معلوم کریں\(     \frac{dM}{dr}   \) جبکہ \عددی{M} مقناطیس سے فاصلے \عددی{r} پر مقناطیسی قوت ہے۔
\item
معلوم کریں \(    \frac{dv}{dt}   \) جبکہ \عددی{v} ایک زرے کی رفتار ہے جو  وقت \عددی{t} کے ساتھ ایک سیدھی لکیر میں حرکت کر رہا ہے۔
\item
معلوم کریں\(    \frac{dq}{dS}   \) جبکہ \عددی{q} گاڑی میں استعمال ہونے والے پیٹرول کی شرح ہے، اور \عددی{S} کلومیٹر فی گھنٹہ میں گاڑی کی رفتار ہے۔
\end{enumerate}
\انتہا{سوال}
\ابتدا{سوال}
درج ذیل تمام جملوں کو موزوں اکائیوں اور علامات کا استعمال کرتے ہوۓ متفرق کی شکل میں لکھیں۔
\begin{enumerate}
\item
سطح سمندر سے بلنری کی نسبت سے فضائ دباؤ میں تبدیلی کی شرح
\item
دن کے وقت کی نسبت سے درجہ حرارت میں تبدیلی کی شرح
\item
وقت کے ساتھ جوار میں بڑھنے کی شرح
\item
زندگی کے پہلے ہفتے میں بچے کے وزن میں اضافے کی شرح
\end{enumerate}
\انتہا{سوال}

\ابتدا{سوال}
\begin{enumerate}[a.]
\item
معلوم کریں\( \frac{dz}{dt} \)جبکہ\( z=3t^2+7t-5 \)
\item
معلوم کریں\( \frac{d\theta}{dx} \)جبکہ\( \theta =x-\sqrt{x} \)
\item
معلوم کریں\(  \frac{dx}{dy} \)جبکہ\(  x=y+\frac{3}{y^2}\)
\item
معلوم کریں\( \frac{dr}{dt} \)جبکہ\(  r=t^2+\frac{1}{\sqrt{t}}\)
\item
معلوم کریں\( \frac{dm}{dt} \)جبکہ\( m=(t+3)^2 \)
\item
معلوم کریں\( \frac{df}{ds} \)جبکہ\( f=2s^6-3s^2 \)
\item
معلوم کریں\( \frac{dw}{dt} \)جبکہ\( w=5t \)
\item
معلوم کریں\(  \frac{dR}{dr}\)جبکہ\( R=\frac{1-r^3}{r^2} \)
\end{enumerate}
\انتہا{سوال}

\ابتدا{سوال}
ایک ذرہ \عددی{x}-محور کے گرد حرکت کرتا ہے۔وقت \عددی{t} پر اس کی منتقلی\( x=6t-t^{2} \) ہے۔
\begin{enumerate}[a.]
\item
\( \frac{dx}{dt} \)
کیا ظاہر کرتا ہے؟
\item
  \عددی{x} بڑھ رہا ہے یا کم ہو رہا ہے؟ جب \(x=1\) اور \(x=4\) ہے؟

\item
ذرے کی سب سے بڑی مثبت منتقلی معلوم کریں۔ اور بتائیں کے کس طرح یہ آپ کے پہلے حصے کے جواب سے جڑا ہوا ہے؟
\end{enumerate}
\انتہا{سوال}
\ابتدا{سوال}

مندرجہ ذیل میں سے ہر ایک کو ریاضیاتی شکل میں ڈھالنے  کے لۓ مناسب علامت نویسی وضع کریں۔
\begin{enumerate}[a.]
\item
موٹروے پر طے کردہ فاصلہ مستقل شرح سے بڑھ رہا ہے۔
\item
سیونگ بینک ڈپازٹ میں اضافے کی شرح جمع کی گئی رقم کے متناسب ہے۔
\item
 درجہ حرارت کے تفاعل کے متناسب درخت کے تنے کا قطر بڑھتا ہے۔ 
\end{enumerate}
\انتہا{سوال}
%110
\ابتدا{سوال}
ایک گاڑی ہر ایک کلو میٹر کے لۓ \عددی{S} کلومیٹر فی گھٹہ کی رفتار پر چلتے ہوۓ \عددی{y} کلومیٹر فی لیٹر پیٹرول استعمال کرتی ہے۔  جبکہ \[ y=5+\frac{1}{5}S-\frac{1}{800}S^2 \]
وہ رفتار معلوم کریں جس کے لیے کار کم خرچ میں ذیادہ فاصلہ طے کرے۔
\انتہا{سوال}
\ابتدا{سوال}
ایک گیند عمودی طور پر اوپر کی طرف پھینکی گئی۔ وقت \عددی{t} پر اس کی بلندی \عددی{h} ہے اور ان دونوں کے بیچ کا تناسب اس مساوات\(h=20t-5t^2 \) سے ملتا ہے۔ گیند کی زمین سے اوپر زیادہ سے زیادہ بلندی معلوم کریں۔
\انتہا{سوال}
\ابتدا{سوال}
دو حقیقی اعداد \عددی{x} اور \عددی{y} کا مجموعہ \عددی{   12 } ہے۔ اس ضرب \( xy \) کی زیادہ سے زیادہ قیمت معلوم کریں۔
\انتہا{سوال}

\ابتدا{سوال}
دو حقیقی مثبت اعداد \عددی{x} اور \عددی{y} کا ضرب  \عددی{ 20 }  ہے۔ان کے جمع کی کم سے کم قیمت  معلوم کریں۔
\انتہا{سوال}

\ابتدا{سوال}
ایک سیلنڑر کے حجم کا کلیہ \(V=\pi r^{2}h \)  ہے، حجم  \عددی{V} کی سب سے بڑی اور سب سے چھوٹی قیمت معلوم کریں۔
\انتہا{سوال}

\ابتدا{سوال}
ایک رسی جو کہ \عددی{ 1 } سینٹی میٹر لمبی ہے، سے دائرے  بناۓ گۓ ہیں، اس مربعی دائرے میں مخالف سمتوں کے ایک جوڑے کی لمبائ \عددی{x} سینٹی میٹر ہے۔ اس   \عددی{x} کی قیمت معلوم کریں یہ خیال رکھتے ہوۓ کہ اس دائرے کا رقبہ بڑے سے بڑا ہوگا۔
\انتہا{سوال}
%question 12
\ابتدا{سوال}
بھیڑوں کے  ایک مستطیل باڑے کی ایک سمت میں رکاوٹ لگائ گئ ہے باقی کی تین سمتوں میں باڑ لگائ گئ ہے۔  مستطیل کی لمبائی \عددی{x} میٹر ہے؛ \عددی{ 120  }میٹر جنگلا  دستیاب ہے۔
\begin{enumerate}[a.]
\item
ظاہر کریں کہمستطیل  کا رقبہ\( \frac{1}{2}x(120-x) {m^2} \) ہے۔
\item
بھیڑ کے باڑے کا زیادہ سے زیادہ رقبہ معلوم کریں۔
\end{enumerate}
\انتہا{سوال}
%question 13
\ابتدا{سوال}
دھات کے ایک مستطیل ٹکڑے کی لمبائ 50 سینٹی میٹر اور چوڑائ  40 سینٹی میٹر ہے۔ ہر ایک کونے سے لمبائی  \عددی{x} سینٹی میٹر کے برابر مربع کاٹے اور پھینک دیے گۓ۔ اب شیٹ کو ایک تہ کر کے \عددی{x} سینٹی میٹر گہرائی کی ایک ٹرے بنائی گئی۔\عددی{x} کی ممکنہ قیمتوں کا دائرہ کار کیا ہے؟ ٹرے کی  سکت یا حجم  کو زیادہ سے زیادہ بنانے والی \عددی{x}  کی قیمت معلوم کریں۔
\انتہا{سوال}
%question 14
\ابتدا{سوال}
ایک مربع بنیاد کا استعمال کرتے ہوۓ ایک کھلا مستطیل بنانا ہے جس کا حجم 4000 سینٹی میٹر کیوب ہے۔ اس بنیاد کی ایک سمت کی لمبائ معلوم کریں جب مستطیل بنانے کے لیے درکار مواد کو کم سے کم استعمال میں لایا جاۓ۔
\انتہا{سوال}
%question 15
\ابتدا{سوال}
ایک بیلن کار ردی کی ٹوکری، جس کا رداس\عددی{r} سینٹی میٹر ہے اور سکت \عددی{V} سینٹی میٹر کیوب ہے۔ سطح کا رقبہ 5000  مربع سینٹی میٹر  ہے۔
\begin{enumerate}[a.]
\item
ثابت کریں کہ \( V=\frac{1}{2}r(5000-\pi r^2) \)
\item
ٹوکری کی زیادہ سے زیادہ   سکت معلوم کریں۔
\end{enumerate}

\انتہا{سوال}
%question 16
\ابتدا{سوال}
رداس 10 سینٹی میٹر کے کرہ کی اندر ایک گعل اسطوانہ پڑا ہوا ہے۔  اسطوانہ کے زیادہ سے زیادہ حجم کا حساب لگائیں۔
\انتہا{سوال}

%miscellanous exercise 7

\ابتدا{سوال}
تفرق کا استعمال کرتے ہوۓ وکر پر غیر متحرک نقطوں کے محدد  تلاش کریں۔
\[ y=x+\frac{4}{x} \]
اور  دریافت کریں کہ ہر غیر متحرک نقطہ زیادہ سے زیادہ نقطہ ہے یا کم سے کم نقطہ ہے۔ \عددی{x} کے اقدار کا مجموعہ تلاش کریں جس کے لئے \عددی{y}  بڑھتا ہے جیسے  جیسے \عددی{x} کی قیمت بڑھتی ہے۔
\انتہا{سوال}

%page 111
\ابتدا{سوال}
ایک تابکار مادہ کے سڑنے کی شرح، اس وقت تک بچے ہوئے مادہ کے متناسب سمجھا جاتا ہے۔ اگر وقت \عددی{t} پر، بچا ہوا مادہ \عددی{m} ہے، تو اس کا مطلب ہے کہ \عددی{m} اور \عددی{t} مساوات کو پورا کرتے ہیں۔
\[ \frac{dm}{dt}=-km \]
جبکہ \عددی{k} ایک مثبت مستقل ہے۔(منفی نشان مادہ کے گھٹنے کی نشاندہی کرتا ہے۔)
اسی طرح کی مساوات بنائیں جو مندرجہ ذیل بیانات کی نمائندگی کرتی ہوں۔
\begin{enumerate}[a.]
\item
بیکٹیریا کی آبادی نڑھنے کی شرح، بیکٹیریاکی موجودہ آبادی میں موجود بیکٹیریا کی تعداد \عددی{،n} کے متناسب ہے
\item
جب گرم سوپ کا ایک پیالہ فریزر میں رکھا جاۓ تو ،درجہ حرارت، \(\theta^0 C\) کے گھٹنے کی شرح موجودہ درجہ حرارت کے متناسب ہے۔
\item
ایک کافی  کپ کا درجہ حرارت  \(\theta^0 C\) گھٹنے کی شرح، کمرے کے درجہ حرارت اور اس کافی کے پیالے کے درجہ حرارت میں فرق کے ساتھ  متناسب ہے ۔
\end{enumerate}
\انتہا{سوال}
%question 3
\ابتدا{سوال}
ایک گاڑی نے ایک ٹرک کو تیز رفتاری سے پیچھے چھوڑا۔ اس کی ابتدائی رفتار \عددی{u} ہے، اور وقت \عددی{t} پر جب اس کار نے رفتار بڑھانا شروع کی تو  \عددی{x} فاصلہ طے کیا ہے، جبکہ \( x=ut+kt^2 \)
تفرق کا استعمال کرتے ہوۓ دکھائیں کہ گاڈی کی رفتار\( x=ut+kt^2 \)ہے، اور یہ ظاہر کریں کہ اس کی تیز رفتاری مستقل ہے۔
\انتہا{سوال}
%question 4   left here.
\ابتدا{سوال}
جب ڈرائیور نے گاڑی کے بریک لگاۓ تو گاڑی \( 20ms^{-1}\) کی رفتار سے چل رہی تھی۔ بریک لگانے کے \عددی{t} سیکنڈز بعد گاڑی مزید \عددی{x} میٹرز کا فاصلہ طے کر چکی تھی۔جبکہ \( x=20t-20t^2 \)۔ تفرق کی مدد سے اس وقت کار کی رفتار اور اسراع معلوم کریں۔یہ بھی معلوم کریں کہ یہ کلیے کب تک لاگو ہوں گے؟
\انتہا{سوال}
 %question 5
\ابتدا{سوال}
ایک لڑکا ایک پہاڑ کی \عددی{60} میٹر اونچی چوٹی پر کھڑا ہے۔ وہ سیدھا اوپر کی جانب ایک پتھر پھینکتا ہے ، کہ اس پتھر کا فاصلہ \عددی{ h }  میٹر ، اس چٹان کی چوٹی سے اس مساوات\( h=20t-5t^2 \) کی مدد  سے معلوم کیا جا سکتا ہے۔
 
\begin{enumerate}[a.]
\item
 پہاڑ کے اوپر پتھڑ کی زیادہ سے زیادہ اونچائی معلوم کریں۔
\item
پتھر لڑکے اور پہاڑ کی چوٹی سے تھوڑا سا چوک کر چوٹی سے نیچے کر جاتا ہے، معلوم کریں وہ وقت کہ جب پتھر ساحل سے ٹکراۓ گا۔
\item
وہ رفتار معلوم کریں جس سے پتھر ساحل سمندر سے ٹکرایا۔
\end{enumerate}
\انتہا{سوال}
%question 6

\ابتدا{سوال}
مساوات \( x^2+y^2 \) کی کم سے کم قیمت معلوم کریں جب  \(x+y=10 \) ہو۔
\انتہا{سوال}
%question 7

\ابتدا{سوال}
ایک قائمہ زاویہ مثلث کی دو چھوٹی اطراف کی لمبائ کا  مجموعہ  \عددی{ 18 } سینٹی میٹر ہے، معلوم کریں؛
\begin{enumerate}[a.]
\item
ڈھلوان کی کم سے کم لمبائی۔
\item
مثلث کا زیادہ سے زیادہ ممکنہ رقبہ
\end{enumerate}
\انتہا{سوال}

%question 8
\ابتدا{سوال}
\begin{enumerate}[a.]
\item
مساوات\( y=12x+3x^2-2x^3 \) کے خاکہ پر مستقل نقطے تلاش کریں اور خاکہ بنائیں۔
\item 
یہ خاکہ کیسے دیکھاۓ گا کہ مساوات\(12x+3x^2-2x^3 =0\) کے صرف تین  حقیقی حل ہیں۔
\item
اپنے خاکے کو استعمال کرتے ہوۓ واضع کریں کہ اس مساوات\( 12x+3x^2-2x^3=-5 \)کے بھی صرف تین حقیقی حل ہیں۔
\item
مساوات \( 12x+3x^2-2x^3=k\)کے حل  \عددی{ k } کی کن قیمتوں کے لیے درج ذیل شرائط پوری کریں گے؟
\begin{enumerate}[a.]
\item
بلکل تین حقیقی حل؟
\item
صرف ایک  حقیقی حل؟
\end{enumerate}
\end{enumerate}
\انتہا{سوال}
%question 9 
 \ابتدا{سوال}
مساوات \( y=x^3-12x-12 \) کے خم کے ساکن نقاط معلوم کریں، نیز خم بھی بنائیں۔

\عددی{ k }  کی وہ قیمتیں معلوم کریں کہ جن کے لیے مساوات  \( x^3-12x-12=k \)کے ایک سے ذیادہ حقیقی حل ملیں۔
 \انتہا{سوال}
%question 10
 \ابتدا{سوال}
مساوات  \(   y=x^{3}-12x-12\) کے خم پے موجود ساکن نقاط معلوم کریں، خم بنائیں اور \عددی{ k }  کی وہ تمام قیمتیں معلوم کریں جن کے لیے مساوات \(x^3+x^2=k \) کے تین حقیقی حل موجود ہوں۔
 \انتہا{سوال}
%question 11
\ابتدا{سوال}
مساوات \( y=3x^4-4x^3-12x^2+10 \) کے خم کے ساکن نقاط معلوم کریں ، خم بنائیں،  \عددی{ k }   کی کن قیمتوں کے لیے مساوات  \( 3x^4-4x^3-12x^2+10=k \) کے ؛
\begin{enumerate}[a.]
\item   چار حقیقی حل ہوں گے
\item  دو حقیقی حل ہوں گے
\end{enumerate}
 \انتہا{سوال}
%question12
 \ابتدا{سوال}
مساوات \( y=x(x-1)^2 \) کے خم کے ساکن نقاط کے محدد معلوم کریں ، خم بنائیں۔  \عددی{ k } کی حقیقی قیمتوں کا ایک سیٹ بنائیں جن کے لیے مساوات  \( x(x-1)^2=k \) کا صرف ایک حقیقی حل ہو۔
 \انتہا{سوال}
%question13
\ابتدا{سوال}
کسی جسم کا درمیانی حصہ ایک چوتھای دائرہ ہے، اور جیساک کہ شکل میں دکھایا گیا ہے اسکا رداس \عددی{ r } ہے اور یہ چوتھائ دائرہ ایک مستطیل جسکی لمبائ  \عددی{x} اور اونچائ \عددی{ r } ہے، سے جڑا ہوا ہے۔
\begin{enumerate}[a.]
\item  
 اس حصے کی باہری دیوار  \عددی{ P }اور رقبہ \عددی{ A} ہیں۔ ان دونوں کو \عددی{r }اور   \عددی{ x} کی نسبت سے لکھیں، اور یہ بھی ثابت کریں کہ \(A=\frac{1}{2}Pr-r^2 \)۔
\item  
فرض کریں کہ باہر دیوار کی لمبائ مستقل ہے،  \عددی{ x} معلوم کریں \عددی{r }کی نسبت سے، ایسی صورتحال کے لیے کہ جب رقبہ  \عددی{ A}ذیادہ سے ذیادہ ہو۔ اور ثابت کریں کہ \عددی{x}  کی اس قیمت کے لیے  \عددی{ A}ذیادہ سے ذیادہ ہے نہ کہ کم سے کم۔
 \end{enumerate}
 \انتہا{سوال}
%question14
\ابتدا{سوال}
ایک خم کی مساوات \( y=\frac{1}{x}-\frac{1}{x^2} \) ہے۔ متفرق کی مدد سے اس خم کے ساکن نقاط کے محدد معلوم کریں اور یہ بھی معلوم کریں کہ ساکن نقاط ذیادہ سے ذیادہ قیمت کے حامل نقاط ہیں یا کم سے کم قیمت کے حامل نقاط ہیں۔ اسی خم کے نقاط حاصل کریں وگرنہ ذیل میں دیے گۓ دونوں خموں کے ساکن نقاط کے محدد معلوم کریں۔
\begin{enumerate}[a.]
\item \(    y=\frac{1}{x}-\frac{1}{x^2} +5 \)
\item \(  y=\frac{2}{x-1}-\frac{2}{(x-1)^2}  \)
 \end{enumerate}
 \انتہا{سوال}
%question15
\ابتدا{سوال}
ایک سوپر مارکیٹ کا مینیجر اکثر اوقات \عددی{20}فیصد منافع رکھتا ہے ان تمام اشیاء پر جو کہ وہ بیچتا ہے۔ وہ یہ تسلیم کرتا ہے کہ \عددی{ F} اسکے پکے گاہک ہیں اور اگر وہ اپنا منافع  \عددی{x} فیصد تک لے آۓ تو وہ \( k(20-x) \)مزید گاہکوں کی توجہ اپنی طرف کر سکے گا۔ ہر ہفتے خریدار جو اس سے سامان خرئدتے ہیں انکی تھوک کے حساب سے قیمت\عددی{ A} یورو ہے۔ ثابت کریں کہ منافع \عددی{x} فیصد ہونے پر مینیجر کو ہفتہ وار منافع  \( \frac{1}{100}Ax((F+20k)-kx) \) یورو ہوگا۔
یہ بھی ثابت کریں کہ اگر چہ مینیجر اپنا منافع \عددی{20} فیصد سے کم کر لے وہ منافع کما سکتا ہے، یہ بات ذہن میں رکھتے ہوۓ کہ اسکے پاس گاہکوں کا اضافہ ہوگا۔
 \انتہا{سوال}
%question16
\ابتدا{سوال}
ایک کمپنی جو کہ چڑھائ چڑھنے والے جوتے بناتی ہے اسکے دو طرح کے اخراجات ہیں۔
مستقل اخراجات، (پودوں، قیمتوں اور دفتر کے اخراجات) \عددی{2000} یورو فی ہفتہ۔
مصنوعات بنانے کی لاگت، (مواد اور مزدوروں پر آنے والی لاگت)\عددی{20} یورو فی جوڑا۔
مارکیٹ پر کی گئ تحقیق یہ بتاتی ہے کہ اگر \عددی{30} یورو فی جوتا بیچا جاۓ تو ہفتے میں 500 جوڑے جوتے کے بکیں گے۔ لیکن  \عددی{55}    یوروں میں ایک جوتا بھی نہیں بکے گا۔ اور ان کے درمیان بنائ گئ ترسیم جو کہ بکری اور قیمت کے مابین ہے وہ ایک سیدھی لکیر ہے۔
اگر کمپنی والے ایک جوڑے کی قیمت \عددی{x} یورو لگا دیں ، تو درض ذیل کے لیے مساوات معلوم کریں
\begin{enumerate}[a.]
\item  ہفتہ وار بکری
\item  ہفتہ وار رسیدیں 
\item  ہفتہ وار لاگت
\end{enumerate}

یہ بھی ثابت کریں کہ ہفتہ وار منافع اس مساوات \( P=-20x^2+1500x-24000 \)سے حاصل کیا جا سکتا ہے اور وہ قیمت بھی ملوم کریں کہ جس پے بوٹ بیچنے سے ذیادہ سے ذیادہ منافع ہوگا۔
 \انتہا{سوال}
%question17
\ابتدا{سوال}
ایک جفت تفاعل کا خم بنائیں جسکا ہر نقطے پر تفرق لینا ممکن ہے۔
فرض کریں اسی خم پے  \عددی{P} کوئ نقطہ ہے جبکہ $ x=p $ بشرطیکہ  $p>0 $، اسی خم پے نقطہ \عددی{P}  پر ایک خط مماس بنائیں۔ نقطہ  \(P\prime\)پر بھی خط مماس بنائیں جس کے لیے  $x=-p$ 

\begin{enumerate}[a.]
\item  
نقطہ  \عددی{P}  اور  \(P\prime\)کی ڈھلوانوں میں کیا تعلق ہے۔ ان دونوں نقاط کے متفرق    $f'(p)$    اور   $f(p)$میں باہمی تعلق بھی معلوم کریں۔ یہ تعلق آپکو ایک جفت تفاعل کے متفرق کے بارے یں کیا تفصیلات فراہم کر رہا ہے۔
\item 
ثابت کریں کہ کسی بھی تاک تفاعل کا متفرق جفت ہوتا ہے۔
\end{enumerate}
 \انتہا{سوال}
