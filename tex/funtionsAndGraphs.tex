\باب{تفاعل اور خم}\شناخت{باب_تفاعل_اور_ترسیمات}
%page32
اس سبق میں تفاعل کی بات کی گئ ہے اور ان تفاعل سے جڑے ترسیمات کا جائزہ لیا گیا ہے۔ اس سبق کو مکمل کر چکنے کے بعد آپ اس قابل ہونے چاہئیں کہ ؛
\begin{itemize} 
\item 
تفاعل کو لکھنے کے طریقوں سے آگاہ ہوں ، اور سعت اور میدان عمل سے آگاہی حاصل کریں۔
\item 
\عددی{x} کی طاقتوں کے لیے ترسیمات کی شکلوں سے آگاہ ہوں۔
\item  
مساوات \(f(x)=ax^{2}+bx+c\) کے ترسیمات کی شکلیں سمجھ سکیں۔
\item  
ایسے تفاعل اور ترسیمات کی مساوات بنانے کے قابل ہو جائیں۔
\item  
یہ سیکھ سکیں کہ اجزاء کے استعمال سے ترسیمات کیسے بناۓ جاتے ہیں۔
\item  
دو تفاعل کے مشترک نقاط معلوم کر سکیں۔
\end{itemize}

اگر آپ کے پاس ایک ترسیم بنانے والا اعداد ہے یا آپ کے کمپیوٹر میں کوئ ایسا پروگرام ہے جس کی مدد سے آپ ترسیم بنا سکیں تو اپ ذاتی طور پر تقابل کر سکتے ہیں اپنے پپاس بننے والے ترسیمات کا کتاب میں بناۓ گۓ ترسیمات سے۔

 \حصہ{ایک تفاعل کی تعریف}
آپ کلیوں سے پہلے ہی آگاہ ہیں جو کہ حساب کتاب کو بہت کم کر دیتے ہیں اور جن کو اکثر استعمال کرنا پڑتا ہے ۔ مثال کے طور پر؛

ایک دائرہ جسکا رداس \عددی{x} ہوگا اسکا حدود اربعہ \(\pi x^{2}\)مربع میٹر ہوگا۔\\ ایک مکعب جس کی ایک طرف کی لمبائ \عددی{x} میٹر ہے، اسکا حدود اربعہ \( x^{3} \) مکعب میٹر ہوگا۔\\ اگر ہم \عددی{x }کلومیٹر فی گھنٹہ کی رفتار سے سفر کر رہے ہوں گے تو،  \عددی{ k} کلومیٹر فاصلے طے کرنے کے لیے، \(\frac{k}{x}\) گھنٹے کا وقت درکار ہوگا۔

اکثر آپ نے ان کلیوں میں \عددی{x } کے بجاۓ دیگر حروف بھی دیکھے ہوں گے جیسے کہ رداس کے لیے \عددی{ r}، رفتار کے لیے \عددی{ s}لیکن اس سبق میں ہم ہمیشہ \عددی{x } ہی استعمال کریں گے ان کلیوں میں۔ اور جو مقدار آپ کو معلوم کرنا ہوگی اس کے لیے  ہم \عددی{y} استعمال کریں گے۔ ذرا غور کرین کہ ان کلیوں میں بعض اوقات  دیگر حروف یا علامات بھی ہوتی ہیں، انہیں ہم مستقل اعداد کہتے ہیں، یہ یا تو \( \pi \) کی طرح کا کوئ عدد ہو سکتی ہیں جو کہ غیر ناطق ہوں اور آپ انکی پوری عددی قیمت نہیں لکھ سکتے ، لیکن ان کی قیمت ایک مستقل ہے جسے بدلا نہیں جا سکتا۔ یا تو کوئ ایسی مقدار ہو سکتی ہے جسے آپ خود بیان کریں گے جیسے کہ فاصلہ \عددی{ k}، جس کی قیمت آپ نے خود سے طے کرنی ہے کہ آپ نے کتنا فاصلہ طے کیا۔

ریاضیاتی بیانیے جیسا کہ \( \pi x^{2}\)، \( x^{3} \) اور \( \frac{k}{x} \) انہیں ہم \عددی{x } کے تفاعل کہہ سکتے ہیں۔ \عددی{x } کی ایک قیمت، جو بھی آپ چنیں، کے لیے آپ کو \عددی{y} کی ایک قیمت ملے گی۔

عموماً تفاعل لکھنے کے طریقوں پر ذیادہ توجہ دی جاتی ہے ، بجاۓ چند مخصوص تفاعل کی مثال دینے کے، تفاعل کے لیے جو علامے استعمال کی جاتی ہے وہ ہے \(f(x) \)، جسے یوں پڑھا جاۓ گا (ایف آف ایکس یا بعض اوقات صرف ایف ایکس کہنا بھی کافی ہوگا۔) ایف \عددی{ f} تفاعل کے خود کے لیے استعمال ہوتا ہے جبکہ \عددی{x } وہ مقدار ہے جس کے لیے تفاعل کا استعمال کرنا ہے۔
 
اگر آپ \عددی{x } کی کسی قیمت کے لیے تفاعل کی کسی قیمت کی طرف اشارہ کرنا چاہتے ہیں ، کہہ لیں کہ \( X=2          \) کے لیے تفاعل کی قیمت کیا ہوگی تو آپ اسے \(f(2) \) ایسے لکھیں گے۔ اور اگر آپکا تفاعل \( F(X) =X^{3}         \) ہے تو آپ کہہ سکتے ہیں کہ \(f(2)=2^{3}=8\)
 %PAGE 33
اگر کسی مثال میں ایک سے ذیادہ تفاعل استعمال ہو رہے ہوں تو آپ ان تمام تفاعل کے لیے مختلف حروف استعمال کر سکتے ہیں۔ جیسا کہ \( f(x)         \) اور \(g(x)  \)۔

یہ لازمی نہیں کہ ہمیشہ ایک الجبرائ کلیہ ہی تفاعل کو بیان کرے، بعض اوقات ایک جملہ ،  ایک عملی خاکہ   یا ایک کمپیوٹر کا پروگرام  ایک تفاعل کو ذیادہ بہتر طور پر   کر سکتا ہے۔ آپ تفاعل کو کیسے بھی بیان کریں بس اس بات کا خیال رکھیں کہ \عددی{x } کی ایک قیمت کے لیے \عددی{y} کی ایک ہی قیمت اور وہ بھی نایاب قیمت آۓ۔
 \حصہ{ترسیم، عملی میدان اور سعت}
آپ جانتے ہیں کہ ترسیم کیسے بناۓ جاتے ہیں۔ آپ کارتییسی نظام محدد کے لیے محدد چنتے ہیں۔ اور پھر افقی اور عمودی محوروں کے لیے پیمانے بناتے ہیں۔

یہ محور پرچے یا کاغذ کو چار حصوں میں تقسیم کر ددیتے ہیں، جیسا کہ شکل \حوالہ{شکل3.1} میں دکھایا گیا ہے۔ پہلے خانے میں \عددی{x } اور \عددی{y} دونوں مثبت ہیں ،دوسرے خانے میں \عددی{x } منفی اور \عددی{y}   مثبت  ہے،   تیسرے خانے میں \عددی{x } اور \عددی{y} دونوں منفی ہیں جبکہ چوتھے خانے میں \عددی{x } مثبت اور \عددی{y}  منفی  ہے۔

%example3.2.1
 \ابتدا{مثال}
کس خانے میں \(xy>0\)؟

اگر دو اعداد کا ضرب مثبت ہے تو اسکا مطلب یہ ہوا کہ یا تو دونوں اعداد مثبت ہیں یا دونوں منفی ہیں، لہٰذہ ہم کہہ سکتے ہیں کہ \(    x>0  \) اور \(   y>0 \) یا \( x<0 \) اور \( y<0 \)، لہٰذہ ہم وثوق سے کہہ سکتے ہیں کہ نقطہ \( (x,y)\) یا تو پہلے خانے میں ہے یا تیسرے خانے میں۔
 \انتہا{مثال}

اکثر \عددی{y}-محور کو عمودی محور کہہ دیا جاتا ہے اور اسی طرح \عددی{x }-محور کو افقی محور کہہ دیا جاتا ہے۔ لیکن اگر آپ ترسیم ایک افقی سطح جیسا کہ ایک میز یا ایک کاغذ پر بنا رہے ہیں تو اوپر بتایا  گیا بیان سو فیصد درست نہیں ہوگا۔

ایک تفاعل \(f(x)  \) کی ترسیم بنانے میں ان تمام نقطوں کو استعمال کیا جاتا ہے جنکے محدد \( (x,y)\)، ہماری مساوات \(y=f(x)  \) کو درست ثابت کریں۔ جب آپ ہاتھ سے ترسیم بنا رہے ہوتے ہیں تو آپ   \عددی{x } کی قیمتیں چنتے ہین اور انکے لیے \(y=f(x)  \) کو حل کرتے ہیں۔ اور پھر آپ ان نقطوں پر نشان لگا دیتے ہیں جن کے محدد \( (x,y)\) ہونگے اور ان نقطوں کو جوڑنے سے آپ کا ترسیم بن جاۓ گا۔ اگر آپ نے یہ صحیح سے کیا ہے تو اس ترسیم پر موجود دوسرے نقطوں کے محدد بھی اس مساوات کو درست ثابت کریں گے جس کے لیے آپ نے ترسیم بنایا ہے۔ کمپیوٹر اور ترسیمی اعداد بھی اسی طرح سے ہی ترسیم بناتے ہیں لیکن وہ  ذیادو نقطوں کی مدد سے کم وقت میں ترسیم بنانے کے قابل ہوتے ہیں۔

چند مساوات جیسے کہ \( y=mx+c\) اور \(y=x^{2}\) کے ترسیمات پوری طرح سے نہیں بناۓ جا سکتے ، آپ پیمانہ جتنا بھی چھوٹا کر لیں اور کاغذ جتنا بھی بڑا کر لیں ترسیم کناروں سے باہر نکلے گا۔ اور ایسا اس لیے ہو رہا ہے کہ ایسے ترسیماات میں \عددی{x } کوئ بھی حقیقی عدد ہو سکتا ہے جتنا برا آپ سوچ سکیں دونوں مثبت اور منفی محورووں میں تو ایسے ترسیم کو کاغذ میں مقید کر پانا بہت مشکل ہے، بلکہ نا ممکن ہے۔ اب ایسے ترسیمات بنانے ہوں تو آپکی مہارت کا سوال ہے کیونکہ آپ نے \عددی{x } کی وہ قیمتییں چنننی ہیں کہ ترسیم کے ساارے نمایاں خدوخال واضع ہو جائیں۔

آپ کا چند ایسے تفاعل سے بھی پالا پڑا ہوگا جو کہ تمام حقیقی اعداد پر معین نہیں ہیں۔مچال کے طور پر \(\frac{1}{x}\)، جسکا کوئ مطلب نہیں ہے جب  \عددی{x } صفر ہوگا، اسی طرح سے \(\sqrt{x}\) جسکا کوئ مطلب نہیں ہے جب \عددی{x } منفی ہوگا۔
%PAGE 34
 ہم ایک اور مثال دیکھیں گے اور اس مثال میں \عددی{x } کی قیمت پر پابندی بھی لگائ گئ ہے۔
  \ابتدا{مثال}
مساوات \(f(x)=\sqrt{4-x^{2}}\) کا مکمل ترسیم بنائیں۔

آپ اس تفاعل کی جوابی قیمتیں معلوم کر سکتے ہیں بشرطیکہ \عددی{x } کی قیمت \عددی{-2} اور \عددی{2} کے درمیان ہو۔ اگر  \(x>2 \) یا  \(x<-2 \) تو تفاعل  \(4-x^{2} \) کی ایک منفی قیمت آۓ گی اور منفی قیمت کا جذر نہیں لیا جا سکتا۔

 تفاعل \(y=f(x)\) کبھی بھی منفی نہیں ہو سکتا (جزر ہمیشہ مثبت ہوتے ہیں یا صفر ہوتے ہیں) اور یہ \عددی{2} سے بڑا بھی نہیں ہو سکتا۔ لہٰذہ تفاعل \(f(x)=\sqrt{4-x^{2}}\) کا ترسیم ،جیسا کہ شکل \حوالہ{شکل3.2} میں دکھایا گیا ہے، افقی محور پے \عددی{2} اور \عددی{-2} کے درمیان ہوگا جبکہ عمودی محور میں اسکی بلندی صفر سے لے کر \عددی{ 2} تک ہوگی۔
 \انتہا{مثال}

اگر آپ ایسا تفاعل بھی استعمال کر رہے ہوں جو کہ تمام حقیققی \عددی{x } کے لیے معین ہو ، آپ اس میں صرف اسی وقت دلچسپی لیں گے جب \عددی{x } پر کسی قیم کی پابندی لگی ہوئ ہو۔ مثال کے طور پر  ایک مکعب کے حجم کا کلیہ \(V=x^{3}\) ہے، اگرچہ آپ \عددی{x } کی قیمت کسی بھی حقیقی عدد کے لیے معلوم کر سکتے ہیں لیکن ایسا آپ صرف\عددی{x } کی مثبت  قیمتوں کے لیے کرتے ہیں۔

اب ہم ایک ایسی مثال حل کرتے ہیں جس میں \عددی{x } کو ایک متناہی فاصلے کے لیے پابند کیا گیا ہے۔
 
 
 \ابتدا{مثال}
ایک تار جسکی لمبائ  \عددی{ 4} میٹر ہے اسے دو حصوں میں کاٹا گیا ہے ، اور ہر حصے کو موڑ کر ایک مربع بنایا گیا ہے، اس تار کو کسطرح سے کاٹا جاۓ کہ دونوں مربع بیقوقت ،

\begin{enumerate}[a.]
\item 
کم سے کم حدود اربعہ رکھتے ہوں
\item 
ذیادہ سے ذیادہ حدود اربعہ رکھتے ہوں
\end{enumerate}

فرض کریں دونوں ٹکڑوں کی لمبائ \عددی{x }میٹر اور \(    (4-x)\) میٹر ہے۔ اگر لمبائیاں یہ فرض کر لی جائیں تو دونوں مربعوں کا حدود اربعہ ، جیسا کہ شکل \حوالہ{شکل3.3} میں دکھایا گیا ہے، \(\big(\frac{1}{4}x\big)^{2}\) اور \(\big(\frac{1}{4}(4-x)\big)^{2}\) ہوگا۔ لہٰذہ ککل حدود اربعہ جسے \عددی{y} سے ظاہر کیا جاۓ گا ہوگا؛ 
\[y=\frac{1}{16}(x^{2}+(16-8x+x^{2}))=\frac{1}{8}(x^{2}-4x+8)\]
 غور کریں کہ کیونکہ \((x-2)^{2}=x^{2}-4x+4\)، ہم یہ بھی کہہ سکتے ہیں کہ \(y=\frac{1}{8}\big((x-2)^{2}+4\big)\)۔
آپ اس ریاجیاتی بیانیے کی قیمت کسی بھی \عددی{x } کے لیے معلوم کر سکتے ہیں ، لیکن اگر مسئلہ کو سمجھیں تو ہمیں صرف \(0<x<4\) کے درمیان کی کسی قیمت کے لیے بیانیے کو حل کرنا ہے۔
شکل \حوالہ{شکل3.4} میں اسی کا ایک ترسیم بنایا گیا ہے، کیونکہ \((x-2)^{2}\ge0\)، حدود اربعہ کم سے کم ہے جب \(x=2\) ہے اور اس نقطے پر حدود اربعہ \عددی{ 0.5} مربع میٹر ہے۔

ترسیم دیکھنے سے اییسا معلوم ہوتا ہے کہ ذیادہ سے ذیادہ حدود اربعہ یعنی \عددی{1} مربع میٹر اس وقت ملے گا جب \(x=0\) یا \(x=4\) ہو ، لیکن ہم ان قیمتوں کو اس لیے اہمیت نہیں دے رہے کہ سوالل میں بتایا گیا تھا تار کو دو ٹکروں میں تقسیم کرنے کا اگر ہم \عددی{x } کی یہ قییمتیں لیں تو تار کٹی ہی نہیں۔ اس لیے ذیادہ سے ذیادہ حدود اربعہ کے مربع بنا ہی نہیں، یا ہم یوں بھی کہہ سکتے ہیں کہ ذیادہ سے ذیادہ حدود اربعہ کا مربع کا حدود اربعہ ہم معلوم نہیں کر سکتے۔
 \انتہا{مثال}
%PAGE 35
 ایک تفاعل \عددی{x } کی کسی بھی قیمت کے لیے معین کیوں نہیں ہوگاا اسکی دو وجوہات ہو سکتی ہیں۔
\begin{enumerate}[a.]
\item  
تفاعل \( f(x)  \) کو بیان کرنے والا ببانیہ صرف ایک یا چند \عددی{x } کی قیمتوں کے لیے کوئ معنی رکھتا ہوگاا یا معین ہوگا۔
\item  
جن معنی میں یہ تفاعل لیا جا رہا ہے ان میں صرف  چند \عددی{x } ہی کام کے ہوں گے
\end{enumerate}

تفاعل \عددی{x } کی جن حقیقی قیمتوں کے لیے معین ہوتا ہے اور کوئ جواب بھی دیتا ہے ان تمام قیمتوں کو تفاعل کا میدان عمل کہا جاتا ہے۔ مثلاً مثال \حوالہ{مثال3.2.2} میں تفاعل کا میدان عمل \(-2\le x\le 2\) تھا، اور اسی طرح مثال \حوالہ{مثال3.2.3} میں تفاعل کا میدان عمل \(0<x<4\)تھا۔ تفاعل \(\frac{1}{x}\) کا وسیع ترین میدان عمل تمام حقیقی اعداد ہیں سواۓ صفر کے، لیکن اگر اس تفاعل کو کسی عام مسئلے میں استعمال کیاا جا رہا ہوگا تو آپ اسکا قدرے چھوٹا میدان عمل استعمال کریں گے جیسے کہ تمام مثبت حقیقی اعداد۔


ایک بار آپ نے ایک تفاعل \( f(x)  \) کا میدان عمل طے کر لیا تو اب نیا سوال یہ اٹھتا ہے کہ میدان عممل کی قیمتوں کے جواب میں کونسی قیمتیں آئیں گی؟ کسی تفاعل ککے میدان عمل کی جوابی قیمتوں کو اس تفاعل کی سعت کہا جاتا ہے۔  مثلاً مثال \حوالہ{مثال3.2.2} میں تفاعل کی سعت \(0\le y\le 2 \) تھی، اور اسی طرح مثال \حوالہ{مثال3.2.3} میں تفاعل کی سعت \(\frac{1}{2}\le y\le 1\) تھی۔ غور کریں کہ قیمت \(y=\frac{1}{2}\) آتی ہے جب \(x=2\) ہو، لیکن قیمت \(y=1\) نہیں حاصل کی جا سکتی اگر \(0<x<4\)۔
 تفاعل \(f(x)=\frac{1}{x}\) جسکا میدان عمل تمام حقیقی اعداد ہیں سواۓ صفر کے تو اسکی سعت بھی تمام حقیقی اعداد ہیں سواۓ صفر کے۔

%exercise 3A
    \ابتدا{سوال}
ہمیں بتایا گیا ہے کہ تفاعل \(f(x)=2x+5,\) ہے، درجذیل کی قیمتیں معلوم کریں،
\begin{multicols}{2}
\begin{enumerate}[a.]
\item \( f(3) \)
\item \( f(0) \)
\item \(  f(-4)\)
\item \(f(-2\frac{1}{2})  \)
 \end{enumerate}
\end{multicols}
\انتہا{سوال}
\ابتدا{سوال}
 ہمیں بتایا گیا ہے کہ تفاعل \(f(x)=3x^{2}+2\) ہے، درجذیل کی قیمتیں معلوم کریں،
\begin{multicols}{2}
\begin{enumerate}[a.]
\item \( f(4) \)
\item \( f(\pm 1) \)
\item \(  f(\pm 3) \)
\item \(f(3)  \)
\end{enumerate}
\end{multicols}
 \انتہا{سوال}
\ابتدا{سوال}
 ہمیں بتایا گیا ہے کہ تفاعل\(f(x)=x^{2}+4x+3\) ہے، درجذیل کی قیمتیں معلوم کریں،
\begin{multicols}{2}
\begin{enumerate}[a.]
\item \( f(2) \)
\item \( f(\frac{1}{2}) \)
\item \( f(\pm 1) \)
\item \(f(\pm 3)  \)
\end{enumerate}
\end{multicols}
 \انتہا{سوال} 
\ابتدا{سوال} 
ہمیں بتایا گیا ہے کہ \(   g(x)=x^{3} \) اور \(   h(x)=4x+1 \) ہے۔ درجذیل معلوم کریں۔
\begin{multicols}{2}
\begin{enumerate}[a.]
\item \( g(2)+h(2)  \)
\item \(3g(-1)-4h(-1)   \)
\item \(g(5)=h(31)  \)
\item \( h(g(2))  \)
\end{enumerate}
\end{multicols}
 \انتہا{سوال}
\ابتدا{سوال}
ہمیں بتایا گیا ہے کہ \(f(x)=x^{n}\) اور \(f(3)=81\) ، \عددی{n} کی قیمت معلوم کریں۔
\انتہا{سوال}
\ابتدا{سوال}
ہمیں بتایا گیا ہے کہ \(f(x)=ax+b\) اور \(f(2)=7\)اور \(f(3)=12\) ،\عددی{a} اور \عددی{b} کی قیمت معلوم کریں۔
 \انتہا{سوال}
\ابتدا{سوال}
درج ذیل تفاععل کا وسیع ترین میدان عمل بیان کریں۔
\begin{multicols}{4}
\begin{enumerate}[a.]
\item \( \sqrt{x} \)
\item \( \sqrt{-x} \)
\item \(\sqrt{x-4}  \)
\item \( \sqrt{4-x} \)
\item \( \sqrt{x(x-4)} \)
\item \( \sqrt{2x(x-4)} \)
\item \( \sqrt{x^{2}-7x+12} \)
\item \( \sqrt{x^{3}-8}\) 
\item \(\frac{1}{x-2}  \)
\item \( \frac{1}{\sqrt{x-2}} \)
\item \(\frac{1}{1+\sqrt{x}}  \)
\item \( \frac{1}{(x-1)(x-2)} \)
\end{enumerate}
\end{multicols}
\انتہا{سوال}
\ابتدا{سوال}
 درج ذیل تفاعل کا میدام عمل تمام حقیقی اعداد ہیں، انکی سعت معلوم کریں۔
\begin{multicols}{3}
\begin{enumerate}[a.]
\item \( f(x)=2x+7 \)
\item \( f(x)=-5x \)
\item \(f(x)=3x-1  \)
\item \( f(x)=x^{2}-1 \)
\item \( f(x)=(x+2)^{2}-1 \)
\item \( f(x)=(x-1)^{2}+2 \)
\end{enumerate}
\end{multicols}
 \انتہا{سوال}
%PAGE 36
\ابتدا{سوال}
درج ذیل تمام تفاعل کی سعت معلوم کریں، تمام تفاعل \عددی{x }کی تمام حقیقی قیمتوں کے لیے معین ہیں۔
 \begin{multicols}{2}
\begin{enumerate}[a.]
\item \(  f(x)=x^{2}+4\)
\item \( f(x)=2(x^{2}+5) \)
\item \( f(x)=(x-1)^{2}+6 \)
\item \( f(x)=-(1-x)^{2}+7 \)
\item \(f(x)=3(x+5)^{2}+2  \)
\item \( f(x)=2(x+2)^{4}-1 \)
\end{enumerate}
\end{multicols}
 \انتہا{سوال}
\ابتدا{سوال}
دیے گۓ تمام تفاعل بتاۓ گۓ عملی میدان کے لیے معین ہیں، ان تمام تفاعل کی سعت معلوم کریں۔

\begin{multicols}{2}
\begin{enumerate}[a.]
 \item \( f(x)=2x  \) اور \( 0\le x\le 8 \)
\item \( f(x)=3-2x \) اور \( -2\le x\le 2 \)
\item \(f(x)=x^{2}  \) اور \( -1\le x\le 4 \)
\item \(  f(x)=x^{2} \) اور \(-5\le x\le -2 \)
\end{enumerate}
\end{multicols}
 \انتہا{سوال}
\ابتدا{سوال}
درجذیل تمام تفاعل کی سعت معلوم کریں، تما تفاعل \عددی{x } کے عملی میدان کی بڑی سے بڑی قیمتوں کے لیے معین ہیں۔
\begin{multicols}{4}
\begin{enumerate}[a.]
\item \( f(x)=x^{8} \)
\item \( f(x)=x^{11} \)
\item \(f(x)=\frac{1}{x^{3}}  \)
\item \( f(x)=\frac{1}{x^{4}} \)
\item \(f(x)=(x)^{4}+5  \)
\item \( f(x)=\frac{1}{4}x+\frac{1}{8} \)
\item \( f(x)=\sqrt{4-x^{2}} \)
\item \( f(x)=\sqrt{4-x} \)
\end{enumerate}
\end{multicols}
 \انتہا{سوال}
\ابتدا{سوال}
تار کا ایک ٹکڑا جو کہ \عددی{24} سینٹی میٹر لمبا ہے، اور ایک مستطیل کی شکل کا ہے۔ ہمیں بتایا گیا ہے کہ اسکی چوڑائ \عددی{w} سینٹی میٹر ہے ۔ ثابت کریں کہ حدود اربعہ \عددی{A} مربع سینٹی میٹر تفاعل \(A=36-(6-w)^{2}\) کے براربر ہے۔ اس تفاعل کے عملی دائرے کی سب سے بڑی قیمت کا پتی لگائیں اور ساتھ میں اس کی اسی قیمت پے سعت کا بھی۔
\انتہا{سوال}
\ابتدا{سوال}
تفاعل \(y=x(8-2x)(22-2x)\) کی ترسیم بنائیں۔ ہمیں بتایا گیا ہے کہ اس مکعب کا حجم \عددی{y } مکعب سینٹی میٹر ہے اگر اسکی بلندی \عددی{x} سینٹی میٹر ہو، لمبائ \((22-2x)\) اور چوڑائ \((8-2x)\) ہو تو۔ اوپر بتاے گۓ تفاعل کے لیے عملی دائرہ بھی بتائیں۔
\انتہا{سوال}

%graphs of powers of x
 \حصہ{\عددی{x}کی طاقتوں کے ترسیم}
 \جزوحصہ{مثبت صحیح عددی طاقتیں}
 تفاعل \(f(x)=x^{n}\) کی طرح کے ترسیم کو ذہن میں لائیں، جبکہ \عددی{n} ایک مثبت عدد ہے۔ غور کریں\(  (0,0)  \) اور  \((1,1)\) سے تفاعل \(y=x^{n}\) معین ہوتا ہے \عددی{n} کی تمام قیمتوں کے لیے، اور اسی وجہ سے تمام ترسیمات میں \(  (0,0)  \) اور  \((1,1)\) نقطے لازمی موجود ہیں۔

پہلے ترسیم کو دیکھیں جب \عددی{x} مثبت ہے، تب \(x^{n}\) بھی مثبت ہے اور ترسیم پورے کا پورا پہلے خانے میں آ جاتا ہے۔ شکل \حوالہ{شکل3.5} میں \(  n=1,2,3 \) اور \عددی{4} کے لیے ترسیمات دکھاۓ گۓ ہیں جن میں \عددی{x} کی قیمت صفر سے \عددی{1} کے بیچ رہ رہی ہے۔

خیال رکھنے کے نقطے یہ ہیں کہ ؛
\begin{itemize} 
\item  
اگر \(  n=1  \) ہو تو یہ ایک مخصوص معاملہ ہے اور  جسکی ترسیم \(y=x\) والی بنتی ہے، جو کہ مبدا سے گزرتی ہے اور دونوں محوروں کے ساتھ \(  45^{0}  \) درجے کو زاویہ بناتی ہے۔
%PAGE 37
\item   
اگر \(n>1\) تو افقی محور مبدا پہ ترسیم کا خط مماس بن جاۓ گا، اور یہ اس لیے کہ جب \عددی{x} کی قیمت چھوٹی ہو تو \(x^{n}\)  بھی  چھوٹا ہوگا، مثال کے طور پر \((0.1)^{2}=0.01,\,\,(0.1)^{3}=0.001,(0.1)^{4}=0.0001\)
\item  
طاقت \عددی{n} میں ہونے والی ہر بڑھوتری کے ساتھ ، ترسیم  افقی محور کے قریب ہی رہتا ہے \(x=0  \)اور \(x=1 \) کے درمیان، لیکن پھر \(x=1 \) کے بعد بہت کم ڈھلوان کے ساتھ بلندی کی طرف بڑھتا ہے۔ اور یہ اس لیے ہوتا ہے کہ \( x^{n+1}=x\times x^{n}\) اور اسی لیے \( x^{n+1}<x^{n}\) جب \(    0<x<1 \) اور \(  x^{n+1}>x^{n}  \) جب \(  x>1  \)۔
 \end{itemize}
اگر \عددی{x} منفی ہو تو آگے کیا ہوگا یہ منحصر ہوتا ہے کہ آیا \عددی{n} جفت ہے یا طاق۔ فرض کریں کہ \(x=-a\) جبکہ \عددی{a } ایک مثبت عدد ہے۔

اگر طاقت \عددی{n} جفت ہے  تو \( f(-a)=(-a)^{n}=a^{n}=f(a)   \) ، یعنی ترسیم پر  \عددی{y} کی قیمت وہی رہے گی مساوات  \(x=-a  \)   اور \(x=a \)   کے لیے۔ یعنی ہمارا ترسیم عمودی محور کے ساتھ تشاکل کی خصوصیت رکھتا ہے۔ یہی ببات شکل \حوالہ{شکل3.6} میں ترسیمات \( y=x^{2}\)  اور \( y=x^{4}\)   کے لیے بیان کی گئ ہے۔ وہ تفاعل جن میں \عددی{a } کی تمام قیمتوں کے لیے یہ خصوصیت ہو کہ   \(   f(-a)=f(a)  \)  تو ایسے تما تفاعل کو جفت تفاعل کہا جاۓ گا۔

اگر \عددی{n} تاک ہوگا تو، \(f(-a)=(-a)^{n}=-a^{n}=-f(a)\)۔  \(x=-a\) کے لیے \عددی{y} کی قیمت \(x= a\) کی قیمت کے برابر ہے لیکن ایک منفی کی علامت کے ساتھ۔۔ غور کریں کہ نقاط جن کے محدد  \(   (a,a^{n})  \) اور \((-a,-a^{n})\) ہیں مبدا کے دونوں اطراف تشاکل کی خصوصیت پر عمل پیرا ہوتے ہوۓ موجود ہیں۔ اسکا مظلب یہ پوری ترسیم مبدا کے ساتھ تشاکل میں ہے۔ یہ شکل \حوالہ{شکل3.7} میں  \(  y=x \) اور  \( y=x^{3}  \)  کے ترسیمات کے لیے واضع کی گئ ہے۔ تفاعل جن میں یہ خصوصیت ہو کہ \(   f(-a)=-f(a)\) تو ایسے تفاعل کو تاک تفاعل کہا جاتا ہے۔
 
 \جزوحصہ{منفی صحیح عددی طاقتیں}
آپ کسی بھی منفی عدد صحیح \عددی{n} کو \عددی{-m } لکھ سکتے ہیں جبکہ \عددی{ m } ایک مثبت عدد صحیح ہے۔ اور یوں \( x^{n}  \) بن جاۓ گا\(  x^{-m}   \) یا  \(   \frac{1}{x^{m}}\)۔
ایک مرتبی پھر ہم ابدا کریں گے ترسیم کے مثبت یعنی سادہ حصے سے، یعنی جب \عددی{x} مثبت ہوگا، ایسی صورتحال میں \( \frac{1}{x^{m}}   \)  بھی مثبت ہوگا اور ترسیم پہلے خانے میں بنے گا ۔بالکل ایسے ہی جیسے کہ \عددی{n} مثبت ہو نقطہ \( (1,1)    \)  ترسیم پر ہی موجود ہوگا لیکن اگر \(  x=0    \) ہو تو ایک بڑا فرق ہوگا، کیونکہ \(  x^{m}=0   \)  ہو جاۓ گا اور ایسی صورتحال میں \(   \frac{1}{x^{m}}   \)  غیر معین ہوگا، لب لباب ساری بات کا یہ ہے کہ ہمارے ترسیم پے کوئ بھی ایسا نقطہ نہیں ہوگا جسکے لیے \(      x=0 \) ہو۔
%PAGE 38
 اس پر اگر مزید گہری نگاہ ڈالی جاۓ، \عددی{x} کی کوئ قیمت فرض کریں جو کہ صفر کے قریب ہو جیسے کہ \عددی{0.01} ، تب اگر \( n=-1  \)  ہوا  تو \عددی{ Y } کی قیمت \(  0.01^{-1}=\frac{1}{0.01^{1}}=\frac{1}{0.01}=100\) ہوگی اور اسی طرح \(n=-2    \)  کے لیے \(0.01^{-2}=\frac{1}{0.01^{2}}=\frac{1}{0.0001}=10000\) اگر آپ بہت چھوٹا پیمانہ بھی استعمال کریں تو \(     x^{n} \)  کا ترسیم صفحے کےاوپری حصے سے بالکل غائب ہو جاۓ گا جیسے جیسے \عددی{x} کی قیمت کو صفر کے مزید قریب لایا جاۓ گا۔

کیا ہوگا اگر \عددی{x} بہت بڑا ہو ؟ مثال کے طور پے فرض کر لیں کہ \(x=100  \)  پھر \(   n=-1 \)  کے لیے \عددی{ y } کی قیمت \(  n=-1    100^{-1}=\frac{1}{100^{1}}=\frac{1}{100}=0.01\) ہوگی۔ اور اگر \(   n=-2 \)  کر دیا جاۓ تو یہ \(n=-2    100^{-2}=\frac{1}{100^{2}}=\frac{1}{10000}=0.0001\) ہو جاۓ گی۔لہٰذہ ہم کہہ سکتے ہیں کہ جیسے جیسے \عددی{x} کی قیمت کم ہوتی ہے ترسیم افقی محور کے قریب تر آتا جاتا ہے۔

اب ترسیم کے اس حصے پر غور کریں جس کے لیے \عددی{x} منفی ہے۔ ہم نے اوپر دیکھا کہ مثبت \عددی{ n } کے لیے ، انحصار اس بات پر ہوتا ہے کہ آیا \عددی{ n } جفت ہے یا تاق۔ اور اگر \عددی{ n } منفی ہو تب بھی یہی بات درست ہے اور اسی وجہ سے درست ہے۔ اگر \عددی{ n } جفت ہے تو \( x^{n}\)  ایک جفت تفاعل ہے اور اسکا ترسیم عمودی محور کے ساتھ تشاکل میں ہوگا۔ اور اگر  \عددی{ n } تاق ہوگا تو تفاعل \( x^{n}\) 
 بھی تاق ہوگا اور یہ مبدا کے ساتھ تشاکل میں ہوگا۔

شکل \حوالہ{شکل 3.8} میں ترسیم \(y=x^{n}  \) دکھایا گیا ہے اور اس میں    \(  n=-1  \)  اور  \(  n=-2\) کو مدنظر رکھتے ہوۓ یہ ترسیم  بناۓ  گۓ ہے۔
  \جزوحصہ{کسر کی صورت میں طاقتیں}
جب \عددی{ n } کسر کی صورت میں ہو ، تفاعل \( x^{n}  \)، \عددی{x} کی منفی قیمتوں کے لیے معین ہو بھی سکتا ہے اور نہیں بھی ہو سکتا۔مثال کے طور پر \(x^{\frac{1}{3}}\) اور \(x^{-\frac{4}{5}}\) معین ہیں جب \(x<0\)، لیکن\(x^{-\frac{1}{2}}\) اور  \(x^{-\frac{3}{4}}\)کے ساتھ ایسا معاملہ نہیں ہے۔ ب شک  \( x^{n}  \) کسی  منفی  \عددی{x} ککے لیے معین بھی ہو لیکن ذیادہ تر اعداد اور  کمپیوٹر انہیں حل نہیں کرتے۔ لہٰذہ ہمارے لیے ذیادہ بہتر ہے کہ ہم اس سارے بیانیے کو \(x\ge 0\) تک مختصر کر لیں۔ ان تفاعل کے ترسیمات بنانے کا سب سے آسان طریقہ ہے ککہ انکا تقابل کیا جاۓ اعداد صحیح کے ترسیمات سے۔ ذیل میں اسکی دو مثالیں دی گئ ہیں۔

تفاعل \(y=x^{\frac{5}{2}}\) کی ترسیم کو کہیں نا کہیں  \(y=x^{2}   \) اور  \(  y=x^{3}\) کی ترسیمت کے درمیان میں مووجوود ہونا چاہئیے۔
%PAGE 39

تفاعل  \( y=x^{-\frac{1}{2}}   \) کی تترسیم غیر معین ہو جاتی ہے جب \(   x=0\) ہو تو۔ اسکی ترسیم تفاعل \(y=x^{-1}  \) کی ترسیم سے ملتی جلتی ہے لیکن یہ اس سے نیچے بنتی ہے جب  \( x<1  \) اور اوپر بنتی ہے جب \(     x>1\)

اگر آپ کے پاس ترسیم بنانے والا اعداد ہے تو آپ خود بھی اس چیز کا تجربہ کر سکتے ہیں کسر کی صورت میں طاقت رکھنے والاے تفاعل کی ترسیما بنا بنا کے ، اگر آپ ایسا کر لیں تو آپ بھی مندرجہ ذیل نتائج تک پہنچیں گے۔
\begin{itemize}
\item  
تفاعل  \( x^{n}  \) کے ترسیم میں ابھی بھی نقطہ \(   (1,1)\) موجود ہے۔ 
\item  
اگر \عددی{ n } مثبت ہے تو تترسیم میں ایک اور نقطہ\((0,0)\) بھی لازمی موجود ہوگا 
\item 
 اگر \(n>1   \) ہے تو افقی محور ترسیم کا خط مماس ببن جاۓ گی، اگر \(   0<n<1\) تو  عمودی محور ترسیم کا خط مماس بن جاے گی۔ (اسے ذیادہ بہتر طور پر دیکھ پانے کے لیے آپ کو ترسیم کے مبدا کے قریبی حصے کو بڑا کر کے دیکھنا ہوگا۔)
\end{itemize}
  
ان تمام ترسیمات میں سب سے ذیادہ اہمیت کے حامل \(y=x^{\frac{1}{2}}  \)  اور \(   y=\sqrt{x}\) کے ترسیم ہیں۔ ااس ترسیم کی شکل کیسی ہوگی / یہ معللوم کرنے ککے لیے ذرا سوچیں کہ اگر \( y=x^{\frac{1}{2}}  \) تب \( x=y^{2}  \)، یعنی ترسیم \(   y=x^{2}\) کی ہی بنانی ہے لیکن مھوروں کو آپس میں ادلہ بدلی کر کے یعنی افقی محور کو عمودی محور بنا دیں اور عمودی محور کو افقی محور بنا دیں۔ اور یوں ہمارا ترسییم اوپر کی جانب کھلا ہونے کے بجاۓ دائیں جانب کھلا ہوا ہے۔ 

لیکن ابھی داستان مکمل نہین ہوئ ، کہ داستان مکلمل ہوتے ہوتے ہو گی، اگر \( x=y^{2}  \) تب یا تو \(  y=+\sqrt{x}  \) یا پھر \(  y=-\sqrt{x}  \) اور چونکہ آپ ان میں سے پہلے معاملے کے ساتھ ہی خوش ہیں تو آپ کو تفاعل \( x=y^{2}  \) کی ترسیم کا ایک حصہ ، جوو کہ افقی محور کے نیچے کا ہے اسے متتا  دینا ہوگا، صرف اوپری حصے کو باقی رہہنے دینا ہے جیسا کہ شکل \حوالہ{شکل3.9} میں دکھایا گیا ہے تفاعل \( y=x^{\frac{1}{2}}   \) یا \(      y=\sqrt{x} \) کا ترسیم۔
 

%modulus of a number
 \حصہ{ایک عدد کا مقیاس}
 فرض کریں آپ دو لوگوں کی بلندیون مین فرق کو ماپنا چاہتے ہیں، عددی معلومات مہیا کی گئ ہوں تو جواب سیدھا سا ہوگا، اگر انکی بلندیاں \عددی{90} سینٹی میٹر اور \عددی{100 } سینٹی میٹر ہیں تو آپکا جواب ہوگا \عددی{10 } سینٹی میٹر، اور اگر انکی بلندیاں \عددی{100 } اور \عددی{90} سینٹی میٹر ہوتیں تب بھی آپ کا جواب \عددی{10  } سینٹی میٹر ہی ہوتا۔لیکن ککیا ہو اگر انکی بلندیاں \عددی{H} سینٹی میٹر اور \عددی{h  } سینٹی میٹر ہوں، اسکا جواب اس بات پر منحصر ہوگا کہ کونسی قیمت بڑی ہے اگر \( H\g h \) ہو تو آپکا جواب ہوگا \((H-h)  \) سینٹی میٹر۔  اگر \( h\g H \) ہو تو آپکا جواب ہوگا \((h-H)  \)،  اور اگر\((h=H)  \) ہو تو آپکا جواب ہوگا صفر سینٹی میٹر، جو کہ یا تو \((h-H)  \) سینٹی میٹر ہے یا  \((H-h)  \) سینٹی میٹر ہے۔

اس طرح کے سوالات جن کے جوابات ہمیشہ مثبت یا منفی میں ہوتے ہیں مقیاس کی ضرورت کاا بڑھا دیتے ہیں۔

\عددی{x} کا مقیاس جیسے یوں \(\abs{x}\) لکھا جاتا ہے اور اسے "ماڈ ایکس" پڑھا جاتا ہے، اسکی تعریف کچھ یوں ہے۔
\[\abs{x}=x\quad x\ge 0\]
\[\abs{x}=-x\quad x<0\]


مقیاس ککی علامت استعمال کرتے ہوۓ آپ اب بلندی کو \( \abs{H-h}  \) لکھ سکتے ہیں جس میں \( h\g H \)، \( h\l H \) یا \( h= H \) ہو سکتے ہیں۔
 
%40



 ایک اور صورتحال جس میں معیار یا مقیاس کارآمد ہوتا ہے وہ ہے جب کوئ ہندسہ عددی اعتبار میں تو بہت بڑا ہو، لیکن منفی ہو جیسا کہ \عددی{-1000} یا \عددی{-1000000} ، ایسے اعددا کو آپ بڑے مقیاس والے منفی اعداد کہہ سکتے ہیں۔

مثال کے طور پر \عددی{x} کی قیمت جوں جوں بڑھے گی ، \(\frac{1}{x}\)کی قیمت گھٹتی جاۓ گی یہاں تک کہ یہ صفر کے قریب ترین پینچ جاۓ گی۔ اور \عددی{x}  کے بڑے مقیاس والی منفی قیمتوں کے لیے بھی یہ بات درست ہے۔ اور اسی لیے آپ کہہ سکتے ہیں کہ جب \(\abs x\) بڑا ہوگا تو \(\abs{\frac{1}{x}}\) صفر کے قریب ترین ہوگا۔ اور اگر عددی مثال دیکھیں تو جب \(\abs x>1000\) تو \(\abs{\frac{1}{x}}<0.001\) ہوگا۔ شکل \حوالہ{شکل3.10} دیکھیں۔

کچھ حساب کتاب کے آلات میں ایک بٹن موجود ہوتا ہے جو کسی بھی عدد کا مقیاس بتاتا ہے۔ اس بٹن پر اکثر \( [ABS]\) درج ہوتا ہے جو کہ مقیاس قیمت کا مآخذ ہے۔

%excercise 3B

اگر آپ کے پاس ترسیم بنانے والا اعداد ہے تو آپ سوالات \حوالہ{سوال3ب4}،سوال\حوالہ{3ب5}،سوال\حوالہ{3ب6} کی ترسیم بھی بنا کر دیکھیں۔

\ابتدا{سوال}
درج ذیل مساوات کی ترسیم بنائیں۔
\begin{multicols}{2}
\begin{enumerate}[a.]
\item \(  y=x^{5}\)
\item \( y=x^{6}\)
\item \( y=x^{10}\)
\item \( y=x^{15}\)
\end{enumerate}
\end{multicols}

\انتہا{سوال}
\ابتدا{سوال}
تین ترسیم جنکی مساوات بالترتیب
 \((p)\,y=x^{-2}\quad (q)\,y=x^{-3}\quad (r)\,y=x^{-4}\) 
ہیں۔ایک لکیر \عددی{A} جسکی مساوات \(x=k\)  ہے ، ان تینوں ترسیموں کو نقاط \عددی{P},\عددی{Q}  اور  \عددی{R} پر آپس میں ملاتی ہے۔ان تینوں نقاط \عددی{P},\عددی{Q}  اور  \عددی{R} کو بالترتیب (نیچے سے اوپر) لکھیں جبکہ  \عددی{k} درج ذیل قیمتیں لے گا۔
\begin{multicols}{2}
\begin{enumerate}[a.]
\item \(  2\)
\item \( \frac{1}{2} \)
\item \( \frac{-1}{2} \)
\item \( -2 \)
\end{enumerate}
\end{multicols}
\انتہا{سوال}
\ابتدا{سوال}
\عددی{x} کی کن قیمتوں کے لیے درج ذیل عدم مساوات صحیح ثابت ہوں گی؟اپنے جوابات کی وضاحت کے لیے ترسیم بھی بنائیں۔
\begin{multicols}{2}
\begin{enumerate}[a.]
\item \( 0<x^{-3}<0.001 \)
\item \( x^{-2}<0.0004 \)
\item \( ,x^{-4}\ge 100 \)
\item \( 8x^{-4}<0.00005 \)
\end{enumerate}
\end{multicols}
\انتہا{سوال}
\ابتدا{سوال}
درج ذیل مساوات کی ترسیم بنائیں ، اس شرط کے ساتھ کہ  \(x>0  \).
\begin{multicols}{3}
\begin{enumerate}[a.]
\item \( y=x^{\frac{3}{2}} \)
\item \(y=x^{\frac{1}{3}}  \)
\item \( y=-2x^{\frac{1}{2}}  \)
\item \(y=4x^{-\frac{1}{4}}  \)
\item \(  y=x^{-\frac{4}{3}}\)
\item \(  y=x^{\frac{2}{3}}-x^{-\frac{2}{3}}\)
\end{enumerate}
\end{multicols}
\انتہا{سوال}
\ابتدا{سوال}
درج ذیل مساوات کی ترسیم بنائیں، جہاں ضرورت پڑے \عددی{x} کی منفی قیمتیں بھی استعمال کریں۔
\begin{multicols}{3}
\begin{enumerate}[a.]
\item \(  y=x^{\frac{2}{3}}\)
\item \( y=x^{\frac{3}{4}} \)
\item \( y=x^{\frac{4}{5}} \)
\item \( y=x^{-\frac{1}{3}} \)
\item \( y=x^{\frac{4}{3}} \)
\item \(  y=x^{-\frac{3}{2}} \)
\end{enumerate}
\end{multicols}
\انتہا{سوال}
\ابتدا{سوال}
درج ذیل مساوات کے ساتھ ترسیم بنائیں۔
\begin{multicols}{3}
\begin{enumerate}[a.]
\item \( y=x^{2}+x^{-1} \)
\item \(y=x+x^{-2}  \)
\item \(y=x^{2}-x^{-1}  \)
\item \(y=x^{-2}-x^{-1}  \)
\item \( y=x^{-2}-x^{-3} \)
\item \( y=x^{-2}-x^{-4}  \)
\end{enumerate}
\end{multicols}
\انتہا{سوال}
\ابتدا{سوال}
درج ذیل تفاعل میں سے ایک جفت ہے اور دو تاک ہیں، آپ معلوم کریں کہ کونسا تفاعل جفت اور کونسے تاک ہیں؟
\begin{multicols}{3}
\begin{enumerate}[a.]
\item \( y=x^{7} \)
\item \( y=x^{4}+3x^{2} \)
\item \( y=x(x^{2}-1) \)
 \end{enumerate}
\end{multicols}
\انتہا{سوال}

%page 41
\ابتدا{سوال}
درج ذیل کی قیمتیں معلوم کریں۔
\begin{multicols}{3}
\begin{enumerate}[a.]
\item \(  \abs{-7}\)
\item \( \abs{-\frac{1}{200}} \)
\item \( \abs{9-4} \)
\item \(\abs{4-9}  \)
\item \( \abs{\pi -3} \)
\item \( \abs{\pi -4} \)
\end{enumerate}
\end{multicols}
\انتہا{سوال}

\ابتدا{سوال}
مقیاس \(\abs{x-x^{2}}\) کی قیمت معلوم کریں، جبکہ \عددی{x}کی قیمتیں درج ذیل میں سے ہوں۔
\begin{multicols}{3}
\begin{enumerate}[a.]
\item \( 2 \)
\item \( \frac{1}{2} \)
\item \( 1 \)
\item \( -1 \)
\item \( 0 \)
 \end{enumerate}
\end{multicols}
\انتہا{سوال}
\ابتدا{سوال}
آپ کو بتا دیا گیا ہے کہ \(y=\frac{1}{x^{2}}\)، آپ \عددی{y} کے بارے میں کیا کہیں گے اگر؛
\begin{multicols}{3}
\begin{enumerate}[a.]
\item \( \abs{x}>100 \)
\item \(\abs{x}<0.01  \)
 \end{enumerate}
\end{multicols}
\انتہا{سوال}
 \ابتدا{سوال}
آپ کو بتا دیا گیا ہے کہ \(y=\frac{1}{x^{3}}\)؛
\begin{multicols}{3}
\begin{enumerate}[a.]
\item 
آپ \عددی{y}کے بارے میں کیا کہیں گے اگر \( \abs{x}<1000 \)
\item  
آپ \عددی{y}کے بارے میں کیا کہیں گے اگر \( \\abs{y}>1000\)
 \end{enumerate}
\end{multicols}
\انتہا{سوال}
 \ابتدا{سوال}
ایک فٹبال کا میچ دیکھنے آۓ تماشائیوں کی تعداد \عددی{N}  قریب ترین ہزار میں\عددی{37000} گنی گئ۔ اس بیان کو مقیاس کی علامت استعمال کرتے ہوۓ ایک عدم مساوات کی صورت میں لکھیں۔
\انتہا{سوال}
 \ابتدا{سوال}
دو جڑواں بچوں کے ریاضی نمبروں بالترتیب \عددی{m} اور \عددی{n} کا فرق کبھی بھی \عددی{5} سے ذیادہ نہیں ہوا۔ اس بیان کو مقیاس کی علامت استعمال کرتے ہوۓ ایک عدم مساوات کی صورت میں لکھیں۔
\انتہا{سوال}
\ابتدا{سوال}
ایک لکیر کی لمبائ \عددی{x} سینٹی میٹر ہے، آپ کو بتایا گیا ہے کہ \(\abs{x-5.23}<0.005\)، آس اس کو ایک بیان کی صورت میں کیسے بتائیں گے۔
\انتہا{سوال}

 \حصہ{مساوات \(y=ax^{2}+bx+c\) کے ترسیم}

سبق \حوالہ{نام سبق ایک} میں آپ نے سیدھی لکیروں کے ترسیم بنانا سکیھے،اور آپ نے یہ بھی جانا کہ مساوات \(y=mx+c\) میں مستقل \عددی{m} اور\عددی{c} کا کیا مطلب ہے۔ 

مشق \حوالہ{مشق3c} میں آپ کو موقع ملے گا \(y=ax^{2}+bx+c\) کے جیسی مساوات کے ترسیمات بنانے کا۔

اگر آپ کے پاس ترسیم بنانے والا اعداد ہے تو اسے بروۓ کار لاتے ہوۓ تمام سوالات حل کریں ورنہ ایک گروپ کی شکل میں دیگر لوگوں کے ساتھ مل کر سوالات حل کریں۔

اہم نقاط  کا تزکرہ مشق حل کرنے کے بعد کریں گے۔
%exercise 3C

\ابتدا{سوال}
ایک ہی نظام محدد میں درج ذیل مساوات کے ترسیم بنائیں۔
\begin{multicols}{3}
\begin{enumerate}[a.]
\item \( y=x^{2}-2x+5 \)
\item \( y=x^{2}-2x+1 \)
\item \( y=x^{2}-2x \)
 \end{enumerate}
\end{multicols}
\انتہا{سوال}
\ابتدا{سوال}
ایک ہی نظام محدد میں درج ذیل مساوات کے ترسیم بنائیں۔
\begin{multicols}{3}
\begin{enumerate}[a.]
\item \(y=x^{2}+x-4 \)
\item \( y=x^{2}+x-1 \)
\item \( y=x^{2}+x+2 \)
 \end{enumerate}
\end{multicols}
\انتہا{سوال}
%page42
\ابتدا{سوال}
دی گئ شکل میں مساوات \(y=ax^{2}-bx\) کی ترسیم بنائ گئ ہے، اسی شکل پے درج ذیل مساوات کی ترسیم بنائیں۔
\begin{multicols}{3}
\begin{enumerate}[a.]
\item \( y=ax^{2}-bx+4 \)
\item \( y=ax^{2}-bx-6 \)
 \end{enumerate}
\end{multicols}
\انتہا{سوال}
\ابتدا{سوال}
مساوات \(y=ax^{2}+bx+c\) کے ترسیم میں \عددی{c} کی قیمت کو تبدیل کرنے سے ترسیم پر کیا فرق پڑے گا؟
\انتہا{سوال}
\ابتدا{سوال}
درج ذیل مساوات کے ترسیم بنائیں۔
\begin{multicols}{3}
\begin{enumerate}[a.]
\item \( y=x^{2}-4x+1\)
\item \(y=x^{2}-2x+1  \)
\item \( y=x^{2}+1 \)
\item \( y=x^{2}+2x+1 \)
\end{enumerate}
\end{multicols}
\انتہا{سوال}
\ابتدا{سوال}
مساوات \(y=2x^{2}+bx+4\) کی ترسیم بنائیں لیکن  \عددی{b} کی مختلف قیمتوں کے لیے، \عددی{b} کی قیمتوں کو بدلنے سے مساوات \(y=ax^{2}+bx+c\) پر کیا فرق پڑے گا۔
\انتہا{سوال}
 \ابتدا{سوال}
درج ذیل   مساوات کی ترسیم   بنائیں۔
\begin{multicols}{2}
\begin{enumerate}[a.]
\item \(y=x^{2}+1  \)
\item \( y=3x^{2}+1 \)
\item \(y=-3x^{2}+1  \)
\item \( y=-x^{2}+1 \)
 \end{enumerate}
\end{multicols}
\انتہا{سوال}
\ابتدا{سوال}
درج ذیل   مساوات کی ترسیم   بنائیں۔
\begin{multicols}{2}
\begin{enumerate}[a.]
\item \(,y=-4x^{2}+3x+1  \)
\item \( y=-x^{2}+3x+1 \)
\item \(y=x^{2}+3x+1  \)
\item \(y=4x^{2}+3x+1 \)
 \end{enumerate}
\end{multicols}
\انتہا{سوال}
\ابتدا{سوال}
مساوات  \(y=ax^{2}+bx+c\) میں \عددی{a} تبدیل کرنے سے مساوات  کی ترسیم پر کیا فرق پڑے گا؟
\انتہا{سوال}
\ابتدا{سوال}
شکل میں ایک ترسیم دکھائ گئ ہے۔ معلوم کریں کہ درج ذیل مساوات میں سے کونسی مساوات اس ترسیم کی ہے؟
\begin{multicols}{3}
\begin{enumerate}[a.]
\item \(y=x^{2}-2x+5  \)
\item \( y=-x^{2}-2x+5 \)
\item \(y=x^{2}+2x+5  \)
\item \( y=-x^{2}+2x+5 \)
\end{enumerate}
\end{multicols}
\انتہا{سوال}
\ابتدا{سوال}
شکل میں ایک ترسیم دکھائ گئ ہے۔ معلوم کریں کہ درج ذیل مساوات میں سے کونسی مساوات اس ترسیم کی ہے؟
\begin{multicols}{3}
\begin{enumerate}[a.]
\item \(y=-x^{2}+3x+4\)
\item \( y=x^{2}-3x+4 \)
\item \(y=x^{2}+3x+4  \)
\item \( y=-x^{2}-3x+4 \)
\end{enumerate}
\end{multicols}
\انتہا{سوال}

 %page43 
 \حصہ{مساوات \(y=ax^{2}+bx+c\)سے بنے ترسیموں کی اشکال}
مشق حل کرتے ہوۓ آپ نے کئ نتائج حاصل کیے ہوں گے، ان تمام نتائج کا نچوڑ ذیل میں بیان کیا گیا ہے۔

تمام ترسیم کی شکل لگ بھگ ایک جیسی تھی اور اس شکل کو قطع مکافی یا پیرابولا کہا جاتا ہے۔ انن تمام قطع مکافی کی محور تشاکل عمودی ہے۔ جس نقطے پر ایک قطع مکافی اپنے محور تشاکل سے ملتا ہے اس نقطے کو راس کہتے ہیں۔ اگر  \عددی{c} کی قیمت بدلی جاۓ تو ترسیم عمودی محور پر اوپر نیچے حرکت کرتی ہے، جبکہ  \عددی{a} بدلنے سے محور تشاکل افقی محور میں آگے پیچھے حرکت کرتی ہے۔ اگر  \عددی{a} اور \عددی{b} کے ساتھ موجود ریاضیاتی علامات ایک جیسی ہیں یعنی دونوں کے ساتھ جمع یا منفی کی علامت ہے تو محور تشاکل عمودی محور کے بائیں جانب ہوگی اور اگر  \عددی{a} اور \عددی{b} کے ساتھ مخالف ریاضیاتی علامات ہوں تو محور تشاکل عمودی محور کے دائیں جانب ہوگی۔

اگر \عددی{a} مثبت ہے تو راس ترسیم کے سب ساے نچلے حصے پے موجود ہوگا اور اگر \عددی{a} منفی ہے تو راس ترسیم کے سب سے بلند حصے پے موجود ہوگا۔\(\abs{a}\)  جتنا بڑا ہوگاترسیم اتنی ہی لمبوتری ہوگی، یعنی عمودی محور میں لمبی۔
  \حصہ{دو ترسیموں کا مشترکہ نقطہ}
دو ترسیموں کا مشترکہ نقطہ معلوم کرنے کا بھی وہی طریقہ ہے جو کہ دو سیدھی لکیروں کا مشترکہ نقطہ معلوم کرنے کا طریقہ ہے۔ فرض کریں کہ آپ کے پاس دو ترسیم ہیں، جن کی مساوات \(    y=f(x)        \)  اور \(    y=g(x)       \) ہیں۔ آپ کو ایک نقطے \(   (x,y)    \)  کی تلاش ہے جو کہ دونوں ترسیموں میں موجود ہو، اسکا مطلب محدد \(   (x,y)    \) سے دونوں مساوات درست ثابت ہوں گی۔ یہاں سے یہ بات سامنے آتی ہے کہ ہمیں ایک ایسا \عددی{x} چاہئیے جس کے لیے \(   f(x)=g(x)  \) ۔
   \ابتدا{مثال}
لکیر \(  y=2\) اور ترسیم \(  y=x^{2}-3x+4\) کا مشترکہ نقطہ معلوم کریں۔
دونوں مساوات کو ایک ساتھ حل کرنے سے ہمیں حاصل ہوا \(  x^{2}-3x+4=2\) اور ہمیں اسے یوں بھی لکھ سکتے ہیں؛ \(  x^{2}-3x+2=0 \)۔
اس مساوات کو حل کرنے سے اسکے اجزاء ملے جن سے اس مساوات کا حل نکالا جا سکتا ہے جو کہ یہ ہیں؛ \(x=1   x=2\) اب ان قیمتوں کو کسی ایک مساوات میں ڈال کر \عددی{y} معلوم کرنا یقیناً نہایت آسان ہے، اور یوں ہمیں جو مشترکہ نقاط ملے ہیں وہ \(    (1,2)  \)اور \(     (2,2)  \) ہیں۔
 \انتہا{مثال}
%page44
 \ابتدا{مثال}
لکیر جسکی مساوات \( y=2x-1 \) ہے ، اس لکیر اور ترسیم \(   y=x^{2} \) کا مشترکہ نقطہ معلوم کریں۔
دونوں مساوات کو حل کریں تو \(   2x-1=x^{2} \) ملتا ہے، جو کہ دراصل \(    x^{2}-2x+1=0 \) ہے اور اسکے اجزاء معلوم کیے جائیں تو \(   (x-1)^{2}=0  \)ملے گا جہاں سے \(     x=1  \) ملے گا۔
\عددی{x} کی اس قیمت کو دونوں مساوات میں سے ایک میں ڈالیں تو ہمیں \عددی{y} کی قیمت مل جاۓ گی ، اور یوں ہمیں اس نقطے کے محدد معلوم ہو جائیں گے جو کہ لکیر اور ترسیم میں مشترک ہے، اس سوال میں وہ نقطہ \((1,1)\) ہے۔ 
یہاں مشرتک نقطہ صرف ایک ہی ہے اور اسکی وجہ یہ ہے کہ یہ لکیر ترسیم کا خط مماس ہے، اور خط مماس ایک ترسیم کو ایک ہی نقطے پر چھوتی ہے، اگر آپ کے پاس ترسیم بنانے والا اعداد ہے تو آپ اس بیان کی تصدیق بھی کر سکتے ہیں۔
 \انتہا{مثال}
 \ابتدا{مثال}
ترسیم \(   y=x^{2}-2x-6  \) اور  ترسیم \(   y=12+x-2x^{2}  \) کا مشترکہ نقطہ معلوم کریں۔

دونوں مساوات کو حل کریں تو \(   x^{2}-2x-6=12+x-2x^{2}  \)ملتا ہے، جو کہ دراصل \(     3x^{2}-3x-18=0   \)ہے  اسے \عددی{3}سے تقسیم کریں تو ہمیں ملے گا \(        x^{2}-x-6=0   \) اور اسکے اجزاء معلوم کیے جائیں تو   \(        (x+2)(x-3)=0   \)ملے گا جہاں سے\(            x=-2    x=3\) ملے گا۔

\عددی{x}کی ان قیمتوں کو کسی بھی ایک مساوات میں ڈال کر \عددی{y} کی دو قیمتیں معلوم کی جا سکتی ہیں، اور اسطرح سے ہمارے پاس دو نقاط کے محدد ہوں گے جوکہ \((-2,2)(3,-3)\) ہیں
 \انتہا{مثال}

%Exercise 3D

\ابتدا{سوال}
درج ذیل سوالاات میں بتائ گئ لکیروں اور ترسیمات  کے مشترکہ نقاط معلوم کریں
\begin{multicols}{2}
\begin{enumerate}[a.]
\item \(    x=3     \)           اور  \(  y=x^{2}+4x-7 \)
\item \(,y=3  \) اور  \(  y=x^{2}-5x+7 \)
\item \( y=8\) اور  \(  y=x^{2}+2x \)
\item \( y+3=0 \) اور  \( y=2x^{2}+5x-6 \)
 \end{enumerate}
\end{multicols}
\انتہا{سوال}

\ابتدا{سوال}
درج ذیل سوالاات میں بتائ گئ لکیروں اور ترسیمات  کے مشترکہ نقاط معلوم کریں
\begin{multicols}{2}
\begin{enumerate}[a.]
\item \(  y=x+1    \)           اور  \(  y=x^{2}-3x+4 \)
\item \(y=2x+3 \) اور  \(   y=x^{2}+3x-9 \)
\item \( y=3x+11\) اور  \(  y=2x^{2}+2x+5 \)
\item \( y=4x+1  \) اور  \(y=9+4x-2x^{2} \)
\item \( 3x+y-1=0 \) اور  \(  y=6+10x-6x^{2} \)
 \end{enumerate}
\end{multicols}
\انتہا{سوال}
\ابتدا{سوال}
درج ذیل سوالوں میں یہ ثابت کریں کہ ترسیم اور لکیر ایک ہی نقطے پر ملتی ہیں، اور وہ مشترک نقطہ بھی معلوم کریں۔
 
\begin{enumerate}[a.]
\item \(y=2x+2   \)         اور \(  y=x^{2}-2x+6 \)      
\item \( y=-2x-7 \)  اور \(y=x^{2}+4x+2  \)    
 \end{enumerate}
\انتہا{سوال}
\ابتدا{سوال}
ترسیم \(y=x^{2}-x\) اور ذیل میں دی ہوئ لکیروں کے مشترک نقاط معلوم کریں۔
\begin{enumerate}[a.]
\item \(y=x  \)      
\item \(y=x-1\)  
 \end{enumerate}
اگر آپ کے پاس ترسیم بنانے والا اعداد ہے تو دیکھین کہ ترسیم اور للکیر کا باہمی رشتہ یا تعلق کیا ہے؟
\انتہا{سوال}
\ابتدا{سوال}
ترسیم \(y=x^{2}+5x+18\) اور ذیل میں دی ہوئ لکیروں کے مشترک نقاط معلوم کریں۔
\begin{enumerate}[a.]
\item \(y=-3x+2 \)      
\item \(y=-3x+6\)  
 \end{enumerate}
اگر آپ کے پاس ترسیم بنانے والا اعداد ہے تو دیکھین کہ ترسیم اور للکیر کا باہمی رشتہ یا تعلق کیا ہے؟
\انتہا{سوال}
%page45
 
\ابتدا{سوال}
لکیر، جسکی مساوات \(y=x+5\) ہے ، اور ذیل میں دیے گۓ دو ترسیمات کا الگ الگ ،مشترک نقطہ معلوم کریں 
\begin{enumerate}[a.]
\item \(y=2x^{2}-3x-1 \)      
\item \(y=2x^{2}-3x+7\)  
 \end{enumerate}
اگر آپ کے پاس ترسیم بنانے والا اعداد ہے تو دیکھین کہ ترسیم اور للکیر کا باہمی رشتہ یا تعلق کیا ہے؟
\انتہا{سوال}
\ابتدا{سوال}
درج ذیل مساوات سے بننے والے ترسیمات کے مشترک نقاط معلوم کریں۔ 
\begin{enumerate}[a.]
\item \( y=x^{2}+5x+1    \)           اور  \(   y=x^{2}+3x+11 \)
\item \(y=x^{2}-3x-7 \) اور  \(  y=x^{2}+x+1 \)
\item \( y=7x^{2}+4x+1\) اور  \(   y=7x^{2}-4x+1 \)
  \end{enumerate}
\انتہا{سوال}

\ابتدا{سوال}
درج ذیل مساوات سے بننے والے ترسیمات کے مشترک نقاط معلوم کریں۔ 
\begin{enumerate}[a.]
\item \(    y=\frac{1}{2}x^{2}  \)           اور  \(  y=1-\frac{1}{2}x^{2}   \)
\item \(   y=2x^{2}+3x+4    \) اور  \(    y=x^{2}+6x+2               \)
\item \(   y=x^{2}+7x+13     \) اور  \(      y=1-3x-x^{2}    \)
\item \(   y=6x^{2}+2x-9   \)           اور  \( y=x^{2}+7x+1    \)
\item \(   y=(x-2)(6x+5)    \) اور  \(     y=(x-5)^{2}+1               \)
\item \(    y=2x(x-3)    \) اور  \(   y=x(x+2)       \)
  \end{enumerate}
\انتہا{سوال} 
 \ابتدا{سوال}
ذیل میں دی گئ مساوات کی جوڑیوں کے ترسیمات بنائیں اور انکے مشترکہ نقاط بھی معلوم کریں۔
 \begin{enumerate}[a.]
\item \(y=8x^{2}\,,y=8x^{-1}  \)
\item \( y=x^{-1}، y=3x^{-2} \)
\item \(y=x\,,y=4x^{-3} \)
\item \( y=8x^{-2}\,,y=2x^{-4} \)
\item \( y=9x^{-3}\,,y=x^{-5} \)
\item \(y=\frac{1}{4}x^{4}\,,y=16x^{-2} \)
\end{enumerate}
\انتہا{سوال}
  \حصہ{اجزاء کی مدد سے ترسیمات بنانا}
کچھ تفاعل جیسا کہ \(f(x)=ax^{2}+bx+c\) کے ترسیمات ایسے ہوتے ہیں کہ ان کے اجزاء کی مدد سے بھی ترسیم بنایا جا سکتا ہے۔ مثال کے طور پر  ذیل میں دیے گۓ تفاعل ہی کو یکھ لیں 
\begin{align*}
f(x)&=x^{2}-6x+5=(x-1)(x-5)\\
g(x)&=12x-4x^{2}=-4x(x-3)\\
\end{align*}
پہلے حصے میں \( f(1)=0 \) اور \(  f(5)=0  \) اور اسی لیے نقاط \(   (1,0)    \) اور \(  (5,0)    \) پہلے تفاعل کی ترسیم پر ہی موجود ہیں۔
یہ شکل \حوالہ{شکل3.13} میں دکھایا گیا ہے۔
اسی طرح \(  g(0)=g(3)=0   \) اسی لیے \(  (0,0)  \) اور \( (3,0)  \) دوسرے تفاعل کی ترسیم پر موجود ہیں ۔ یہ شکل \حوالہ{شکل3.14} میں دکھایا گیا ہے۔
 آپ ایسے کسی بھی تفاعل کا ترسیم بنا سکتے ہیں جو کچھ اس طرح سے اجزاء میں بٹے، \(a(x-r)(x-s)\)۔ 

پہلے یہ ذہن نشین کر لیں کہ یہ افقی محور کو \((r,0)  \) اور \((s,0)   \) پر کاٹے گا، مستقل \عددی{a} کی علامت آپکو یہ بتاۓ گی کہ آیا ترسیم اوپر کی طرف مڑے گی یاا نیچے کی جانب۔

%%%%%%%%%%%%%%%%%%%%%%%%%%%%%%%%%%%%%%%%%%%%%%%%%%%%%%%%%%%%%%%%%%%%%%%%%%%%%
\section{مثال 3.8.1}
مندرجہ ذیل ترسیم کا خاکہ بنائیں
$f(x)=3x^2-2x-1$
آپ اس مساوات کی تجری اس طرح نکال سکتے ہیں۔
$f(x)=(3x+1)(x-1)$
مگر اجزاےٗ ضربی سے حل کرنے کے لیٗے ہمیں اس طرح سے لکھنا ہوگا۔
$$f(x)=3(x+\frac{1}{3})(x-1)$$
اس سے ہمیں معلوم ہوتا ہے کہ ترسیم نکات
$(\frac{-1}{3},0)$  
اور
  $1,0$
  سے گزرتی ہے۔مستقل قیمت
3
سے معلوم ہوتا ہے کہ خاکہ کس قدر بڑھا ہوا ہے۔
\\
اس سے ہمیں یتنی معلومات ملتی یے کہ ہم ترسیم کی شکل کا اندازہ کر سکتے ہیں اور آپ 
%figure
میں خاکہ دیکھ سکتے ہیں۔
اگر ہم غور کریں تو 
$f(0)=-1$
جس سے معلوم ہوتا کہ ترسیم 
$(0,1)$
پر
y-axis
کو جاملتی ہے۔

اگر غور کریں کہ خاکے پر محور کے خلاف نشانات نہیں ہیں ، سوائے یہ کہنے کے کہاں
ترسیم ان کو کاٹتی ہے۔
\\
اجزاےٗ ضربی کے طریقے کو استعمال کرتے ہوےٗ ہم دو سے زیادہ اجزاء کے لیے لکھ سکتے ہیں۔
$$f(x)=a(x-r)(x-s)(x-t)$$
یہ تفاعل کسی ایسی ترسیم کی وضاحت کرتا ہے جس کی مساوات ، جب ضرب ہوجاتی ہے  
$f(x)=ax^3-... .$
کے ساتھ شروع ہوتی ہے۔
\\
یہ ترسیم نکات 
$(r,0),(s,0)$
اور
$(t,0).$
سے گزرتی ہے۔ مستقل      a    سے معلوم ہوتا ہے کہ     x   کی بڑی قیمتوں کے لیے ترسیم(کارتیسی نزامِ محدد کے تہد) اول اور چوٹھے خانے موجود میں ہوگی۔
غور کریں کہ 
figure3.17
میں اجراء
$x-1$
کا مربع (کارتیسی نزامِ محدد کے تہد)   x- محور پر یوگا۔ 
\section{ترسیم سے مساوات کا اندازہ لگانا}
اجراء ضربی کا استعمال کرتے ہوئے ہم ترسیم کی مساوات کا باخوبی اندازہ لگا سکیے ہیں۔
$ f(x)=ax^2+bx+c $
اگر وہ نقطے معلوم ہوں جہاں کو ترسیم اُفکی محور پر کاٹتی ہے اور کم از کم ایک نقطے کی جگہ معلوم ہو۔ 
\newpage
%3.9 predicting functions from their graphs 
 ایک ایسی مساوات بنائیں جو 
\(y=ax^{2}+bx+c\)
سے تعلق رکھتی ہو۔اَفقی مہور کو دو جگہوں پر کاٹتا ہے جو 
$   (1,0) $
اور
$   (4,0)  $ 
۔جبکہ نکتے کو 
$ (3,-4)$
پر کاتتا ہے۔
\\
چونکہ وکر محور کو
\((1,0)\)  
اور
 \( (4,0)\)
پر  کاٹتا ہے۔ 
 \\
 %fig 3.18
 مساوات کچھ اس طرح کی ہوگی۔
\[y=a(x-1)(x-4)\]\\
جیسے کہ نکتہ 
\((3,-4)\)
وکر میں آتا ہے۔
\( -4=a(3-1)(3-4)\)
جس سے ہمیں ملتا ہے۔
\(    -4=-2a     a=2 \)\\
وکر کی مساوات یہ بنتی ہے۔
\(y=2(x-1)(x-4)     y=2x^{2}-10x+8\)\\


%Exercise 3E

1.
مندرجہ ذیل ترسیمات کا خاکہ بنائیں۔
\begin{enumerate}[a.]
\item  \((a)y=(x-2)(x-4)\) 
\item \(y=(x+3)(x-1)\) 
\item \(y=x(x-2)\)
\item \(y=(x+5)(x+1)\) 
\item \(y=x(x+3)\) 
\item \(y=2(x+1)(x-1)\)
\end{enumerate}


2.
مندرجہ ذیل ترسیمات کا خاکہ بنائیں۔
\begin{enumerate}[a.]
\item \(y=3(x+1)(x-5)\)
\item \(y=-2(x-3)(x-1)\)
 \item \(y=-(x+3)(x+5)\)
\item \(y=2(x+\frac{1}{2})(x-3)\)
  \item \(y=-3(x-4)^{2}\quad \)
  \item \(y=-5(x-1)(x+\frac{4}{5} \)
\end{enumerate}
3.
پہلے ضرب کریں پھر ترسیمات کا خاکہ بنائیں۔
\begin{enumerate}[a.]
\item   \(y=x^{2}-2x-8 \) 
\item   \(y=x^{2}-2x \)
\item   \(y=x^{2}+6x+9\)
\item   \(y=2x^{2}-7x+3 \)
\item   \(y=4x^{2}-1\)
\item    \(y=-(x^{2}-x-12)\)
\item    \(y=-x^{2}-4x-4\) 
\item  \(y=-(x^{2}-7x+12)\) 
 \item \(y=11x-4x^{2}-6\)
\end{enumerate}
4.
مساوات بنائیں جو 
\(y=x^{2}+bx+c\)
قطع مکافی  کی صورت پر مبنی ہو۔\\
\begin{enumerate}[a.]
\item \(  \text{اَفقی مہور کو دو جگہوں پر کاٹتا ہے جو}\,(2,0)\text{اور عمودی مہور کو } (5,0)\)
\item \( \text{اَفقی مہور کو دو جگہوں پر کاٹتا ہے جو}\,(-7,0)\text{اور عمودی مہور کو } (-10,0)\)
\item \( \text{اَفقی مہور کو دو جگہوں پر کاٹتا ہے جو}\,(-5,0)\text{اور عمودی مہور کو } (3,0)\)
\item \( \text{اَفقی مہور کو دو جگہوں پر کاٹتا ہے جو}\,(-3,0)\text{اور عمودی مہور کو } (1,-16)\)
\end{enumerate}


5.
مندرجہ ذیل ترسیمات کا خاکہ بنائیں۔
\begin{enumerate}[a.]
\item \( y=(x+3)(x-2)(x-3) \)
\item \(y=x(x-4)(x-6)\)
\item \(y=x^{2}(x-4)]\)
\item \(y=x^{2}(x-4)^{2}\)
\item \(y=-x(x+4)(x+6)(x+2)\)
\item \(y=-3(x+1)(x-3)^{2} \)
\end{enumerate}
6.
مساوات بنائیں جو
\(y=ax^{2}+bx+c\)
قطع مکافی کی صورت پر مبنی ہو۔\\
\begin{enumerate}[a.]
\item \( \text{اَفکی  مہور کو دو جگہوں پر کاتتا ہے جو}
\,(1,0)(5,0)  
\text{اور عمودی مہور کو }  
 (0,15)
  \text{  پر کاتتا ہے }\)
\item \( \text{اَفکی مہور کو دو جگہوں پر کاتتا ہے جو}\,
(-2,0)(7,0)   
\text{اور عمودی مہور کو }  
(0,-56) 
\text{  پر کاتتا ہے }\)
\item \( \text{اَفکی مہور کو دو جگہوں پر کاتتا ہے جو}\,
(-6,0)(-2,0)  
\text{اور عمودی مہور کو }  
 (0,-6) 
 \text{  پر کاتتا ہے }\)
\item \( \text{اَفکی مہور کو دو جگہوں پر کاتتا ہے جو}\,
(-3,0)(2,0)   
 \text{اور عمودی مہور کو }   
 (1,16)
  \text{  پر کاتتا ہے }\)
\item \( \text{اَفکی مہور کو دو جگہوں پر کاتتا ہے جو}\,
(-10,0)(7,0)   
 \text{اور عمودی مہور کو } 
   (8,90 )
   \text{  پر کاتتا ہے })\)\\
\end{enumerate}
%%%%%%%%%%%%%%%%%%%%%%%%%%%%%%%%%%%%%%%%%%%%%%%%%%%%%%%%%%%%%%%%%%%%%%%%%%%%%%

 %PAGE 48-50
\ابتدا{سوال}
درج ذیل ترسیمات بنائیں؛
\begin{multicols}{2}
\begin{enumerate}[a.]
\item \( y=x^{2}-4x-5 \)
\item \( y=4x^{2}-4x+1 \)
\item \( y=-x^{2}-3x+18 \)
\item \( y=2x^{2}-9x+10 \)
\item \(y=-(x^{2}-4x+9  \)
\item \( y=3x^{2}+9x\)
\end{enumerate}
\end{multicols}
\انتہا{سوال}
\ابتدا{سوال}
ذیل میں \عددی{9} قطع مکافی کی مساوات ہیں،
\begin{multicols}{3}
\begin{enumerate}[a.]
\item \(  y=(x-3)(x-8\)
\item \(y=14+5x-x^{2}  \)
\item \(y=-x^{2}-3x+18  \)
\item \( y=x(3-x) \)
\item \(y=(x+2)(x-7)  \)
\item \(y=-3(x+3)(x+7)  \)
\item \( y=x^{2}+2x+1 \)
\item \(y=x^{2}+8x+12  \)
\item \( y=x^{2}-25 \)
\end{enumerate}
\end{multicols}
درج ذیل سوالات کے جواب دین بغیر ترسیمات بناۓ۔
\begin{itemize} 
\item  
کونسا قطع کافی عمودی محور سے \عددی{y  } کی مثبت قیمت پر سے گزرتا ہے۔؟
\item  
کس قطع مکافی کا راس ترسیم کے بلند ترین مقام پر موجود ہے۔؟
\item  
کس قطع مکافی کا راس عمودی محور کے بائیں جانب ہے؟
\item  
کونسا قطع مکافی مبدا سے گزرتا ہے؟
\item  
کونسا قطع مکافی افقی محور کو \عددی{x} کی دو الگ الگ قیمتوں سے نہیں کاٹتا۔
\item  
کس قطع مکافی کے لیے عمودی محور تشاکلی محور کا کام کر رہی ہے؟
\item  
کونسے دو قطع مکافی کی ایک ہی تشاکلی محور ہے؟
\item
کس قطع مکافی کا راس چوتھے کانے  میں ہے؟
\end{itemize}
\انتہا{سوال}
\ابتدا{سوال}
 درج ذیل میں دیے گۓ  ترسیمات کے لیے مناسب مساوات بنائیں۔
\انتہا{سوال}
 

%Miscellaneous Exercise 3
\ابتدا{سوال}

ایک تفاعل کی تعریف کچھ یوں ہے؛\(f(x)=7x-4\)
\begin{enumerate}[a.]
\item  
تفاعل \( f(7)  \)، \( f(\frac{1}{2}) \) اور \(  f(-5) \) کی قیمت معلوم کریں۔
\item  
\عددی{x} کی قیمت معلوم کریں جبکہ \( f(x)=10 \)
\item  
\عددی{x} کی قیمت معلوم کریں جبکہ \( f(x)=x \)
\item  
\عددی{x} کی قیمت معلوم کریں جبکہ \( f(x)=f(37)\)
 \end{enumerate}
 \انتہا{سوال}
\ابتدا{سوال}
ایک تفاعل \( f(x)=x^{2}-3x+5 \) کے لیے \عددی{x} کی دو قیمتیں معلوم کریں جن کے لیے \(      f(x)=f(4)\)
 \انتہا{سوال}
\ابتدا{سوال}
شکل میں مساوات \(  y=x^{n}\) کا ترسیم بنایا گیا ہے، جس میں \عددی{n  } ایک عدد صحیح ہے۔ ہمیں بتایا گیا ہے کہ ترسیم نقاط \(  (2,200)\) اور \(  (2,2000)\) میں سے گزرتا ہے ۔ \عددی{n  } کی قیمت معلوم کریں۔
\انتہا{سوال}
\ابتدا{سوال}
دو تترسیمات \( f(x)=x^{2}-7x+5\) اور \(      y=1+2x-x^{2}\) کا مشترک نقطہ معلوم کریں۔
\انتہا{سوال}
\ابتدا{سوال}
خط مستقیم \( y=2x+3   \) اور  ترسیم \(   y=2x^{2}+3x-7\) کا مشترک نقطہ معلوم کریں۔
\انتہا{سوال}
\ابتدا{سوال}
اس نقطے کے محدد معلوم کریں جس پے خط مستقیم \(3x+y-2=0 \) اور ترسیم \(      y=(4x-3)(x-2)\) آپس میں ملتے ہیں۔
\انتہا{سوال}
\ابتدا{سوال}
ترسیم \(y=(x-4)(x-2) \) اور \(      y=x(2-x)\) کے مشترک نقطے کے محدد معلوم کریں اور دونوں ترسیمات کو بنا کے انکا آپسی تعلق بھی بیان کریں۔
\انتہا{سوال}
\ابتدا{سوال}
ہمیں بتایا گیا ہے کہ \عددی{k} ایک مثبت مستقل عدد ہے، درج ذیل کے ترسیمات بنائیں۔
\begin{multicols}{2}
\begin{enumerate}[a.]
\item \( y=(x+k)(x-2k) \)
\item \( y=(x+4k)(x+2k) \)
\item \(y=x(x-k)(x-5k)  \)
\item \( y=(x+k)(x-2k)^{2} \)
\end{enumerate}
\end{multicols}
\انتہا{سوال}
\ابتدا{سوال}
تفاعل \عددی{f} کی مساوات \(f(x)=ax^{2}+bx+c \) ہے ۔ہمیں بتایاا گیا ہے کہ \( f(0)=6,f(-1)=15\) اور \(   f(1)=1\)،
آپ \عددی{a}،  \عددی{b} اور  \عددی{c  } کی قیمتیں معلوم کریں۔
\انتہا{سوال}
\ابتدا{سوال}
خط \(y=3-4x \)  اور ترسیم \(      y=4(4x^{2}+5x+3)\) کا مشترک نقطہ معلوم کریں۔
\انتہا{سوال}
\ابتدا{سوال}
درج ذیل کی ترسیمات بنائیں
\begin{enumerate}[a.]
\item \(y=(x+4)(x+2)+(x+4)(x-5)  \)
\item \( y=(x+4)(x+2)+(x+4)(5-x) \)
\end{enumerate}
\انتہا{سوال}
\ابتدا{سوال}
تفاعل \عددی{f} کی مساوات\((f(x)=ax+b \)  ہے ۔ہمیں بتایاا گیا ہے کہ\(f(-2)=27 \) اور \( f(1)=15  \) ،
آپ \عددی{ x}،  کی قیمتیں معلوم کریں جب \( f(x)=-5\)۔
\انتہا{سوال}

%faltoooo       \( y=ax^{2}+bx+c \) 

%13
\ابتدا{سوال}
ایک ترسیم جسکی مساوات \( y=ax^{2}+bx+c \)  ہے، افقی محور کے نقطوں  \( (-4,0) \) اور \(  (9,0)\)  سے  گزرتا ہے، ایک اور نقطے \(     (1,120)\) سے بھی گزرتا ہے۔یہ ترسیم عمودی محور کے کس نقطے سے گزرے گا؟
\انتہا{سوال}

%14
\ابتدا{سوال}
ایک ترسیم جسکی مساوات  \( y=ax^{2}+bx+c \)  ہے، اور یہ نقطوں \(     (-1,22)(1,8)(3,10) (-2,p) (q,17)\) سے گزرتا ہے \عددی{p} اور \عددی{q} کی قیمتیں معلوم کریں۔
\انتہا{سوال}\\
%faltoooo       \( y=x^{2}+3x+14 \) 
%15
\ابتدا{سوال}
ترسیمات \( y=2x^{2}+5x\) ، \( y=x^{2}+4x+12\)  اور \(  y=3x^{2}+4x-6 \) میں ایک نقطہ مشترک ہے، اس نقطے کے محدد معلوم کریں۔
\انتہا{سوال}
%16
\ابتدا{سوال}
ترسیمات  \( y=x^{2}-3x+c\)  اور  \( y=k-x-x^{2} \) ایک نقطے \( (-2,12) \)  پر ملتے ہیں،\عددی{c  } اور \عددی{k} کی قیمت معلوم کریں اور وہ دوسرا نقطہ بھی معلوم کریں جو ان میں مشترک ہے۔
\انتہا{سوال}


%faltoooo       \(y=10x-9\) 
%17
\ابتدا{سوال}
اگر ترسیمات   \( y=x^{2}+3x+14 \) ،  \( y=x^{2}+2x+11 \)  اور \( y=px^{2}+px+p \)\ مین ایک مشترک نقطہ ہے، آپ \عددی{p} کی قیمت معلوم کریں۔
\انتہا{سوال}
%18
\ابتدا{سوال}
ایک سیدھی لکیر    \( y=x-1\) ایک ترسیم   \(y=x^{2}-5x-8\)  سے ملتی ہے نقاط  \عددی{A}اور \عددی{B} پر۔ ایک اور ترسیم \(   y=p+qx-2x^{2} \)  بھی انہی نقاط سے گزرتا ہے۔ \عددی{p  } اور \عددی{q} کی قیمت معلوم کریں۔
\انتہا{سوال}


%19  
\ابتدا{سوال}
لکیر \(y=10x-9\)  ایک ترسیم \( y=x^{2}\) سے ملتی ہے ۔ دونون کے مشترک نقطے کے محدد معلوم کریں۔
\انتہا{سوال} 
 
\ابتدا{سوال}
ذیل میں دیے گے ترسیمات کے لیے مساوات تجویز کریں۔
\انتہا{سوال}

\ابتدا{سوال}
ترسیمات   \(y=x^{2}-5x-3\)  اور  \(y=3-5x-x^{2}\)  میں ایک نقطہ مشترک ہے، اس نقطے کو معلوم کریں اور جزر کی شک میں لکھیں۔
\انتہا{سوال}
 
%22   

\ابتدا{سوال}
ایک لکیر  \(y=6x+1\)  ایک ترسیم \(y=x^{2}+2x+3 \)  سے ملتی ہے اور ان میں دو نقاط مشترک ہیں۔ ثابت کریں کہ ایک نقطے کے محدد \( (2-\sqrt{2},13-6\sqrt{2})\) ہیں ، جبکہ دوسرے نقطے کے محدد معلوم کریں۔
\انتہا{سوال}
\ابتدا{سوال}
ثابت کریں کہ ترسیمات  \(y=2x^{2}-7x+14 \)  اور \(     y=2+5x-x^{2}\)صرف ایک ہی نقطے پر ملتے ہیں۔ مزید حساب کتاب کیے بغیر  اور ترسیمات بناۓ بغیر بتائیں کہ درج ذیل ترسیمات میں کتنے مشترک نقاط ہیں؟
\begin{enumerate}[a.]
 \item \( y=2x^{2}-7x+12 \) اور \( y=2+5x-x^{2} \)
\item \( y=2x^{2}-7x+14 \) اور \( y=1+5x-x^{2}  \)
\item \( y=2x^{2}-7x+34 \) اور \( y=22+5x-x^{2} \)
\end{enumerate}
\انتہا{سوال}
 \ابتدا{سوال}
آپ  \(\frac{\abs x}{x} \) کے بارے میں کیا راۓ  رکھتے ہیں اگر ؛
\begin{enumerate}[a.]
\item \(x>0  \)
\item \(x<0  \)
 \end{enumerate}
\انتہا{سوال}
