\باب{مثلث}
%138
اس سبق میں ہم سائن ، کوسائن اور ٹینجنٹ کے بارے میں پڑھیں گے، جب آپ یہ سبق مکمل کر لیں گے تو آپ اس قابل ہوں گے کہ؛
\begin{enumerate}
\item
تمام زاویوں کے لیے سائن ، کوسائن اور ٹینجنٹ کے ترسیموں کی شکل پہچانیں  
\item
خاص زاویوں کے لیے سائن ، کوسائن اور ٹینجنٹ کی قیمتیں معلوم ہوں یا معلوم کرنے کا طریقہ آتا ہو۔
\item
سادہ مثلثی مساوات حل کر سکیں
\item
\(\tan{\theta}^{0}\)اور\(\cos{\theta}^{0}\)،\(\sin{\theta}^{0}\)
کی مماثل کا استعمال آتا ہو۔
\end{enumerate}

\حصہ{\(\cos{\theta}^{0}\)کی ترسیم }

زاویے کی علامت کے طور پر اکثر یونانی زبان کے خط استعمال کیے جاتے ہیں، ہم اس سبق میں \(\theta\) (تھیٹا) اور \(\phi\) (فائ) استعمال کریں گے۔

غالباً آپ نے \(\cos{\theta}^{0}\) پہلے قائم مثلث میں زاویوں کا حساب لگاتے ہوۓ استعمال کیا ہوگا، کہ جب زاویہ صفر سے بڑا اور 90 سے چھوٹا تھا۔ اور پھر آپنے اسے کسی اور مثلث میں استعمال کیا ہوگا جب زاویہ \(0<\theta<180\) تھا۔ تاہم اگر آپکے پاس ایک ترسیم بنانے والا حساب کتاب کا آلہ ہے تو آپ دیکھیں گے کہ یہ \(\cos{\theta}^{0}\) کی ایسی ہی ترسیم بناتا ہے جیسی کہ شکل 10.3 میں بنی ہوئ ہے۔ یہ حصہ \(\cos{\theta}^{0}\) کی تعریف بیان کرتا ہے ہر طرح کے زاویوں کے لیے بیشک وہ مثبت ہوں یو منفی۔

شکل 10.1 میں ایک دائرہ دکھایا گیا ہے جسکا رداس 1 اکائ ہے اور جسکا مبدا O پر ہے۔ x محدد پر ایک زاویہ بناتاے ہوۓ ایک خط OP کھینچیں کہ یہ دائرے کی حد کو چھو لے اور اس نقطے کو P کہہ دیں۔  سے P ایک عمودی خط کھینچیں کہ وہ  OA کو پا لے اور جس نقطے پر وہ خط  OA    کو چھوۓ اس نقطے کو N کہہ دیں ۔ فرض کریں کہ  ON=x  ہے اور  NP=y   ہے 
جبکہ نقطہ   P کے محدد  (x,y) ہیں۔

مثلث ONP کو دیکھیں، تعریف استعمال کرتے ہوۓ \(\cos\theta=\frac{ON}{OP}\) اور ہمیں معلوم ہوتا ہے کہ \(\cos\theta=\frac{x}{1}=x\)۔

نتیجہ \(\cos{\theta}^{0}=x\) دراصل \(\cos{\theta}^{0}\)  کی تعریف کے طور پر استعمال ہو رہا ہے زاویے کی تمام قیمتوں کے لیے۔

آپ اس تعریف کی اثرات دیکھیں گے جب زاویہ 90 کا مضرب ہوگا۔
\ابتدا{مثال}
مثلثی تناسب \(\cos{\theta}^{0}\) کی قیمت معلوم کریں جب ؛
\begin{multicols}{2.}
\begin{enumerate}
\item \(\theta=180\)
\item  \(\theta=270\)
\end{enumerate}
\end{multicols}
\begin{enumerate}
\item
  جب \(\theta=180\)، P ایک نقطہ ہے جسکے محدد(-1,0) ہیں ۔ جیسا کہ x محدد نقطہ P  کا -1 ہے لہٰذہ \(\cos^{0}180=-1\) ہے۔
\item  
جب زاویہ \(\theta=270\) ،P  ایک نقطہ ہے \((0,-1)\) اسی لیے \(\cos270^{0}=0\)
\end{enumerate}
\انتہا{مثال}


%page 139 
جیسے جیسے زاویہ بڑھتا ہے نقطہ P  دائرے کے گرد گھومتا ہے, اور جب  \(\theta=360\) ہوتا ہے نقطہ  P    پورا دائرہ مکمل کر کے دوبارہ   A  پر پہنچ جاتا ہے۔ اور جب زاویہ 360 سے بڑھتا ہے تو نقطہ  P      دوبارہ چکر شروع کر دیتا ہے ۔ یہاں سے ہم بآسانی یہ کہہ سکتے ہیں کہ \(\cos(\theta-360)^{0}=\cos\theta^{0}\) اور جب بھی زاویہ 360 ہوتا ہے \(\cos{\theta}^{0}\)اپنی قیمت دہراتا ہے ۔

اگر زاویہ 0 سے چھوٹا ہو تو \(\theta\) مخالف سمت میں گھومے گا لیکن شروع  A سے ہی ہوگا۔ شکل 10۔2 میں زاویہ -150 دکھایا گیا ہے۔ یعنی اگر  \(\theta=-150\) تو   P    تیسرے خانے میں  ہوگا اور چونکہ P    کا  x محدد منفی ہے لہٰذہ \(\cos(-150)^{0}\) منفی ہوگا۔

حساب کتاب کا ایک آلہ آپکو زاویے کی ہر قیمت کے لیے \(\cos{\theta}^{0}\) کی قیمت دے گا۔ اگر آپکے پاس ترسیم بنانے والا حساب کتاب کا آلہ ہے تو ان قیمتوں کو استعمال کرتے ہوۓ \(\cos{\theta}^{0}\) کی ترسیم بنائیں وہ ایسی ہی دکھے گی جیسی کہ شکل 10.3 میں نظر آ رہی ہے۔

اگر آپ \(\cos{\theta}^{0}\) کی ترسیم بنانا چاہتے ہیں تو آپ کو حساب کتاب کے آلے میں مساوات \(y=\cos x\) ڈالنی ہوگی اور یہ بھی خیال رکھیں کہ حساب کتاب کا آلہ ڈگری موڈ میں ہے۔

کوسائن تفاعل کی ترسیم خود کو دہراتی رہتی ہے۔  تفاعل  کی اس خصوصیت کو دوری خصوصیت کہتے ہیں۔ اور ان تفاعل کا دور وہ کم سے کم وقفہ ہے کہ جس کے لیے تفاعل خود کو دہراتا ہے۔اسی لیے کوسائن کے تفاعل کا دور 360 درجے ہے۔ اور خصوصیت \(\cos(\theta\pm360)^{0}=\cos\theta^{0}\)
کو دوری خصوصیت کہیں گے۔ کئ قدرتی رجحانات بھی دوری خصوصیت دکھاتے ہیں۔ اور اکثر انکی خصوصیات سمجھنے کے لیے کوسائن تفاعل کا ہی استعمال کیا جاتا ہے۔
\ابتدا{مثال}
ایک بندرگاہ میں پانی کی گہرائ میٹرز میں ناپی جاتی ہے اور اس گہرائ کو ماپنے کا کلیہ\(d=6+3\cos30t^{0}\) ہے. جبکہ  t وقت کے لیے ہے جو گھنٹوں میں ناپا جاۓ گا دوپہر کے بعد سے۔ معلوم کریں؛
\begin{enumerate}
\item 
رات کے      پے پانی کی گہرائ معلوم کریں
\item
پانی کی کم سے کم اور ذیادہ سے ذیادہ گہرائ اور یہ کس وقت ہوگی۔
\end{enumerate}
\begin{enumerate}
\item
رات کے \(9.45\)  جب \(t=9.75\) تاکہ  \(d=6+3\cos(30+9.75)=6+3\cos292.5=7.148\dotsc\)
اسی لیے پانی کی گہرائ 7.15 میٹرز ہے۔ اور آپکا جواب 3 معنی خیز ہندسوں تک ہونا چاہیے۔
\item
مستقل  d     کی قیمت ذیادہ سے ذیادہ تب ہو گی جب کوسائن تفاعل کی قیمت 1 ہے۔ اور اسی لیے \(6+3\times1=9\) ۔ اسی طرح کم سے کم قیمت بھی \(6+3\times(-1)=3\)، ذیادہ سے ذیادہ گہرائ 9 میٹر اور کم سے کم گہرائ 3 میٹر ہے ۔پہلی دفعہ جب دوپہر میں یہ واقع وقوع پزیر ہوگا \(30t=360\) اور \(30t=180\) ، جسکا مطلب رات کا درمیان اور شام کے 6 بجے ہے۔
\end{enumerate}
\انتہا{مثال}

% page 140
 \حصہ{\(\sin{\theta}^{0}\)اور\(\tan{\theta}^{0}\) کی ترسیم }

جیسے ہم نے کوسائن کے تفاعل کے لیے ایک شکل  10.1 بنائ اسی کو استعمال کرتے ہوۓ سائن کی تعریف کچھ یوں ہو گی۔
\[\sin\theta=\frac{NP}{OP}=\frac{y}{1}=y\]
کوسائن کی ترسیم کی طرح سائن کی ترسیم (شکل 10.4) دوری ہے، جسکا دورانیہ 360 درجے ہے۔اور اسکی ترسیم بھی -1 اور 1 کے درمیان ہی رہتی ہے۔

اگر آپ شکل 10.1 کی طرف لوٹیں تو آپ دیکھیں گے کہ \(\tan\theta=\frac{NP}{OP}=\frac{y}{x}\)، اور اسے \(\tan{\theta}^{0}\) کی تعریف کی طرح لیا جاتا ہے۔  \(\tan{\theta}^{0}\) کے میدان عمل میں وہ زاویے شامل نہیں ہیں جن کے لیے x    صفر ہو۔ جیسا کہ  \(\theta=\pm90,\pm270\dotsc\)۔ شکل 10.5 میں \(\tan{\theta}^{0}\) کی ترسیم دکھائ گئ ہے۔
 
سائن اور کوسائن کی ترسیم کی طرح ٹینجنٹ کی ترسیم بھی دوری ہے لیکن اسکا دورانیہ 180 ہے ،اسی لیے \(\tan(\theta\pm180)=\tan\theta\)



% 141 to 147 by members
%PAGE 141-143

\ابتدا{تعریف}
 ھم جانتے ھیں کہ \(\cos \theta^o =x \, , \sin \theta^o =y\) ہے اور ھم یہ بھی جانتے ہیں کہ\(\tan \theta^o =\frac{y}{x}\) ہےـ ان تمام  حقائق کو جمع کر یں تو ہم کہہ سکتے ہیں کہ\(\tan \theta^o =\frac{\sin \theta^o}{\cos \theta^o}\)
آپ ، \(\tan \theta^o\) کی متبادل تعریف کے طور پر استمعال کر سکتے ہیں۔ 
\انتہا{تعریف}
\حصہ {چند مثلق تفاعل کی درست قیمیتیں}\شناخت{چن ـ مثلق ـ تفاعل ـ کی ـ درست ـ قیمیتیں}
\ابتدا{تعریف}
صرف چند ہی ایسے زاویے ہیں جن کی درست قیمت عدد صحیح ہے اور جن کے\[\cos 45^o =\frac{1}{\sqrt{2}} اور \,\sin 45^o =\frac{1}{\sqrt{2}}, \, \tan 45^o =1 \] آپ درست معلوم کرسکتے ہیں۔ ان زاویوں میں     ° 30 ,  ° 45 اور ° 60  زیادہ اہم ہیں۔  ° 45  زاویےکی مثلثی  تناسب  معلوم کرنے کے لیے ایک قائمہ زاویہ کے سلتھ مسادی الساقین تکون بتائیں ۔ جس کی اطراف کی لمبائی 1 اکائی ہو۔ جیسا کہ شکل 6 -10  میں ھے وتر کی لمبائ-- ھو گی۔ تب \[\cos 45^o =\frac{\sqrt{2}}{2}, \,\sin 45^o =\frac{\sqrt{2}}{2}, \, \tan 45^o =1 \]
اگر آپ نسب نما کو اسثولالی بنائیں تو\[\cos 45^o =\frac{\sqrt{2}}{2}, \,\sin 45^o =\frac{\sqrt{2}}{2}, \, \tan 45^o =1 \]
-  ° 30 اور ° 60 درجے کی مثلی تناسب معلوم کرنے کے لیے ایک یکرطرفہ مثلث(تکون) بنائیں جس کی اطراف 2 اکائیوں جتنی لمبی ہیں۔ جیسے کہ شکل 7۔10 میں دکھایا گیا ہے۔ راس سے ایک خط عمودی خط کھینچیں جو قائدہ کو دو مساوی حصوں میں تقسیم کر دے۔ اس عمودی خط کی لمبائی \(\sqrt{3}\) اکائیاں ہیں۔ اس عمودی خط نے راس کو بھی دو برابر حصوں میں تقسیم کر دیا ہے۔
\[\cos 60^o =\frac{1}{2}, \, \, \sin 60^o =\frac{\sqrt{3}}{2}, \, \, \tan 60^o =\sqrt{3}; \]
\[\cos 30^o =\frac{\sqrt{3}}{2}, \, \, \sin 30^o =\frac{1}{2}, \, \, \tan 30^o =\frac{1}{\sqrt{3}}=\frac{\sqrt{3}}{3} \]
آپ کو یہ نتائج ازبر ہونے چاہئیں۔
\انتہا{تعریف}
 \ابتدا{مثال}
مندرجہ ذیل کی درست قیمتیں معلوم کریں۔
\begin{multicols}{3}
\begin{enumerate}[aـ]
\item \( \cos 135^o\)

\item \(\sin 120^o\)

\item \(\tan 495^o\)

\end{enumerate}
\end{multicols}
-- شکل 3۔10 کے مطابق ---------
-- شکل 4-10 کے مطابق----------
-- شکل 5۔10 کے مطابق-
\begin{enumerate}[aـ]
\item \(\cos 135^o=-\cos 45^o =-\frac{1}{2}\sqrt{2}\)
\item \(\sin 120^o=\sin 60^o =\frac{1}{2}\sqrt{3}\)
\item \(\cos 495^o=\tan (495-360)^o =\tan 135^o =-\tan 45^o =-1\)
\end{enumerate}
\انتہا{مثال}

                                                                      \quad \quad \quad  \quad    \quad    \quad    \quad \quad   \quad                                 \موٹا{  مشق 10-ا }

1)ذیل میں دیے گئے \(\theta\)  زاویوں کے لیے  4 اعشاری نقطوں تک درست قیمت معلوم کریں(تمام سوالات کی مساوات یہاں لکھیں)

\begin{multicols}{3}
\begin{enumerate}[iـ]
\item \( \cos \theta^o\)

\item \(\sin \theta^o\)

\item \(\tan \theta^o\)

\end{enumerate}
\end{multicols}

\begin{multicols}{4}
\begin{enumerate}[aـ]
\item 25

\item 125

\item 225

\item 325

\item 250-

\item 67.4

\item 124.9

\item 554

\item 225

\end{enumerate}
\end{multicols}



2)ذیل میں دیے گئے تمام تفاعل کی کم اور زیادہ ترین قیمت معلوم کریں۔ نیز--- کی شرح کی وہ کم از کم مثبت قدر بھی معلوم کریں جس پے آپ قیمیتیں معلوم کریں گے۔
\begin{enumerate}[aـ]
\begin{multicols}{2}
\item \( 2+\sin x^o \)
\item \(7-4\cos x^o\)
\item \( 5+8 \cos 2x^o\)
\item \( \frac{8}{\sin x^o}\)
\item \( 9+\sin (4x-20)^o\)
\item \( \frac{30}{11-5\cos \big( \frac{1}{2}x -45\big)^o}\)
\end{multicols}
\end{enumerate}
3) (اس سوال کے لیے حساب و کتاب کے کسی آلے کا استعمال نہ کریں) سوال کے ہر حصے میں اعداد کے مثلثی تفاعل دیے گئے ہیں' باقی تمام اعداد معلوم کریں' \(x,\, 0\leq x \leq 360\)اس شرط کے ساتھ کہ معلوم کیے گئے اعداد کا مثلثی تفاعل دیے گئے تفاعل کے مساوی ہو۔ مثال کے طور پراگر\(\sin 80^o\)دیا گیا ہے تو ہمارا جواب \(x=100\)ہونا چاہیئے کیونکہ \(\sin 100^o =\sin 80^o\)۔
\begin{enumerate}[aـ]
\begin{multicols}{4}
\item \(\sin 20^o\)
\item \(\cos 40^o\)
\item \(\tan 60^o\)
\item \(\sin 130^o\)
\item \(\cos 140^o\)
\item \(\tan 160^o\)
\item \(\sin 400^o\)
\item \(\cos (-30)^o\)
\item \(\tan 430^o\)
\item \(\sin (-260)^o\)
\item \(\cos (-200)^o\)
\item \(\tan 1000^o\)
\end{multicols}
\end{enumerate}
4) (اس سوال کے لیے  بھی حساب و کتاب کے کسی آلے کا استعمال نہ کریں) سوال کے ہر حصے میں اعداد کے مثلثی تفاعل دیے گئے ہیں' باقی تمام اعداد معلوم کریں،  \(x\) \(x,\, -180\leq x \leq 180\) بشرطیکہ کہ معلوم کیے گئے اعداد کا مثلثی تفاعل دیے گئے تفاعل کے مساوی ہو۔ مثال کے طور پر اگر\(\sin 80^o\) دیا گیا ہے تو ہمارا جواب \(x=100\) ہونا چاہیئے کیونکہ \(\sin 100^o =\sin 80^o\)


\begin{enumerate}[aـ]
\begin{multicols}{4}
\item \(\sin 20^o\)
\item \(\cos 40^o\)
\item \(\tan 60^o\)
\item \(\sin 130^o\)
\item \(\cos 140^o\)
\item \(\tan 160^o\)
\item \(\sin 400^o\)
\item \(\cos (-30)^o\)
\item \(\tan 430^o\)
\item \(\sin (-260)^o\)
\item \(\cos (-200)^o\)
\item \(\tan 1000^o\)
\end{multicols}
\end{enumerate}
5) حساب و کتاب کا آلہ استعمال کیے بغیر  درج ذیل کی درست قیمیت معلوم کریں۔
\begin{enumerate}[aـ]
\begin{multicols}{4}

\item \(\sin 135^o \)
\item \( \cos 120^o\)
\item \( \sin (-30)^o\)
\item \( \tan 240^o\)
\item \( \cos 225^o\)
\item \( \tan (-330)^o\)
\item \( \cos 900^o\)
\item \( \tan 510^o\)
\item \( \sin 225^o\)
\item \( \cos 630^o\)
\item \( \tan 405^o\)
\item \( \sin (-315)^o\)
\item \( \sin 210^o\)
\item \( \tan 675^o\)
\item \(\cos (-120)^o \)
\item \(\sin 1260^o \)
\end{multicols}
\end{enumerate}
 6) حساب و کتاب کا آلہ استعمال کیے بغیر وہ کم ترین زاویہ معلوم کریں کہ دی گئی مساوات درست ہو جائیں۔
\begin{enumerate}[aـ]
\begin{multicols}{4}
\item \(\cos \theta^o =\frac{1}{2}\)
\item \(\sin \phi^o =-\frac{1}{2}\sqrt{3}\)
\item \(\tan \theta^o  =-\sqrt{3}\)
\item \(\cos \theta^o =\frac{1}{2}\sqrt{3}\)
\item \(\tan \theta^o  =\frac{1}{3} \sqrt{3}\)
\item \(\tan \phi^o =-1\)
\item \(\sin \theta^o =-\frac{1}{2}\)
\item \(\cos \theta^o =0\)
\end{multicols}
\end{enumerate}


7) حساب و کتاب کا آلہ استعمال کیے بغیر طبیعات مقباس کا حاصل کم ترین زاویہ معلوم کریں کہ مساوات برابر ہو جائیں- (اگر دو زاویہ ہوں تو مثبت کو چنیں)۔
\begin{enumerate}[aـ]
\begin{multicols}{4}
\item \(\cos \theta^o =-\frac{1}{2}\)
\item \(\tan \phi^o =\sqrt{3}\)
\item \(\sin \theta^o  =-1\)
\item \(\cos \theta^o =-1\)
\item \(\sin \phi^o  =\frac{1}{2} \sqrt{3}\)
\item \(\tan \theta^o =- \frac{1}{3}\sqrt{3}\)
\item \(\sin \phi^o =-\frac{1}{2 \sqrt{2}}\)
\item \(\tan \phi^o =0\)
\end{multicols}
\end{enumerate}

8) گودمی میں پانی کی سطح(تقربیا 12 گھنٹے بعد چکر دہراتی ہے اور اس کی مساوات \(D=A+B \sin 30 t^o\) ہے، یہاں \(D\) گہرائی کو ظاہر کرتا ہے اور اس کی اکائی میٹر ہے۔ \(A\) اور   \(D\) حثیت مستقل ہیں۔ t وقت ہے ۔ جیسے کہ گھنٹوں میں ناپا جائے گا اور یہ کام صبح کے 8:00 بجے کے بعد سے شروع ہوا ہے۔
ہمیں معلوم ہوا کہ پانی کی زیادہ سے زیادہ 7.60 میٹر ہے جبکہ کم سے کم گہرائی 2.2 میٹر ہے۔  \(B\)اور \(A\) کی قیمت معلوم کریں' دوپہر کے وقت گودمی میں پانی کی ایک گہرائی ہو گی۔ آپ کا جواب سینٹی میٹر کی حد تک درست بتائیں-

\حصہ{\( \cos \theta^o, \,\sin \theta^o \) اور \(\tan \theta^o\) کی ترانیم کی تشاکل کی خصوصیات}
\ابتدا{تعریف}
اگر آپ \( \cos \theta^o, \,\sin \theta^o \) اور  \(\tan \theta^o\) کی ترانیم کا بغور جائزہ لیں تو آپ ان میں سے تساکل کی خصوصیات کو دستیاب پائیں گے۔ شکل 5۔10 میں \(\cos \theta^o\) کی ترسیم دکھائی گئی ہے۔\(\cos \theta^o\)  کی ترنیم عمودی خط کے ساتھ تشاکل میں ہے۔ اس کا مطلب ہے آپ \(\theta\) کو \(-\theta\) سے بدل دیں تو  ترسیم پر کوئی اثر نہیں پڑے گا۔
\[ \cos (-\theta)^o = \cos \theta^o \]
اس کا مطلب \(\cos \theta^o\) کی ترنیم \(\theta\)
 کا ایک جفت تفاعل ہے۔ (جیسا کہ حصہ 3۔3 میں بیان کیا گیا ہے) 
تشاکل کی دیگر خصوصیات بھی ہیں' مثال کے طور پر شکل 8۔10 میں آپ دیکھ سکتے ہیں کہ اگر آپ تفاعل میں 180 درجے جمع یا منفی کریں تو آپ کے تفاعل کا نشان بدل جائے گا۔ یعنی اگر تفاعل مثبت تھا تو منفی ہو جائے گا جبکہ منفی تفاعل مثبت ہو جائے گا۔
\[\cos (\theta -180)^o =-\cos \theta^o\]
 ہم اسے مستقیم حرقت کی خصوصیات کہتے ہیں۔ 
\انتہا{تعریف}
یہاں ایک مزید کارآمد خصوصیات بھی موجود ہیں۔ جیسے ہم جفت اور اور مستقیم حرکت کی خصوصیات کے ملاپ سے وجود میں لائے۔
\[\cos (180 - \theta)^o =\cos (\theta -180)^o=-\cos \theta^o\]
مثلث میں\(\cos \theta^o\) کا کلیہ استعمال کرتے ہوئے آپ کا اس خصوصیت سے واسطہ پڑا ہو گا ۔   \(\sin \theta^o\) کی ترمیم جو شکل 9۔10 میں دکھائی گئی ہے،کے لیے بھی ایسی ہی خصوصیات ہیں۔ مشق 10-- کے ایک سوال میں آپ ان خصوصیات کے وجود کو ثابت گے۔ ان کو ثابت کرنے کا طریقہ-- کی خصوصیات کو ثابت کرنے کے طریقے سے مماثلت رکھنا ہے۔  \(\cos \theta^o\)   اور  \(\sin \theta^o\)
 کے تفاعل کے خصوصیات درج ذیل ہیں۔
\begin{description}
\item [تواتر کی خصوصیات] \( \sin (-\theta)^o= -\sin \theta^o \), \(\cos (-\theta)^o= \cos \theta^o\)
\item [تاک کی خصوصیات] \(\cos (\theta -180)^o=-\cos \theta^o\),\(\sin (\theta - 180)^o=-\sin \theta^o\)
\item [مستقیم حرکت کی خصوصیات]
\item \(\cos ( \theta \pm 360)^o=\cos \theta^o\)
\item \( \cos (180- \theta)^o=-\cos \theta^o\)
\item  \(\sin (\theta \pm 360)^o=\sin \theta^o \)
\item \( \sin (180-\theta )^o=\sin \theta^o\)
\end{description}



%page 144-145
اگر آپ شکل 10.5میں \(\tan \theta^o\)کی ترسیم کا حوالہ لیں اور\(\sin \theta^o\)اور\(\cos \theta^o\)کی ترسیم کے انداز میں اسکا بھی جائزہ لیں تو آپ کو\(\sin \theta^o\) اور\(\cos \theta^o\) جیسےہی جوابات ملیں گے۔
\(\tan \theta^o\)کے تفاعل کی خصوصیت مندرجہ ذیل ہیں۔
تواتر کی خصوصیت:   \begin{equation*}
\tan (\theta \pm 180)^o=\tan \theta^o
\end{equation*} 
ناک خصوصیت: 
 \begin{equation*}
 \tan (-\theta)^o=-\tan \theta^o 
\end{equation*}
 \begin{equation*}
 \tan (180 -\theta)^o=-\tan \theta^o
\end{equation*}
اس بات پر غور کریں کہ\(\tan \theta^o\) کی ترسیم 180 درجے کے بعد خود کو دہراتی ہے لہذا اس کی مستبقم حرکت کی  خصوصیت اور تواتر کی خصوصیت ایک سی ہیں۔

\ابتدا{مثال}
خصوصیت ثابت کریں کہ:-\( \cos (90- \theta)^o=\sin \theta^o\)۔
یہ آسان ہو جاۓگا اگر وقفہ\(0<\theta <90\) یہ تصور کیا جاۓ۔ ایک قائم زاوے کی حاصل مثلث بنائیں، زاویہ صرف آسانی کے لیے چنا گیا ہے البتہ یہ خصوصیت کسی بھی زاویے کے لیے ثابت کی جاسکتی ہے۔
اگر آپ \(\cos \theta^o\)کے کی ترسیم کو زاویے کے مثبت غور میں 90 درجے مستبقم حرکت دیں تو آپ کو \(\sin \theta^o\)کا ترسیم ملے گا۔لہذاہم کہہ سکتےہیں کہ \( \cos (\theta-90)^o=\sin \theta^o\)اور چونکہ \(\cos \theta^o\)ایک جفت تفاعل ہے\(\cos (90- \theta)^o=\cos (\theta-90)^o\)اسی لیے \(\cos (90- \theta)^o=\sin \theta^o\)ثابت ہوگیا۔
\انتہا{مثال}

مشق 10B میں ایک اور خصوصیت جو آپ کو ثابت کرنی ہوگی وہ \(\sin (90- \theta)^o=\cos \theta^o\)ہے۔
\ابتدا{مشق}
\ابتدا{سوال}
\(\sin \theta^o\), \(\cos \theta^o\) اور\(\tan \theta^o\) کی تشاکل اور تواتر کی خصوصیات استعمال کرتے ہوۓ مندرجہ ذیل نتائج اخز کریں۔
\begin{multicols}{2}
\begin{enumerate}[a.]
\item \(\sin (90 -\theta)^o =\cos \theta^o\)
\item \(\sin (270 +\theta)^o =-\cos \theta^o\)
\item \(\sin (90 +\theta)^o =\cos \theta^o\)
\item \(\cos (90 +\theta)^o =-\sin \theta^o\)
\item \(\tan (\theta-180)^o =\tan \theta^o\)
\item \(\cos (180 -\theta)^o =\cos (180+ \theta)^o\)
\item \(\tan (360 -\theta)^o =-\tan (180+ \theta)^o\)
\item \(\sin (-90 -\theta)^o =-\cos \theta^o\)
\end{enumerate}
\end{multicols}
\انتہا{سوال}
\ابتدا{سوال}
\(y= \tan \theta^o\)
اور \(y=\frac{1}{\tan \theta^o}\) کی ترسیم بنائیں اور انہی عور پر ثابت کریں کہ\(\tan (90 -\theta)^o=\frac{1}{\tan \theta^o}\)-
\انتہا{سوال}
\ابتدا{سوال}
مندرجہ ذیل تمام مساوات کیلۓ \(\alpha\)کی ایسی قیمتیں معلوم کریں کہ جن سے درج ذیل مساوات درست ثابت ہو جائیں۔


\begin{multicols}{2}
\begin{enumerate}[a.]
\item \(\cos (\alpha -\theta)^o =\sin \theta^o\)
\item \(\sin (\alpha -\theta)^o =\cos (\alpha+\theta)^o\)
\item \(\tan \theta^o =\tan (\theta + \alpha)^o\)
\item \(\sin (\theta +2 \alpha)^o =\cos (\alpha -\theta)^o\)
\item \(\cos (2\alpha -\theta)^o =\cos (\theta -\alpha)^o\)
\item \(\sin (5\alpha +\theta)^o =\cos ( \theta -3\alpha)^o\)
\end{enumerate}
\end{multicols}
\انتہا{سوال}
\انتہا{مشق}
\حصہ{مثلثی تفاعل کی مساوات کا حل}
\textbf{\(\cos \theta^o = k\)کی مساوات کا حل}


\(\cos \theta^o = k\)کی مساوات حل کرنے کے لیے فرض کریں کہ ۔\(-1 \leq k \leq 1\)
 اگر \(k\)اس شرط پر پورا اترے تو مساوات کا کوئی حل نہیں ہوگا۔ شکل 10.10 میں\(k\) کی منفی قیمت دکھائی گئی ہے۔ یاد رکھیں ہر 360 درجے کے وقفے میں \(\cos \theta^o = k\)کے دو جزر ہوتے ہیں سواۓ جب \(k= \pm 1\)ہو۔

حساب کتاب کے آلے پر \([\cos^{-1}]\)کابٹن دبائیں تو آپکو وہ زاویہ ملے گا جس سے مساوات درست ثابت ہو گی۔ کچھ آلات پر الٹ کوسائن کا بٹن ہوگا۔ لیکن بدقسمتی سے اس طریقے میں ہمیں صرف ایک جزر ملے گا۔ عموما آپ  دیے گئے وقفے میں  \(\cos \theta^o = k\)کے تمام جزر حاصل کرنا چاہتےہیں۔

 \(\cos \theta^o = k\)کی مساوات کو حل کرنے کے لیے 3 افدام ہیں:-

\begin{enumerate}[a.]
\item
\([\cos^{-1}k]\)معلوم کریں۔
\item
 تشاکل کی خصوصیت استعمال کرتے ہوۓ مزید ایک جزر حاصل کریں۔ تشاکل کی خصوصیت یہ ہے
 \(\cos (-\theta)^o=\cos \theta^o\) ۔

\item
 تواتر کی خصوصیت یعنی \(\cos (\theta \pm 360)^o = \cos \theta^o\) کا استعمال کرتے ہوۓ مزید جزر معلوم کریں۔
\end{enumerate}
\ابتدا{مثال}

مساوات \(\cos \theta^o =\frac{1}{3}\)کو حل کریں اور \(0 \leq \theta \leq 360\)میں آنے والے تمام جزر ایک اعشاری نقطع تک درست معلوم کریں۔


\begin{enumerate}[a.]
\item
حساب کتاب کے آلے کا استعمال کریں اور\(\cos^{-1} \frac{1}{3}= 70.52...\)معلوم کریں کہ یہ بتاۓ گۓ وقفے کا پہلا جزر ہے.
\item
تشاکل کی خصوصیت \(\cos (-\theta)^o=\cos \theta^o\)کا استعمال کریں اور اس خصوصیت سے آپ حاصل کریں گے -70.52چوکہ دوسرا جذر ہے۔ لیکن یہ بتاۓ گۓ وقفے کا حصہ نہیں ہے.
\item
تواتر کی خصوصیت \(\cos (\theta \pm 360)^o = \cos \theta^o\)اور اس سے آپ کو ملے گا-70.52---+360=289.47 اور یہ جزر بتاۓ گۓ وقفے میں ہی ہے۔
\end{enumerate}
لہزا\(0 \leq \theta \leq 360\)اس وقفے میں 70.52 اور 289.5 ایک اعشاری نقطے تک درست جوابات ہیں۔ 

\انتہا{مثال}




 \( -180\leq \theta\leq180 \)
میں مساوات
 \( \cos 3\theta^{\circ} =-\frac{1}{2}\)
 کے تمام جز معلوم کریں۔ یہ مثال بھی پچھلی مثال جیسی ہے۔ فرق صرف اتنا ھے کہ اس میں دو فالتو اقدام ھیں ایک ابتداء میں اور ایک انتہا پہ۔
فرٖٖض کریں  کہ
\( 3\theta=\phi\) 
 اب مساوات
\(\cos \phi^{\circ} = -\frac{1}{2} \) 
کو حل کرنا ھو گا اوراب یہ مساوات کافی حد تک سادہ ہو چکی ھے ۔ لیکن اگر
\( 3\theta=\phi\) 
ھے  تو 
\(\times\left(-180\right)\leq3\theta\leq\times180\)
 اسی لیے اب نیا وقفہ 
\( -540\leq\phi\leq 540 \) 
ھو گا ۔ اس طرح ھم اصل مسلے تک آ پہچے ھیں کہ
 \( \cos\phi^{0} =-\frac{1}{2} \)
 کی مساوات ھل کرتی ہے کچھ اس طرح کہ جوابات اسی وقفے میں ہوں(آپ تقریبا 6 جز کے لیے تیار رہیں)
پہلا قدم
\[ \cos^{-1}( -\frac{1}{2} )=120 \]

دوسرا قدم : دوسرا جز ہو گا 
\(-120\)

تیسرا قدم : تیسری کی خصو صیت کے مطابق دونوں معلام شدہ جز  میں \( 360 \) جمع اور منفی کرتے ہوۓ 

\( -120-360=-480, \, -120+360=240,\,120-360=-240 \) 

\( 120+360=480 \) 


لہزا دیۓ گۓ وقفے میں \( \cos \phi^{\circ} = -\frac{1}{2} \) کے جز \( -480,\,-240,\,-120,\,120,\,240 \) 	\( 480 \) 
 یہ ھیں

اصل مساوات کی طرف لوٹتے ھوۓ

اور یہ \( \theta = \frac{1}{3}\phi \)  حقیقت مدنظر رکھتے ھوۓ اصل جز \( -160,\,-80,\,-40,\,40,\,80 \) \( 160 \) ہوں گے

\( \sin\theta^{\circ}=k \) 
  کی مساوات کا حل 

 \( \sin\theta^{\circ}=k \) 
کی مساوات اگر دیۓ گۓ وقفے میں ہو تو اسی طریقے سے ہی حل ہو گافرق صرف اتنا ہے کے  \( \sin\theta^{\circ} \)  کے لیے تشاکل کی خصوصیت \( \sin\left(180-\theta\right)^{\circ} \) ہے ۔ وقفہ \( -1\leqslant k \leqslant1 \) 
 ہے

قدم 1:  \( \sin^{-1}k \) معلوم کریں

قدم 2:  تشاکل کی خصوصیت \( \sin\left(180-\theta\right)^{\circ}=\sin\theta^{\circ} \)  کو استعمال کرتے ہوۓ دیگر جز معلوم کریں
   
قدم 3: تواتر کی خصو صیت \(\sin\left(\theta\pm360\right)^{\circ}=\sin\theta^{\circ} \) 
 کا استعمال کرتے ہوۓ دیگر جز معلوم کریں
مثال  : 3۔5۔10

 \( -18-\leqslant\theta\leqslant180 \) میں \( \sin\theta^{\circ}=-0.7 \) کے تمام جز ایک اعشاری نقطے تک درست معلوم کریں 

قدم :1 حساب و کتاب کے آلے کا استعمال کرتے ہوۓ \( \sin^{-1}\left(-0.7\right)=-44.42\cdots \) معلوم کریں۔ دی گئ مساوات کا پہلا جز ہے  

قدم : تشاکل کی خصوصیت \( \sin\left(180-\theta\right)^{\circ}=\sin\theta^{\circ} \) کا استعمال کرتے ہوۓ  یہ \( 180-\left(-44.42\cdots\right)=224.42\cdots \) دوسرا جز ھے ۔ بد قسمتی سے یہ بناۓ گۓ وقفے میں نہیں ھے

قدم 3 : تواتر کی خصو صیت\( \sin(\theta\pm360)^{\circ}=\sin\theta^{\circ} \)  کا استعمال کر کے \( 224.42\cdots-360=-135.57\cdots \) حاصل کریں گے یہ جز بناۓ گۓ وقفے میں ہی شامل ہے

مثال : 4۔5۔10

وقفہ : \( 0\leqslant\theta\leqslant360 \) میں مساوات \( \sin\frac{1}{3}\left(\theta-30\right)^{\circ}=\frac{1}{2}\sqrt{3} \) کو حل کریں اور تمام جز معلوم کریں۔

فرض کریں کہ \( \frac{1}{3}\left(\theta-30\right) = \phi \) اور یوں دی گئ مساوات \( \sin\phi^{\circ}=\frac{1}{2}\sqrt{3} \)  سادہ ہو گئ اور اب ہم اس نئ مساوات کے حل تلاش کریں گے

قدم 1 : \( \sin^{-1}\left(\frac{1}{2}\sqrt{3}\right)=60 \) یہ بتاۓ گۓ حصہ میں پہلا جزر ہے

قدم 2: دوسرا جزر  \( 180-60=120 \) لیکن یہ بتاۓ گۓ وقفے مہں نہی آتا۔ 

قدم 3: \( 360 \)  کے مضرب کو جمع نفی کرنے سے بھی ھمیں اس وقفے میں ھمیں مزید جزر نہی ملیں گے 

اسی وجہ سے مساوات\( \sin\phi^{\circ}=\frac{1}{2}\sqrt{3} \)   کا وقفہ \( -10\leqslant\phi\leqslant110 \) میں ایک ہی جز ہے اور وہ ہے \( 60 \) ۔ اصل مساوات کی طرف لوٹتے ہوۓ جبکہ ہم جانتے ہیں کہ \( \theta = 3\phi+30 \) تو مساوات کا اصل  جزر \( \theta =210 \)  ہو گا 

\( \tan\theta^{\circ}=k \) کی مساوات حل کرتے ہوۓ

\( \tan\theta^{\circ}=k \) کی مساوات بھی ویسے ہی حل ہو گی جیسے ہم نے باقی مثلی تناسب کی مساوات کو حل کیا ۔ یہاں یہ بات اہم ہے کہ ہر  \( 180 \) درجے کے وقفے میں صرف ایک ہی جزر  ملے گا اور مزید جزر کے لیے ہمیں تواتر کی خصوصیت کا سہارا لینا پڑتے گا

قدم 1: \( \tan^{-1}k \) معلوم کریں

قدم 2 : تواتر کی خصوصیت \( \tan\left(180+\theta\right)^{\circ}=\tan\theta^{\circ} \)  کا استعمال کرتے ہوۓ دیگر جزر تلاش کریں









% page 148
\ابتدا{سوال}
زاویے کی دو کم سے کم قیمتیں معلوم کریں کہ جن کے لیے درج ذیل مساوات درست ثابت ہوں۔ آپکا جواب ایک اعشاری نقطے تک درست ہونا چاہیے۔
\begin{multicols}{3}
\begin{enumerate}[a.]
\item \( \cos\frac{1}{2}\theta^{\circ}=\frac{2}{3} \) 
\item \( \tan\frac{2}{3}\theta^{\circ}=-3 \) 
\item \( \sin\frac{1}{4}\theta^{\circ}=-\frac{1}{4} \) 
\item \( \cos\frac{1}{3}\theta^{\circ}=\frac{1}{3} \) 
\item \( \tan\frac{3}{4}\theta=0.5 \) 
\item \( \sin\frac{2}{3}\theta^{\circ}=-0.3 \) 
\end{enumerate}
\end{multicols}
\انتہا{سوال}


\ابتدا{سوال}
 بغیر حساب و کتاب کے آلے کی مدد لیے درج ذیل مساوات کے  وقفہ
\( 0 \leqslant t\leqslant360 \) 
میں جذر ( اگر کوئ ہیں تو) معلوم کریں۔
%here is the error
\begin{multicols}{3}
\begin{enumerate}[a.]
\item \(\sin \left(2t-30\right)^{\circ}=\frac{1}{2} \) 
\item \( \tan\left(2t-45\right)^{\circ}= 0 \) 
\item \( \cos\left(3t+135\right)^{\circ}=\frac{1}{2}\sqrt{3} \) 
\item \( \tan\left(\frac{3}{2}t-45\right)^{\circ}=-\sqrt{3} \) 
\item \( \cos \left(2t-50\right)^{\circ}=-\frac{1}{2}  \) 
\item \( \sin\left(\frac{1}{2}t+50\right)^{\circ}=1 \) 
\item \( \cos\left(\frac{1}{5}t-50\right)^{\circ}=0 \) 
\item \( \tan \left(3t-180\right)^{\circ}=-1 \) 
\item \( \sin\left(\frac{1}{4}t-20\right)^{\circ}=0 \) 
\end{enumerate}
\end{multicols}
\انتہا{سوال}

\ابتدا{سوال}
ایک اعشاری نقطے تک    z    کی تمام قیمتیں معلوم کریں ، بشرطیکہ ذیل میں میں دی گئ مساوات درست ثابت ہوں اور تمام قیمتیں اس وقفے \( -180 \leqslant z \leqslant 180 \) میں ہوں۔
%Q7


\begin{multicols}{3}
\begin{enumerate}[a.]
\item \( \sin z^{\circ}=-0.16 \) 
\item \( \cos z^{\circ}\left(1+\sin z^{\circ}\right)=0 \) 
\item \( \left(1-\tan z^{\circ}\right)\sin z^{\circ} = 0 \) 
\item \( \sin z^{\circ}=0.23 \) 
\item \( \cos\left(45+z\right)^{\circ}=0.832 \) 
\item \( \tan\left(3z-17\right)^{\circ}=3 \) 
\end{enumerate}
\end{multicols}

 
\انتہا{سوال}
\ابتدا{سوال}
  وقفے   \( 0 \leq \theta \leq 360 \) میں موجود درج ذیل مساوات کے لیے زاویے \( \theta \) 	کی قیمت معلعم کریں۔%Q8 


\begin{multicols}{3}
\begin{enumerate}[a.]
\item \( \sin 2\theta^{\circ}=\cos 36^{\circ} \) 
\item \( \cos 5\theta^{\circ}=\sin 70^{\circ} \) 
\item \( \tan 3\theta^{\circ}=\tan 60^{\circ} \) 
\end{enumerate}
\end{multicols}


\انتہا{سوال}
\ابتدا{سوال}

وقفے \( 0 \leq \theta \leq \)  میں زاویے کی تمام قیمتیں معلوم کریں جنکے لیے مساوات
 \( \sin\theta^{\circ}\cos\theta^{\circ}=\frac{1}{2}\tan\theta^{\circ} \) درست ثابت ہو۔
\انتہا{سوال}
\ابتدا{سوال}
درجہ ذیل قیمتوں کے لیے مثلثی تفاعل سائن، کوسائن اور ٹینجنٹ کی ایک مثال بنائیں اور اس طرح کے بتائ گی قیمت پے یہ تفاعل خود کو دہراتا ہو۔
%Q10 
\begin{multicols}{3}
\begin{enumerate}[a.]
\item \( 90 \) 
\item \( 20 \) 
\item \( 48 \) 
\item \( 120 \) 
\item \( 720 \) 
\item \( 600 \) 
\end{enumerate}
\end{multicols}
\انتہا{سوال}
\ابتدا{سوال}
وقفے \( 0 \leq \phi \leq 360 \) میں درج ذیل کی ترسیم بنائیں ، ہر ایک سوال میں تفاعل کے دورانیے کا بھی بتائیں ۔ 

\begin{multicols}{3}
\begin{enumerate}[a.]
\item \( y=\sin 3\phi^{\circ} \) 
\item \( y=\cos 2\phi^{\circ} \) 
\item \( y=\sin 4\phi^{\circ} \) 
\item \( y=\tan \frac{1}{3}\phi^{\circ} \) 
\item \( y=\cos \frac{1}{2}\phi^{\circ} \) 
\item \( y=\sin\left(\frac{1}{2}\phi+30\right)^{\circ} \) 
\item \( y=\sin\left(3\phi-20\right)^{\circ} \) 
\item \( y=\tan 2\phi^{\circ} \) 
\item \( y=\tan \left(\frac{1}{2}\phi+90\right)^{\circ} \) 
\end{enumerate}
\end{multicols}

\انتہا{سوال}
\ابتدا{سوال}
قطب شمالی کے ایک مخصوص علاقے میں  پورے سال کے تمام  دنوں  میں روشن گھنٹے \( d \)  معلوم کرنے کا کلیہ \( d=A+B\sin kt^{\circ} \) جسمیں A,B,        اور  k
مثبت مستقل ہیں اور \( t \)  دن میں وقت ہے موسم بہار کے بدلاؤ کے بعد سے۔
\begin{enumerate}
\item        یہ تصور کرتے ہوۓ کہ دن میں روشن گھنٹوں کی عددی قیمت 365 دنوں بعد خود کو دہراتی ہے
k ۔  کی قیمت معلعم کریں  
آپ کا جواب 3 اعشاری نقطوں تک درست ہو۔
\item        یہ بتایا گیا ہے کہ سب سے چھوٹے دن میں 6 گھنٹے روشن جبکہ سب سے لمبے دن میں 18 روشن گھنٹے ہیں 
AاورB
کی قیمت معلوم کریں۔ سال کے نۓ دن میں روشن وقت کتنا ہوگا گھنٹوں اور منٹوں میں بتائیں یہ مانتے ہوۓ کہ سال کا نیا دن موسموں کی اس تبدیلی سے 80 دن پہلے آتا ہے۔
\item
اسی علاقے میں ایک قصبہ ہے جہاں کے لوگ سال میں سو دفعہ تہوار مناتے ہیں اور ان دونوں دن روشن دن 10 گھنٹے کا ہوتا ہے۔                    موسموں کے تغیر کو مد نظر رکھتے ہوۓ بتائیں کہ یہ کونسے دو دن ہیں
\end{enumerate}

\انتہا{سوال}



%Page 150 




\حصہ{مثلثی تفاعل کے باہمی روابط}
الجبرا میں مساوات حل کرنا  آپ کی عادت بن جاتی ہے، جن میں ہم ایک نا معلوم غیر مستقل مقدار ، جسے ہم عموماً \( x \) ، کہتے ہیں ، کی قیمت معلعم کرتے ہیں جیسے اس مساوات میں \( 2x+3-x-6=7 \) ۔ آپ الجبرائ مساوات کو سادہ کرنے میں بھی میارت رکھتے ہیں جیسے مساوات \( 2x+3-x-6 \) سادہ ہو کے \( x-3 \) بن جاتی ہے، آپ کو اندازہ نہیں ہوا لیکن یہ دونوں بالکل الگ طریقہ کار ہیں۔

جب آپ مساوات \( 2x+3-x-6=7 \)  کو حل کرتے ہیں تو آپ کو معلوم ہوتا ہے کہ اسکا صرف ایک ہی حل ہے  \( x=10 \) ، لیکن 
\( x-3 \) 
اور \( 2x+3-x-6 \) بالکل ایک جیسے ہیں \( x \) کی تمام قیمتوں کے لیے، بعض اوقات ان دونوں طرح کی  صورتحال میں فرق کرنا ضروری ہوتا ہے۔

اگر دو تراکیب \( x \) کی ہر قیمت کے لیے ایک سا جواب دیں تو ایسی تراکیب کو ہو بہو برابر کہا جاۓ گا۔ اور ایسی تراکیب کو ظاہر کرنے کے لیے (    )  علامت استعمال کی جاتی ہے اور اسے پڑھا جاۓ گا "ہو بہو برابر ہے". یہ جملہ \[ 2x+3-x-6=x-3 \]
ایک مماثل کہلاۓ گا۔ لہٰذہ \( x \)  میں ایک مماثل ایک ایسی مساوات ہے جو\( x \)  کی تمام قیمتوں کے لیے درست ہے۔

مثلثی تناسب میں بھی ایسا ہی ہوتا ہے، حصہ 10.2 کے آخر میں یہ دیکھا گیا تھا کہ \( \tan\theta^{\circ}=\frac{\sin\theta^{\circ}}{\cos\theta^{\circ}} \) بشرطیکہ 
\( \cos\theta^{\circ} \neq 0 \) ۔

\[ \tan\theta^{\circ}=\frac{\sin\theta^{\circ}}{\cos\theta^{\circ}} \]

مماثل کی علامت استعمال کی جاتی ہے تب بھی جبکہ قوت نمائ قیمتیں موجود ہوں جنکے لیے دونوں اطراف معین نہ ہوں، دہ گئ مثال میں اگر زاویہ 90 کا تاک مضرب ہو تو کوی بھی طرف معین نہیں ہے لیکن مماثل کی علامت وہاں موجود ہے۔


حصہ 10.1 اور 10.2 میں کی گئ 
\( \cos\theta^{\circ}=x \)  اور \( \sin\theta^{\circ}=y \) 
کی تعریف سے ایک اور تعلق فوراً سے ذہن میں آتا ہے 
اگر  P  ایک  اکائ کے ایک دائرے کی باہری حد بندی پر موجود ایک نقطہ ہے ۔ فیثا غورث کے قانون کے مطابق \( x^2 = y^2 = 1 \) ہے یا ہم کہہ سکتے ہیں کہ  \( \left(\cos\theta^{\circ}\right)^{2} + \left( \sin\theta^{\circ}\right)^{2} = 1 \) 

غلط العام میں ہم \( \left(\cos\theta^{\circ}\right)^{2} \)  کو 
 \( \cos^{2}\theta^{\circ} \)  
کہتے ہیں اور ایسے ہی \( \left(\sin\theta^{\circ}\right)^{2} \)  کو 
\( \sin^{2}\theta^{\circ} \) 
کہتے ہیں , 
زاویے کی ہر قیمت کے لیے \( \cos^{2}\theta^{\circ}+\sin^{2}\theta^{\circ}\equiv 1 \)۔ ہم اسے بعض اوقات مثلثیات کا فیثاغورث کا کلیہ بھی کہتے ہیں۔

زاویے کی ہر قیمت کے لیے؛
\( \tan\theta^{\circ}\equiv\frac{\sin\theta^{\circ}}{\cos\theta^{\circ}} \) 
بشرطیکہ \( \cos\theta^{\circ} \neq 0 \) 

\[ \cos^{2}\theta^{\circ}+\sin^{2}\theta^{\circ}\equiv1 \] 

غلط العام  \( \cos^{n}\theta^{\circ} \)  جسکا ہم نے ذکر کیا 
یہ مثبت طاقتوں کی حد تک تو بہترین ہے ۔ کسی بھی صورت میں \( n=-1 \) استعمال نہیں کیا جا سکتا کیونکہ یہاں ایک خطرہ ہے آپ اسے یہ 
\( \cos^{-1}x \) 
سمجھ سکتے ہیں، جبکہ یہ ان زاویوں کے لیے استعمال ہوتا ہے جنکے cosine کی قیمت   x
 ہوتی ہے۔ اگر آپ شک میں گرفتار ہوں تو \( \left( \cos\theta^{\circ} \right)^{n} \)  یا 
\( \left(\cos\theta\right)^{-n} \) 
استعمال کریں کیونکہ انکا ایک ہی مطلب ہے جو واضع ہے
% page 151

آپ اس مساوات \(\cos^2\theta\degree+\sin\theta\degree \equiv1\)\\ کو استعمال کرتے ہوۓ کسی بھی مثلث کے کوسائن کلیے کو ثابت کر سکتے ہیں۔

فرض کریں ABC ایک  مثلث ہے جسکی اطراف ،   CA=b  ،BC=a
اور  AB=c
ہیں ۔ فرض کریں کہ نقطہ A کارتیسی نظام محدد کے مبدا پے ہے۔ اور   AC  ایک خط ہے جو کہ x محدد پے x کی  سمت میں ہے ۔ جیسا کہ شکل 10.11 میں دکھایا گیا ہے۔ 

نقطہ   C کے محدد  (b,0) ہیں، جبکہ B کے محدد  \((c\cos A\degree,c\sin A\degree)\)    یہ ہیں ، جبکہ A  زاویے   BAC  کے لیے ہے۔ اور تب فاصلے کے کلیے کا استعمال کرتے ہوۓ  
\begin{eqnarray*}
a^{2} &=&(b-c\cos A\degree)^2+(c\sin A\degree)^2\\ 
&=& b^2-2bc\cos A\degree+c^2 \cos^2 A\degree +c^2\sin^2 A\degree\\
&=&b^2-2bc\cos A\degree +c^2 (\cos^2 A\degree+\sin A\degree)\\
&=&b^2+c^2-2bc\cos A\degree,
\end{eqnarray*}
 اب آخر میں \(\cos^2 A\degree+\sin A\degree=1\) کا استعمال کرتے ہوۓ۔

\ابتدا{مثال}
 بتایا گیا ہے کہ \(\sin\theta\degree=\frac{3}{5}\) اور زاویہ منفرجیہ ہے۔ حساب و کتاب کے آلے سے  پرہیز کرتے ہوۓ \( \cos \theta^{0}\)  اور   \( \tan \theta^{0}\) کی قیمت معلعم کریں۔

جیسا کہ \(\cos^2\theta\degree+\sin^2\theta\degree = 1, \cos^2\theta\degree=1-\big(\frac{3}{5})^2=\frac{16}{25}\) اور اس سے ہمیں ملے گا  \(\cos\theta\degree=\pm\frac{4}{5}\)۔ جیسا کہ ہم جانتے ہیں زاویہ منفرجیہ ہے . \( 90\l \theta \l 180\) لہٰذہ \( \cos \theta^{0}\) منفی ہے ، اسی لیے \(\cos\theta\degree=-\frac{4}{5}\)۔

جیسا کہ \(\sin\theta\degree=\frac{3}{5}\) اور \(\cos\theta\degree=-\frac{4}{5},\tan\theta\degree=\cfrac{\sin\theta\degree}{\cos\theta\degree}=\cfrac{3/5}{-4/5}=-\frac{3}{4}\)
\انتہا{مثال}

\ابتدا{مثال}
مساوات \(3\cos^2\theta\degree+4\sin\theta\degree=4\) کو حل کریں اور وقفہ 
\(-180\l \theta \le 180\)
میں آنے والے تمام جذر ایک اعشاری قیمت تک درست معلوم کریں۔

جیسا کہ نظر آ رہا ہے ہم اس مساوات کع حل نہیں کر سکتے لیکن اگر یم اس مساوات میں\(\cos^2\theta\degree\)  کو  \(1-\sin^2\theta\degree\) سے بدل دیں تو، ہمیں نئ مساوات \(3(1-\sin^2\theta\degree)+4\sin\theta\degree=4\) ملے گی جو کہ مزید سادہ ہو کہ درج ذیل شکل اختیار کر لے گی؛
\[3\sin^2\theta\degree-4\sin\theta\degree+1=0\]
یہ \( \sin \theta^{0}\)  میں ایک دو طاقتی مساوات ہے جس کے  آپ اجزاۓ ضربی بنا سکتے ہیں\((3\sin\theta\degree-1)(\sin\theta\degree-1)=0\)اور اس سے ہمیں ملے گا\(\sin\theta\degree=\frac{1}{3}\) یا \(\sin\theta\degree=1\)

ایک جذر تو \(\sin^{-1}\frac{1}{3}=19.47\ldots,\) ہے اور باقی جذر \( \sin \theta^{0}\)   کی تشاکل کی خصوصیت کی مدد سے جو ہمیں ملے ہیں وہ ہیں 
\( (180-19.47\ldots)=160.52\ldots\)
مساوات \(\sin\theta\degree=1\) کا اکلوتا جذر \(\theta=90,\) ہے، لہٰذہ تمام جذر 19.5،90 اور 160.5 ہیں ۔
\انتہا{مثال}

%page 152

\ابتدا {سوال}
نیچے بنی ہر ایک مثلث کے لیے 
\begin{enumerate}
\item  
فیثا غورث کے کلیے کا استعمال کریں اور تیسری سمت کی لمبائ معلوم کریں۔
\item
\( \sin \theta^{0}\)   ،\( \cos \theta^{0}\)   
اور \( \tan \theta^{0}\) کی درست قیمتیں معلعم کریں ۔
\end{enumerate}
\انتہا{سوال}

\ابتدا{سوال}
\begin{enumerate}
\item
یہ بتایا گیا ہے کہ زاویہ  A  ایک منفرجیہ زاویہ ہے اور یہ کہ \(\sin A\degree = \frac{5}{14}\sqrt{3}\) آپ \( \cos A  ^{0}\)  کی درست قیمت معلعم کریں۔
\item

ہمیں وقفہ   \(180\l B \le 360\) معلوم ہے اور ہم جانتے ہیں کہ \( \tan B\degree=-\frac{21}{20}\)   آپ \( \cos B  ^{0}\)  کی قیمت معلعم کریں۔
\item
  \( \sin C  ^{0}\)  کی وہ تمام قیمتیں معلوم کریں جن کے لیے  \(\cos C\degree=\frac{1}{2}\)

\item
کی وہ تمام قیمتیں معلوم کریں جن کے لیے اس وقفے  \(-180<D<180\) میں مساوات \(\tan D\degree=5\sin D\degree\) درست ثابت ہو۔                                                                                                                                                                                                                                                                                                                                                                                                                                                                                                                                                                                                                                                                                                                                                                                                                                                                                                                                                                                                                                                                                                                                                                                                                                                                                                                                                                                                                                                                                                                                                                                                                                                                                                                                                                                                                                                                                                                                                                                                                                                                                                                                                                                                                                                                                                                                                                                                                                                                                                                                                                                                                                                                                                                                                                                                                                                                                                                                                                                                                                                                                                                                                                                                                                                                                                                                                                                                                                                                                                                                                                                                                                                                                                                                                                                                                                                                                                                                                                                                                                                                                                                                                                                                                                                                                                                                                                                                                                                                                                                                                                                                                                                                                                                                                                                                                                                                                                                                                                                                                                                                                                                                                                                                                                                                                                                                                                                                                                                                                                                                                                                                                                                                                                                                                                                                                                                                                                                                                                                                                                                                                                                                                                                                                                                                                                                                                                                                                                                                                                                                    
\end{enumerate}
\انتہا{سوال}

\ابتدا {سوال}
اور اس مساوات  \(\cos^2\theta\degree+\sin^2\theta\degree \equiv1\) اور اس مساوات \(\tan \theta\degree\equiv\cfrac{\sin\theta\degree}{\cos\theta}\)کا استعمال کریں بشرطیکہ\(\cos\theta\degree \neq 0 \) اور نیچے دی گئ مساوات کو ثابت کریں۔
\begin{multicols}{2.}
\begin{enumerate}[a.]
\item
\(\cfrac{1}{\sin\theta\degree}-\cfrac{1}{\tan\theta\degree}\equiv\cfrac{1-\cos\theta\degree}{\sin\theta\degree}\)
\item
 \(\cfrac{\sin^2\theta\degree}{1-\cos\theta\degree}\equiv 1+\cfrac{1}{\cos\theta\degree}\)
\item
\(\cfrac{1}{\cos\theta\degree}+\tan\theta\degree\equiv\cfrac{\cos\theta\degree}{1-\sin\theta\degree} \)
\item
\(\cfrac{\tan\theta\degree\sin\theta\degree}{1-\cos\theta\degree}\equiv1+\cfrac{1}{\cos\theta\degree} \)
\end{enumerate}
\end{multicols}
\انتہا{سوال}
\ابتدا{سوال}
دی گئ تمام مساوات کو زاویے کی قیمت کے لیے حل کریں ، اور وقفے  \(  0\le \theta\le 360\) میں زاویے کے جوابات دیں اس بات کو خیال رکھتے ہوۓ کہ آپکے جوابات 0.1 کے قرئب ترین درست ہوں۔ 
\begin{multicols}{2.}
\begin{enumerate}[a.]
\item     \(4\sin^2\theta\degree-1=0\)
\item      \(\sin^2\theta\degree+2\cos^2\theta\degree=2 \)
\item \(10\sin^2\theta\degree-5\cos^2\theta\degree+2=4\sin\theta\degree \)
\item    \(4\sin^2\theta\degree\cos\theta\degree=\tan^2\theta\degree\)
\end{enumerate}
\end{multicols}
\انتہا{سوال}
\ابتدا{سوال}
 دیے گۓ وقفے 
\(-180\le \theta\le 180\) میں   زاویے کی قیمتیں معلوم کریں کہ جن کے لیے  \(2\tan\theta\degree-3=\cfrac{2}{\tan\theta\degree}\)۔
\انتہا{سوال}

% miscellaneous exercise 10

\ابتدا{سوال}
درج ذیل کی دہرائ کا نقطہ معلوم کریں
\begin{multicols}{2.}
\begin{enumerate}[a.]
\item \(\sin x\degree\)
\item   \(\tan2x\degree \)
\end{enumerate}
\end{multicols}
\انتہا{سوال}
\ابتدا{سوال}
\( y=\cos x^{0} \)
کی ترسیم کو ذہن میں رکھتے ہوۓ یا پھر درج ذیل کو \( \cos x^{0} \) کی صورت میں لکھیں 
\begin{multicols}{2.}
\begin{enumerate}[a.]
\item  \(\cos(360-x)\degree \)
\item    \(\cos(x+180)\degree \)
\end{enumerate}
\end{multicols}
\انتہا{سوال}
\ابتدا{سوال}
مساوات \(y=\cos\frac{1}{2}\theta\degree \) کی ترسیم بنائیں اور وقفے \(-360\le \theta\le 360\) میں زاویے کی قیمت معلوم کریں۔ ان نقطوں کے محدد بھی واضع کریں کہ جن پے ترسیم  \( \theta   \) اور \(   y \) محدد کو کاٹے گا۔
\انتہا{سوال}
\ابتدا{سوال}
درج ذیل مساوات کو زاویے کے لیے حل کریں . آپکا جواب وقفے \(0\le \theta\le 360\) میں ہونا چاہیے 
\begin{multicols}{2.}
\begin {enumerate}[a.]
\item  \(\tan\theta\degree=0.4 \)
\item    \(\sin2\theta\degree=0.4\)
\end{enumerate}
\end{multicols}
\انتہا{سوال}


% page    153

\ابتدا{سوال}
مساوات \(3\cos2x\degree=2 \) کو حل کریں اور وقفے \(0\le \theta\le 180\) میں تمام جوابات تحریر کریں۔ آپکے جوابات 0.1 کے قریب ترین ہونے چاہئیں۔
\انتہا{سوال}
\ابتدا{سوال}
\begin{enumerate}
\item
ایک ایسے مثلثی تفاعل کی مثال دیں جو ہر 180 درجے بعد خود کو دہراتا ہو۔
\item
مساوات  \(\sin3x\degree=0.5 \) کو  وقفے \(0\l x\l 180\) میں آنے والے   x کے تمام جوابات معلوم کریں۔
\end{enumerate}
\انتہا{سوال}
\ابتدا{سوال}
وقفے \(0\leq\theta\leq360\) میں زاویے کی وع تمام قیمتیں معلوم کریں کہ جن کے لیے مساوات  \(2\cos(\theta+30)\degree \) درست ثابت ہو۔
\انتہا{سوال}
\ابتدا{سوال}
\begin{enumerate}
\item
مساوات \(\sin2x\degree+\cos(90-2x)\degree \) کو کسی ایک مثلثی تفاعل کی صورت میں لکھیں۔
\item
وقفے \(0\leq x \leq360\) میں مساوات \(\sin2x\degree+\cos(90-2x)\degree =-1\)کی  x کی تمام قیمتیں معلوم کریں۔

\end{enumerate}
\انتہا{سوال}

\ابتدا{سوال}
زاویہ A کی وہ کم ترین قیمت معلوم کریں کہ جس کے لیے
\begin{multicols}{2.}
\begin{enumerate}[a.]
\item
\(\sin A\degree=0.2\)
اور \(\cos A\degree\) منفی ہوں۔
\item
 \(\tan A\degree=-0.5 \)
اور \(\sin A\degree \) منفی ہوں۔
\item
\(\cos A\degree=\sin A\degree \)
دونوں منفی ہوں۔
\item
\(\sin A\degree=-0.2275 \)
اور \(A>360 \)۔
\end{enumerate}
\end{multicols}
\انتہا{سوال}
\ابتدا{سوال}
درج ذیل مماثل کو ثابت کریں۔
\begin{multicols}{2.}
\begin{enumerate}[a.]
\item   \(\cfrac{1}{\sin\theta\degree}-\sin\theta\degree\equiv\cfrac{\cos\theta\degree}{\tan\theta\degree} \)
\item  \(\cfrac{1-\sin\theta\degree}{\cos\theta\degree}\equiv\cfrac{\cos\theta\degree}{1+\sin\theta\degree}\)
\item    \(\cfrac{1}{\tan\theta\degree}+\tan\theta\degree\equiv\cfrac{1}{\sin\theta\degree\cos\theta\degree} \)
\item     \(\cfrac{1-2\sin^2\theta\degree}{\cos\theta\degree+\sin\theta\degree}\equiv \cos\theta\degree-\sin\theta\degree \)
\end{enumerate}
\end{multicols}
\انتہا{سوال}
\ابتدا{سوال}
درج ذیل تفاعل کے لیے y کی کم ترین اور ذیادہ ترین قمتیں جبکہ x کی کم ترین مثبت قیمت معلوم کریں کہ جس کے لیے یہ تفاعل درست ثابت ہوں۔
\begin{multicols}{2.}
\begin{enumerate}[a.]
\item  \(y=1+\cos2x\degree \)
\item   \(y=5-4\sin(x+30)\degree \)
\item     \(y=29-20\sin(3x-45)\degree \)
\item  \(y=8-3\cos^2x\degree \)
\item   \(y=\cfrac{12}{3+\cos x\degree} \)
\item  \(y=\cfrac{60}{1+\sin^2(2x-15)\degree} \)
\end{enumerate}
\end{multicols}

\انتہا{سوال}
\ابتدا{سوال}
درج ذیل مساوات کو زاویے کے لیے حل کریں اور آپنا جواب اس وقفے \(0\le x\le 360\)  میں دیں۔
\begin{multicols}{2.}
\begin{enumerate}[a.]
\item   \(\sin\theta\degree=\tan\theta\degree \)
\item    \(2-2\cos^2\theta\degree=\sin\theta\degree \)
\item     \(\tan^2\theta\degree-2\tan\theta\degree=1 \)
\item    \(\sin2\theta\degree-\sqrt{3}\cos2\theta\degree=0 \)
\end{enumerate}
\end{multicols}
\انتہا{سوال}
\ابتدا{سوال}
t 
کا تفاعل \(t(x)=\tan3x\degree \) ہے۔
\begin{enumerate}
\item
تفاعل  کب t(x)  خود  کو  دہراۓ گا۔ 
\item
 وقفے   \(0\leq x\leq180\)  کے لیے  مساوات  \(t(x)=\frac{1}{2} \) حل کریں 
\item
درج ذیل مساوات کے لیے کم سے کم مثبت حل تلاش کریں۔
\begin{enumerate}
\item  \(t(x)=-\frac{1}{2}\)
\item  \(t(x)=2\)
\end{enumerate}

\end{enumerate}
\انتہا{سوال}

%page 154
\ابتدا{سوال}
درج ذیل مسائل میں سے ھر ایک کے لیے ایک مثلثی تفاعل بنائیں جس سے بتائ گئ صورت حال واضع ہو سکے۔
\begin{enumerate}
\item
 ایک نہر میں پانی کی گہرائ کم سے کم 3.6 میٹر اور ذیادہ سے ذیادہ 6 میٹر کے درمیان تبدیل ہوتی رہتی ہے 24 گھنٹے کے اوقات میں۔
\item
ایک کیمیائ کارخانے جو کہ دس دن کے وقفے میں کام کرتا ہے ، دن میں کم سے کم 1500 بیرل تیل صاف کرتا ہے جبکہ ذیادہ سے ذیادہ 2800 بیرل صاف کر پاتا ہے۔
\item
دائرہ قطب شمالی کے جنوب کے کچھ قصبوں میں روشن دن 2 سے 22 گھنٹوں کا ہوتا ہے 360 دنوں کے ایک مدار میں۔
\end{enumerate}
\انتہا{سوال}

\ابتدا{سوال}
\انتہا{سوال}
\ابتدا{سوال}
ایک فولادی دوشاخہ مرتعش ہے۔ اسکی ایک شاخ کے آخری سرے کا ہٹاؤ  y رکی ہوئ حالت سے ذیادہ سے ذیادہ ہٹاؤ تک وقت t میں بیان کرنے کے لیے کلیہ ہے۔\[y=0.1 \sin(100000t)\degree \]
معلوم کریں؛
\begin{enumerate}
\item
 سب سے ذیادہ ہٹاؤ اور کس وقت یہ وقوع پزیر ہوگا۔
\item    
 ایک مکمل چکر کے لیے کتنا وقت لگے گا۔
\item    
ایک سیکنڈ میں کتنے دائرے مکمل کرے گا فولادی دوشاخے کا  ارتعاش۔
\item  
 پہلے مکمل دائرے کے دوران وہ وقت بتائیں کہ جب فالادی دوشاخے کا دوسرا سرا اپنی رکی ہوئ حالت سے 0.06 سینٹی میٹر ہٹتا ہے۔ 
\end{enumerate}
 \انتہا{سوال}
\ابتدا{سوال} 

ایک لچک دار رسی کا ایک کنارہ ایک چوکھٹ سے باندھا گیا ہے جبکہ دوسرا سرا لٹک رہا ہے۔ کھلے سرے پر ایک چھوٹی سی گیند بندھی ہوئ ہے۔ اس لٹکتی ہوئ گیند کو تھوڑا سا نیچے کھینچا جاتا ہے اور پھر چھوڑ دیا جاتا ہے، اس سے بال اس اس لچک دار رسے پر اوپر نیطے مرتعش ہو جاتی ہے۔ گیند کی گہرائ چوکھٹ سے   d وقت t کے بعد اس کلیے کی مدد سے معلوم کی جا سکتی ہے
\[d=100+10\cos500t\degree\]
معلوم کریں کہ؛
\begin{enumerate}
\item   گیند کی ذیادہ سے ذیادہ اور کم سے کم گہرائ
\item  وہ وقت جب گیند اپنے اونچے ترین مقام پے ہوگی۔
\item   ایک مکمل ارتعاش کے لیے درکار وقت۔
\item  ایک ارتعاش میں  وقت کا وہ حصہ کہ جسکے لیے رسی کی لمبائ 99 سینٹی میٹر سے کم رہتی ہے 
\end{enumerate}
\انتہا{سوال}
\ابتدا{سوال} 
ایک مرتوش ذرے کا ہٹاؤ y ہے، جوکہ میٹرز میں ماپا جاتا ہے اور جسکے لیے تفاعل  \(y=a\sin(kt+\alpha)\degree \) ہے۔ جسمیں a میٹرز میں ، وقت t سیکنڈز میں جبکہ k اور \(\alpha \) دونوں مستقل ہیں ۔ ایک مکمل ارتعاز کے لیے وقت   T  سیکنڈز ہے۔
معلوم کریں کہ؛
\begin{enumerate}
\item   مستقل   k کو  T  کی اکایئوں میں
\item   ایک سیکنڈ میں مکمل ہونے والی دائروی ارتعاش،   k کی اکائیوں میں۔
\end{enumerate}
\انتہا{سوال}

%page 155

\ابتدا{سوال}
ایک جزیرے پر ایک خاص قسم کے پرندوں کی آبادی P تبدیل ہوتی رہتی ہے، اور یہ منحصر کرتی ہے ان کی خوراک، ہجرت، موسم اور شکار پر۔ ایک ماہر ارضیات جو ان پر تحقیق کر رہا تھا اسنے سال میں انکی آبادی کے لیے ایک کلیہ بنایا  \[P=N-C\cos\omega t\degree, \]
اس کلیے میں N،C اور \(\omega \)   مستقل   ہیں۔ جبکہ   t   وقت ہے جسکی اکاے ایک ہفتہ رکھی گئ ہے اور یہ وقت صفر سے شروع ہو رہا ہے یعنی یکم جنوری رات 12 بجے سے۔
\begin{enumerate}
\item  
   فرض کریں کہ تفاعل خود کو 50 ہفتوں بعد دہراتا ہے   \(\omega \)    کی قیمت معلوم کریں
\item
مساوات کا استعمال کریں اور Nاور  C  کی اکائیوں میں جواب دیں
\begin{enumerate} 
\item 
سال کے شروع میں  اس نسل کے کتنے پرندے پاۓ جاتے ہیں 
\item
اس نسل کے پرندوں کی ذیادہ سے آبادی اور یہ سال کے کس حصے میں پائ جاۓ گی
\end{enumerate}
\end{enumerate}

\انتہا{سوال}
\ابتدا{سوال}
صحرا کے قریبی ایک جزیرے تک جانے والی سڑک اکثر پانی سے ڈھکی ہوتی ہے۔ سمندر کا پانی جب سڑک کے برارب آتا ہے تو سڑک بند ہو جاتی ہے۔ ایک خاص دن پانی کی سطح سمندر سے بلندی 4.6 میٹرز ہے.   لہر کی بلندی  h  بیان کرنے کی لیے یہ  \(h=4.6\cos kt\degree \) کلیہ استعمال کیا جا سکتا ہے۔ وقت  t  سے ظاہر کیا گیا ہے اور یہ وہ وقت ہے جو شروع ہوتا ہے اونچی لہر کے آنے کے بعد سے۔ اور یہ بھی دیکھنے میں آیا ہے کہ اونچی لہر 12 گھنٹے میں ایک بار آتی ہے۔
\begin{enumerate} 
\item 
  مستقل   k    کی قیمت معلوم کریں
\item
اسی دن ایک عبارت لگا دی گئ کہ سڑک تین گھنٹے کے لیے بند ہے ۔ یہ مانتے ہوۓ کہ حکم نامہ درست ہے ، سڑک کی سطح سمندر سے اونچائ معلوم کریں اور آپکا جواب دو اعشاری نقطوں تک درست ہونا چاہئیے
\item
دراصل سڑک کی بحالی کے کام میں اسکی سطح بڑھی ہے، اب سڑک صرف 2 گھنٹے 40 منٹ کے لیے بند ہوئ ہے ، یہ بتائیں کہ سڑک کی سطح کتنی بلند ہوئ۔
\end{enumerate}
 
\انتہا{سوال}

\ابتدا{سوال}
سمندر میں بننے والی لہروں کے لیے سب سے سادہ نظریہ یہ ہے کہ یہ سورج اور چاند کی کشش ثقل کی وجہ سے معرض وجود میں آتی ہیں۔ چاند کی کشش ثقل سورج کی نسبت 9 گناہ ذیادہ ہے۔ سورج کی وجہ سے ہونے والا تغیر خود کو 360 دنوں بعد دہراتا ہے جبکہ چاند کے زیر اثر سلسلہ 30 دنوں بعد خود کو دہراتا ہے ۔ لہروں کی اونچائ h ، وقت کی علامت  t    ہے جسکی اکائ دن لیا گیا ہے
اور تفاعل  \[h= A\cos\alpha t\degree+B\cos\beta t\degree, \]  ہے۔
اس تفاعل میں \(A\cos\alpha t\degree \)  یہ سورج کے اثر کے لیے ہے جبکہ کلیے کا دوسرا حصہ \(B\cos\beta t\degree \) چاند کی کشش ثکل سے پیدا ہونے والی لہروں کے لیے ہے۔ ہمیں بتایا گیا ہے کہ 
 h=5   
ہے اور  t=0 آپ ،B، A،\(\alpha \)  اور  \(\beta \)، کی قیمت معلوم کریں۔ 
     
\انتہا{سوال}









 









