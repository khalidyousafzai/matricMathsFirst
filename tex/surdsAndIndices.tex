\باب{غیر معقول اور طاقتیں}\شناخت{باب_غیر_معقول_اور_طاقتیں}

اس باب کا پہلا حصہ مربع اور مکعب جذر والی تراکیب کے بارے میں اور دوسرا حصہ طاقتی بیانیوں کے بارے میں ہے۔ اس کو مکمل کرنے کے بعد آپ کو اس قابل ہو جانا چاہیے کہ۔ 
\begin{itemize}
\item
مربع، مکعب اور دیگر جذروں والی تراکیب کو سادہ بنا سکیں
\item
طاقت کے قوانین جانتے ہوں
\item
منفی، صفر، اور کسری طاقتوں کا مطلب جانتے ہوں
\item
طاقت کی حامل تراکیب کو سادہ کر سکیں
\end{itemize}
\حصہ{اعداد کی اقسام}\شناخت{حصہ_غیر_معقول_اعداد_کی_اقسام}
آغاز میں اعداد فقط گنتی کے لیے استعمال ہوتے تھے اور \(1,2,3,\dotsc\) ہماری اس ضرورت کے لیے کافی تھے۔ یہ طبعی اعداد یا مثبت صحیح عدد کہلاتے ہیں۔ 

آہستہ آہستہ ہمیں معلوم ہوا کہ اعداد پیمائش اور تجارتی مقاصد کے لیے بھی ضروری ہیں، اور اس کے لیے ہمیں کسروں کی ضرورت بھی پڑنے لگی۔ صحیح عدد اور کسروں کو ملا کر منطقی اعداد بنائے گئے۔ یہ وہ اعداد ہیں کہ جنھیں \(\frac{p}{q}\) کی شکل میں لکھا جا سکتا ہے۔ جب کہ p اور q  دونوں صحیح اعداد ہیں اور q صفر نہیں ہو گا۔
یونانی ریاضی دانوں کی بے شمار شاندار دریافتوں میں سے ایک دریافت یہ بھی تھی کہ ایسے اعداد موجود ہیں جنھیں اس ہئیت میں نہیں لکھا جا سکتا۔ ایسے اعداد کو غیر منطقی اعداد کہا جاتا ہے۔ پہلا ایسا عدد جو دریافت کیا گیا\(\sqrt{2}\) تھا، جو فیثاغورس کے قانون کے مطابق ایک ایسے مربع کے وتر کی لمبائی بنتی ہے جس کی ہر طرف کی لمبائی 1 ہو۔ یونانیوں نے جس دلیل سے ثابت کیا کہ \(\sqrt{2}\) کو کسری صورت میں نہیں لکھا جا سکتا ، اسی دلیل سے یہ بھی ثابت کیا جا سکتا ہے کہ کوئی بھی جزر، مربع، مکعب یا کوئی بھی، یا تو صحیح عدد ہو گی یا غیر منطقی عدد۔ اب ہم بہت سے غیر منطقی اعداد جان چکے ہیں جن میں سب سے مشہور \(\pi\) ہے۔

منطقی اور غیر منطقی اعداد مل کر حقیقی اعداد بناتے ہیں۔ اعداد صحیح، منطقی، غیر منطقی اور حقیقی اعداد مثبت،منفی یا صفر ہو سکتے ہیں۔

جب کسی منطقی عدد کو اعشاریہ کی صورت میں لکھا جائے تو یا تو اعشاریے کے ایک درجے تک رک جاتے ہیں یا ہندسوں کی ایک مخصوص وضع یا ترتیب میں  بار باار دہرایا جانے لگتا ہے۔  مثال کے طور پر۔ 

\[\frac{7}{10}=0.7,\quad \frac{7}{11}=0.6363\dotsc,\quad\frac{7}{12}=0.5833\dotsc,\quad\frac{7}{13}=0.538 461 538 461 53\dotsc\]
\[\frac{7}{14}=0.5,\quad\frac{7}{15}=0.466\dotsc,\quad\frac{7}{16}=0.4375,\quad\frac{7}{17}=0.411 764 705 882 352941176\dotsc\]

اس کا معکوس بھی درست ہے، یعنی اگر ایک اعشاری عدد رک جائے یا محدود بار دہرایا جائے تو وہ منطقی عدد کہلائے گا۔ لہٰذا اگر ایک غیر منطقی عدد کو اعشاری صورت میں لکھا جائے تو آپ جتنا مرضی پھیلا لیں، اس کے ہندسوں کی ترتیب کبھی دہرائی نہیں جائے گی۔

\حصہ{نامعقولیے اور ان کی خصوصیات}
آج سے پہلے جب ہم  \(\sqrt{2}\) \(\sqrt{8}\) \(\sqrt{12}\)یا ایسی کسی ترکیب کو دیکھتے تھے تو ہم کیلکولیٹر کی مدد سے اسے اعشاری صورت میں بدل کر لکھ لیا کرتے تھے۔ مثلاً کچھ اس طرح

\(\sqrt{2}=1.414\dotsc\)یا 
\(\sqrt{2}=1.414\) تین اعشاری ہندسوں تک درست یا
\(\sqrt{2}\approx 1.414\)
لیکن \(\sqrt{2}=1.414\) خود یہ ترکیب کیوں درست نہیں ہے؟
\(\sqrt{2}\)\(\sqrt[3]{9}\) ایسی تراکیب کو نامعقولیہ کہا جاتا ہے۔ اس جزو میں ہم انھی نا معقولیوں سے حساب کرنا سیکھیں گے۔
آپ کو یاد رکھنا ہو گا کہ \(\sqrt{x}\) ہمیشہ x کی مثبت مربع جذر(یا \(x=0\) ہونے کی صورت میں صفر) کے معنوں میں لکھاجاتا ہے۔
نامعقولیوں کی اہم خاصیتیں، جو ہم بار بار استعمال کریں گے، یہ ہیں:

\(\sqrt{xy}=\sqrt{x}\times\sqrt{y}\)
 اور 
\(\sqrt{\frac{x}{y}}=\frac{\sqrt{x}}{\sqrt{y}}\)
آپ دیکھ سکتے ہیں کہ \((\sqrt{x}\times\sqrt{y})\times(\sqrt{x}\times\sqrt{y})=(\sqrt{x}\times\sqrt{x})\times(\sqrt{y}\times\sqrt{y})=x\times y=xy\)  ہے اور چونکہ \(\sqrt{x}\times\sqrt{y}\) مثبت ہے، لہٰذا یہ \(xy\) کا جزر ہے۔ اسی طرح \(\sqrt{xy}=\sqrt{x}\times\sqrt{y}\) ہے۔ اور اسی دلیل سے ہم سمجھ سکتے ہیں کہ \(\sqrt{\frac{x}{y}}=\frac{\sqrt{x}}{\sqrt{y}}\)۔

درج ذیل مثالیں ان خصوصیات کو سمجھنے میں مدد دے سکتی ہیں۔

\[\sqrt{8}=\sqrt{4\times 2}=\sqrt{4}\times\sqrt{2}=2\sqrt{2};\quad\sqrt{12}=\sqrt{4\times 3}=\sqrt{4}\times\sqrt{3}=2\sqrt{3};\]
\[\sqrt{18}\times \sqrt{2}=\sqrt{18\times 2}=\sqrt{36}=6;\quad \frac{\sqrt{27}}{\sqrt{3}}=\sqrt{\frac{27}{3}}=\sqrt{9}=3.\]

اس حساب کو اپنے کیلکولیٹر سے دوبارہ کر کے دیکھنا شاید آپ کے یقین میں اضافے کا باعث ہو۔

\ابتدا{مثال}\شناخت{مثال_غیر_معقول_سادہ_کریں}
سادہ کریں 
(ا)
\(\sqrt{28}+\sqrt{63}\)
(ب) 
\(\sqrt{5}\times \sqrt{10}\)
ان کا حل متبادل طریقے سے بھی نکالا جا سکتا ہے، جیسے جزو ب کے لیے نکالا گیا ہے۔
(ا) \[\sqrt{28}+\sqrt{63}=(\sqrt{4}\times\sqrt{7})+(\sqrt{9}\times\sqrt{7})=2\sqrt{7}+3\sqrt{7}=5\sqrt{7}\]
(ب) پہلا طریقہ \(\sqrt{5}\times\sqrt{10}=\sqrt{5\times10}=\sqrt{50}=\sqrt{25\times 2}=5\sqrt{2}\)
دوسرا طریقہ \(\sqrt{5}\times\sqrt{10}=\sqrt{5}\times(\sqrt{5}\times\sqrt{2})=(\sqrt{5}\times\sqrt{5})\times\sqrt{2}=5\sqrt{2}\)
بعض اوقات کسر کے نسب نما سے نا معقولیوں کو ہٹا دینا مفید ہوتا ہے جیسے \(\frac{1}{\sqrt{2}}\) کے نسب نما سے نا معقولیہ ہٹانے کے لیے ہم اوپر نیچے دونوں کو \(\sqrt{2}\) سے ضرب دے سکتے ہیں۔ \(\frac{1\times\sqrt{2}}{\sqrt{2}\times\sqrt{2}}=\frac{\sqrt{2}}{2}\)
\انتہا{مثال}

کچھ نتائج جو اکثر ہماری مدد کریں گے۔
\(\frac{x}{\sqrt{x}}=\sqrt{x}\) اور اسی کا بالعکس \(\frac{1}{\sqrt{x}}=\frac{\sqrt{x}}{x}\)
غیر معقول کو نسب نما سے ہٹا دینا نسب نما کو معقول بنانا کہلاتا ہے۔

\ابتدا{مثال}
درج ذیل ترکیب میں نسب نما کو معقول بنائیں۔

(ا)
\(\frac{6}{\sqrt{2}}\)

(ب)
\(\frac{3\sqrt{2}}{\sqrt{10}}\)

حل:
(ا)
\(\frac{6}{\sqrt{2}}=\frac{3\times 2}{\sqrt{2}}=3\times\frac{2}{\sqrt{2}}=3\sqrt{2}\)

(ب)
\(\frac{3\sqrt{2}}{\sqrt{10}}=\frac{3\times\sqrt{2}}{\sqrt{5}\times\sqrt{2}}=\frac{3}{\sqrt{5}}=\frac{3\sqrt{5}}{5}\)

مربع جذر کے لیے استعمال ہونے والے قوانین ہی مکعب جذر اور اس سے بالائی جذروں کے لیے استعمال ہوتے ہیں۔
\انتہا{مثال}

\ابتدا{مثال}
شکل \حوالہء{2.1} میں ایک عمارت کی چھت کا قطع عمودی کو ایک قائم مثلث \(ABC\) کی صورت میں دکھایا گیا ہے۔ جس میں
 \(AB=\SI{15}{\meter}\)ہے۔چھت کی بلندی
 \(BD\) \(10m\) ہے۔ x اور y  معلوم کریں۔
ہم مثلث \(ABD\) سے شروع کرتے ہیں۔ فیثا غورس کے قانون کے مطابق ہم جانتے ہیں کہ
 \(z^{2}+10^{2}=15^{2}\)
لہٰذا \(z^{2}=225-100=125\) ہو گا۔

\[z=\sqrt{125}=\sqrt{25\times 5}=5\sqrt{5}\]
توجہ کیجیے کہ مثلث \(ABC\) اور \(ABD\) مماثل ہیں۔ اس مماثلت کو بہتر طور پر سمجھنے کے لیے ہم  شکل  \حوالہء{2.2} میں \(ABD\) کو الٹا کر دکھاتے ہیں۔ اب \(ABC\) اور \(ABD\) دونوں مثلثوں کی طرفیں ایک ہی تناسب میں ہوں گی۔ لہٰذا\(\frac{x}{15}=\frac{y}{10}=\frac{15}{z}\)

۔ اور جیسا کہ ہم جانتے ہیں\(\frac{15}{z}=\frac{15}{5\sqrt{5}}=\frac{3}{\sqrt{5}}=\frac{3\sqrt{5}}{5}\)
\(x=15\times\frac{3\sqrt{5}}{5}=9\sqrt{5}\)

\(y=10\times\frac{3\sqrt{5}}{5}=6\sqrt{5}\)

آپ فیثاغورس کے قانون سے مثلث  \(ABC\) میں \(x^{2}=15^{2}+y^{2}\) کی تصدیق کر سکتے ہیں۔
\انتہا{مثال}

\ابتدا{سوال}
کیلکولیٹر استعمال کیے بغیر ان تراکیب کو سادہ کریں۔
\begin{multicols}{3.}
\begin{enumerate}
\item
\(\sqrt{3}\times\sqrt{3}\)
\item
\(\sqrt{10}\times\sqrt{10}\)
\item
\(\sqrt{16}\times\sqrt{10}\)
\item
\(\sqrt{8}\times\sqrt{2}\)
\item
\(\sqrt{32}\times\sqrt{2}\)
\item
\(\sqrt{3}\times\sqrt{12}\)
\item
\(5\sqrt{3}\times\sqrt{3}\)
\item
\(2\sqrt{5}\times 3\sqrt{5}\)
\item
\(3\sqrt{6}\times 4\sqrt{6}\)
\item
\(2\sqrt{20}\times 3\sqrt{5}\)
\item
\((2\sqrt{7})^{2}\)
\item
\((3\sqrt{3})^{2}\)
\item
\(\sqrt[3]{5}\times\sqrt[3]{5}\times\sqrt[3]{5}\)
\item
\((2\sqrt[4]{3})^4\)
\item
\((2\sqrt[3]{2})^{6}\)
\item
\(4\sqrt{125}\times4\sqrt{5}\)
\end{enumerate}
\end{multicols}
\انتہا{سوال}
\ابتدا{سوال}
درج ذیل تراکیب کو کیلکولیٹر استعمال کیے بغیر سادہ کریں۔
\begin{multicols}{3.}
\begin{enumerate}
\item
\(\sqrt{18}\)
\item
\(\sqrt{20}\)
\item
\(\sqrt{24}\)
\item
\(\sqrt{32}\)
\item
\(\sqrt{40}\)
\item
\(\sqrt{45}\)
\item
\(\sqrt{48}\)
\item
\(\sqrt{50}\)
\item
\(\sqrt{54}\)
\item
\(\sqrt{72}\)
\item
\(\sqrt{175}\)
\item
\(\sqrt{675}\)
\end{enumerate}
\end{multicols}
\انتہا{سوال}
\ابتدا{سوال}
درج ذیل تراکیب کو کیلکولیٹر استعمال کیے بغیر سادہ کریں۔

\begin{multicols}{2.}
\begin{enumerate}
\item
\(\sqrt{8}+\sqrt{18}\)
\item
\(\sqrt{3}+\sqrt{12}\)
\item
\(\sqrt{20}-\sqrt{5}\)
\item
\(\sqrt{32}-\sqrt{8}\)
\item
\(\sqrt{50}-\sqrt{18}-\sqrt{8}\)
\item
\(\sqrt{27}+\sqrt{27}\)
\item
\(\sqrt{99}+\sqrt{44}+\sqrt{11}\)
\item
\(8\sqrt{2}+2\sqrt{8}\)
\item
\(2\sqrt{20}+3\sqrt{45}\)
\item
\(\sqrt{52}-\sqrt{13}\)
\item
\(20\sqrt{5}-5\sqrt{20}\)
\item
\(\sqrt{48}+\sqrt{24}-\sqrt{75}+\sqrt{96}\)
\end{enumerate}
\end{multicols}
\انتہا{سوال}
\ابتدا{سوال}
درج ذیل تراکیب کو کیلکولیٹر استعمال کیے بغیر سادہ کریں۔

\begin{multicols}{4.}
\begin{enumerate}[a.]
\item
\(\frac{\sqrt{8}}{\sqrt{2}}\)
\item
\(\frac{\sqrt{27}}{\sqrt{3}}\)
\item
\(\frac{\sqrt{40}}{\sqrt{10}}\)
\item
\(\frac{\sqrt{50}}{\sqrt{2}}\)
\item
\(\frac{\sqrt{125}}{\sqrt{5}}\)
\item
\(\frac{\sqrt{54}}{\sqrt{6}}\)
\item
\(\frac{\sqrt{3}}{\sqrt{48}}\)
\item
\(\frac{\sqrt{50}}{\sqrt{200}}\)
\end{enumerate}
\end{multicols}
\انتہا{سوال}
\ابتدا{سوال}
درج ذیل تراکیب کو کیلکولیٹر استعمال کیے بغیر سادہ کریں
\انتہا{سوال}
\begin{multicols}{5.}
\begin{enumerate}[a.]
\item
\(\frac{1}{\sqrt{3}}\)
\item
\(\frac{1}{\sqrt{5}}\)
\item
\(\frac{4}{\sqrt{2}}\)
\item
\(\frac{6}{\sqrt{6}}\)
\item              
\(\frac{11}{\sqrt{11}}\)
\item
\(\frac{2}{\sqrt{8}}\)
\item
\(\frac{12}{\sqrt{3}}\)
\item
\(\frac{14}{\sqrt{7}}\)
\item
\(\frac{\sqrt{6}}{\sqrt{2}}\)
\item
\(\frac{\sqrt{2}}{\sqrt{6}}\)
\item
\(\frac{3\sqrt{5}}{\sqrt{3}}\)
\item
\(\frac{4\sqrt{6}}{\sqrt{5}}\)
\item
\(\frac{7\sqrt{2}}{2\sqrt{3}}\)
\item
\(\frac{4\sqrt{2}}{\sqrt{12}}\)
\item
\(\frac{9\sqrt{12}}{2\sqrt{18}}\)
\item
\(\frac{2\sqrt{18}}{9\sqrt{12}}\)
\end{enumerate}
\end{multicols}
\ابتدا{سوال}
درج ذیل تراکیب سادہ بنائیں اور ہر ایک کا جواب $ k\sqrt{3}$ کی شکل میں لکھیں۔
\انتہا{سوال}
\begin{multicols}{2.}
\begin{enumerate}[a.]
\item 
\(\sqrt{75}+\sqrt{12}\)
\item 
\(\sqrt{6}+\sqrt{3}(4-2\sqrt{3})\)
\item
\(\frac{12}{\sqrt{3}}-\sqrt{27}\)
\item
\(\frac{2}{\sqrt{3}}+\frac{\sqrt{2}}{\sqrt{6}}\)
\item
\(\sqrt{2}\times\sqrt{8}\times\sqrt{27}\)
\item
\((3-\sqrt{3})(2-\sqrt{3})-\sqrt{3}\times\sqrt{27}\)
\item
\(AB=4\sqrt{5}cm\)
\item
\( BC=\sqrt{10}\)
\end{enumerate}
\end{multicols}
\ابتدا{سوال}
\(ABCD\)ایک چوکور ہے، جس میں \(AB=4\sqrt{5}cm\) اور \(BC=\sqrt{10}\)۔ درج ذیل سوال کا جواب سادہ غیر معقول جذر کی شکل میں  لکھیں۔
(ا) چوکور کا رقبہ معلوم کریں
(ب) وتر \(AC\) کی لمبائی معلوم کریں
\انتہا{سوال}

\ابتدا{سوال}
درج ذیل تراکیب سادہ بنائیں اور ہر ایک کا جواب $k \sqrt{2}$  کی شکل میں لکھیں۔
\انتہا{سوال}
\begin{multicols}{2.}
\begin{enumerate}
\item
\(x\sqrt{2}=10\)
\item
\(2y\sqrt{2}-3=\frac{5y}{\sqrt{2}}+1\)
\item
\(z\sqrt{32}-16=z\sqrt{8}-4\)
\end{enumerate}
\end{multicols}

\ابتدا{سوال}
درج ذیل تراکیب  کو\(k\sqrt[3]{3}\) کی شکل میں لکھیں۔
\انتہا{سوال}

\begin{multicols}{2.}
\begin{enumerate}
\item
\(\sqrt[3]{24}\)
\item
\(\sqrt[3]{81}+\sqrt[3]{3}\)
\item
\((\sqrt[3]{3})^{4}\)
\item
\(\sqrt[3]{3000}-\sqrt[3]{375}\)
\end{enumerate}
\end{multicols}
\ابتدا{سوال}
درج ذیل قائم مثلثوں کی تیسری نا معلوم طرف معلوم کریں۔ اپنے جواب کو سادہ غیر معقول کی شکل میں لکھیں
\انتہا{سوال}


\ابتدا{سوال}آپ کو بتایا جائے کہ اعشاریے کے بارہ ہندسوں تک لکھیے، مثلاً \(\sqrt{26}=\num{5.099019513593}\)
\انتہا{سوال}
\begin{enumerate}
\item
\(\sqrt{104}\) کی ایسی قیمت معلوم کریں جو دس اعشاری ہندوسں تک درست ہو۔
\item
\(\sqrt{650}\) کی ایسی قیمت معلوم کریں جو دس اعشاری ہندسوں تک درست ہو۔
\item
\(\frac{13}{\sqrt{26}}\) کی ایسی قیمت معلوم کریں جو دس اعشاری ہندوسں تک درست ہو۔
\end{enumerate}

\ابتدا{سوال}
دی گئی بیک وقت مساواتوں کو حل کریں،\(7x-(3\sqrt{5})y=9\sqrt{5}\) اور \((2\sqrt{5})x+y=34\)
\انتہا{سوال}

\ابتدا{سوال}
درج ذیل کو سادہ بنائیں
\انتہا{سوال}

\begin{multicols}{3}
\begin{enumerate}[a.]
\item
\((\sqrt{2}-1)(\sqrt{2}+1)\)
\item
\((2-\sqrt{3})(2+\sqrt{3})\)
\item
\((\sqrt{7}+\sqrt{3})(\sqrt{7}-\sqrt{3})\)
\item
\((2\sqrt{2}+1)(2\sqrt{2}-1)\)
\item
\((4\sqrt{3}-2)(4\sqrt{3}+2)\)
\item
\((10+\sqrt{5})(10-\sqrt{5})\)
\item
\((4\sqrt{7}-5)(4\sqrt{7}+5)\)
\item
\((2\sqrt{6}-3\sqrt{3})(2\sqrt{6}+3\sqrt{3})\)
\end{enumerate}
\end{multicols}
\ابتدا{سوال}
سوال نمبر \(13\) میں ہر جواب ایک عدد صحیح ، نقل کر کے درج ذیل کو مکمل کریں
\انتہا{سوال}
\begin{multicols}{2}
\begin{enumerate}[a.]
\item
\((\sqrt{3}-1)(\quad )=2\)
\item
\((\sqrt{5}+1)(\quad)=4\)
\item
\((\sqrt{6}-\sqrt{2})(\quad)=4\)
\item
\((2\sqrt{7}+\sqrt{3})(\quad)=25\)
\item
\((\sqrt{11}+\sqrt{10})(\quad)=1\)
\item
\((3\sqrt{5}-2\sqrt{6})(\quad)=21\)
\end{enumerate}
\end{multicols}
سوال نمبر \(15\)اور\(16\) میں دی گئی مثالیں ہمیں نسب نما کو منطقی بنانے کے طریقے کی طرف متوجہ کرتی ہیں، جو سوال نمبر \(5\) کی ترکیبوں سے زیادہ پیچیدہ ہوں۔
\ابتدا{سوال}
(ا) وضاحت کریں\(\frac{1}{\sqrt{3}-1}=\frac{1}{\sqrt{3}-1}\times\frac{\sqrt{3}+1}{\sqrt{3}+1}\) اور ثابت کریں \(\frac{1}{\sqrt{3}-1}=\frac{\sqrt{3}+1}{2}\)

(ب) ثابت کریں \(\frac{1}{2\sqrt{3}+3}=\frac{2\sqrt{2}-\sqrt{3}}{5}\)
\انتہا{سوال}
\ابتدا{سوال}
نسب نما کو معقول بنا کر درج ذیل کسروں کو سادہ کریں
\begin{multicols}{3}
\begin{enumerate}[a.]
\item
\(\frac{1}{2-\sqrt{3}}\)
\item
\(\frac{1}{3\sqrt{5}-5}\)
\item
\(\frac{4\sqrt{3}}{2\sqrt{6}+3\sqrt{2}}\)
\end{enumerate}
\end{multicols}
\انتہا{سوال}

\حصہ{طاقتوں کا استعمال}
سولہویں صدی میں جب ریاضی کی کتب چھپنے لگیں، تو ریاضی دان مکعب اور مربع مساواتوں کا حل ڈھونڈ رہے تھے۔ انھیں لگا کہ \(xxx\)اور \(xxxx\) کو \(x^{3}\) اور 
\(x^{4}\) لکھنا زیادہ آسان اور مفید رہے گا۔

طاقت نویسی کا آغاز تو اس انداز میں ہوا تھا لیکن وقت گزرنے کے ساتھ ساتھ اندازہ ہوا کہ یہ صرف مختصر نویسی ہی کا ایک انداز نہیں تھا، بلکہ اس انداز سے لکھنا مستقبل میں اہم دریافتوں کا باعث بنا اور ریاضی کی موجودہ شکل اس انداز کے بغیر مبہم اور ناقابل استفہام ہوتی۔
آپ نے اس انداز بیان کی سادہ مثالیں تو استعمال کی ہی ہوں گی۔ عام طور پہ علامت\(a^{m}\) ، \(a\) کو \(m\)بار ضرب دینے کے لیے لکھی جاتی ہے, اس کو یوں سمجھا جا سکتا ہے۔
\[a^{m}=\overbrace{a\times a\times a\times\dotsc\times a}^{\text{\RL{ان کی تعداد $m$ ہے}}}\]

اس میں \(a\) کو اساس کہا جاتا ہے اور \(m\) کو طاقت کہا جاتا ہے۔ یہاں توجہ دلانا ضروری ہے کہ \(a\) کسی بھی قسم کا عدد ہو سکتا ہے لیکن\(m\) لازمی طور پر مثبت عدد صحیح ہی ہو گا۔ اسکو عام طور پہ \(a\) کی طاقت \(m\) کہا جاتا ہے۔ طاقتی بیانیوں میں لکھی جانے والی تراکیب کو درج ذیل سادہ قوانین سے آسان بنایا جا سکتا ہے۔
ان میں سے ایک ضرب کا قانون ہے۔
\[a^{m}\times a^{n}=\overbrace{a\times a\times\dotsc\times a}^{\text{\RL{تعداد \(m\)}}}\times\overbrace{a\times a\times\dotsc\times a}^{\text{\RL{تعداد \(n\)}}}=\overbrace{a\times a\times\dotsc\times a}^{\text{\RL{تعداد \(m+n\)}}}=a^{m+n}\]
یہ بہت سی جگہوں پہ استعمال ہوتا ہے، مثلاً ایسے مکعب کا ہجم معلوم کرنے کے لیے جس کی ہر طرف کی لمبائی $a$
 ہو۔ ہم جانتے ہیں کہ اساس کے رقبے کو بلندی سے ضرب دے کر ہجم دریافت کیا جاتا ہے۔
\(=a^{2}\times a=a{2}\times a^{1}=a^{2+1}=a^{3}\)

اس سے ملتا جلتا تقسیم کا قانون

\begin{align*}
a^{m}\div a^{n}&=\overbrace{(a\times a\times\dotsc\times a)}^{\text{\RL{تعداد \(m\)}}}\div\overbrace{(a\times a\times\dotsc\times a)}^{\text{\RL{تعداد \(n\)}}}\\
&=\overbrace{a\times a\times\dotsc\times a}^{\text{\RL{تعداد \(m-n\)}}}\\
&=a^{m-n}\\
\end{align*}

 اسی طرح طاقت پہ طاقت کا قانون ہے
\begin{align*}
(a^{m})^{n}&=\overbrace{\overbrace{a\times a\times\dotsc\times a}^{\text{\RL{تعداد \(m\)}}}\times \overbrace{a\times a\times\dotsc\times a}^{\text{\RL{تعداد \(m\)}}}\times\dotsc\overbrace{a\times a\times\dotsc\times a}^{\text{\RL{تعداد \(m\)}}}}^{\text{\RL{تعداد \(n\)}}}\\
&=\overbrace{a\times a\times\dotsc\times a}^{\text{\RL{تعداد \(m\times n\)}}}=a^{m\times n}\\
\end{align*}

ایک اور قانون جو جز کا قانون ہے کہ جس میں دو اساسیں اور ایک طاقت ہوتی ہے۔ 

\begin{align*}(a\times b)^{m}&=\overbrace{(a\times b)\times(a\times b)\times\dotsc\times
 (a\times b)}^{\text{\RL{تعداد \(m\)}}}\\
&=\overbrace{a\times a\times\dotsc\times a}^{\text{\RL{تعداد \(m\)}}}\times \overbrace{b\times b\times\dotsc\times b}^{\text{\RL{تعداد \(m\)}}}\\
&=a^{m}\times b^{m}
\end{align*}
ان قوانین کو بیان کرنے کے لیے ضرب کی علامت استعمال کی گئی ہے، لیکن الجبرا کے دیگر حصوں میں اگر غلطی کی گنجائش نہ ہو تو یہ ہٹا دی جاتی ہے۔ اسے مکمل کرنے کےلیے یہاں یہ قوانین دوبارہ دیے جا رہے ہیں۔
ضرب کا قانون
\(a^{m}\times a^{n}=a^{m+n}\)
تقسیم کا قانون
\(a^{m}\div a^{n}=a{m-n}\)
طاقت پہ طاقت کا قانون
\((a^{m})^{n}=a^{m\times n}\)
جز کا قانون
\((a\times b)^{m}=a^{m}\times b^{m}\)
\ابتدا{مثال}
دی گئی ترکیب کو سادہ بنائیں۔
\((2a^{2}b)^{3}\div (4a^{4}b)\)

حل:
\begin{align*}
(2a^{2}b)^{3}\div (4a^{4}b)&=(2^{3}(a^{2})^{3}b^{3})\div
 (4a^{4}b)\\
جز کا قانون
&=(8a^{2\times 3}b^{3})\div (4a^{4}b)\\
طاقت پہ طاقت کا قانون
&=(8\div 4)\times (a^{6}\div a^{4})\times (b^{3}\div b^{1})\\
دوبارہ ترتیب دے کر لکھیں
&=2a^{6-4}b^{3-1}\\
تقسیم کا قانون
&=2a^{2}b^{2}
\end{align*}
\انتہا{مثال}


\حصہ{صفر اور منفی طاقت}

پچھلے حصے میں ہم نے ترکیب \({a}^m\) کی تعریف بیان کی جس میں ہم  \(m\) مرتبہ ضرب دیتے ہیں،لیکن اگر  \(m\) صفر یا منفی ہو تو یہ تعریف اپنے معنی کھو دیتی ہے ۔ ہم کسی بھی چیز کو \(3-\) یا صفر مرتبہ ضرب نہیں دے سکتے۔ لیکن \({a}^m\) کے معنی کو وسعت دے کر دیکھا جائے تو صفر یا منفی طاقت کی صورت میں بھی نہ صرف یہ کہ یہ معنی درست ہے بلکہ مفید بھی ہے۔ اس کے ساتھ اہم بات یہ کہ مثبت طاقت کے تمام قوانین منفی اور صفر طاقتوں کے لیے بھی درست ہیں۔
اس تسلسل پہ غور کریں۔

دائیں سمت پہ اساس ہمیشہ \(2\)  ہے جب کہ طاقت ہر مرتبہ ایک کم ہوتی جا رہی ہے۔ جبکہ بائیں طرف عدد آدھے ہوتے جا رہے ہیں۔ لہذا اس تسلسل کو یوں بڑھایا جا سکتا ہے۔

 اور ہم اس طرح لا محدود حد تک جا سکتے ہیں۔  اب ان کا آپس میں موازنہ کریں

یوں لگتا ہے جیسے \({2}^m-\) کو \({{m}frac{1}}\) لکھنا چاہیے، اور صفر کی طاقت کے لیے ایک خصوصی قیمت \(1={2}^0\) رکھنی چاہیے۔
ہم اپنے پہلے مشاہدے کو  صفر کے علاوہ تمام اساسوں اور کسی بھی مثبت عدد صحیح \(m\) کے لیے پھیلائیں تو منفی طاقت کے قوانین تک پہنچ سکتے ہیں۔

منفی طاقت کا قانون

 ہم یہاں کچھ مثالوں سے ثابت کریں گے کہ مثبت طاقتوں کے لیے بنائے گئے قوانین منفی طاقتوں کے لیے بھی درست ہیں۔ اسی طرح آپ اپنےلیے بہت سی اور مثالیں بھی بنا سکتے ہیں۔

ضرب کا قانون:

طاقت پہ طاقت کا قانون:

جز کا قانون:

\ابتدا{مثال}
اگر  $ 5= a$ ہے تو  کی قیمت معلوم کریں۔
یہاں اہم نکتہ یہ ہے کہ طاقت 
$-2$
صرف
$a$
کے ساتھ ہے، یعنی $4$ پہ نہیں ہے۔ لہٰذا  
$4a^{-2}$
کا مطلب ہے
\(4\times{\frac{1}{2^{a}}}\). اب جب کہ \(a=5\) ہے،  \(4a^{-2}\) \(=\) \(=4\times{\frac{1}{25}}\)
\انتہا{مثال}
\ابتدا{مثال}
ان تراکیب کو سادہ کریں

(ا) 
\(4a^{2}b\times\)
(b)

(ا) پہلا طریقہ ہر چیز کو مثبت طاقت میں لے آئیں

دوسرا طریقہ
مثبت اور منفی دونوں طاقتوں کے لیے قوانین استعمال کر لیں۔


(ب) زیر نظر مثال میں میکینکس کا ایک استعمال دیکھیے۔ (M,L,T)،لزوجیت کی پیمائش کے لیے ماس، لمبائی اور وقت کی جہتیں ہیں۔ بریکٹس کو الگ الگ کر کے


منفی طاقتوں کو بہت چھوٹے اعداد لکھنے کے لیے بھی استعمال کیا جاتا ہے۔ یقیناً آپ بہت بڑے اعداد کو معیاری شکل یا سائنسی بیانیے میں لکھنا جانتے ہوں گے، مثلاً روشنی کی رفتار کو  \num{300000000} میٹر فی سیکنڈ لکھنے کی بجائے
\(\SI{3e8}{\meter\per\second}\)
 لکھا جاتا ہے۔ بالکل اسی طرح سرخ روشنی کا طول موج جو تقریباً 
\num{0.00000075} میٹر ہے، کو بھی آسانی سے 
\(\SI{7.5e-7}{\meter}\)  لکھا جا سکتا ہے۔
کمپیوٹر اور کیلکولیٹر میں لوگوں کے لیے سائنسی اعتبار سے لکھنے کا امکان موجود ہوتا ہے اور اگر کوئی عدد عام عدد سے زیادہ بڑا یا زیادہ چھوٹا ہو جائے تو وہ اسے میعاری شکل میں بدل دیتا ہے۔ مثلاً  یا ۔ علامت
(E) ایکسپونینٹ کے لیے استعمال ہوتی ہے جو طاقت ہی کے لیے استعمال ہونے والا دوسرا لفظ ہے
\انتہا{مثال}

\ابتدا{مثال}
اس ترکیب \(G=\frac{gR^{2}}{M}\) سے کشش ثقل کے مستقل \(G\) کا حساب لگائیں، جبکہ \(g\approx 9.81\)،
\(R\approx 6.37\times 10^{6}\) اور
\(M\approx 5.97\times 10^{24}\)۔ 
(\(R\) اور \(M\) زمین کا رداس اور ماس ہے، اور 
\( g \) کشش ثقل کے سبب پیدا ہونے والا اسراع ہے۔
\begin{align*}
G&\approx \frac{9.81\times (6.37\times 10^{6})^{2}}{5.97\times 10^{24}}=\frac{9.81\times(6.37)^{2}}{5.97}\times \frac{(10^{6})^{2}}{10^{24}}\\
&\approx 66.7\times\frac{10^{12}}{10^{24}}=6.67\times 10^{1}\times 10^{-12}=6.67\times 10^{1-12}=6.67\times 10^{-11}\\
\end{align*}
\انتہا{مثال}

\ابتدا{سوال}
درج ذیل تراکیب کو سادہ کریں
\begin{multicols}{3}
\begin{enumerate}[.a]
\item
\(a^{2}\times a^{3}\times a^{7}\)
\item
\((b^{4})^{2}\)
\item
\(c^{7}\div c^{3}\)
\item
\(d^{5}\times d^{4}\)
\item
\((e^{5})^{4}\)
\item
\((x^{3}y^{2})^{2}\)
\item
\(5g^{5}\times3g^{3}\)
\item
\(12h^{12}\div 4h^{4}\)
\item
\((2a^{2})^{3}\times(3a)^{2}\)
\item
\((p^{2}q^{3})^{2}\times(pq^{3})^{3}\)
\item
\((4x^{2}y)^{2}\times(2xy^{3})^{3}\)
\item
\((6ac^{3})^{2}\div(9a^{2}c^{5}\)
\item
\((3m^{4}n^{2})^{3}\times(2mn^{2})^{2}\)
\item
\((49r^{3}s^{2})^{2}\div(7rs)^{3}\)
\item
\((2xy^{2}z^{3})^{2}\div(2xy^{2}z^{3})\)

\end{enumerate}
\end{multicols}

\انتہا{سوال}

\ابتدا{سوال}
درج ذیل تراکیب کو سادہ کریں، ہر جواب 
\(2^{n}\) کی  ہیئت میں لکھیں۔

\begin{multicols}{2}
\begin{enumerate}[a.]
\item
\(2^{11}\times(2^{5})^{3}\)
\item
\((2^{3})^{2}\times(2^{2})^{3}\)
\item
\(4^{3}\)
\item
\(8^{2}\)
\item
\(\frac{2^{7}\times 2^{8}}{2^{13}}\)
\item
\(\frac{2^{2}\times 2^{3}}{(2^{2})^{2}}\)
\item
\(4^{2}\div 2^{4}\)
\item
\(2\times 4^{4}\div 8^{3}\)
\end{enumerate}
\end{multicols}
\انتہا{سوال}

\ابتدا{سوال}
درج ذیل تراکیب کو صحیح عدد یا کسر کی صورت میں لکھیں
\begin{multicols}{4}
\begin{enumerate}[a.]
\item
\(2^{-3}\)
\item
\(4^{-2}\)
\item
\(5^{-1}\)
\item
\(3^{-2}\)
\item
\(10^{-4}\)
\item
\(1^{-7}\)
\item
\(\big(\frac{1}{2}\big)^{-1}\)
\item
\(\big(\frac{1}{3}\big)^{-3}\)
\item
\(\big(2\frac{1}{2}\big)^{-1}\)
\item\
\(2^{-7}\)
\item\
\(6^{-3}\)
\item
\(\big(1\frac{1}{3}\big)^{-3}\)
\end{enumerate}
\end{multicols}
\انتہا{سوال}

\ابتدا{سوال}
\(x=2\) کے ساتھ درج ذیل میں سے ہر ایک کی قیمت معلوم کریں
\begin{multicols}{3}
\begin{enumerate}[a.]
\item
\(4x^{-3}\)
\item
\((4x)^{-3}\)
\item
\(\frac{1}{4}x^{-3}\)
\item
\(\big(\frac{1}{4}x\big)^{-3}\)
\item
\((4\div x)^{-3}\)
\item
\((x\div 4)^{-3}\)
\end{enumerate}
\end{multicols}
\انتہا{سوال}
\ابتدا{سوال}
\(y=5\) کے ساتھ درج ذیل میں سے ہر ایک کی قیمت معلوم کریں
\begin{multicols}{3}
\begin{enumerate}[a.]
\item
\((2y)^{-1})\)
\item
\(2y^{-1}\)
\item
\(\big(\frac{1}{2}y\big)^{-1}\)
\item
\(\frac{1}{2}y^{-1}\)
\item
\(\frac{1}{(2y)^{-1}}\)
\item
\(\frac{2}{(y^{-1})^{-1}}\)
\end{enumerate}
\end{multicols}
\انتہا{سوال}


\ابتدا{سوال}
درج ذیل تراکیب کو ممکنہ سادہ ترین شکل میں لکھیں
\begin{multicols}{3}
\begin{enumerate}[a.]
\item
\(a^{4}\times a^{-3}\)
\item
\(\frac{1}{b^{-1}}\)
\item
\((c^{-2})^{3}\)
\item
\(d^{-1}\times 2d\)
\item
\(e^{-4}\times e^{-5}\)
\item
\(\frac{f^{-2}}{f^{3}}\)
\item
\(12g^{3}\times(2g^{2})^{-2}\)
\item
\((3h^{2})^{-2}\)
\item
\((3i^{-2})^{-2}\)
\item
\(\big(\frac{1}{2}j^{-2}\big)^{-3}\)
\item
\((2x^{3}y^{-1})^{3}\)
\item
\((p^{2}q^{4}r^{3})^{-4}\)
\item
\((4m^{2})^{-1}\times 8m^{3}\)
\item
\((3n^{-2})^{4}\times (9n)^{-1}\)
\item
\((2xy^{2})^{-1}\times (4xy)^{2}\)
\item
\((5a^{3}c^{-1})^{2}\div (2a^{-1}c^{2})\)
\item
\((2q^{-2})^{-2}\div \big(\frac{4}{q}\big)^{2}\)
\item
\((3x^{-2}y)^{2}\div(4xy)^{-2}\)
\end{enumerate}
\end{multicols}
\انتہا{سوال}
\ابتدا{سوال}
درج ذیل تراکیب کو حل کریں
\begin{multicols}{3}
\begin{enumerate}[a.]
\item
\(3^{x}=\frac{1}{9}\)
\item
\(5^{y}=1\)
\item
\(2^{z}\times 2^{z-3}=32\)
\item
\(7^{3x}\div 7^{x-2}=\frac{1}{49}\)
\item
\(4^{y}\times 2^{y}=8^{120}\)
\item
\(3^{t}\times 9^{t\div 3}=27^{2}\)
\end{enumerate}
\end{multicols}
\انتہا{سوال}
\ابتدا{سوال}
ایک مکعب کی ہر طرف کی لمبائی \(3\times 10^{-2}\) میٹر ہے۔
(ا) مکعب کا ہجم معلوم کریں
(ب) مکعب کا کل سطحی رقبہ معلوم کریں
\انتہا{سوال}
\ابتدا{سوال}
ایک کھلاڑی\(7.5\times 10 ^{-3}\) گھنٹے میں  \(\SI{2e-1}{\kilo\meter}\) کلومیٹر کا فاصلہ طے کرتا ہے۔ اس کی اوسط رفتار کلومیٹر فی گھنٹہ معلوم کریں۔
\انتہا{سوال}
\ابتدا{سوال}
ایک 
\(L\)
لمبائی رکھنے والی تار کا ہجم
 \(V\,m^{3}\)
 یوں بیان کیا گیا ہے۔ جبکہ اس کے عمودی تراش کا رداس
 \(r\)
  ہے۔
(ا) 
\(80\)
میٹر لمبائی اور \(2\times 10^{-3}m\) عمودی تراش کے رداس کی تار کا ہجم معلوم کریں۔

(ب) ایک اور تار جس کی عمودی تراش کا رداس \(5\times 10^{-3}m^{3}\)  اور ہجم \(8\times 10^{-3}m^{3}\) ہے، کی لمبائی معلوم کریں۔

(ج) ایک تار جس کی لمبائی
 \(60m\) اور ہجم
 \(6\times 10^{-3}m^{3}\) ہے۔ اس کی عمودی تراش کا رداس معلوم کریں۔
\انتہا{سوال}
\ابتدا{سوال}
ایک مساوات جو موج کو سمجھتے ہوئے سامنے آتی ہے یہ ہے۔
\(y=\frac{\lambda d}{a}\)

(ا) \(y\) معلوم کریں، جبکہ 
\(\lambda=7\times 10^{-7}\)،
\(d=5\times 10^{-1}\) اور 
\(a=8\times 10^{-4}\) ہے۔

(ب)\(\lambda\) معلوم کریں، جبکہ \(y=10^{-3}\)، 
\(d=0.6\) اور \(a=2.7\times 10^{-4}\)
 ہے۔
\انتہا{سوال}
\ابتدا{سوال}
حل کریں

\(\frac{3^{5x+2}}{9^{1-x}}=\frac{27^{4+3x}}{729}\)
\انتہا{سوال}

\حصہ{کسری طاقتیں}
گزشتہ حصے میں آپ دیکھ چکے ہیں کہ طاقت کے قوانین صحیح اعداد 
\(m\)
 اور
 \(n\) 
کی مثبت اور منفی دونوں طاقتوں کے لیے ٹھیک کام کرتے ہیں۔ لیکن اگر 
\(m\) اور
 \(n\) اعداد صحیح ہی نہ ہوں تو کیا ہو گا۔
اگر ہم طاقت پہ طاقت کے قانون میں \(m=\frac{1}{2}\) اور \(n=2\) مانیں  تو ہم اس نتیجے پہ پہنچیں گے 
\[(x^{\frac{1}{2}})^{2}=x^{\frac{1}{2}\times 2}=x^{1}=x\]

\(x^{\frac{1}{2}}=y\) سمجھنے سے یہ مساوات\(y^{2}=x\) بن جائے گی۔
لہٰذا \(y=\sqrt{x}\) یا \(y=-\sqrt{x}\) جس سے 
\(x^{\frac{1}{2}}=\sqrt{x}\)
 یا \(-\sqrt{x}\)۔ 
\(x^{\frac{1}{2}}\)  کو \(x\) کی مثبت جذر ماننے سے ہمیں 
\(x^{\frac{1}{2}}=\sqrt{x}\) ملے گا۔
اسی طرح اگر ہم 
\(m=\frac{1}{3}\) اور \(n=3\) رکھیں تو ہم ظاہر کر سکتے ہیں کہ \(x^{\frac{1}{3}}=\sqrt[3]{x}\)۔ 
اس سے زیادہ وسیع طریقے سے ہم دیکھ سکتے ہیں کہ \(m=\frac{1}{n}\)، ہم دیکھیں گے \((x^{1}{n})^{n}=x^{\frac{1}{n}\times n}=x\)
جو ہمیں ایک بڑا نتیجہ دے گا جو کہ 
\[x^{\frac{1}{n}}=\sqrt[n]{x}\]

توجہ کیجیے کہ \(x^{1}{2}=\sqrt{x}\) کی صورت میں لازمی طور پہ \(x\le0\) ہو گا، لیکن \(x^{1}{3}=\sqrt[3]{x}\) کی صورت میں لازمی طور پہ \(x\le0\) کی ضرورت نہیں ہو گی، کیونکہ ہم کسی منفی نمبر کا مکعب جذر تو بہرحال لے سکتے ہیں۔
\(x^{\frac{1}{n}}=\sqrt[n]{x}\)
کو ذرا سا بڑھا کر دیکھیں تو ہم دیکھ سکتے ہیں کہ \(x^{\frac{2}{3}}\) کی قسم کی تراکیب کو کیسے حل کرنا ہے۔
اس کے دو متبادل ہو سکتے ہیں۔

\(x^{\frac{2}{3}}=x^{\frac{1}{3}\times 2}=(\sqrt[3]{x})^{2}\)
اور
\(x{\frac{2}{3}}=x^{2\times \frac{1}{3}}=(x^{2})^{\frac{1}{3}}=\sqrt[3]{x^{2}}\)

(اگر \(x\) کی قطعی مکعب جذر ہو تو اس کے لیے پہلی شکل بہتر ہے، ورنہ دوسری قسم بہتر ہے)
عمومی طور پہ یہی منطق ہمیں کسری طاقتوں کے اصولوں تک لے جاتی ہے۔

جذری طاقت کا قانون 
\[x^{\frac{m}{n}}=(\sqrt[n]{x})^{m}=\sqrt[n]{x^{m}}\]

جذری طاقتوں کو \(x^{1/2}\)، \(x^{m/n}\) بھی لکھا جا سکتا ہے اور اسی طرح مزید بھی۔
\ابتدا{مثال}
سادہ کریں۔ (ا) \(9^{\frac{1}{2}}\)، (ب)\(3^{\frac{1}{2}}\times 3^{\frac{3}{2}}\)، (ج)\(16^{-\frac{3}{4}}\)

حل:
(ا)
\(9^{\frac{1}{2}}=\sqrt{9}=3\)

(ب)\(3^{\frac{1}{2}}\times 3^{\frac{3}{2}}=3^{\frac{1}{2}+\frac{3}{2}}=3^{2}=9\)

(ج) پہلا طریقہ \(16^{-\frac{3}{4}}=(2^{4})^{-\frac{3}{4}}=2^{-3}=\frac{1}{8}\)

دوسرا طریقہ 
\(16^{-\frac{3}{4}}=\frac{1}{16^{\frac{3}{4}}}=\frac{1}{(\sqrt[4]{16})^{3}}=\frac{1}{2^{3}}=\frac{1}{8}\)
\انتہا{مثال}

طاقت کے معمے حل کرنے کے لیے بہت سے متبادل طریقے بھی موجود ہیں اور آپ کو ان کا بھی تجربہ کرنا چاہیے۔ بہت سے لوگ مثبت طاقت میں سوچنا آسان سمجھتے ہیں لہذا وہ منفی طاقت کو مثبت بنا کر آسانی سے حل کر سکتے ہیں، اگر آپ بھی ایسے ہی ہیں تو آپ پہلا مرحلہ  \(16^{-\frac{3}{4}}=\frac{1}{16^{\frac{3}{4}}}\) یوں لکھ سکتے ہیں، بالکل جیسے ہم نے دوسرے طریقے میں دیکھا۔

\ابتدا{مثال}
سادہ کریں
(ا) \((2\frac{1}{4})^{-\frac{1}{2}}\)، (ب)\(2x^{\frac{1}{2}}\times 3x^{-\frac{5}{2}}\)، (ج)\(\frac{(2x^{2}y^{2})^{-\frac{1}{2}}}{(2xy^{-2})^{\frac{3}{2}}}\)

حل:
(ا)\((2\frac{1}{4})^{-\frac{1}{2}}=(\frac{9}{4})^{-\frac{1}{4}}=(\frac{4}{9})^{\frac{1}{2}}=\sqrt{\frac{4}{9}}=\frac{2}{3}\)

(ب)
\(2x^{\frac{1}{2}}\times 3x^{-\frac{5}{2}}=6x^{\frac{1}{2}-\frac{5}{2}}=6x^{-2}=\frac{6}{x^{2}}\)

(ج)  پہلا طریقہ
\((2x^{2}y^{2})^{-\frac{1}{2}}=\frac{1}{(2x^{2}y^{2})^{\frac{1}{2}}}=\frac{1}{2^{\frac{1}{2}xy}}\)

\(\frac{(2x^{2}y^{2})^{-\frac{1}{2}}}{(2xy^{-2})^{\frac{3}{2}}}=\frac{1}{2^{\frac{1}{2}xy}}\times \frac{1}{2^{\frac{3}{2}}x^{\frac{3}{2}}y^{-3}}= \frac{1}{2^{2}x^{\frac{5}{2}}y^{-2}}=\frac{y^{2}}{4x^{\frac{5}{2}}}\)

دوسرا طریقہ \((2xy^{-2})^{\frac{3}{2}}\) سے تقسیم کر نا ایسا ہی ہے جیسا \((2xy^{-2})^{-\frac{3}{2}}\) سے ضرب دینا۔

\(\frac{(2x^{2}y^{2})^{-\frac{1}{2}}}{(2xy^{-2})^{\frac{3}{2}}}=(2x^{2}y^{2})^{-\frac{1}{2}}(2xy^{-2})^{-\frac{3}{2}}=(2^{-\frac{1}{2}}x^{-1}y^{-1})(2^{-\frac{3}{2}}x^{-\frac{3}{2}}y^{3})=2^{-2}x^{\-\frac{5}{2}}y^{2}\)

جز ج میں ایک نکتہ قابل توجہ ہے اور وہ یہ کہ دونوں طریقوں سے جواب مختلف آ رہا ہے، اور ہم سمجھ سکتے ہیں کہ مساوات کا سادہ ہونا ہر ایک کے مزاج کے مطابق مختلف ہو سکتا ہے۔
\انتہا{مثال}


\ابتدا{سوال}

کیلکولیٹر کی مدد کے بغیر درج ذیل تراکیب کا مساوی لکھیں

\begin{multicols}{4}
\begin{enumerate}[a.]
\item
\(25^{\frac{1}{2}}\)
\item
\(8^{\frac{1}{3}}\)
\item
\(36^{\frac{1}{2}}\)
\item
\(32^{\frac{1}{5}}\)
\item
\(81^{\frac{1}{4}}\)
\item
\(9^{-\frac{1}{2}}\)
\item
\(16^{-\frac{1}{4}}\)
\item
\(49^{-\frac{1}{2}}\)
\item
\(1000^{-\frac{1}{3}}\)
\item
\((-27)^{\frac{1}{3}}\)
\item
\(64^{\frac{2}{3}}\)
\item
\((-125)^{-\frac{4}{3}}\)
\end{enumerate}
\end{multicols}
\انتہا{سوال}

\ابتدا{سوال}

کیلکولیٹر کی مدد کے بغیر درج ذیل تراکیب کا مساوی لکھیں

\begin{multicols}{4}
\begin{enumerate}[a.]
\item
\(4^{\frac{1}{2}}\)
\item
\((\frac{1}{2})^{2}\)
\item
\((\frac{1}{4})^{-2}\)
\item
\(4^{-\frac{1}{2}}\)
\item
\((\frac{1}{2})^{-\frac{1}{2}}\)
\item
\((\frac{1}{4})^{\frac{1}{2}}\)
\item
\(4^{2}\)
\item
\(((\frac{1}{4})^{\frac{1}{4}})^{2}\)
\end{enumerate}
\end{multicols}
\انتہا{سوال}


\ابتدا{سوال}

کیلکولیٹر کی مدد کے بغیر درج ذیل تراکیب کا مساوی لکھیں

\begin{multicols}{4}
\begin{enumerate}[a.]
\item
\(8^{\frac{2}{3}}\)
\item
\(4^{\frac{3}{2}}\)
\item
\(9^{-\frac{3}{2}}\)
\item
\(27^{\frac{4}{3}}\)
\item
\(32^{\frac{2}{5}}\)
\item
\(64^{-\frac{5}{6}}\)
\item
\(4^{2\frac{1}{2}}\)
\item
\(\num{10000}^{-\frac{3}{4}}\)
\item
\((\frac{1}{125})^{-\frac{4}{3}}\)
\item
\((3\frac{3}{8})^{\frac{2}{3}}\)
\item
\((2\frac{1}{4})^{-\frac{1}{2}}\)
\end{enumerate}
\end{multicols}
\انتہا{سوال}

\ابتدا{سوال}
درج ذیل مساواتوں کو سادہ بنائیں
\begin{multicols}{3}
\begin{enumerate}[a.]
\item
\(a^{\frac{1}{3}}\times a^{\frac{5}{3}}\)
\item
\(3b^{\frac{1}{2}}\times 4b^{-\frac{3}{2}}\)
\item
\((6c^{\frac{1}{4}})\times (4c)^{\frac{1}{2}}\)
\item
\((d^{2})^{\frac{1}{3}}\div (d^{\frac{1}{3}})^{2}\)
\item
\((2x^{\frac{1}{2}}y^{\frac{1}{3}})^{6}\times (\frac{1}{2}x^{\frac{1}{4}}y^{\frac{3}{4}})^{4}\)
\item
\((24e)^{\frac{1}{3}}\div (3e)^{\frac{1}{3}}\)
\item
\(\frac{(5p^{2}q^{4})^{\frac{1}{3}}}{(25pq^{2})^{-\frac{1}{3}}}\)
\item
\((4m^{3}n)^{\frac{1}{4}}\times (8mn^{3})^{\frac{1}{2}}\)
\item
\(\frac{(2x^{2}y^{-1})^{-\frac{1}{4}}}{(8x^{-1}y^{2})^{-\frac{1}{2}}}\)
\end{enumerate}
\end{multicols}
\انتہا{سوال}

\ابتدا{سوال}
درج ذیل مساواتوں کو حل کریں
\begin{multicols}{4}
\begin{enumerate}[a.]
\item
\(x^{\frac{1}{2}}=8\)
\item
\(x^{\frac{1}{3}}=3\)
\item
\(x^{\frac{2}{3}}=4\)
\item
\(x^{\frac{2}{3}}=27\)
\item
\(x^{-\frac{3}{2}}=8\)
\item
\(x^{-\frac{2}{3}}=9\)
\item
\(x^{\frac{3}{2}}=x\sqrt{2}\)
\item
\(x^{\frac{3}{2}}=2\sqrt{x}\)
\end{enumerate}
\end{multicols}
\انتہا{سوال}
\ابتدا{سوال}
\(L\)
میٹر لمبائی کی ایک لٹکن کو ایک گردش مکمل کرنے کے لیے 
\(T\) وقت درکار ہے، جسے یوں لکھا جائے گا۔
\(T=2\pi l^{\frac{1}{2}}g^{-\frac{1}{2}}\)
 جبکہ
\(g\approx 9.81 ms^{-2}\)
(ا)ایک
 \(0.9\) 
میٹر لمبی لٹکن کا وقت
 \(T\) 
دریافت کریں۔
(ب) ایک ایسی لٹکن کی لمبائی معلوم کریں کہ جسے ایک گردش کے لیے تین سیکنڈ کا وقت درکار ہے۔
\انتہا{سوال}

\ابتدا{سوال}
ایک کرے کے رداس 
\(r cm\) اور ہجم \(V cm^{3}\) کے درمیان تعلق
\(r=\big(\frac{3V}{4\pi}\big)^{\frac{1}{3}}\)
بنتا ہے۔ ایک ایسے کرے کا رداس معلوم کریں جس کا ہجم
 \(1150cm^{3}\) ہو۔
\انتہا{سوال}
\ابتدا{سوال}
درج ذیل مساواتوں کو حل کریں
\begin{multicols}{4}
\begin{enumerate}[a.]
\item
\(4^{x}=32\)
\item
\(9^{y}=\frac{1}{27}\)
\item
\(16^{z}=2\)
\item
\(100^{x}=1000\)
\item
\(8^{y}=16\)
\item
\(8^{z}=\frac{1}{128}\)
\item
\((2t)^{3}\times 4^{t-1}=16\)
\item
\(\frac{9^{y}}{27^{2y+1}}=81\)
\end{enumerate}
\end{multicols}
\انتہا{سوال}

\ابتدا{سوال}
سادہ کریں
\begin{multicols}{2.}
\begin{enumerate}[a.]
\item
\(5(\sqrt{2}+1)-\sqrt{2}(4-3\sqrt{2})\)
\item
\((\sqrt{2})^{4}+(\sqrt{3})^{4}+(\sqrt{4})^{4}\)
\item
\((\sqrt{5}-2)^{2}+(\sqrt{5}-2)(\sqrt{5}+2)\)
\item
\((2\sqrt{2})^{5}\)
\end{enumerate}
\end{multicols}

\انتہا{سوال}
\ابتدا{سوال}
سادہ کریں
\begin{multicols}{2.}
\begin{enumerate}[a.]
\item
\(\sqrt{27}+\sqrt{12}-\sqrt{3}\)
\item
\(\sqrt{63}-\sqrt{28}\)
\item
\(\sqrt{\num{100000}}+\sqrt{1000}+\sqrt{10}\)
\item
\(\sqrt[3]{2}+\sqrt[3]{16}\)
\end{enumerate}
\end{multicols}
\انتہا{سوال}
\ابتدا{سوال}
درج ذیل کے نسب نما کو ناطق بنائیں
\begin{multicols}{4}
\begin{enumerate}[a.]
\item
\(\frac{9}{2\sqrt{3}}\)
\item
\(\frac{1}{5\sqrt{5}}\)
\item
\(\frac{2\sqrt{5}}{3\sqrt{10}}\)
\item
\(\frac{\sqrt{8}}{\sqrt{15}}\)
\end{enumerate}
\end{multicols}
\انتہا{سوال}
\ابتدا{سوال}
سادہ کریں
\begin{multicols}{2}
\begin{enumerate}[a.]
\item
\(\frac{4}{\sqrt{2}}-\frac{4}{\sqrt{8}}\)
\item
\(\frac{10}{\sqrt{5}}+\sqrt{20}\)
\item
\(\frac{1}{\sqrt{2}}(2\sqrt{2}-1)+\sqrt{2}(1-\sqrt{8})\)
\item
\(\frac{\sqrt{6}}{\sqrt{2}}+\frac{3}{\sqrt{3}}+\frac{\sqrt{15}}{\sqrt{5}}+\frac{\sqrt{18}}{\sqrt{6}}\)
\end{enumerate}
\end{multicols}
\انتہا{سوال}
\ابتدا{سوال}
\(\frac{5}{\sqrt{7}}\) کو \(k\sqrt{7}\) شکل میں بنا کر دکھائیں، جبکہ \(k\) ایک ناطق عدد ہے۔
\انتہا{سوال}

\ابتدا{سوال}
اس نتیجے کو درست ثابت کریں \(\sqrt{12}\times\sqrt{75}=30\)

(ا) غیر معقول اعداد کو استعمال کرتے ہوئے
(ب) کسری طاقتیں استعمال کرتے ہوئے
\انتہا{سوال}

\ابتدا{سوال}
اس شکل میں زاویہ \(ABC\) اور \(ACD\) قائم زاویے ہیں۔
اگر ہم جانتے ہوں کہ \(AB=CD=2\sqrt{6}\) اور \(BC=7cm\) تو ظاہر کریں کہ \(AD\) کی لمبائی \(4\sqrt{6}\) اور \(7\sqrt{2}\) کے درمیان ہے۔
\انتہا{سوال}

\ابتدا{سوال}
مثلث \(PQR\) میں \(Q\) ایک قائمہ زاویہ ہے۔ \(PQ=(6-2\sqrt{2})cm\) اور \(QR=(6+2\sqrt{2})cm\)
(ا) مثلث کا رقبہ دریافت کریں
(ب) ظاہر کریں کہ \(PR\) کی لمبائی\(2\sqrt{22}cm\) ہے۔
\انتہا{سوال}

\ابتدا{سوال}
ترکیب \(\sqrt[3]{36}\times\sqrt[6]{\frac{4}{3}}\times\sqrt{27}\)
 کے ہر جز کو طاقت میں لکھ کر سادہ بنائیں
\انتہا{سوال}

\ابتدا{سوال}
ایک مثلث \(ABC\) میں، \(AB=4\sqrt{3}cm\)،\(BC=5\sqrt{3}cm\) اور زاویہ \(B\) \(60\degree\) ہے۔ کوسائن قاعدے کی مدد سے \(AC\) کی لمبائی سادہ غیر معقول اعداد میں نکالیں۔
\انتہا{سوال}

\ابتدا{سوال}
درج ذیل ہمزاد مساواتوں کو حل کریں
\(5x-3y=41\) اور \(7\sqrt{2})x+(4\sqrt{2})y=82\)
\انتہا{سوال}

\ابتدا{سوال}
اپنے کیلکولیٹر پہ موجود طاقت بنائیں والا بٹن استعمال کرتے ہوئے 5 معین اعداد تک جواب ڈھونڈیں
(ا) 
\(\frac{1}{3.7^{5}}\)
(ب)
 \(\sqrt[5]{3.7}\)
\انتہا{سوال}

\ابتدا{سوال}
نقاط \(A\) اور \(B\) کے محدد، بالترتیب \((2,3)\) اور
\((4,-3)\) ہیں۔ \(AB\) کی لمبائی اور اس کے درمیانی نقطے کے محدد معلوم کریں۔
\انتہا{سوال}

\ابتدا{سوال}
(ا) ایک خط \(l\) کی مساوات دریافت کریں، جو نقطہ \(A(2,3)\) سے ڈھلوان\(\frac{-1}{2}\) کے ساتھ گزر رہی ہو۔
(ب)ظاہر کریں کہ نقطہ \(P\) جس کے محدد \((2+2t,3-t)\) ہیں، ہمیشہ \(l\) پر رہے گا، بھلے \(t\)کی قیمت کچھ بھی ہو۔
(ج) \(t\) کی قیمت دریافت کریں، ایسے کہ \(AP\) کی لمبائی \(5\) رہے۔
(د) \(t\) کی قیمت دریافت کریں جب کہ \(OP\) \(l\) کے عمودی ہے۔ \(O\) کو نقطہ آغاز مانتے ہوئے \(OL\)عمودی خط کی لمبائی معلوم کریں
\انتہا{سوال}



\ابتدا{سوال}
\(P\) اور \(Q\) ایک خط کے انقطاع کے نقطے ہیں
اور \(x\) اور \(y\) محور بالترتیب یہ ہیں۔
\[\frac{x}{a}+\frac{y}{b}=1\quad(a>0,b>0)\]
\(PQ\) کا درمیانی فاصلہ \(20\)ہے اور اس کی ڈھلوان \(-3\) ہے۔ اس سب کے ساتھ \(a\) اور \(b\) کی قیمت معلوم کریں۔

\انتہا{سوال}
\ابتدا{سوال}
ایک چوکور کی اطراف ان خطوط پر موجود ہیں \(y=2x-4, y=2x-13, x+y=5\)۔\(x+y=-4\)۔
اس کی ایک سمت اور اس کے اور اس کے متوازی سمت کا درمیانی فاصلہ معلوم کریں۔ نیز اس چوکور کا رقبہ بھی دریافت کریں۔
\انتہا{سوال}

\ابتدا{سوال}
درج ذیل کو عداد کی مدد کے بغیر حل کریں
\begin{multicols}{4}
\begin{enumerate}[a.]
\item
\(\Big(\frac{1}{2}\Big)^{-1}+\Big(\frac{1}{2}\Big)^{-2}\)
\item
\(32^{-\frac{4}{5}}\)
\item
\(\Big(4^{\frac{3}{2}}\Big)^{-\frac{1}{3}}\)
\item
\(\Big(1\frac{7}{9}\Big)^{1\frac{1}{2}}\)
\end{enumerate}
\end{multicols}

\انتہا{سوال}

\ابتدا{سوال}
ترکیب \(\big(9a^{4}\big)^{-\frac{1}{2}}\)
کو الجبرائی کسرے کی شکل میں لکھ کر سادہ بنائیں
\انتہا{سوال}
\ابتدا{سوال}
\(y=x^{\frac{1}{3}}\) سمجھتے ہوئے، \(x\) کی قیمت معلوم کریں، جس کے لیے \(x^{\frac{1}{3}}-2x^{-\frac{1}{3}}=1\)
\انتہا{سوال}


\ابتدا{سوال}
مساوات\(4^{2x}\times 8^{x-1}=32\) کو حل کریں
\انتہا{سوال}


\ابتدا{سوال}
ترکیب \(\frac{1}{(\sqrt{a})^\frac{4}{3}}\) کو \(a^n\) کی شکل میں لکھیں اور  \(n\) کی قیمت بتائیں۔
\انتہا{سوال}


\ابتدا{سوال}
سادہ کریں
\begin{multicols}{2.}
\begin{enumerate}[.a]
\item
\((4p^{\frac{1}{4}}q^{-3})^{\frac{1}{2}}\)
\item
\(\frac{(5b)^{-1}}{(8b^{6})^{\frac{1}{3}}}\)
\item
\((2x^{6}y^{8})^{\frac{1}{4}}\times (8x^{-2})^{\frac{1}{4}}\)
\item
\((m^{\frac{1}{3}}n^{\frac{1}{2}})^2\times (m^{\frac{1}{6}}n^{\frac{1}{3}})^{4}\times(mn)^{-2}\)
\end{enumerate}
\end{multicols}
\انتہا{سوال}

\ابتدا{سوال}
یہ نظر میں رکھتے ہوئے کہ معیاری شکل میں \(3^{236}\approx4\times10^{112}\) اور \(3^{-376}\approx4\times 10^{-180}\)
، درج زیل تراکیب کے لیے معیاری شکل میں اندازے معلوم کریں

\begin{multicols}{4}
\begin{enumerate}[.a]
\item
\(3^{376}\)
\item
\(3^{612}\)
\item
\((\sqrt{3})^{236}\)
\item
\((3^{-376})^{\frac{5}{2}}\)
\end{enumerate}
\end{multicols}
\انتہا{سوال}


\ابتدا{سوال}
ذیل میں دیا گیا جدول تین سیاروں کا سورج سے اوسط فاصلہ اور ایک گردش کے لیے درکار وقت بتا رہا ہے

(ا) دکھائیں کہ 
\(r^{3}T^{-2}\) تینوں سیاروں کے لیے تقریباً ایک سی قیمت رکھتا ہے۔
(ب) زمین سورج کے گرد ایک چکر مکمل کرنے میں ایک سال لگاتی ہے، زمین کے مدار کا اوسط رداس معلوم کریں
\انتہا{سوال}


\ابتدا{سوال}
سادہ کریں

(ا)\(2^{-\frac{3}{2}}+2^{-\frac{1}{2}}+2^{\frac{1}{2}}+2^{\frac{3}{2}}\)، اپنے جواب کو \(k\sqrt{2}\)
 کی شکل میں لکھیں۔

(ب) \((\sqrt{3})^{-3}+(\sqrt{3})^{-2}+(\sqrt{3})^{-1}+(\sqrt{3})^{0}(\sqrt{3})^{1}+(\sqrt{3})^{2}+(\sqrt{3})^{3}\)
، اپنے جواب کو \( a+b\sqrt{3}\)
 کی شکل میں لکھیں۔
\انتہا{سوال}

\ابتدا{سوال}
درج ذیل میں سے ہر ایک کو \(2^{n}\) کی شکل میں ظاہر کریں

\begin{multicols}{2}
\begin{enumerate}[.a]
\item
\(2^{70}+2^{70}\)
\item
\(2^{-400}+2^{-400}\)
\item
\(2^{\frac{1}{3}}+2^{\frac{1}{3}}+2^{\frac{1}{3}}+2^{\frac{1}{3}}\)
\item
\(2^{100}-2{99}\)
\item
\(8^{0.1}+8^{0.1}+8^{0.1}+8^{0.1}+8^{0.1}+8^{0.1}+8^{0.1}+8^{0.1}\)
\end{enumerate}
\end{multicols}
\انتہا{سوال}


\ابتدا{سوال}
مساوات کو حل کریں
\(\frac{125^{3x}}{5^{x+4}}=\frac{25^{x-2}}{3125}\)
\انتہا{سوال}

\ابتدا{سوال}
ایک کرے کے سطحی رقبے اور ہجم کے کلیے بالترتیب
 \(S=4\pi r^{2}\) 
اور 
\(V=\frac{4}{3}\pi r^{3}\) 
ہیں۔ جبکہ
 \(r\) کرے کا رداس ہے۔ درجذیل کے لیے موزوں تراکیب بنائیے۔

(ا)سطحی رقبے کو ہجم کے ذریعے لکھیں

(ب)ہجم کو سطحی رقبے کے ذریعے لکھیں
\انتہا{سوال}


\ابتدا{سوال}
\(mKg\) وزن کے حامل  اور\(vms^{-1}\) رفتار سے حرکت کرنے والے ایک جسم کی حرکی توانائی
 \(K\)کے لیے کلیہ \(K=\frac{1}{2}mv^{2}\) ہے۔
اس کلیے کو مدنظر رکھتے ہوئے
 \(2.5\times10^{-2}kg\) وزن رکھنے والی اور
 \(8\times10^{2}ms^{-1}\) رفتار سے حرکت کرنے والی گولی کی حرکی توانائی معلوم کریں۔
\انتہا{سوال}


