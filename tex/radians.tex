\باب{ریڈیئن}\شناخت{باب_ریڈیئن}
%Page 266\\
ایک دائرے کا مرکز 0 اور رداس 6 سم ہے۔ ایک مستقیم خط
PG
جس کی لمبائی 8 سم ہے اس میں ایک قطع بناتا ہے۔ اس قطعے کا احاطہ اور رقبہ دریافت کریں۔ آپکا جواب تین نمایاں ہندسوں تک درست ہو۔\\
اس طرح کے مسائل میں یہ مفید ہوتا ہے کہ آعاز کے لیے شکل
\(18.3\)
میں دکھائے نیم تاریک ضلع کی بجائے پورے احاطے
OPQ
پر غور کیا جائے۔\\
اس قطع کا احاطہ دو حصوں پر مشتمل ہے۔
8 سم لمبائی والا سیدھا حصہ اور خط منحنی والا حصہ۔ منحنی حصہ کی لمبائی معلوم کرنے کے لیے زاویہ
POQ
کو جاننے کی ضرورت ہو گی۔\\
اس زاویے کو آپ
\(\theta\)
کا نام دے لیں۔ چونکہ یہ ایک مساوی الساقین (دو اغلاع برابر) تکون ہے۔ لہزہ مرکز 0 سے خط 
PQ
تک کھینچا گیا ایک خط سے خط 
PQ
اور زاویہ 
POQ
دونوں دو برابر حصوں میں تقسیم ہو جاتے ہیں۔ 
\[\sin\frac{1}{2}\theta=\frac{4}{6}=0.666...\]
لہذا 
\(\frac{1}{2}\theta=0.7297...\)
اور 
\(\theta=1.459...\)
\\اپنے عداد کو لازماً ریڈئین انداز میں کر لیں۔ 
اب اس کا احاطہ 
\(d=8+6\theta=16.756...\)
ٹھہرتا ہے۔ 
اسکا احاطہ 16.8 سم ہے جوکہ تین ہندسوں تک درست ہے\\
مذکورہ قطعے کا رقبہ معلوم کرنے کے لیے آپکو حلقہ 
OPQ
کا رقبہ معلوم کرنا ہو گا 
پھر اس رقبے میں سے مثلث 
OPQ
کا رقبہ نفی کرنا ہو گا۔ \\
اگر ہم کسی تکون کے رقبے کے لیے کلیہ 
\(\frac{1}{2}bc\sin A\)
کو استعمال میں لائیں تو تکون 
PQR
کا رقبہ 
\(\frac{1}{2}r^2\sin\theta\)
بنے گا۔ لہٰذا نیم تاریک حصے کا رقبہ یوں ہو گا 
\[\frac{1}{2}r^2\theta-\frac{1}{2}r^2\sin\theta=\frac{1}{2}\times6^2\times1.459...-\frac{1}{2}\times6^2\times\sin1.459...=8.381\]\\
اس موقع پر 
\(\theta\)
کی جو بھی قیمت آئے اس اپنے عداد میں محفوظ کر لینا آپکے لیے مفید ثابت ہو گا تا کہ بعد کے حساب کتاب میں اسے استعمال کر سکیں۔ \\
مثال 18.2.1 میں 
\(\sin\frac{1}{2}\theta=\frac{4}{6}==0.666...\)
\(\sin\theta\)
اور 
\(\sin1.459...\)
استعمال ہوئے ہیں تاہم یہ نشاندہی نہیں کی گئی کہ یہ زاویے ریڈئین میں تھے۔ روایتی طور پر ایسے حالات میں آپ ریڈئین اکائی مانتے ہیں مثلاً اگر 
\(\sin12\)
درج ہو تو آپ اسے 12 ریڈئین کا 
sine
سمجھیں گے۔ اگر یہ 120 کا 
Sine
ہوتا تو اسے 
\(\sin120\)
لکھا جاتا۔ \\
مثال 18.2.2
ایک مستقیم خط دائرے کے مرکز پر 
\(\theta\)
زاویہ بناتا ہے۔ اور اس طرح دائرہ کا ایک حصہ قطع کرتا ہے۔ اس حصے کا رقبہ دائرے کے کل رقبے کا 
\(\frac{1}{3}\)
ہے۔ \\
(ا) اخز کریں کہ 
\(\theta-\sin\theta=\frac{2}{3}\pi\)\\
(ب) ثابت کریں کہ 
\(\theta=2.61\)
دو اعشاری نقطوں تک درست ہے۔ \\
%Page 267\\
(ا) رداس کو 
r
مان لیں۔ 
\\اگر ہم مثال 18.2.1 میں استعمال کردی طریقہ سے فائدہ اٹھائیں تو اس حصے کا رقبہ درج ذیل ہو گا 
\[{1/2}r^2\theta-{1/2}r^2\sin\theta\]
یہ دائرے کے کل رقبے کا 
\(1/3\)
تب قرار پائے گا جب 
\[1/2r^2\theta-1/2r^2\sin\theta=1/3\pi r^2\]
اس مساوات کو 2 سے ضرب دیں اور 
\(r^{2}\)
سے تقسیم کر دیں تو آپکو درج زیل نتیجہ حاصل ہو گا\\
\[\theta-\sin\theta=2/3\pi\]
(ب) اگر ہم مساوات 
\(f(\theta)=\theta\sin{\theta}\)
میں 
\(\theta\)
کی قیمت 2.61 لگائیں تو 
\(f(2.61)=2.103\)
بنے گا جو 
2.094
کے بہت قریب ہے۔ \\
اس سے ہمیں یہ اندازہ ہوتا ہے کہ 
\(\theta\)
کی قیمت 2.61 کے بہت قریب ہے۔ لیکن یہ دع اعشاری نقطوں تک درست 2.61 بیان کو ثابت کرنے کے لیے ناکافی ہے 
اس مقصد کے لیے آپکو یہ ثابت کرنا پڑے گا کہ 
\(\theta\)
کی قیمت 2.605 اور 2.165 کے درمیان ہے۔ \\
شکل 18.5 سے عیاں ہے کہ 
\(\theta\)
کی قیمت 
0
اود 
\(\pi\)
کے درمیان ہے اور 
\(\theta\)
کے بڑھنے سے نیم تاریک حصہ بھی بڑھ جاتا ہے۔ چنانچہ آپکو یہ دکھانا ہے کہ 
\(\theta=2.605\)
ہونے سے یہ حصہ بہت چھوٹا ہو جاتا ہے جبکہ 
\(\theta=2.515\)
ہونے سے بہت بڑا ہو جاتا ہے۔ \\
\[f(2.605)=2.605-\sin2.605=2.093\dotsc\]
اور 
\[f(2.615)=2.615-\sin2.615=2.112\dotsc\]\\
پہلا جواب 
\(2/3\pi=2.094\dotsc\)
سے چھوٹا ہے جبکہ دوسرا جواب بڑا ہے۔ اس سے یہ ثابت ہوتا ہے کہ مساوات کا جزر  2.605 اور 2.615 کے درمیان ہے۔ لہٰذا جزر دو اعشاری نقطوں تک درست 2.61 ہے\\
مشق 18\\
درج ذیل میں سے ہر زاویے کو ریڈئین میں تبدیل کریں۔ آپ جواب کو 
\(\pi\)
کے مضرب چھوڑ سکتے ہیں۔ \\
\[90\]
\[135\]
\[45\]
\[30\]
\[72\]
\[18\]
\[120\]
\[22.5\]
\[720\]
\[600\]
\[270\]
\[1\]
مندرجہ ذیل تمام زاویے ریڈئین میں ہیں۔ عداد استعمال کیے بغیر انہیں درجوں میں تبدیل کریں۔ \\
\[(1/3)\pi\]
\[(1/20)\pi\]
\[(1/5)\pi\]
\[(1/8)\pi\]
\[(1/9)\pi\]
\[(2/3)\pi\]
\[(5/8)\pi\]
\[(3/5)\pi\]
\[(1/45)\pi\]
\[(6)\pi\]
\[(-1/2)\pi\]
\[(5/18)\pi\]
عداد استعمال کیے بغیر مندرجہ ذیل کو عین درست قیمتیں لکھیں۔ \\
\[\sin(1/3)\pi\]
\[\cos(1/4)\pi\]
\[\tan(1/6)\pi\]
\[\cos(3/2)\pi\]
\[\sin(7/4)\pi\]
\[\cos(7/6)\pi\]
\[\tan(5/3)\pi\]
\[\sin^2(2/3)\pi\]
%Page 268\\
زیریں مساواتیں اس شکل کے حوالے ہیں جبکہ 
R
دائرے کا رداس (سم) ہے \\
S
قوس کی لمبائی (سم) ہے۔ \\
اور 
A
ضلعے کا رقبہ (سم) ہے \\
جبکہ 
\(\theta\)
مرکز پر بننے والا زاویہ (ریڈئین) میں ہے\\
\(r=7\)
اور 
\(\theta=1.2\)
ہے۔ 
S
اور 
A
کی قیمتیں معلوم کریں۔ \\
\(r=3.5\)
اود 
\(\theta=2.1\)
ہے۔ 
S
اور 
A
کی قیمتیں معلوم کری\\
\(s=12\)
اود 
\(r=8\)
ہے۔ 
\(\theta\)
اور 
A
کی قیمتیں معلوم کریں\\
\(s=14\)
اود 
\(\theta=0.7\)
ہے۔
r
اور 
A
کی قیمتیں معلوم کریں\\
\(A=30\)
اود 
\(r=5\)
ہے۔
\(\theta\)
اور 
s
کی قیمتیں معلوم کری\\
\(A=64\)
اود 
\(s=16\)
ہے۔
\(\theta\)
اور 
r
کی قیمتیں معلوم کریں۔\\
\(A=30\)
اود 
\(s=10\)
ہے۔
\(\theta\)
کی قیمتیں معلوم کریں۔\\
درج ذیل ہر صورت میں نیم تاریک حصے کا رقبہ دریافت کریں 
\(r=5\)
\(\theta=(1/3)\pi\)\\
\(r=3.1\)
\(\theta=(2/5)\pi\)\\
\(r=28\)
\(\theta=(5/6)\pi\)\\
\(r=6\)
\(s=9\)\\
\(r=9.5\)
\(s=4\)\\
ایک دائرے کا رداس 13 سم ہے۔  10 سم لمبا ایک مستقیم خط، اس دائرے کا جو حصہ قطع کرتا ہے اس کا رقبہ معلوم کریں۔\\
ایک دائرہ جس کا رداس 25 سم ہے، 4 سم والا ایک مستقیم خط اس کے حصے کو منقطع کرتا ہے۔ اس حصے کا احاطہ دریافت کریں\\
ایک مستقیم خط دائرے کو اس طرح منقطع کرتا ہے کہ مرکزہ پر زاویہ 
\(\theta\)
بنائے اور منقطع حصےکا رقبہ دائرہ کے کل رقبے کا
\((1/4)\)
بناتا ہے \\
(ا) واضع کریں کہ 
\(\theta-\sin\theta=(1/2)\pi\)\\
(ب) ثابت کریں کہ 
\(\theta=2\pi\)
جبکہ یہ قیمت دو اعشاری نقطوں تک درست ہو۔ \\
دو دائرے جن کے رداس 5 سم اور 24 سم ہیں جزوی طور پر ایک دوسرے کے کونے ہیں۔ ان کے مراکز باہم 13 سم دور ہیں۔ دونوں میں مشترک رقبہ معلوم کریں۔ \\
اس شکل میں دو ایسے دائرے دکھائے گئے ہیں جو ایک دوسرے کو قطع کرتے ہیں۔ دائروں کے رداس 6 سم اور 4 سم ہیں جبکہ ان کے \\مراکز کا درمیانی فاصلہ 7 سم ہے۔ دونوں دائروں میں مشترک نیم تاریک حصے کا احاطہ اور رقبہ معلوم کریں۔ \\

%page 269-70 
سورج کے کرّے کا 10٪ حصّہ چاند کے کرّے سے ڈھک جائے تو اسے 10٪ سورج گرہن کہتے ہیں 
ایک بچہ اس کی دو کرّوں کی مدد سے تصویر کشی کرتا ہے ہر کرّے کا رداس r cm ہے جیسا کہ یہاں دکھایا گیا ہے۔\\
1۔ ان دونوں کرّوں کے مراکز کے درمیانی فاصلے کی پیمائش r کے حوالے سے کیجیے۔\\
2۔ انہی دو کرّوں کے مراکز کے درمیانی فاصلے کی پیمائش 80٪ سورج گرہن کے مطابق بھی کیجیے۔\\
18.3 مثلثاتی افعال کے ترسیمے\\
اگر کسی زاویہ کی ریڈئین میں ناپا جائے تو 
\(y=\cos\theta\)،
\(y=\sin\theta\)
اور
\(y=\tan\theta\)
کی اشکال ویسی ہی ہوں گی جیسی کہ 
\(y=\cos\theta\degree\)،
\(y=\sin\theta\degree\)
اور
\(y=\tan\theta\degree\)
کی ہوں گی صرف یہ فرق ہوگا کہ محور کے ساتھ ان کا پیمانہ مختلف ہوگا۔\\
\(\theta\)
کو ریڈئین میں رکھ کر 
\(y=\cos\theta\)،
\(y=\sin\theta\)
اور
\(y=\tan\theta\)
کے ترسیمات اشکال 18.6٫ 18.7٫ اور 18.8 میں دکھائے گئے ہیں۔\\


اگر آپ حصہ جات 10.1 اور 10.2 والے ترسیمات ہر سمت میں ایک جیسی پیمائش کے ساتھ بنائیں تو وہ یہاں دکھائے گئے ترسیمات کے مقابلے میں بہت زیادہ چوڑے اور چپٹے ہوں گے۔\\ 
در حقیقت اگر آپ کو 
\(y=\cos\theta\)،
\(y=\sin\theta\)
اور
\(y=\tan\theta\) 
کے ترسیمات دریافت کرنے ہوں تو اس کے لئے تقریباً ہمیشہ ریڈئین ہی استعمال کئے جاتے ہیں
ان ترسیمات میں توازن کی وہی خصوصیات موجود ہوتی ہیں جو کہ
\(y=\cos\theta^o\)،
\(y=\sin\theta^o\)
اور
\(y=\tan\theta^o\)
میں ہوتی ہیں۔
دورعا خصوصیات: 
\begin{enumerate}
\item
\(\cos(\theta\pm2\pi)=\cos\theta\)
\item
\(\sin(\theta\pm2\pi)=\sin\theta\)
\item
\(\tan(\theta\pm2\pi)=\tan\theta\)\\
\end{enumerate}
طارق/جفت خصوصیات: 
\begin{enumerate}
\item
\(\tan(\theta\pm2\pi)=\tan\theta\)
\item
\(\sin(-\theta)=-\sin\theta\)
\item
\(\tan(-\theta)=\tan\theta\)\\
\end{enumerate}
مستقیم حرکت کی خصوصیات:
\begin{enumerate}
\item
\(\cos(\theta-\pi)=-\cos\theta\)
\item
\(\sin(\theta-\pi)=-\sin\theta\)
\item
\(\cos(\pi-\theta)=-\cos\theta\)
\item
\(\sin(\pi-\theta)=-\sin\theta\)
\item
\(\tan(\pi-\theta)=-\tan\theta\)\\
\end{enumerate}
مشق 18ب 
1۔ 
\(y=\cos\theta\)
اور
\(y=\sin\theta\)
کے ترسیمات استعمال کر کے یہ دکھائے کہ 
\(\sin(\frac{1}{2}\pi-\theta)=\cos\theta\)\\
اس خصوصیت اور اس کے ساتھ اوپر خانے میں دی گئی cosine , sine اور tangent افعال کی توازن کی خصوصیات کو استعمال کر کے مندرجہ ذیل نتائج کو ثابت کیجئے
\begin{multicols}{2}
\begin{enumerate}[a]
\item\(\sin(\frac{3}{2}\pi+\theta)=-\cos\theta\)
\item\(\sin(\frac{1}{2}\pi+\theta)=-\cos\theta\)
\item\(\cos(\frac{1}{2}\pi+\theta)=-\sin\theta\)
\item\(\sin(-\theta-\frac{1}{2}\pi)=-\cos\theta\)
\end{enumerate}
\end{multicols}
%Page 271\\
اپنی محوروں کو استعمال کرتے ہوئے 
\(y=\tan\theta\)\
اور
 \(y=\frac{1}{\tan\theta}\)
کے خاکے بنائیں۔ نیز یہ دکھائیں کہ
\(\tan\Big(\frac{1}{2}-\theta\Big)=\frac{1}{\tan\theta}\)\\
\(\alpha\)
کی ایسی کم از کم قیمت تلاش کریں جس کے لیے\\
\begin{enumerate}
\item
\(\cos(\alpha-\theta)=\sin\theta\)
\item
\(\sin(\alpha-\theta)=\cos(\alpha+\theta)\)
\item
\(\tan\theta=\tan(\theta+\alpha)\)
\item
\(\sin(\theta+2\alpha)=\cos(\alpha-\theta)\)
\item
\(\cos(2\alpha-\theta)=\cos(\theta-\alpha)\)
\item
\(\sin(5\alpha+\theta)=\cos\theta-3\alpha\)
\end{enumerate}
الٹ تکونیاتی تفاعل\\
آپ اب تک کئی بار علامات
\(\sin^-1\)
، 
\(\cos^-1\)
، 
\(\tan^-1\)\
کو دیکھ چکے ہوں گے۔ اب وقت آگیا ہے کہ آپ کو الٹ تکونی تفاعل(تفالات) کی ایک جامع تعریف سے آگاہ کیا جائے۔\\
آپ حصہ 18.3 سے دیکھ سکتے ہیں کہ تفاعل
\(\cos x\)
، 
\(\sin x\)
اور 
 \(\tan x\)
ایک ایک نہیں ہوتے۔ حصہ 11.6 سے یہ نتیجہ نکالا جا سکتا ہے کہ جب تک ان کی تعریف کے دائرہ کار کو محدود نہ کر دیں ان کے الٹ نہیں ہوتے۔ یہاں ہم فرض کر رہیں کا آپ ریڈ ئین اکائی کو استعمال کر رہے ہیں۔\\
شکل 18.9 سے ہم دیکھ سکتے ہیں کہ 
\(\cos^{-1}\)
کی تعریف کے لیے 
cosine
تغاعل کے دائرہ کار کو
\( 0\le x \le \pi\) 
تک محدود کیا گیا ہے۔\\
%Page 272\\
ایک بار پھر دیکھیں کہ
\(y=\sin{x}\) 
کی ترسیم کا موٹا حصہ 
\( y=x\) 
میں 
 \(y=\sin^-1 x \) 
کا عکس ہے۔
اس کا الٹ  بھی درست ہے۔\\
\\مشق
سوالات 1 سے 5 میں آلہء حساب استعمال نہ کریں\\
\\دریافت کریں
\begin{enumerate}
\item
\(\cos^{-1 }\frac{1}{2}\sqrt{3}\)
\item
\(\tan^-{1} 1\)
\item
\(\cos^{-1} 0 \)
\item
\(\sin^{-1} \frac{1}{2}\sqrt{3}\)
\item
\(\tan^-{1} -\sqrt{3}\)
\item
\(\sin^-{1}-1\)
\item
\(\tan^{-1} -1\)
\item
\(\cos^-{1} -1\)
\end{enumerate}
ریافت کریں\\
\begin{enumerate}
\item
\(\cos^-1 \frac{1}{\sqrt{2}}\)
\item
\(\sin^-1 -0.5\)
\item
\(\cos^-1 -0.5 \)
\item
\(\tan^-1 \frac{1}{\sqrt{3}}\)
\item
\(\tan^-1 -\sqrt{3}\)
\end{enumerate}
دریافت کریں\\
\begin{enumerate}
\item \(\sin(\sin^-1 0.5)\)
\item \(\cos(\cos^-1 (-1))\)
\item \(\tan(\tan^-1\sqrt{3}) \)
\item \(\cos( \cos^-1 0)\)
\end{enumerate}
دریافت کریں\\
\begin{enumerate}
\item
\(\cos^-1(\cos \frac{3}{2}\pi)\)
\item
\(\sin^-1(\sin\frac{13}{6}\pi)\)
\item
\(\tan^-1(\tan \frac{1}{6}\pi) \)
\item
\(\cos^-1(\cos 2\pi)\)
\end{enumerate}
دریافت کریں\\
\begin{enumerate}
\item \(\sin(\cos^-1 \frac{1}{2}\sqrt{3})\)
\item \(\frac{1}{\tan(\tan^-1 2)}\)
\item \(\cos(\sin^-1 (-0.5)) \)
\item\(\tan(\cos^-1\frac{1}{2}\sqrt{2})\)
\end{enumerate}

کوئ
ی ترسیمی طریقہ استعمال کر کے مساوات
\(\cos x=\cos^-1 x \) 
کو حل کریں۔ آپ کا جواب 3 اشاری درجوں تک درست ہو۔ یہ کسی سادہ تر مساوات کا واحد جزر ہے؟\\
%Page 273\\
ریڈئیں کو استعمال کرتے ہوئے تکونیاتی مساواتوں کو حل کرنا\\
بعض اوقات تکونیاتی مساواتوں کو حل کرتے ہوئے آپ کسی زاویے کو ریڈئین میں تلاش کرنا چاہیں گے۔ اس کے اصول وہی ہوں گے جو آپ نے حصہ 10.5 میں درجوں (درجات) میں کام کرنے کے لیے استعمال کیے تھے۔ تاہم تغاعل 
\(\cos^-1\)
، 
\(\sin^-1 \)
اور
\(\tan^-1 \)
کے معانی وہی ہوں گے جو انہیں حصہ 18.4 میں تفویض کیے گئے تھے۔\\ 
مثال \\
مساوات 
\(\cos\theta=-0.7 \) 
کواس طرح حل کریں کا وقفہ
\(0 \le\theta \le 2\pi\) 
میں تمام جزر دو اشاری نقطوں تک درست آئیں۔\\ 
قدم 1\\
\(\cos^-1 (-0.7)=2.346...\) 
پہ وقفہ
\(0\le\theta \le2\pi\)
میں ایک جزر ہے \\
قدم 2\\
\(\cos(-\theta)=\cos\theta\) 
کی تشاکل کی خصوصیت کو استعمال کر کہ یہ دکھائیں کہ
-2.346..... 
ایک اور جزر ہے۔ توجہ کریں کہ
-2.346..... 
مطلوبہ وقفے میں نہیں ہے۔ \\
قدم 3\\
دوری خصوصيت 
\((\cos{\theta\pm 2\pi})=\cos{\theta}\)
کے استعمال سے واضع کریں
\(-2.346....+2\pi=3.936...\)
مطلوبہ وقفے میں ایک جزر ہے۔\\ 
وقفہ 
\(0\le\theta \le2\pi\) 
میں مساوات 
\(\cos(\theta)=-0.7\)    
کے جزر 2.35 اور 3.94 ہیں جو کہ دو اشاری نقطوں تک درست ہیں۔ \\
مثال \\
مساوات
\(\sin\theta=(-0.2) \) 
کو اس طرح حل کریں کہ وقفہ
\(-\pi \le\theta \le \pi\) 
میں دو اشاری نقطوں تک درست ہوں۔ \\
قدم 1\\
\(\sin^{-1}{0.2}=-0.201\)\\
مذکورہ بالا  وقفے میں ایک جزر ہے۔ \\
قدم 2\\
اس مساوات کا ایک جزر 
\(\pi-(-0.201\dotsc)=3.342\)
ہے۔ تاہم یہ مطلوبہ وقفے میں نہیں ہے۔\\ 
قدم 3\\
\(2\pi\)
کو تفریق کرنے سے ہمیں
-2.490
حاصل ہوتا ہے۔ یہ وقفہ
\(-\pi \le\theta \le \pi\)
میں ایک اور جزر ہے۔\\ 
لہٰذا 
\(-\pi \le\theta \le \pi\)
میں 
\(\sin\theta=(-0.2) \)
کے جزر
-2.94
اور
-0.20
ہیں جو دو اشاری نقطوں تک درست ہیں۔\\ 
مثال \\
مساوات 
\(\cos3\theta-0.1)=0.3\) 
کو اس طرح حل کریں کہ وقفہ
\(-\pi \le\theta \le \pi\)
میں تمام جزر دو اشاری نقطوں تک درست ہوں۔ \\
\(3\theta -0.1 =\)
کو 
\(\phi\)
فرض کر لیں تا کہ مساوات کی شکل 
\(\cos\phi=0.3\)
ہو جائے۔ چونکہ 
\(\theta\)
وقفہ 
\(-\pi \le\theta \le \pi\)
میں آتا ہے لہذا 
\(\phi=3\theta-0.1\)
وقفہ
\(-3\pi-0.1\le\phi\le3\pi-0.1\)   
میں آئے گا جو کہ
\(-9.524...\le\phi\le9.324\)
ہے۔ \\
اس مسئلے کا پہلا حصہ یہ ہے کہ
\(\cos\phi=0.3\)
کو حل کر کہ
\(-9.524...\le\phi\le9.324...\)
حاصل کیا جائے۔ \\
قدم 1\\
\(\cos^-1 0.3= 1.266...\) 
میں یہ وقفہ
\(-9.524..\le\phi\le9.324...\)
میں ایک جزر ہے۔ \\
قدم 2\\
اس حقیقت کے مطابق cosine تغاعل جفت ہے اور ایک جزر
1.266...
ہے۔


%274

درج ذیل صورت دے سکتے ہیں 
\(\sin^2\theta+\sin\theta-1=0\)
اب یہ 
\(\sin\theta\)
میں ایک دو درجی الجبرائی مساوات ہے ۔آپ اسے حصہ 
4.4
میں درج دو درجی الجبرائی کلیے کو استعمال کر کے حل کر سکتے ہیں ۔
\(\sin\theta=\frac{-1\pm\sqrt{1^2-4\times 1 \times(-1)}}{2}\)
جس سے 
\(\sin\theta=0.618\)
یا
\(\sin\theta=-1.618\)
حاصل ہوتا ہے ۔
ایک جزر 
\(\sin^-1 0.618...=0.666....\) 
ہے۔
\(\sin\theta =0.618...\)
کے لیے اسی وقفے میں دوسرا جزر 
\(\pi-0.666\cdot\cdot\cdot=2.475\cdot\cdot\cdot\)
ہے جو 
\(\sin\theta\)
کے تشاکل کے استعمال سے حاصل ہوتا ہے۔
چونکہ 
\(\sin\theta\)
کی یہ خصوصیت ہے کہ 
\(-1 \le \sin \theta\le1\)
اس لیے مساوات 

\(\sin\theta=-1.618\)
کا کوئی جزر نہیں ہے۔
لہذا مطلوبہ جزر 
0.67
اور 
2.48
ہیں جو کہ دو اعشاری نقطوں تک درست ہیں۔
مشق 18
ریڈیئن میں،دو اعشاری نقطوں تک درست،
\(\theta\)
کی وہ کم از کم مثبت قیمتیں تلاش کریں جن کے لیے 
\begin{enumerate}
\item
\( \sin\theta= 0.12\)
\item
 \(\sin\theta = -0.86\)
\item
\( \sin\theta = 0.925\)
\item
\( \cos\theta = 0.81\)
\item
\( \cos \theta = -0.81\)
\item
\( \cos \theta =\sqrt{\frac{1}{3}}\)
\item
 \(\tan\theta = 4.1\)
\item
 \(\tan\theta =-0.35\)
\item
 \(\tan\theta = 0.17\)
\item
 \(\sin(\pi+\theta)=0.3\)
\item
 \(\sin(2\pi+\frac{1}{3})=0.123\)
\item
 \(\sin(\frac{1}{6}-\theta)=0.5\)
\item
 \(\cos(3\theta-\frac{2}{3}\pi)=0\)
\end{enumerate}
%275-276 

وقفہ 
\(-\pi \le \theta \le \pi\)
میں
\(\theta\)
کی وہ تمام قیمتیں تلاش کریں جو مندرجہ ذیل کو حل کر سکیں۔ آپ کے جوابات جتنا ممکن ہو دو اعشاری نقطوں تک درست ہونے چاہئیں۔\\
\begin{multicols}{3}
\begin{enumerate}[a]
\item\(\sin\theta = 0.84\)
\item \(\cos\theta = 0.27\)
\item \(\tan\theta =1.9\)
\item \(\sin\theta = -0.73\)
\item \(\cos\theta = -0.15\)
\item \(4 \tan\theta + 5 = 0\)
\item \(4\sin\theta =3\cos \theta\)
\item \(3\sin\theta = \frac{1}{\sin\theta}\)
\item \(3\sin\theta =\tan\theta\)
\end{enumerate}
\end{multicols}
3. مندرجہ ذیل مساواتوں کے لیے وقفہ
\(0 < x \le 2\pi\)
میں تمام حل نکالئے
\begin{multicols}{3} 
\begin{enumerate}[a]
\item \(\cos2x =\frac{1}{4}\)
\item \(\tan3x=3\)
\item \(\sin 2x =-0.62\)
\item \(\cos 4x = - \frac{1}{5}\)
\item \(\tan 2x = 0.5\)
\item \(\sin 3x = -0.45\)
\end{enumerate}
\end{multicols}
4. وقفہ
\(-\pi<t \le \pi\)
میں مندرجہ ذیل تمام مساواتوں کے جزر تلاش کریں۔\\
\begin{multicols}{3} 
\begin{enumerate}[a]
\item \(\cos 3t =\frac{3}{4}\)
\item \(\tan2t=-2\)
\item \(\sin 3t = -0.32\)
\item \(\cos 2t = 0.264\)
\item \(\tan 5t =0.7\)
\item \(\sin 2t = -0.42\)
\end{enumerate}
\end{multicols}
5. وقفہ
\(-\pi < \theta \le \pi\)
میں مندرجہ ذیل تمام مساواتوں کے جزر ، اگر کوئی ہوں، تلاش کریں۔\\
\begin{multicols}{3}
\begin{enumerate}[a]
\item \(\cos \frac{1}{2}\theta =\frac{1}{3}\)
\item \(\tan\frac{2}{3}\theta=-5\)
\item \(\sin \frac{1}{5}\theta=-\frac{1}{5}\)
\item \(\cos \frac{1}{3}\theta=\frac{1}{2}\)
\item \(\tan\frac{2}{3}\theta=0.5\)
\item \(\sin \frac{2}{5}\theta = -0.4\)
\end{enumerate}
\end{multicols}
6. آلئہ حساب، کتاب، استمعال کئے بغیر مندرجہ ذیل تمام مساواتوں کے عزاد کے جزر تلاش کریں اگر کوئی ہوں ۔ آپ کے جواب میں وقفہ 
\(0<\theta\le 2\pi\)
میں
\(\pi\)
کے مضربوں میں ہوں۔\\
\begin{multicols}{3}
\begin{enumerate}[a]
\item \(\sin(2\theta - \frac{1}{3}\pi)=\frac{1}{2}\)
\item \(\tan(2\theta - \frac{1}{6}\pi)=0\)
\item \(\cos(3\theta+\frac{1}{4}\pi) =\frac{1}{2}\sqrt{3}\)
\item \(\tan(\frac{3}{2}\theta-\frac{1}{6}\pi)=-\sqrt3\)
\item \(\cos(2\theta-\frac{5}{18}\pi)=-\frac{1}{2}\)
\item \(\sin (\frac{1}{2}\theta+\frac{5}{18}\pi)=1\)
\item \(\cos(\frac{1}{5}-\frac{5}{18}\pi)=0\)
\item \(\tan(3\theta-\pi)=-1\)
\item \(\sin (\frac{1}{4}\theta -\frac{1}{9}\pi)=0\) 
\end{enumerate}
\end{multicols}
7. مندرجہ ذیل مساواتوں کے لیے وقفہ 
\(-\pi<\theta\le\pi\)
میں جزر تلاش کریں اگر کوئی ہوں۔\\
\begin{multicols}{3}
\begin{enumerate}[a]
\item \(\tan\theta = 2\cos\theta\)
\item \(\sin^2\theta=2\cos\theta\)
\item \(\sin^2\theta=2\cos^2\theta\)
\item \(\sin^2\theta = 2\cos^2\theta -1\)
\item \(2\sin\theta =\tan\theta\)
\item \(\tan^2\theta=2\cos^2\theta\)
\end{enumerate}
\end{multicols}
متفرق مشق 18 \\
1. اس خاکے میں آپ کو ایک دائرے کا ایک حلقہ دکھایا گیا ہے جس کا مرکز
\( \kvec{O}\) 
اور رداس
6cm
ہے۔ زاویہ 
\(\kvec{POQ}\)
کی قیمت
0.6
ریڈیئن ہے۔\\
قوس
\(\kvec{PQ}\)
کی لمبائی اور احاطہ
\(\kvec{POQ}\)
کا رقبہ معلوم کریں ۔\\
2. ایک دائرہ جس کا رداس
\(\kvec{a}\)
اور مرکز
\( \kvec{O}\)
ہے۔ اس دائرے کے ایک خطے
\(\kvec{OAB}\)
میں
\(\angle \kvec{AOB}\)
کی قیمت
\(\theta\)
ریڈیئن ہے۔\\
احاطہ
\( \kvec{AOB}\)
کا رقبہ قوس
\(\kvec{AB}\)
کی لمبائی کے مرتبع کا دوگنا ہے۔
\(\theta\)
کی قیمت تلاش کریں۔\\
3. اس شکل میں ایک دائرے، جس کا مرکز
\( \kvec{O}\)
اور رداس
\( \kvec{r}\)
ہے، کا ایک حلقہ دکھایا گیا ہے۔ قوس کی لمبائی حلقے کے احاطے کا نصف ہے۔
\( \kvec{r}\)
کو اکائی مان کر اس حلقے کا رقبہ معلوم کریں ۔\\
4. اس شکل میں آپ کو دو دائرے دکھائے گئے ہیں جن کے مراکز
\(\kvec{A}\)
اور
\(\kvec{B}\)
ہیں۔ یہ دائرے ایک دوسرے کو نقاط
\(\kvec{C}\)
اور
\(\kvec{D}\)
پر اس طرح قطع کرتے ہیں کہ ان میں سے ہر ایک کا مرکز دوسرے کے محیط پر آتا ہے۔ ہر دائرے کا رداس ایک اکائی ہے۔\\
زاویہ 
\(\kvec{CAD}\)
کی قیمت معلوم کریں۔\\
اس شکل میں ایک نیم تاریک علاقہ ہے جس کی حدود قوس
\(\kvec{CBD}\)
اور عمودی خط
\(\kvec{CD}\)
ہیں۔
اس علاقے کا رقبہ دریافت کریں۔
نیز واضح کریں کہ ان دونوں دائروں کے اندر واقع مشترکہ علاقے کا رقبہ ہر دائرے کے رقبے کا قم و بیش
39
فیصد ہے۔\\
5۔ اس خاکے میں آپ کو ایک دائرے کی قوس 
\( \kvec{ABC}\)
دکھائی گئی ہے۔ دائرے کا مرکز
\( \kvec{O}\)
اور رداس
5cm
سم ہے۔ خطوط
\( \kvec{AD}\)
اور
\( \kvec{CD}\)
بالترتیب نقاط
\( \kvec{A}\)
اور
\( \kvec{C}\)
پر اس دائرے کے tangents ہیں۔ زاویہ
\( \kvec{AOC}\)
کی قیمت
\(\frac{2}{3} \pi\)
ریڈیئن ہے۔\\
خطوط
\( \kvec{AD}\)،
\( \kvec{DC}\)
اور قوس
\( \kvec{ABC}\)
کے اندر محدود علاقے کا رقبہ معلوم کریں۔ آپ کا جواب دو نمایاں ہندسوں تک درست ہونا چاہیئے۔\\
6. وقفہ
\(-\pi<x\le\pi\)
میں
\( \kvec{x}\)
کی وہ تمام قیمتیں دریافت کریں جو درج ذیل مساواتوں کے تسلی بخش جواب ہوں۔ آپ کے جواب یا دو اعشاری نقطوں تک درست ہوں یا پھر
\(\pi\)
کے مکمل مضربوں میں ہوں۔
\begin{enumerate}
\item \(\sin{x}=-0.16\)
\item \(\cos{x} (1+\sin{x})=0\)
\item \((1-\tan{x})\sin{x}=0\)
\item \(\sin{2x}=0.23\)
\item \(\cos\Big{(\frac{3}{4}\pi-x}\Big)=0.832\)
\item \(\tan(3x-17)=3\)
\end{enumerate}
7. ایک تار میں برقی رو، amperes c کو مندرجہ ذیل مساواتوں کے ذریعے واضح کیا جا سکتا ہے۔\\
\(c=5\sin(100\pi t+\frac{1}{6}\pi)\)
جہاں t سیکنڈوں میں وقت کا اظہار کرتا ہے۔\\
ارتعاش کا عرصہ دریافت کریں۔ ہر ایک سیکنڈ میں ارتعاشات کی تعداد معلوم کریں۔\\




%276
کی وہ پہلی تین مثبت قیمتیں تلاش کریں جن کے لئے 
c
 کی قیمت 
\(2\)
 ہو۔
آپ کے جوابات 
\(3\)
 اعشاری نظروں تک درست ہونے چا ہئیں۔\\
ایک ذرہ جو ارتعاش میں ہے، کا ہٹاؤ 
y
میٹر ہے جہاں 
y
کی وضاحت 
\(y=a\sin({kt+\alpha})\)
سے ہوتی ہے جبکہ  
\(\alpha\)
کی پیمائش میڑوں میں ہے اور 
t
کی پیمائش سیکنڈوں میں ہوتی ہے پر 
h
اور 
\(\alpha\)
مستقلات ہیں۔ ایک ارتعاش کی قیمت 
T
سیکنڈ ہے۔\\
مندرجہ ذیل جوابات تلاش کریں ۔\\
h
 کی قیمت 
T
کی اکایئوں ہیں۔\\
h
 کو اکائی مان کر ایک سیکنڈ میں
مکمل ہونے والے ارتعاشات کی قیمت\\
%page 277
اس شکل میں آپ کو ایک دائرہ دکھایا گیا ہے جس کا مرکز
0
رداس
r
ہے۔
نیز ایک خط مستقیم
AB
جو مرکز
0
پر ایک دائرہ
\(\theta\)
بنایا ہے جس کی پیمائش ریڈیئن میں ہے۔ خط مستقیم دائرہ کے ایک نیم تاریک حصے کی حد بندی کر رہا ہے۔\\
اس نیم تاریک حصے کا رقبہ
r
اور
\(\theta\)
کی اکائیوں میں معلوم کریں۔\\
یہ متعین ہے کہ اس حصے کا رقبہ تکون
AOB
کے رقبے کا ایک تہائی ہے۔ اس کی روشنی میں واضح کریں کہ\\
\(3\theta-4\sin{\theta}=0\)\\
\(\theta\)
کی وہ مثبت قیمت تلاش کریں جو
0.1
ریڈیئن کے اندر
\(3\theta-4\sin{\theta}=0\)
کو درست ثابت کریں۔ اس مقصد کے لئے 
\(3\theta-4\sin{\theta}=0\)
کی قیمتوں کا جدول بنائیں۔اس دوران میں علامت کی تبدیلی یا عدم تبدیلی پر توجہ دیں۔ \\
اس شکل میں آپ کو دو دائرے دکھائے گئے ہیں جن کے مرکز
A
اور
B
ہیں اور یہ دائرے نقطہ 
C
پر ایک دوسرے کو چھوتے ہیں۔ ہر دائرے کا رداس
r
ہے۔ دونوں دائروں پر ایک نقطہ
D
یا
E
اس طرح سے واقع ہے کہ خط
DE
خطہ
ACB
کہ متوازی ہے۔\\
زاویوں
DAC
اور
EBC
میں سے ہر ایک زاویے کی قیمت
\(\theta\)
ریڈیئن ہے۔جبکہ
\(0<\theta<\pi\)
متعین ہے۔\\
خط
DE
کی لمبائی
r
اور
\(\theta\)
کی اکایئوں میں واضح کریں ۔
خط
DE
کی لمبائی دو قوسوں میں سے کسی کی بھی لمبائی کے برابر ہے۔\\
ثابت کریں کہ
\(\theta-2\cos{\theta}-2=0\)\\
\(0<\theta<1/2\pi\)
کےلئے
\(y=cos\theta\)
کا ترسیم کھینچیں۔ اپنے ترسیم پر ایک موزوں سیدھا خط کھینچ کر، جس کی مساوات بیان کرنا لازمی ہے۔
یہ واضح کریں کہ مساوات
\(\theta+2\cos{\theta}-2=0\)
کا وقفہ
\(0<\theta<1/2\pi\)
میں صرف ایک جزو ہے۔\\
حساب کرتے تصدیق کریں کہ
\(\theta\)
کی قیمت 
1.10
اور
1.11
کے درمیان ہے۔\\
اس شکل میں ایک دائرے کی قوس 
ABC
دکھائی گئی ہے جبکہ دائرے کا مرکز 
0
اور رداس
r
ہے اور
AC
خط مستقیم ہے۔ زاویہ 
AOC
کی قیمت
\(\theta\)
ریڈیئن ہے۔\\
جبکہ قوس 
ABC
کی لمبائی
S
ہے۔\\
\(\theta\)
کی وضاحت 
r
اور
S
کی اکایئوں میں کریں۔ پر یہ اخذ کر کے دکھائیں کہ مثلت
AOC
کے رقبے کا اظہار مندرج ذیل انداز میں کیا جا سکتا ہے۔\\
\[\frac{1}{2}r^{2}\sin{\Big(2\pi-\frac{s}{r}\Big)}\]
کوئی ترسیمی استدلال استعمال کرتے ہوئے جس کی بنیاد
\(y=\sin{x}\)
کے خاکے پر یا کسی اور طریقہ سے یہ دکھائیں کہ
\(\sin{(2\pi-\alpha)}=-\sin{\alpha}\)
جہاں
\(\alpha\)
تینوں زاویوں میں سے کسی بھی زاویے کی قیمت ریڈیئن میں ہے۔
اس تعین کی روشنی میں کہ تکون
AOC
کا رقبہ بڑے حلقے
OABC
کے رقبے کا پانچواں حصہ ہے۔ یہ نتیجہ نکال کر دکھائیں کہ\\
\[\frac{s}{r}+5\sin{\Big(\frac{s}{r}\Big)}=0\]

%278
کوی تریسمی طریقہ استعمال کر کے یا کسی اور طریقہ سے  ان مماثلات کا تعین کیجئے-\\
\begin{equation*}
\sin^{-1}{x}+\cos^{-1}{x}\equiv\frac{1}{2}
\end{equation*}
\begin{equation*}
\tan^{-1}{x}+\tan^{-1}{\Big(\frac{1}{x}\Big)}\equiv\frac{1}{2}\pi \; or \; -\frac{1}{2}\pi
\end{equation*}
  اس شکل میں آپ کو ایک دائرے ایک غوس دکھائی گئی ہے-اس دائرے کا مرکز 0 ہے اور رداس 
  r
  ہے-جبکہ قوس کا خط مستقیم 
  AB
  ہے-خط 
  AB
   داٰٰٰیرے کے مرکز 0 پر
    \(\theta\)
 ریڈین کا زاویہ بناتاہے اس شکل مےں  اپ کو آپک مربع 
 ABCD
 بھی دکھایا گیا ہے یہ متعین ہے کہ نیم تاریک حصے کا رقبہ مربع کے رقبے کا عین آٹھواں حصہ ہے  -یہ ثابت کریں کہ\\
 \begin{equation*}
2\theta-2\sin{\theta}+\cos{\theta}-1=0
\end{equation*}
  یا پھر اس میں یہ دکھائیں کہ 
  \(\theta\)
   کی قیمیت 1اور2 کے درمیان ہے\\
    جدولیاتی طرہقہ کے استعمال سے 
   \(\theta\)
    کی قیمیت دریافت کریں جو ایک اعشاری نقطہ تک درست ہو -\\
     مندرجہ ذیل تفاعلات کے دائرہ ہائے کار اور سعتیں بیان کریں \\
   مساوات 
   \begin{equation*}
2\sin^{-1}{x}-4
\end{equation*}
\begin{equation*}
2\sin^{-1}{(x-4)}
\end{equation*}
   کو حل کرہں جبکہ وقفہ
   \(0\le\theta\le x\)
    مہں تمام جزر کی قیمتیں تلاش کریں جو دو اعشاری نقطوں تک درست ہوں۔\\
     وقفہ
     \(-2\pi\le 0\le 2\pi\)
      میں کسی بھی جزر کی قیمیت 
      \(pi\)
      کی اکائی ہیں دیے ہوئے۔ درج ذیل مساواتوں کو حل کریں \\
   \begin{equation*}
2\cos^{2}{\theta}+\sin^{2}{\theta}=1
\end{equation*}
\begin{equation*}
2\cos^{2}{\theta}+\sin^{2}{\theta}=2
\end{equation*}

%page# 279
نظرثانی مشق 3\\

1) 
\(y=4-x\)
اور
\(y=x^{2}+2x\)
کی ترسیمات اور ان کے نقاط انقطاع پر ان کے محددات کا حساب لگایں ۔
دونوں ترسیمات کے درمیان واقع ممتناہی خِطے کا رقبہ دریافت کریں۔\\
2)
\( y=x^{3}-3x+3\) 
کے ترسیم پر ساکن نقاط کے محددات کا حساب لگایں۔\\
ب)اس نقطے کے محدد کا حساب لگائیں جس کے لیے
\(\frac{d^{2}y}{dx^{2}}\)\\
ج) منحنی پر واقع اس نقطے پر جس کی قیمت
\( x=2\)
 ،منحنی اور اس کے عمودی خط کی مساوات دریافت کریں۔\\
 د)  قوس محور
\(x\)
اور خطوط
\(x=0\)
 اور
\(x=2\)
کی حدود میں واقع رقبہ تلاش کریں۔ \\
3)   عدّاد استعمال کیے بغیر 
\(y=x^4-x^5\)
کا خاکہ بنائے اور ان مقامات کی نشاندہی کریں جن کے لئے
 \(\frac{d^{2}y}{dx^{2}}\)
مثبت ہے نیز جن کے لیے منفی ہے۔\\
4)
\(8\)
ایک صحیح عدد ہے۔
\(y=x^{n}\)
اور
\(y=x^{\frac{1}{n}}\)
کے ترسیمات بنائیں اور اس خطے کا رقبہ معلوم کریں جو ان کے اندر محدود ہے ۔\\
5) ایک منحنی ایک ایسی مساوات کی حامل ہے جو
\(\frac{d^{2}y}{dx^{2}}=5\)
کے تقاضے پورے کرتی ہے ۔
یہ منحنی نقطہ
\((0,4)\)
سے گزرتی ہے ۔ اس نقطے پر اس
tangent
کی تدریج
\(3\)
ہے۔
\(x\)
 کو اکائ بنا کر y کی قیمت تلاش کریں ۔\\
6) ایک منحنی
\(y=k x^{2}\) 
 جہاں
\(k\)
ایک مستقل ہے ،کا
\(y=1\)

اور
\(y=3\)
کا درمیانی حصہ
\(y\)
محور کے گرد
\(360\degree\)
گھمایا جاتا ہے ۔ اس یقین کے ساتھ کہ پیدا شدہ حجم
\(12 \pi\)  
ہے k کی قیمت دریافت کریں۔\\

\(\int_{1}^{3}(x^{3}-6x^{2}+11x-6)\dif{x}\)
کی قیمت دریافت کریں اپنے نتیجے کی طور پر تشریح کریں۔
۔ ایک خطے
R
کی حد بندی درج ذیل سے ہوتی ہے \\
(i)x
محور 
(ii)
خط
\(x=16\)
اور ایک منحنی
\(y=6-\sqrt{x}\)
جہاں
\(0\le x\le 36\)\\
اس جسم طواف کا حجم معلوم کریں جو اس وقت پیدا ہوتا ہے جب
R
 کو
x
مھور کے گرد ایک چکر دےا جاتا ہے۔\\ 
9۔ ربع اول میں ایک خطے کی اطراف محوروں اور ایک منحنی جس کی مساوات
\(y=\sqrt{9-x}\)
ہیں۔ اس خطے کا رقبہ تلاش کریں۔\\
10 ۔ )بالا ایک منحنی
\(y=\frac{1}{\sqrt{x}}\)
کے
\(x=1\)
سے
\(x=4\)
تک حصے کی تصو ےر کشی کریں\\۔
خطے
R
کا رقبہ 
دریافت کریں جبکہ اس خطے کی اطراف اس منحنی 
x
محور 
او ر
خطوط 
\(x=1\)
او ر
\(x=4\)
پر مشتمل ہیں۔\\
ایک بھڑ کے پروں کے متوازی افق
horizon
کے ترجمے افقی کی جگہ استعمال کیا گیا ہے
کے ساتھ بننے
والے زاویے کی م ساوات
\(0.4\sin{600t}\)
ریڈیئن ہے۔ جبکہ اس سے مراد سیکنڈ ہیں۔ اس بھ ڑکے پر ایک سیکنڈ میں کتنی دفعہ ارکعاش کرتے ہیں؟\\

%280
  طے کریں کہ آیا نقطہ
\((1,2,-1)\)
اس خط پر واقع ہے۔ جو
\((3,1,2)\)
اور
\(5,0,5)\)
سے گزرتا ہے ۔\\
اس شکل میں ایک ایسے دائرے کا حصہ دکھایا گیا ہے جس کا مرکز
0
اور رداس
r
ہے۔ نقاط
A,B,C
اس دائرے پر اس طرح واقع ہیں کہ
A,B
دائرے کا قطر ہے جبکہ زاویے کی قیمت
\(\theta\)
ریڈئین ہے۔
زاویہ
AOC
کی قیمت
\(\theta\)
کی اکائی ہے دریافت کریں اور تکون
OAC
میں
cosine
کا قائده استعمال کر کہ
\(AC^{2}\)
کو
r
اور
\(\theta\)
کی اکائیوں میں واضع کریں۔\\
تکون
ABC
کو سمجھ کر
AC
کی لمبائی
r
اور
\(\theta\)
میں تحریر کریں۔ اور یہ نتیجہ اخذ کر کہ دکھائیں کہ
\(\cos{2\theta}=2\cos^{2}\theta-1\)\\
x
کی مثبت قیمتوں کے لیے
\(y=\frac{9}{2x+3}\)
کا ترسیم بنائے۔\\
محن کے اس حصے کو جو
x=0
اور
x=3
کے درمیان واقع ہے۔
x
محور کے گرد
\(2\pi\)
گردش دی جاتی ہے۔ طواف کا حجم دریافت کریں۔\\
x
ک حوالے سے مندرجہ ذیل تفاعلات کو ایک دوسرے سے ممیز کریں۔\\
\[(x^{3}+2x-1)^{3}\]
\[\sqrt{\frac{1}{x^{2}+1}}\]
ایک استاد کو اس کی ملازمت کے پہلے سال کی کل تنخواه
12800
پاؤنڈ ملی اس نے اپنی مستقبل کی تنخواه کا تخمینہ اس بنیاد پہ لگایا کہ اس کی تنخواه میں
950
پاؤنڈ سالانہ کا مستقل اضافہ ہوگا حته کہ اس کی تنخواه
20400
پاؤنڈ سالانہ کی زیاده سے زیادہ حد کو پہنچ جائے گی۔\\
اپنی مدت ملازمت کے پانچویں سالا اس کی کمائی کتنی ہوگی۔
کس سال میں وا پہلی بار زیادہ سے زیادہ تنخواه وصول کرے گا۔
حساب لگائیں کہ اپنی مدت ملازمت کے
nth
سال کے آخر تک وہ کل کتنی رقم وصول کر چکا ہوگا۔ لکھیں کہ کون سی رقم
n
کی کس قیمت کے مطابق ہے۔\\
مذکورہ استاد کی جرطواں بہن نے بھی اسی سال اپنے شعبے میں اپنے کام کا آغاز کیا اس کی پہلے سال کی تنخواہ
13500
پاؤنڈ تھی جبکہ اس کی تنخواه میں مستقلاً
\(5\%\)
کا اضافه ہونا تھا۔\\
کوئی موزوں طریقہ استعمال کر کہ طہ کریں کہ اپنی ملازمت کے
nth
سال اس کی تنخواه کتنی ہوگی۔\\
ثابت کریں کے اپنی ملازمت کے چوتھے سال اس کی آمدن اپنے بھائی سے کم ہوگی۔\\
کس سال میں پہنچ کر پہلی بار اس کی آمون اپنے بھائی سے زیادہ ہوگی ؟\\
ایک جیو میٹرائی عقائد کا پہلا جزو
6
اور مشترک نسبت
0.75
ہے۔ اس عقائد کے پہلے دس اجزاء کا مجموعہ دریافت کریں۔ آپ کا جواب دو اعشاری نقطوں تک درست ہونا چاہی۔\\
اکائی کے اس جال پر دو سمتیں
\(\alpha\)
اور
\(\beta\)
دکھائی گئی ہیں \\
\(\abs{\alpha+\beta}\)
دریافت کریں \\
\(\alpha.\beta\)
دریافت کریں \\
\(\alpha\)
اور
\(\beta\)
کا درمیانہ زاویہ دریافت کریں ۔\\
