\باب{تفرق}\شناخت{باب_تفرقات}
%page 75

یہ سبق کسی بھی  ترسیم پر موجود نقطے کے ڈھلاؤ یا خط مماس معلوم کرنے کے بارے میں ہے۔  جب آپ یہ سبق مکمل کر لیں گے ، آپ کو عبور حاصل ہوگا کہ:
 
 
آپ ایک سمتی مقدار پر ایک نقطہ پہ  ڈھلاؤ معلوم کرنے کے لئے ایک کلیہ کا حساب لگائیں, اس کی مساوات بنائیں
 مربعی اور دیگر قسم کے خم پہ ایک نقطہ پر عین مطابق ڈھلاؤ کا حساب لگائیں 
 

اس سبق کو دو حصوں میں تقسیم کیا گیا ہے۔ پہلے حصے میں حصہ \حوالہء{حصہ6.1} تا حصہ \حوالہء{حصہ6.5}  میں آپ تجربے کی بنیاد پر نتائج اخذ کرتے ہوۓ خط مماس سے ترسیم تک کے مسائل حل کریں گے۔ دوسرے حصے میں حصہ \حوالہء{حصہ6.6}  تاحصہ \حوالہء{حصہ6.7}  تجربے کی بنیاد پر اخذ   نتا ئج کو  ثابت کریں گے۔
آپ اگر چاہیں تو سبق کے دوسرے حصے کو نظرانداذ کر سکتے ہیں لیکن آپ کو چاھئیے کہ سبق کے اختتام پہ موجود مشق کو حل کریں۔

\حصہ{خط مماس کا ڈھلاؤ معلوم کرنا}  
 
ایک سادہ سے خم کے بارے میں سوچیں جیسے کے  \( y=x^{2}\) کی ترسیم۔ جیسے جیسے  آپ کی نظر  \عددی{ x } محور کی سمت بڑھتی ہے ، کیا آپ بیان کر سکتے ھیں ، ریاضی کی زبان میں، خم کی سمت کس طرح سے تبدیل ہوتی ہے۔

جیسے ایک سیدھی لکیر کا ایک عددی ڈھلاؤ ہے، لہٰذا کوئ بھی خم ، بشرطیکہ وہ کافی حد تک (سموتھ)ہو ایک ڈھلوان یا ڈھلاؤرکھتا ہے جو کہ کسی بھی ایک نقطے پہ ماپا جا سکتا ہے۔ فرق صرف اتنا ھے کہ خم کے لیے ڈھلاؤ کی سمت بھی بدلتی ھے جیسے جیسے آپ اسکے ساتھ چلتے ھیں۔ ریاضی دان اس ڈھلاؤ کی مدد سے خم کی سمت کا تعین کرتے ہیں۔


سبق \حوالہء{باب-محدد-نقطے-اور-لکیریں} میں آپ سیکھ چکے ھیں کے اگر آپکے پاس ایک سیدھی لکیر کے دو نقطوں کے محدد دستیاب ہوں تو آپ کیسے اسکا ڈھلاؤ معلوم کر سکتے ہیں۔ آپ اس طریقے کو براہ راست خم پہ استعمال نہیں کر سکتے کیوںکہ وہ ایک سیدھی لکیر نہیں ہے۔ آپ اس خط مماس کا ڈھلاؤ معلوم کرتے ھیں جو کہ خم کے  کسی بھی دو نقطوں کی مدد سے بنایا جاۓ گا۔(جیسا کہ آپ تصویر\حوالہء{شکل6.1} میں دیکھ رھے ھیں) کہ ایک نقطے پر خط مماس اور ڈھلاؤ کی ڈھلوان برابر ہے۔تاھم یہاں ایک نیا مسئلہ کھڑا ہو گیا ہے، وہ یہ کہ آپ ایک لکیر کا ڈھلاؤ صرف تب ہی معلوم کر سکتے ہیں جب آپ کو اسکے دو نقطوں کے محدد  پتہ ہوں۔

تصویر \حوالہء{شکل6.2} میں ہم آہنگ لکیریں (سیدھی لکیریں جو خم کے دو نقطوں میں سے گزریں) دکھائی گئ ہیں جو خط مماس کے قریب تر ہوتی جا رہی ہیں، لہٰذا بہتر یہی ہے کہ ہم ان ھم آھنگ لکیروں کی ڈھلوان معلوم کرنے سے ابتدا کریں، کیونکہ اس طریقے کو پہلے سبق میں سیکھ چکے ھیں۔



%page 76

\ابتدا{مثال}
ایک ھم آھنگ لکیر کی ڈھلوان اور مساوات معلوم کریں جو کہ \( y=x^{2}\) کے خم کے دو نقطوں کو جوڑتی ہے ، ان دو نقطوں کے محدد ہیں (0.4،0.16) اور(0.7،0.16)

حصہ \حوالہء{حصہ1.3} میں ہم آہنگ لکیر کی ڈھلوان معلوم کرنے کے نسخہ کے مطابق؛
\[ \frac{0.49-0.16}{0.7-0.4} = \frac{0.33}{0.3} = 1.1\]

حصہ \حوالہء{حصہ1.5} میں ہم آہنگ لکیر کی مساوات معلوم کرنے کے نسخہ کے مطابق؛
\[ y-0.16 = 1.1(x-0.4), \]
جو کہ؛
\[ y=1.1x-0.28\]
\انتہا{مثال}

 

یہاں مددگار ہوگا کہ ہم کچھ نئ علامات سیکھیں، بڑھوتری کے لئے ہم یونانی علامت \( \delta\)ڈیلٹا) کا استعمال کرتے ہیں۔لہٰذا \(x\)  میں بڑھوتری کو        \(\delta x\)    اور          \(y\)  میں بڑھوتری کو          \(\delta y\) سے ظاہر کرتے ہیں۔ ان مقداروں کو حصہ \حوالہء{حصہ1.3} میں (x-step) اور (y-step) کہا گیا ہے۔ لہٰذہ مثال \حوالہء{مثال6.1.1} میں ھم آھنگ لکیر کے ایک سرے سے دوسرے (x-step)    \( 0.7-0.4=0.3\) جبکہ (y-step) \(0.49-0.16=0.33\) ہے۔لہٰذا ہم کہہ سکتے ھیں کہ؛
\[ \delta x = 0.3 ,\quad \delta y =0.33\]

اس علامت نویسی سے آپ کسی ھم آھنگ لکیر کی ڈھلوان کو \( \frac{\delta y}{\delta x}\) سے ظاہر کر سکتے ہیں۔ 

ایک طبقہ  \( \delta\) کی بجاۓ                 \( \Delta \)             استمعال کرتا ہے، دونوں ہی صحیح ہیں۔   
اس بات کا خیال رکھیں کہ اس \(frac{\delta y}{\delta x}\) میں آپ دونوں \(\delta\) کو آپس میں کاٹ نہیں سکتے کیوںکہ یہ اعداد نہیں ہیں۔
اب جبکہ ہم علامت کو استعمال کر رھے ہیں تو آپ عادت بنا لیں اسکو بڑھوتری کی شرح کہنے کی ، اسطرح آپ اسکو الجبرا کی عام علامت نا سمجھیں۔ یاد رکھیں کہ \(\delta x\) یا\(\delta y\)  منفی بھی ہو سکتے ہیں،اور ایسی صورت میں (x-step)اور(y-step) کم ھونگے۔
 
اگر ہم اس طریقے کو بروے کار لاتے ہوے ھم آھنگ لکیروں کے ڈھلاؤ کو معلوم کریں تو مثال\حوالہء{مثال6.1.1}   کچھ ایسی دکھے گی؛

\[ \frac{\delta y}{\delta x} = \frac{0.49-0.16}{0.7-0.4}=\frac{0.33}{0.3}=1.1 \]

 \ابتدا{مثال}\شناخت{مثال 6.1.1}
خم \( y=x^{2}\) کے دو نقطوں کو جوڑنے والی ھم آھنگ لکیر کا ڈھلاؤ معلوم کریں ۔جسکے  \عددی{ x }   محدد   \عددی{0.4  } اور \عددی{  0.41}  ہیں۔
سب سے پہلے آپکو ان دونوں نقطوں پہ   \عددی{ y }-    
محدد معلوم کرنا ھوگا  جوکہ \( 0.4^{2}=0.16\)  اور \( 0.41^{2} =0.1681\) ہیں۔

% page 77


 اس مثال کو بھی مثال \حوالہء{مثال6.1.1} کی طرح حل کریں گے۔\( \delta x = 0.41-0.4=0.01\)  اور   \(\delta y =0.1681-0.16=0.0081\)
تاہم ھم آھنگ لکیر کا ڈھلاؤ ھے؛
\[ \frac{\delta y}{\delta x} =\frac{0.0081}{0.01}=0.81\]
%done editting till here.
  \انتہا{مثال}
مثال \حوالہء{مثال6.4}  میں  ھم آھنگ لکیر اور \حوالہء{ x}-محدد\عددی{  0.4}  پہ موجود خط مماس کو الگ پہچاننا  مشکل ھے ،یہاں ہمیں ایک طریقہ ملتا ھے کہ ہم  xمحدد 6.4 پہ خط مماس کا ڈھلاؤ کسطرہ معلوم کر سکتے ھیں۔ 
مثال 6.1.3 میں یہ نقطے مزید قریب آگۓ ہیں۔

 \ابتدا{مثال}\شناخت{مثال 6.1.3}
خم \( y=x^{2}\) کے دو نقطوں کو جوڑنے والی ھم آھنگ لکیر کا ڈھلاؤ معلوم کریں ۔جسکے x محدد 0.4 اور0.40001 ہیں۔
دونوں نقطوں کے محدد  \( ( 0.4,0.4^{2} ) \)  اور            \( (0.40001 , 0.40001^{2}) \)           ہیں 

\( \delta x = 0.40001-0.4=0.00001\)  اور   \(\delta y =0.40001^{2}-0.4^{2}=0.0000080001\)
لہٰذہ ھم آھنگ لکیر کا ڈھلاؤ ہے؛
      \(\frac{\delta y}{\delta x}=\frac{0.0000080001}{0.00001}=0.80001\)          

یہ نتیجہ چونکہ 0.8 کے بہت قریب ہے لہٰذہ ایسا نظر آ رہا ہے کہ\( y=x^{2} \) کے خم پہ xمحدد  0.4 پہ موجود     خط مماس کا ڈھلاؤ  0.8 ہے ۔ لیکن اس سے یہ ثابت نہیں ہوتا کیونکہ آپ ابھی تک دو نقطوں کو ملانے والی خط مماس کی مساوات معلوم کر رہے ہیں،قطعی نظر اسکے کہ یہ نقطے کتنے قریب ہیں۔
\انتہا{مثال}

%                             EXERCISE  6A

سوالنمبر\عددی{ 2} اور\عددی { 3} میں سوالوں کو مختلف حصوں میں تقسیم کیا جا سکتا ہے،لہٰذا طلباء کی جماعت اکٹھے کام کرتے ہوۓ ان تمام سوالوں کے جوابات حاصل کر لیں گے،اور پھر ان جوابات کو جمع کیا جا سکتا ھے،

\ابتدا{سوال}\شناخت{سوال6.1}
سیدھی لکیر کی ایک مساوات بنائیں جو \( y=x^{2} \)کے خم پہ دو نقطوں کو ملاۓ جنکی x کی قیمت 1 اور 2 ہو۔
\انتہا{سوال}




\ابتدا{سوال}\شناخت{سوال6.2}
اس سوال کے ہر حصے میں  \( y=x^{2} \)کے خم پہ  درج ذیل  x محدد  کے  دو نقطوں  سے بننے والی ھم آھنگ لکیروں کا ڈھلاؤ معلوم کریں ۔
   \begin{multicols}{2}
\begin{enumerate}[a.]
\item \(1  \) اور \(1.001 \)
\item \( 1\) اور \(    0.9999  \)
\item \( 2 \) اور \(0.002  \)
\item \( 2 \) اور \( 1.999 \)
\item \( 3 \) اور \(  3.000001\)
\item \( 3 \) اور \( 2.99999 \)
\end{enumerate}
\end{multicols}
 
\انتہا{سوال}

\ابتدا{سوال}\شناخت{سوال6.3}
اس سوال کے ہر حصے میں، \( y=x^{2} \)کے خم پہ درج ذیل نقطے اور اسکے قریبی نقطے کی مدد سے بنی    ھم آھنگ لکیر کا ڈھلاؤ معلوم کریں۔ بتاۓ گۓ نقطے اور اسکے قریبی نقطے کے مابین فاصلے کو تبدیل کر کے اسی عمل کو دہرائیں، اس بات کا خیال رکھیں کہ کچھ نقطے بتاۓ گۓ نقطے کی بائیں طرف بھی ہوں۔
\begin{multicols}{3}
\begin{enumerate}[a.]
\item \( (-1،1)   \)
\item \( (-2،4) \)
\item \((10،100)  \)
 \end{enumerate}
\end{multicols}
 


  \انتہا{سوال}


\ابتدا{سوال}\شناخت{سوال6.4}
سوالنمبر \عددی{2} اور\عددی{3} سے حاصل شدہ تجربے کی بنیاد \( y=x^{2} \)کے خم پہ موجود کسی بھی نقطے پے خط مماس کے ڈھلاؤ کا انداذہ کریں۔
 \انتہا{سوال}

\ابتدا{سوال}
 \موٹا{جز ا}
 
سوال نمبر \عددی{ 2} سے \عددی{ 4} تک استعمال شدہ طریقے کی مدد سے \( y=x^{2} +1 \)اور \( y=x^{2} -2 \)کے خم پہ موجود کسی بھی نقطے پہ بنی خط مماس  کا ڈھلاءو معلوم کریں۔
 \موٹا{جز ب}
حصہ \حوالہء{حصہ1} میں حاصل شدہ تجربے کی بنیاد پے
\( y=x^{2} +c \) 
، جبکہ \عددی{c} ایک حقیقی عدد ھے،   کے خم پہ بنی کسی بھی خط مماس کے ڈھلاؤ کو معلوم کرنے کا ایک عالمگیر اصول وضع کریں 

خط مماس کا ڈھلاؤ جو کہ  \( y=x^{2} +c \)کے خم پہ بنی ھو۔
  \انتہا{سوال}

اگر آپ مشق \عددی{6A} کے نتائج کو جمع کریں تو آپکو یہ شبہ گزرے گا کہ کسی بھی نقطے پر\( y=x^{2}\) کے خم پہ بنی خط مماس کا ڈھلاؤ x محدد کا دو گناہ ھے، لب لباب یہ یے کہ \( y=x^{2}\)کے خم کے ڈھلاؤ کا کلیہ\عددی{           2x          }ہے  

مثال کے طور پر ،  \( y=x^{2}\) کے  خم پہ  ایک نقطے (3،9-) کا ڈھلاؤ \( 2 \times (-3) =-6 \) ہے،اسکا مطلب یہ ھوا کہ اس نقطے پہ بنی خط مماس کا ڈھلاؤ -6 ہوگا اور یہ نقطہ   \(  ( 3،9-)  \)سے گزرے گی۔
اس خط مماس کی مساوات معلوم کرنے کے لیے آپ ضیمہ 1.5 کا سہارا لے سکتے ہیں۔ لکیر کی مساوات ہو گی؛
\[y-9=-6(x-(3)),\]
جو کہ ؛
\(y-9 =6x-18\)
یا
\(y=-6x-9\)
ہے۔

غور کریں کہ ڈھلاؤ کا کلیہ  خم \( y=x^{2} +c \) پر بھی لاگو ھو رہا ہے ، جبکہ c مستقل ہے۔  x پہ بھی ڈھلاؤ 2x ہی ہے۔ اسکی وجہ سادہ سی ہے کہ  \( y=x^{2} +c \) کا خم بھی  \( y=x^{2}  \)  کے خم جیسا ہی ھے بس صرف   \عددی{           y         }
محور کی سمت تھوڑا منتقل ہو گیا ہے۔

ایک ثانیے کے لیے یہ مان لیں کہ ان نتائج کو ثابت کیا جا سکتا ہے۔آپکو ثبوت مل جاۓ گا حصہ \حوالہء{حصہ 6.6}میں۔

خم کے کسی نقطے پر عمودی لکیر۔

وہ لکیر جو خم اور خط مماس کے باہمی ملاپ کے نقطے سے کچھ اس طرح گزرے کہ خط مماس کے ساتھ  \( 90^0\) کا زاویہ بناۓ  اس نقطے پر خم کی عمودی لکیر کہلاتی ہے۔

\( 90^0\)


 %page 79

اگر آپکو کسی ایک نقطے پر خط مماس کا ڈھلاؤ  معلوم ہےتو آپ  حصہ\حوالہء{حصہ1.9}کے نتیجے سے  عمودی خط کا ڈھلاؤ معلوم کر سکتے ہیں۔ اگر خط مماس کا ڈھلاؤ  \عددی{           m        }
یے تو عمودی لکیر کا ڈھلاؤ \(- \frac{1}{m}\) ہوگا، بشرطیکہ \(m \ne 0\)

 \ابتدا{مثال}\شناخت{مثال 6.3.1}
 \( y=x^{2} \)   
 کے خم پہ  عمودی لکیر کی ایک مساوات معلوم کریں جبکہ اس نقطے پہ ؛\(x=-3\) اور \(x=0\)  ہے۔

  حصہ\حوالہء{حصہ6.2} میں ہم نے \(  (3،9-)   \) پہ خط مماس  کا ڈھلاؤ معلوم کیا تھا جو کہ \عددی{           -6        } تھا۔ اسی طرح عمودی خط کا ڈھلاؤ \(-\frac{1}{-6} =\frac{1}{6}\) اور یہ بھی\(  (3،9-)   \)سے ہی گزرتا ہے۔ لہٰذا عمودی خط کی مساوات \( y-9=\frac{1}{6}(x-(-3)),\) جسکی سادہ شکل
\( 6y=x+57\)
ہے۔

نقطے \( (0,0)\)  پر خط مماس کا ڈھلاؤ\عددی{         0       } ہے۔ لہٰذا  خط مماس\عددی{        x      } محور کے مساوی بڑھتی رہے گی۔ اسی وجہ سے عمودی خط \عددی{        y      } محور کے مساوی بڑہتی رہے گی۔ اسکی مساوات کچھ یوں ہے۔\(x= کچھ بھی \)   ۔ جب عمودی خط \( (0,0)\)  سے گزرتی یے تو اسکی  مساوات ہوگی \(x= 0 \) 
\انتہا{مثال}

اگر آپکے پاس ترسیم کاری کا آلہ موجود ہے تو ، آپ\( y=x^{2}  \) کے خم کو، \( y=-6x-9  \) کے خط مماس کو، \( 6y=x+57  \) کے عبوری خط کا مشاہدہ کریں، آپ نتائج سے حیران ضرور ہوں گے۔

آپکو یہ بات سمجھنی ھو گی کہ اگر آپ ایک ھی نقطے پر خم، خط مماس اور خط عمودی کشید کریں تو خط عمودی صرف اسی صورت\(90^0\) زاویے پر ہوگا جب \عددی{        x      }  اور  \عددی{        y      } محور دونوں پہ پیمانہ ایک سا ہو، خط مماس پر کوئ فرق نہیں پڑے گا۔  

اس موقع پر آپکو سمجھنا ھو گا کہ آپکو اس نتیجے کو دیگر مساوات کے لئے بھی علمگیر کرنا ھوگا۔ ضیمہ 6.2 میں آپ نے دیکھا کہ \( y=x^{2} +c \) کے خم میں کسی بھی \عددی{        x      }پہ موجود خط مماس کا ڈھلاؤ \عددی{       2 x      } کے برابر ہوگا۔

 \موٹا{مشق 6B}


سوالنمبر\عددی{ 9}  سے \عددی{ 12} میں سوالوں کو مختلف حصوں میں تقسیم کیا جا سکتا ہے،لہٰذا طلباء کی جماعت اکٹھے کام کرتے ہوۓ ان تمام سوالوں کے جوابات حاصل کر لیں گے،اور پھر ان جوابات کو جمع کیا جا سکتا ھے،
%question 1

درج ذیل x \عددی{        x      }محدد کی مدد سے \( y=x^{2}\)  کے خم پہ   بننے والے خط مماس کا ڈھلاؤ معلوم کریں۔ 
\begin{enumerate}[a.]
\item 1
\item4
\item0
\item-2
\item-0.2
\item-3.5
\item p
\item 2p
\end{enumerate}
%question 2


درج ذیل\عددی{        x      } محدد کی مدد سے \(2- y=x^{2}\)  کے خم پہ   بننے والے خط مماس کا ڈھلاؤ معلوم کریں۔ 

\begin{enumerate}[a.]
\item1
\item4
\item0
\item-2
\item-0.2
\item-3.5
\item p
\item 2p
\end{enumerate}

%question 3

خم  \(5+ y=x^{2} ۤۤ\) پہ ایک نقطے  \عددی{        p    }کا   \عددی{        y      } محدد  \عددی{        9  }ہے۔ اسی نقطے پے خط مماس کے ڈھلاؤ کی دو ممکن مقداریں معلوم کریں۔

%page 80
%question 4
 بتاۓ گۓ خم کے دیۓ گۓ\عددی{        x  } یا   \عددی{        y      } محدد سے معرض وجود میں آۓ نقطے پر بنی خط مماس کی مساوات معلوم کریں۔
\begin{enumerate}[a.]
\item \( y=x^{2}\) جبکہ \( x=2\)
\item \( x=2 +2 \)  جبکہ \( x=-1\)
\item \( x=2 -2 \)   جبکہ    \( y=-1\)
\item  \( x=2 -2 \)  جبکہ    \( y=-2\)
\end{enumerate}

%question 05

تاۓ گۓ خم کے دیۓ گۓ x  یا   y محدد سے معرض وجود میں آۓ نقطے پر بنے عمودی خط  کی مساوات معلوم کریں  
\begin{enumerate}[a.]
\item \( y=x^{2}\)        جبکہ      \(x=1\)
\item \( y=x^{2}+1\)   جبکہ       \(x=-2\)        
\item \( y=x^{2}+1\)     جبکہ      \(x=0\)
\item \( y=x^{2}+c\)     جبکہ       \(x=\sqrt{c}\)
\end{enumerate}

 %question 6
خم \( y=x^{2}\) کے ایک نقطے\عددی{       p     } پرخط مماس کا ڈھلاؤ  \عددی{       3  }ہے۔ اس نقطے \عددی{       p     } پر عمودی خط کی مساوات بنائیں۔

%question 7

خم  \( y=x^{2}+1\) کے ایک نقطے P پر عمودی خط  کا ڈھلاؤ\عددی{       -1 } ہے۔ اس نقطے\عددی{       p     } پر  خط مماس  کی مساوات بنائیں۔
%question 8
خم \( y=x^{2}\) میں محدد \( (2,4) \) پہ بننے والا عمودی خط دوبارہ اس خم سے گزرتا ہے، اس نقطے کی نشاندہی کریں۔

%question 9
 سوال کے ہر حصے میں درج ذیل
  خموں کو بروۓ کار لاتے ہوۓ   \( y=2x^{2}\)، \( y=3x^{2}\)اور\( y=-x^{2}\)  بتاۓ گۓ x محدد سے بننے والے نقطوں سے معرض وجود میں آنے والی ھم آھنگ لکیر کا    ڈھلاؤ معلوم کریں 
\begin{multicols}{2}
\begin{enumerate}[a.]
\item \(1  \) اور \(1.001 \)
\item \( 1\) اور \(    0.9999  \)
\item \( 2 \) اور \(0.002  \)
\item \( 2 \) اور \( 1.999 \)
\item \( 3 \) اور \(  3.000001\)
\item \( 3 \) اور \( 2.99999 \)
\end{enumerate}
\end{multicols}
%question 10
اس سوال کے ہر حصے میں، \( y=\frac{1}{2} x^{2}\) اور  \( y=\frac{1}{2} x^{2}+2\)   کے خم پہ درج ذیل                 x محدد سے بننے والے  نقتے اور اسکے قریبی نقتے کی مدد سے بنی    ھم آھنگ لکیر کا ڈھلاءو معلوم کریں۔ بتاۓ گۓ نقتے اور اسکے قریبی نقتے کے مابین فاصلے کو تبدیل کر کے اسی عمل کو دہرائیں، اس بات کا خیال رکھیں کہ کچھ نقتے بتاۓ گۓ نقتے کی بائیں طرف بھی ہوں۔
\begin{enumerate}[a.]
\item \(-1  \)
\item \( -2 \)
\item \( 10 \)
 \end{enumerate}
 
%question 11

سوالنمبر  \عددی { 9 } اور \عددی{ 10 }سے حاصل شدہ تجربے کی بنیاد \( y=ax^{2} \) اور    \( y=ax^{2} +c \) ، جبکہ aایک حقیقی عددہے،  کے خم پے موجود کسی بھی نقطے پے خط مماس کے ڈھلاؤ کا انداذہ کریں۔



%question 12

\ابتدا{سوال}
 \موٹا{حصہ ا}
سوالنمبر\عددی{ 9 }سے \عددی{11} تک استعمال شدہ طریقے کی مدد سے      \( y=x^{2} +3x\)اور\( y=x^{2} -2x\) کے خم پہ موجود کسی بھی نقطے پہ بنی خط مماس  کا ڈھلاؤ معلوم کرنے کا طریقہ وضع کریں۔
 \موٹا{حصہ ب}
حصہ (ا) میں حاصل شدہ تجربے کی بنیاد پے\( y=x^{2} +bx\) ، جبکہ\عددی{       b     }ایک حقیقی عدد ھے،   کے خم پہ بنی کسی بھی خط مماس کے ڈھلاؤ کو معلوم کرنے کا ایک عالمگیر اصول وضع کریں 
\انتہا{سوال}

دو درجی ترسیم کے ڈھلاؤ کا کلیہ؛
سبق\حوالہء{سبق 3}میں عام دو درجی ترسیم کا تذکرہ کیا گیا ،جسکی مساوات کچھ ایسی  \(y=ax^{2}+bx +c\) ہے، جنکہ ،\عددی{     a}،      \عددی{       b     }    اور   \عددی{     c     } مستقل ہیں۔ ایسے خم کی خط مماس کے ڈھلاؤ کے بارے میں آپکا کیا خیال ہے؟

مشق  \حوالہء{مشق6B} کے سوالات کے حل سے آپکو اندازہ ہوا ہوگا \( y=ax^{2}\) کے ڈھلاؤ کا کلیہ \(  2ax   \)یے. اسکا مطلب، مثال کے طور پہ،        \(y=3x^{2}\)    کے خم کا ڈھلاؤ   x کی ہر مقدار پہ تین گناہ ذیادہ ہو گا اگر اسکے مقابل خم \(y=x^{2}\) کا ہو۔ آپنے اس بات کا بھی مشاہدہ کیا ہوگا کہ \(y=x^{2}+bx\)     کے ڈھلاؤ کا کلیہ     \(2x +b\)  یے، لب لباب تمام کلام کا یہ ہے کہ  \(y=x^{2} +4x\)  کے  ڈھلاؤ کا کلیہ   \( 2x+4\)  ہے۔ جوکہ \(x^{2}\) اور \( 4x\) کے ڈھلاؤ کے کلیوں کے جمع کے برابر ہے۔


%page 81
آپکو اس بات کا پہلے سے ہی علم ہے کہ    \(y=x^{2}\)   اور  \( y=x^{2}+c\)   دونوں کے ڈھلاؤ کا کلیہ ایک ہی ہے،    \عددی{     c     }کی مقدار جتنی بھی ہو اس سے فرق نہیں پڑتا۔

لہذا اس بات میں کوئی بعید نہیں ہے کہ؛

مساوات  \( y=ax^{2}+bx+c\)   کے خم کے  ڈھلاؤ کا کلیہ       \(  2ax +b      \)ہے،


یہ نتیجہ اس طرح بھی اہم ہے کہ ھم ایک ایسے تفاعل کا ڈھلاؤ معلوم کر سکتے ہیں جو کئ حصوں پر مشتمل ہو، اور ایسا کرنے کے لیۓ ہم ان تمام حصوں کے ڈھلاؤ کو باہمی طور پہ جمع کر دیتے ہیں۔
آپ ایک ایسے تفاعل کا ڈھلاؤ بھی  معلوم کر سکتے ہیں جسکے ساتھ کوئی مستقل عدد ضرب کھا رہا ہو، اور ایسا کرنے کے لیۓ ہمیں اس تفاعل کے ڈھلاؤ کو بھی اسے عدد سے ضرب دینی ہو گی۔

حصہ  \حوالہء{حصہ 6.6} میں ہم اس بات کا مشاہدہ کریں گے کہ ہم ان نتائج کو ثابت کر سکتے ہیں، وقت ضائع نہ کرتے ہوۓ یہاں ان کے استعمال کی چند مثالیں دی گئ ہیں ، لیکن اس سے قبل ہمیں علامت نویسی کو دیکھنا ہوگا۔


مان لیں کے ایک خم کی مساوات \(y=f(x) \)  ہے، اسکے ڈھلاؤ کا کلیہ \( f\prime (x)\) ہوگا، اور اسکو \عددی{   f     }ڈیش  \عددی{    x} پڑھا جاۓ گا۔

کسی بھی خم کے خط مماس کے ڈھلاؤ کو معلوم کرنا تفریق کاری کہلاتا یے، اور جب ہم اس عمل کو انجام دے رہے ہوتے ہیں تو دراصل ہم تفریق کاری کر رہے ہوتے ہیں۔


جیسے \(  f(2)     \)سے مراد وہ مقدار ہے جب \( x=2\) ہو، اسی طرح  \( f\prime (2)\) اس ڈھلاؤ کے لیے استعمال ہوتا ہے جب \( y=f(x)\) اور \(x=2\) ہو۔ لہٰذا (ڈیش) جو \( f \prime (x)\) پہ بنی ہے ہمیں تفریق کا بتاتی ہے، اس علامت کو دیکھتے ہی آپ جس  \عددی{    x}پہ ڈھلاؤ معلوم کرنا چاہتے اسکے \عددی{    x} کے متبادل سے x کو بدل دیتے ہیں۔

مقدار \( f\prime (2)\) کو تفاعل \( f(x)\) کا تفرق کہا جاۓ گا جب \(x=2\) ہوگا۔

  
پس جب بھی ہمیں مساوات   y=f(x) کے خم کا   \(x=2\) پے ڈھلاؤ  معلوم کرنا ہو ، ہم \(f\prime (x)\) معلوم کریں گے اور پھر \عددی{    x}کو اسکے متبادل مقدار یعنی یہاں \عددی{   2    } سے بدل دیں گے، نتیجے کے طور پر ہمیں \( f\prime (2)\) ملے گا۔


%example 6.4.1

 \ابتدا{مثال}\شناخت{مثال 6.4.1}
مساوات \( 3x^{2}-2x+5\) کو تفریق کریں۔
خم \( 3x^{2}-2x+5\) کے خط مماس اور عمودی خط کے ڈھلاؤ کی مساوات بنائیں، جبکہ اس نقطے پر \(x=1\) ہو۔

مان لیں کہ \(f(x)=3x^{2}-2x+5\)، اس تفاعل کے مطابق    \(  a=3      \)       ،    \(   b=-2     \)       اور    \(  c=5     \) 
ہے ۔ اس تفاعل کا اگر تفرق معلوم کریں،\( f\prime(x) =2\times3\times x-2 = 5x-2\)

وہ نقطہ جس پہ \( x=1    \) اسکا  \عددی{   y   }محدد برابر ہوگا  \(  3-2+5=6  \)

جب\(   x=1 \)ہوگا تو خط مماس کا ڈھلاؤ ہوگا،\(f\prime (1)=6\times 1-2-4\)

اسی لیے خط مماس کی مساوات کچھ یوں بنے گی،         \( y-6=4(x-1)\)    یا\(y=4x+2\)

عمودی خط، خط مماس کے ساتھ عمودی ہوتا ہے اس لیے اسکا ڈھلاؤ   \( \frac{-1}{4}\)  ہوگا، اسی لیے عمودی خط کے ڈھلاؤ کی مساوات      \(y-6=\frac{-1}{4}(x-1)\)  ہوگی جسکی سادہ شکل کچھ یوں \(x+4y=25\) ہے۔
\انتہا{مثال}


%Page 82
\ابتدا{مثال}\شناخت{مثال6.4.2}
درج ذیل تفاعل کا تفرق معلوم کریں۔
\begin{enumerate}[a.]
\item $f(x)=2(x^{2}-3x-2),$
\item  $  g(x)=(x+2)(2x-3)$
\end{enumerate}
\موٹا{طریقہ اول} 
کوسین کو ضرب دینا\\
$f(x)=2(x^2-3x-2)=2x^2-6x-4$
تاہم \(  f'(x)=4x-6     \)\\
\موٹا{طریقہ دوئم}

اگر بتایا گٰیا تفاعل کسی مستقل عدد کے ساتھ ضرب کھا رہا ہو تو، اس تفاعل کا ڈھلاءو بھی اسی عدد کا مضرب ہوگا۔ اس مثال میں وہ عدد \عددی{2} ہے ، اسی لیے $f'(x)=2(2x-3)$

سوال کے دوسرے حصے $g(x)=(x+2)(2x-3)$ کے لیے پہلے حصے میں بتایا گیا طریقہ دوئم کارآمد نہیں ہوگا۔کیونکہ مضرب ایک مستقل عدد نہیں ہے ، لیکن آپ کوسین کو حل کر کے ایک دو درجی مساوات حاصل کر سکتے ہیں جسے بعد میں متفرق کیا جا سکتا ہے، جیسے؛$g(x)=(x+2)(2x-3)=2x^2+x-6 $ اسی لیے  $g\prime(x)=4x+1 $\\
اگر آپ اپنی یاداشت میں آنے والے اصولوں کے تحت ایک تفاعل کا تفرق معلوم کرنے میں ناکام ہو رہے ہیں تو آپ اس تفاعل کو کسی بھی سادہ شکل میں ڈھال کے اس کا تفرق معلوم کر سکتے ہیں۔
\انتہا{مثال}

\ابتدا{مثال}\شناخت{مثال6.4.3}
مساوات $y=x^2-4x+2 $ کے خم پے ایک \عددی{x}-محور کے متوازی بنے خط مماس کی مساوات معلوم کریں۔\\
حصہ  \حوالہء{حصہ1.6} میں ہم سیکھ چکے ہیں کہ  عددی{x}-محور کے متوازی لکیر کا ڈھلاؤ صفر ہوتا ہے۔ فرض کریں کہ  $f(x)=x^2-4x+2 $ اور یوں   $f'(x)=2x-4 $ ۔

 \عددی{x} معلوم کرنے کے لیے کہ جب ڈھلاؤ صفر ہوگا ہمیں مساوات $ 2x-4=0 $ کو حل کرنا ہوگا۔ اور اس سے ہمیں  $x=2 $ حاصل ہوگا۔ جب  $x=2 $ تو  \(   y=2^2-4\times 2+2=-2  \)ہوگا۔

حصہ  \حوالہء{حصہ1.6} میں ہم نے دیکھا کہ عددی{x}-محور کے متوازی لکیر کی مساوات $y=c$ جیسی ہوتی ہے، تاہم ہمارے خط مماس کی مساوات بھی  $y=-2 $ ہے۔
\انتہا{مثال}
%exercise 6C
چار سوالات \عددی{13} تا \عددی{16} میں سوالات کے حصوں کو طلباء کے مابین بانٹا جا سکتا ہے، تاکم طلباء جو کہ ایک ساتھ کام کر رہے ہیں انکے پاس تمام سوالات کے حل موجود ہوں۔
\ابتدا{سوال}\شناخت{سوال6c1} 
درج ذیل تمام تفاعل کے لیے ڈھلاءو کلیہ معلوم کریں۔
\begin{multicols}{4}     
 \begin{enumerate}[a.]
\item  \(  x^2    \)
\item  \(    x^2-x  \)
\item  \(    4x^2   \)
\item  \(   3x^2-2x   \)
\item  \(   2-3x     \)
\item  \(  x-2-2x^2     \)
\item  \(    2+4x-3x^2  \)
\item  \(    \sqrt{2}x-\sqrt{3}x^2   \)
\end{enumerate}
 \end{multicols}
 \انتہا{سوال}
\ابتدا{سوال}\شناخت{سوال6c2}
درض ذیل میں دیے گۓ تمام تفاعل $ f(x)$ کے متفرق  $f'(x)$ معلوم کریں۔ کچھ حصوں میں آپ کو تفاعل کی ترتیب بدلنا ہو گی قبل انکا متفرق معلوم کرنے کے۔
\begin{multicols}{4}     
 \begin{enumerate}[a.]
\item  \( 3x-1    \)
\item  \(   2-3x^2  \)
\item  \(  4\)
\item  \(   1+2x+3x^2 \)
\item  \( x^2-2x^2      \)
\item  \(3(1-2x-x^2)    \)
\item  \(  2x(1-x) \)
\item  \(   x(2x+1)-1   \)
\end{enumerate}
 \end{multicols}
\انتہا{سوال}
%page83 starts here.
\ابتدا{سوال}\شناخت{سوال6c3}  
درج ذیل تمام تفاعل \(f(x)\) کا \(x=-3\) پر تفرق معلوم کریں
 
\begin{multicols}{4}    
\begin{enumerate}[.a]
\item \(-x^{2}\)
\item \(3x\)
\item \(x^{2} + 3x\)
\item \(-x^{2}\)
\item \(2x^{2} + 4x -1\)
\item \(-(3-x^{2} )\)
\item \(-x(2+x)\)
\item \((x-2)(2x-1)\)
\end{enumerate}
 \end{multicols}
\انتہا{سوال} 
\ابتدا{سوال}\شناخت{سوال6c4} 
درج ذیل تمام تفاعل  \(f(x)\) کے لیے  \عددی{x} کی ایسی قیمت معلوم کریں کہ تفرق \(f'(x)\) کی بتائ ہوئ قیمت آ جاۓ۔
\begin{multicols}{4}   
\begin{enumerate}[.a]
\item \(2x^{2} \quad \quad \quad 3\)
\item \(x-2x^{2} \quad \quad \quad -1\)
 \item \(2+3x+x^{2} \quad \quad \quad 0\)
\item \(x^{2} +4x -1 \quad \quad \quad 2\)
\item \((x-2)(x-1) \quad \quad \quad 0\)
\item \((3x+2) \quad \quad \quad 10\)
\end{enumerate}
 \end{multicols}
\انتہا{سوال} 
 \ابتدا{سوال}\شناخت{سوال6c5}  
بتائ گئ مساوات کے خم پے ایک نقطے کا \عددی{x}-محدد بتایا گیا ہے۔ اس نقطے پر بننے والی خط مماس کی مساوات معلوم کریں۔
\begin{multicols}{2}  
\begin{enumerate}[a.]
\item \(y=x^{2}\) \quad\quad \(x= -1\)
\item \(y=3x^{2} -2x -1 \) \quad\quad \(x=1\)
\item \(y=x^{2} -2x +3 \)\quad\quad \(x=2\)
\item \(y=1-x^{2}\)\quad\quad \(x=-3\)
\item \(y=x(2-x)\)\quad\quad  \(x=1\)
\item\(y=(x-1)^{2}\)\quad\quad \(x=1\)
\end{enumerate}
 \end{multicols}
\انتہا{سوال} 
 \ابتدا{سوال}\شناخت{سوال6c6}  
بتائ گئ مساوات کے خم پے ایک نقطے کا \عددی{x}-محدد بتایا گیا ہے۔ اس نقطے پر بننے والے عمودی خط  کی مساوات معلوم کریں۔
\begin{multicols}{2}  
\begin{enumerate}[a.]
\item \(y= -x^{2}\)\quad\quad \(x= 1\)
\item \(y=3x^{2} -2x -1 \)\quad\quad \(x=1\)
\item \(y=1-2x^{2}\)\quad\quad \(x=-2\) 
\item \(y=1-x^{2}\)\quad\quad \(x=0\) 
\item \(y=2(2+x+x^{2})\)\quad\quad \(x=-1\) 
\item \(y=(2x-1)^{2}\)\quad\quad \(x= \frac{1}{2}\)
\end{enumerate}
  \end{multicols}
\انتہا{سوال} 
\ابتدا{سوال}\شناخت{سوال6c7}  
خم \(y=x^{2}\) پے بننے والی خط مماس ، جو کہ مساوات \(y=x\) کے متوازی ہے، کی مساوات معلوم کریں۔
\انتہا{سوال} 

 \ابتدا{سوال}\شناخت{سوال6c8}  
خم \(y=x^{2}\) پے بننے والی خط مماس ، جو کہ \عددی{x}-محور کے متوازی ہے، کی مساوات معلوم کریں۔
\انتہا{سوال} 
 
 \ابتدا{سوال}\شناخت{سوال6c9}  
خم \(y=x^{2} -2x\) پے بننے والی خط مماس ، جو کہ مساوات  \(2y=x-1\)کے  عمودی ہے، کی مساوات معلوم کریں۔
\انتہا{سوال} 

 \ابتدا{سوال}\شناخت{سوال6c10}  
خم \(y=3x^{2} -2x -1\) پے بننے والے عمودی خط  ، جو کہ مساوات  \(y=x-3\)کے  متوازی ہے، کی مساوات معلوم کریں۔
\انتہا{سوال} 

\ابتدا{سوال}\شناخت{سوال6c11}  
خم \(y=(x-1)^{2}\)پے بننے والے عمودی خط  ، جو کہ \عددی{y}-محور    کے  متوازی ہے، کی مساوات معلوم کریں۔
\انتہا{سوال}

\ابتدا{سوال}\شناخت{سوال6c12}  
خم \(y=2x^{2} +3x +4\) پے بننے والے عمودی خط  ، جو کہ مساوات \(y=7x-5\)  کے  عمودی ہے، کی مساوات معلوم کریں۔
\انتہا{سوال}

\ابتدا{سوال}\شناخت{سوال6c13}  
سوال \حوالہء{سوال6b9} اور \حوالہء{سوال6b10} میں استعمال کیے گۓ طریقے کو استعمال کرتے ہوۓ مساوات  \(y=x^3\) اور \(y=x^4\) کے ڈھلاؤ کے کلیے معلوم کریں۔
\انتہا{سوال} 

\ابتدا{سوال}\شناخت{سوال6c14}
  اس سوال کے ہر حصے میں، مساوات \(y=\sqrt{x}\) کے خم کے دو نقطوں کو جوڑنے والی ہم آہنگ لکیر کا ڈھلاؤ معلوم کریں۔ دونوں نقاط کے  \عددی{x}-محدد بتاۓ گۓ ہیں۔ 
\begin{multicols}{3}     
\begin{enumerate}[a.]
\item
  \عددی{1}  اور   \عددی{1.001}  
\item
\عددی{1}  اور   \عددی{0.9999}  
\item
\عددی{4}  اور   \عددی{4.002}  
\item
\عددی{4}  اور   \عددی{3.999}  
\item
\عددی{0.25}  اور   \عددی{0.250001}  
\item
\عددی{0.25}  اور   \عددی{0.249999}  
\end{enumerate}
 \end{multicols}
\انتہا{سوال} 

\ابتدا{سوال}\شناخت{سوال6c15}  
اس سوال کے ہر حصے میں خم \(y=\frac{1}{x}\) پر بننے والئ دو نقاط کو جوڑنے والی    ہم آہنگ لکیر کا ڈھلاؤ معلوم کریں۔ یاد رہے ایک نقطے کے محدد ذیل میں بتاۓ گۓ ہیں اور دوسرا نقطہ آپ کوئ بھی  چن لیں لیکن وہ بتاۓ گۓ نقطے کے قریب ترین ہو۔ اب چنے گۓ نقطے اور بتاۓ گۓ نقطے کا فاصلہ تبدیل کر کے ڈھلاؤ معلوم کریں ، اور اس بات کا خیال رکھیں کہ آپ کچھ نقاط بتاۓ گۓ نقطے کے دائیں طرف چنیں جبکہ کچھ نقاط بائیں طرف۔
\begin{multicols}{3}     
\begin{enumerate}[a.]
\item   \(   (-1,-1)   \)
\item \(    (-2,-0.5)   \)
\item  \(   (10,0.1)   \)
\end{enumerate}
 \end{multicols}
\انتہا{سوال} 



%page84
  16۔ \(y=\sqrt{x}\) اور \(y=\frac{1}{x}\) کے ترسیمات  پر کسی بھی مقام پر خط مماس  کے ڈھلوان کے بارے میں اندازہ لگانے کے لئے سوالات\حوالہء{سوال14}   اور\حوالہء{سوال15}   کے نتائج کا استعمال کریں

 \حصہ{ تفریق کے کچھ اصول}
 

اگر \(f(x)= ax^2 +bx +c\) ہے ، پھر  \(f'(x) = 2ax +b\) ہو گا۔

اگر آپ دو تفاعل کو  جمع کرتے ہیں تو پھر تفرق کی جمع تفرقات کی جمع ہو گی ؛
جیسا کہ اگر 
\(f(x)=g(x)+h(x)\)ہے تو پھر \(f'(x)= g'(x) +h'(x)\) ہو گا۔

اگر آپ تفاعل کو کسی ، مستقل سے مضرب کرتے ہیں تو آپ کو تفریق کو بھی اسی مستقل سے مضرب کرنا پڑے گا۔  جیسا کہ اگر  \(f(x)= ag(x)\) ہے تو \(f'(x) =ag'(x)\) ہو گا ۔

مشق\حوالہء{مشق6C } کے سوال نمبر\حوالہء{سوال13}  سے آپکو پتا لگ جاۓ گا کہ \(f(x) =x^3\) کا تفرق \(f'(x)= 3x^2\) ہے اور \(f(x)=x^4\) کا تفرق \(f'(x) = 4x^3\) ہے ۔ آپ پہلے سے جانتے ہیں کہ اگر \(f(x)=x^2\) ہے تو \(f'(x) = 2x\) یا \(2x^1\) ہو گا۔

اگر \(f(x)=x^n\) ،(جہاں \عددی{  n   }ایک مثبت عدد صحیح ہے) ، تو \(f'(x) = nx^{n-1}\) ہو گا۔

 \ابتدا{مثال}
\(y=x^3 -3x^2 -4x +2\) کی ترسیم پر نقطہ کے نقاط (coordinates) معلوم کریں جس پر ڈھلوان \عددی{    5} ہے

فرض کریں کہ \(f(x)=x^3 -3x^2 -4x +2 \) ، تو پھر \(f'(x)= 3x^2 -6x -4\)  ہو گا   ۔ جب \(f'(x)=5\) تو ڈھلوان بھی \عددی{    5} ہے  ، اس طرح جب 
\(3x^2 -6x -4=5\) ہے تو یہ دودرجی الجبرائی مساوات 
\(3x^2 -6x -9=0\)دیتا ہے جو تسہیل ہو کہ \(x^2 -2x -3=0\) بن جاتی ہے   
 
جز و ضربی کی صورت میں 
\((x+1)(x-3)=0\)ہےتو اس طرح \(x=-1\) یا \(x=3\) ہو گا ۔

ہم پلے کہ نقطے معلوم کرنے کے لیے ان اقدار کو \(y=x^3 -3x^2 -4x +2\) میں متبادل کرنا ہو ہے 
اس طرح آپ کو \(y=(-1)^{3} -3 \times (-1)^{2} -4 \times (-1) +2=-1-3+4+2=2 \) اور \(y=3^3 -3 \times 3^2 -4 \times 3+2 =27 -27 -12 +2 =-10\) حاصل ہو گا ۔ 

اس لئے مطلوبہ نکات کے ہم پلے \((-1,2)\) اور \((3,-10)\) ہیں 
\انتہا{مثال}

مشق\حوالہء{مشق6C}  کے سوال نمبر \حوالہء{سوال14}سے \حوالہء{سوال16} تک دو مزید اصول تجویز کرتے ہیں ۔
اگر \(f(x) = \sqrt{x}\) تو \(f'(x) = \frac{1}{2 \sqrt{x}}\) ہو گا 
اگر \(f(x) = \frac{1}{x}\) تو \(f'(x) = - \frac{1}{x^2}\) ہو گا ۔ 
یں
 %page85

اشاریہ اشارے میں ، یہ نتائج اس طرح شکل اختیار کرتے ہیں  کہ؛
اگر \(f(x) =x^{\frac{1}{2}}\) ، تو \(f'(x)= \frac{1}{2} x^{- \frac{1}{2}}\) ہو گا
اگر \(f(x)=x^{-1}\) ، تو \(f'(x)=-x^{-2} = (-1)x^{-2}\) ہو گا

یہ مندرجہ ذیل اصول تجویز کرتا ہے 

اگر \(f(x)=x^n\) ( جہاں n  ایک نا طق عدد ہے)، تو  \( f'(x) =nx^{n-1}\) ہو گا۔ 

 \ابتدا{مثال}

 \(y=2\sqrt{x}\)  کی ترسیم پر بننے والے خط مماس کی مساوات معلوم کریں جہاں \(x=9\) ہو

فرض کریں کہ \(f(x)=2\sqrt{x}=2x^{\frac{1}{2}}\) ہے 

تو ، خانے میں دیے گئے نتائج کو استعمال کرتے ہوئے   \[f'(x) = 2 \times \frac{1}{2} x^{-\frac{1}{2}}= x^{-\frac{1}{2}}\] ہو گا

 جب \(x=9\) ہے ، تو\(  f'(9) =9^{-\frac{1}{2}}=\frac{1}{\sqrt{9}} = \frac{1}{3}\) ہو گا

جب مماس نقطہ \((9,2\sqrt{9})=(9,6)\) سے گزرے گی، تو یہ مساوات  بنے گی 

\[y-6=\frac{1}{3}(x-9)\]  یا  \[3y-x=9\]
 \انتہا{مثال}
 \ابتدا{مثال}



دیے گئے ہر تفاعل کو تفریق کریں 
\begin{enumerate}[.a]
\item \(x(1+x^2)\)
\item \((1+\sqrt{x})^2\)
\item \(\frac{x^2+x+1}{x}\)
\end{enumerate}

ا۔   فرض کریں کہ   \(f(x)=x(1+x^2)\) ہے،
پھر \(f(x)=x+x^3\) ہو گا ، اور \(f'(x)= 1+3x^2\) ہو گا

ب۔ فرض کریں کہ  \(f(x)=(1+\sqrt{x})^2\) ہے ،
پھر \(f(x)=1+2\sqrt{x} +x=1+2x^{\frac{1}{2}} +x \) ہو گا ، اور \(f'(x)=2 \times \frac{1}{2} x^{-\frac{1}{2}}+1=x^{-\frac{1}{2}}+1=\frac{1}{\sqrt{x}}+1\) ہو گا ۔
 
ج۔ فرض کریں کہ \( f(x)= \frac{x^2+x+1}{x}\)  ہے ،
 پھر تقسیم کے لحاظ سے  \(f(x)=x+1+\frac{1}{x}=x+1+x^{-1}\) ہو گا 
، اور  \(f'(x)=1+(-1)x^{-2} = 1-\frac{1}{x^2}\)
 \انتہا{مثال}


%page86

 \ابتدا{مثال}

نقطہ \((8,2)\) پر \(y=\sqrt[3]{x}\) سے بننے والے خط مماس کی مساوات معلوم کریں ۔ 

اشاریہ اشارہ میں \(\sqrt[3]{x}=x^{\frac{1}{3}}\) ہے ، تو اصول کے مطابق  اس کا تفرق \(\frac{1}{3}x^{(\frac{1}{3}-1)}\) یا \(\frac{1}{3}x^{-\frac{2}{3}}\) ہو گا۔

جو اضافی اشارے میں اس طرح لکھا جاۓ گا \(\frac{1}{3(\sqrt[3]{x})^2}\)۔  نقطہ\((8,2)\) پہ ، یہ اس طرح لکھا جاۓ گا \(\frac{1}{3(\sqrt[3]{x})^2}=\frac{1}{12}\)۔

اس طرح خط مماس کی مساوات \(y-2=\frac{1}{12}(x-8)\) ہا 
\(x-12y+16=0\) ہے
\انتہا{مثال}

اس حصے میں بیان کردہ نتائج کو اس کتاب کے باقی حصے کے لئے سمجھا جاسکتا ہے۔ ان میں سے کچھ حصہ \حوالہء{حصہ6.6}اور \حوالہء{حصہ6.7} میں ثابت ہیں ، لیکن اگر آپ چاہتے ہیں تو آپ ان آخری حصوں کو چھوڑ سکتے ہیں اور ، مشق\حوالہء{مشقD6 }  کے ذریعے کام کرنے کے بعد ، براہ راست متفرق  مشق \حوالہء{مشق6} میں جا سکتے ہیں ۔

%excercise 6D

 
\ابتدا{سوال}
  مندرجہ ذیل تفاعل کو تفریق کریں۔

\begin{enumerate}[.a]
\item \(x^3+2x^2\)
\item \(1-2x^3+3x^2\)
\item \(x^3-6x^2+11x-6\)
\item \(2x^3-3x^2 +x\)
\item \(2x^2(1-3x^2)\)
\item \((1-x)(1+x+x^2)\)
\end{enumerate}
\انتہا{سوال}

\ابتدا{سوال}
  مندرجہ ذیل میں دیۓ گۓ ہر تفاعل \(f(x)\) کے لیے \(f'(-2)\) معلوم کریں ۔

\begin{enumerate}[.a]
\item \(2x-x^3\)
\item \(2x-x^2\)
\item \(1-2x-3x^2+4x^3\)
\item \(2-x\)
\item \(x^2(1+x)\)
\item \((1+x)(1-x+x^2)\)
\end{enumerate}
\انتہا{سوال}

\ابتدا{سوال}
   مندرجہ ذیل میں  دیۓ گۓ ہر تفاعل \(f(x)\) کے لیے x کی رقم معلوم کریں ، اس طرح کے \(f'(x)\) دیۓ گۓ عدد کے برابر ہو۔

\begin{enumerate}[.a]
\item \(x^3 \quad \quad 12\)
\item \(x^3-x^2 \quad \quad 8\)
\item \(3x-3x^2+x^3 \quad \quad 108\)
\item \(x^3 -3x^2 +2x \quad \quad -1\)
\item \(x(1+x)^2 \quad \quad 0\)
\item \(x(1-x)(1+x) \quad \quad 2\)
\end{enumerate}
\انتہا{سوال}
\ابتدا{سوال}
 مندرجہ ذیل میں دیۓ گۓ تفاعل کو تفریق کریں ۔

\begin{enumerate}[.a]
\item \(2\sqrt{x}\)
\item \((1+\sqrt{x})^2\)
\item \(x-\frac{1}{2} \sqrt{x}\)
\item \(x \big( 1-\frac{1}{\sqrt{x}}\big)^2\)
\item \(x-\frac{1}{x}\)
\item \(\frac{x^3+x^2+1}{x}\)
\item \(\frac{(x+1)(x+2)}{x}\)
\item \(\big( \frac{\sqrt{x}+x}{\sqrt{x}} \big)\)
\end{enumerate}
\انتہا{سوال}
\ابتدا{سوال}
 مساوات \(y=x^3+x\) کی ترسیم پر بننے والے خط مماس کی مساوات معلوم کریں ، ایسے نقطہ پہ جہاں \(x=-1\) ہو
\انتہا{سوال}
\ابتدا{سوال}
  مساوات \(y=4x-x^3\) کے ساتھ خط منحنی کی ایک خط مماس میں سے ایک مساوات \(y=x-2\) کی لکیر ہے۔ دوسرے خط مماس کی مساوات  تلاش کریں جو \(y=x-2\) کے متوازی ہو۔ 
\انتہا{سوال}
\ابتدا{سوال}
 مساوات \(y=\sqrt{x}\) کے ساتھ خط منحنی  پہ دئیے ہوئے نقطہ \((4,2)\)  پر خط مماس کی مساوات تلاش کریں
\انتہا{سوال}
\ابتدا{سوال}
  مساوات \(y=\frac{1}{x}\) کے ساتھ خط منحنی  پہ دئیے ہوئے نقطہ \((2,\frac{1}{2})\)  پر خط مماس کی مساوات تلاش کریں
\انتہا{سوال}
\ابتدا{سوال}۔ \(y=x+\frac{1}{x}\) کی ترسیم پر عمودی  مساوات نقطہ \((1,2)\) پر معلوم کریں 
\انتہا{سوال}
\ابتدا{سوال}
مساوات  \(y=x^2-2x\) اور \(y=x^3-3x^2-2x\)   کے ترسیمات،  دونوں مبدا میں سے گزر رہیں ہیں ، ثابت کریں کے کہ یہ ایک ہی خط مماس کا اشتراک کریں گے۔ 
\انتہا{سوال}



%page 87


\ابتدا{سوال}
ایک ترسیم یا خم کی مساوات \(y=x^3-3x^2-2x-6\) ہے، اس خم پر بننے والی خط مماس کی ممساوات معلوم کریں اس نقطے پر کہ جہاں یہ عمودی محور کو کاٹتا ہے۔ 
\انتہا{سوال}
 \ابتدا{سوال}
ایک ترسیم یا خم کی مساوات \(y=x(x-a)(x+a)\) ہے جس میں \عددی{ a  } مستقل ہے، اس خم پر بننے والی خط مماس کی ممساوات معلوم کریں اس نقطے پر کہ جہاں یہ عمودی محور کو کاٹتا ہے۔ 
\انتہا{سوال}
\ابتدا{سوال}
اس مشترک نقطے کے محدد معلوم کریں جو کہ ترسیم \(y=x^2\) کے خط مماس  اور ایک خط مستقیم \(y=x+2\) کے درمیان ہے۔
\انتہا{سوال}
\ابتدا{سوال}
دیے گۓ تمام تفاعل کے متفرق معلوم کریں۔ جس طرح سے سوال دیا گیا ہے ویسی ہی شکل میں جواب دیں لیکن جذر اور منفی علامت کے بغیر۔

\begin{multicols}{3}
\begin{enumerate}[.a]
\item \(\frac{1}{4x}\)
\item \(\frac{3}{x^2}\)
\item \(x^0\)
\item \(\sqrt[4]{3}\)
\item \(6\sqrt[3]{x}\)
\item \(\frac{4}{\sqrt{x}}\)
\item \(\frac{3}{x}+\frac{1}{3x^3}\)
\item \(\sqrt{16x^5}\)
\item \(x\sqrt{x}\)
\item \(\frac{1}{\sqrt[3]{8x}}\)
\item \(\frac{x-2}{x^2}\)
\item \(\frac{1+x}{\sqrt[4]{x}}\)
\end{enumerate}
\end{multicols}
\انتہا{سوال}
\ابتدا{سوال}
ترسیم \(y=\sqrt[3]{x^2}\) کی خط ممماس اور عمودی خط کی مساوات معلوم کریں نقطہ \((8,4)\) پہ۔
\انتہا{سوال}
\ابتدا{سوال}
 ترسیم \(y=\frac{1}{x^2}\) کے نقطے  \(\big( \frac{1}{2},4 \big)\) پر بنی خط مماس محور کو نقطے \عددی{ P } اور \عددی{ Q } پے چھوتی ہے، ان دونوں نقطوں کے محدد معلوم کریں۔
\انتہا{سوال}

 \حصہ{دو درجی ترسیم کے ڈھلاؤ کا کلیہ}
 اگر آپ چاہیں تو آپ ان آخری کے حصوں کو چھوڑ کر سیدھا آخری مشق پے جا سکتے ہیں۔ اس حصے کا مقصد صرف اتنا ہے کہ آپ دو درجی ترسیم کا ڈھلاؤ معلوم کر سکیں وہ بھی اندازے لگاۓ بغیر۔

 \ابتدا{مثال}
ترسیم \(y=x^2\) کے راگ (کارڈ) کا ڈھلاؤ معلوم کریں، جس کے افقی محدد \عددی{ P } اور \(p+h\) ہیں۔ اس نقطے کے عمودی محدد ہوں گے \(p^2\) اور \((p+h)^2\)۔
\[\delta x=h, \delta y=(p+h)^2 -p^2=p^2 +2ph+h^2-p^2=2ph+h^2=h(2p+h)\]
اور یوں ڈھلاؤ ہوگا؛
\[\frac{\delta y}{\delta x}=\frac{h(2p+h)}{h}=2p+h\]
اس بات پر غور کریں کے مثالوں \حوالہء{مثال6.1.1} سے \حوالہء{مثال6.1.3} کے ڈھلاؤ اس جواب کے خاص معاملے تھے۔جیسا کہ ٹیبل \حوالہء{ٹیبل 6.7} میں دکھایا بھی گیا ہے۔
 \انتہا{مثال}
%page 88
الجبرا کے استعمال سے آپ کو یہ فائدہ ہوگا کہ ڈھلاؤ معلوم کرتے وقت آپ کو ہر بار  شروع سے کام نہیں کرنا پڑے گا۔ ذیل میں دیا گیا ٹیبل \( p=0.4  \) کے لیے چند مزید جوابات، جن میں چند منفی اور چند مثبت ہیں، دکھاتا ہے جب \عددی{ h }کی قیمت مسلسل تبدیل کی جاتی ہے۔ 

آگر آپ کے پاس ایک ترسیمی اعدد ہے، یا کوئ کمپوٹر کا ایسا پروگرام جو کہ ترسیم بنا سکتا ہو تو یہ نہایت دلچسپ ہوگا مساوات \(y=x^2\)کا ترسیم بنایا جاۓ  اور اس ترسیم کے دو نقاط \( (0.4 , 0.16) \)کو ملانے والی راگ کا ڈھلاؤ دیکھا جاۓ۔ آپ مشاہدہ کریں گے کہ جب \عددی{ x} صفر کے بہت  قریب ہوگا، یعنی راگ کے دونوں کناروں کے درمیان کا فاصلہ بہت کم ہوگا، تو یہ تقریباً نا ممکن سا ہو جاۓ گا خط مماس اور راگ کو الگ الگ سے پہچاننا۔ اور آپ ٹیبل \حوالہء{ٹییبل6.7} اور \حوالہء{ٹیبل 6.8} میں ان راگوں کے لیے دیکھ سکتے ہیں کہ ان کا ڈھلاؤ \عددی{ 0.8} کے کس قدر قریب ہے۔
 
حقیقت میں \عددی{ h } کو صفر کے قریب لانے سے آپ اس راگ کا ڈھلاؤ \عددی{ 0.8} کت قریب لا سکتے ہیں جتنا آپ چاہیں۔ مثال \حوالہء{مثال6.6.1} اس راگ کا ڈھلاؤ \( (0.8+h) \) تھا، یعنی اگر آپ چاہتے ہیں کہ نقاط \( (0.4 , 0.16) \) سے گزرنے والی خط مماس کا ڈھلاؤ \عددی{ 0.799999} یا \عددی{ 0.800001 } ہو تو آپ کو \عددی{ h} کی قیمت \عددی{ -0.000001 } اور \عددی{  +0.000001} کے درمیان کہیں رکھنی پڑے گی۔
ایک ایسی قیمت جو آپ \عددی{  h} کے لیے نہین لے سککتے وہ صفر خود ہی ہے۔ تاہم آپ کہہ سکتے ہیں کہ ؛

حد میں جیسے جیسے \عددی{  h} صفر کے قریب تر ہوتا جاتا ہے ، راگ کا ڈھلاؤ \عددی{0.8} کے قریب ترین ہوتا جاتا ہے۔

اس بیان کو لکھنے کا عام اور ذیادہ مقبول طریقہ یہ ہے کہ ؛
\[\lim_{h \to 0} \texttt{}=\lim_{h \to 0}(0.8+h)=0.8 \]

\عددی{ p } کی قیمت \عددی{0.4} لینے سے کچھ بھی خاص نہیں ہوگا، اور آپ \عددی{ p } کی کسی بھی اور قیمت کے لیے بھی یہی راۓ رکھیں گے۔مثال \حوالہء{مثال6.6.1} یہ بتاتی ہے کہ راگ کا ڈھلاؤ جو کہ جوڑ رہی ہے \((p,p^2)\) کو \((p+h,(p+h)^2)\) سے، \(2p+h\) ہوگا۔ اگر آپ \عددی{ p } کی قیمت کو مستقل کر دیں اور \عددی{h} کی قیمتوں کو مسلسل تبدیل کریں تو جیسا ہم نے اوپر بیان کیا تھا، 
\[\lim_{h \to 0} \texttt{راگ کا ڈھلاؤ}=\lim_{h \to 0}(2p+h)=2p \]
اور اسی ترسیم \(y=x^2\) کے  لیے نقطہ \((p,p^2)\) پہ راگ کا ڈھلاؤ \(2p\) ہے۔

%page 89
اس سے یہ ثابت ہوتا ہے کہ؛

ترسیم  \(y=x^2\) کے لیے ڈھلاؤ کا کلیہ \(2x\) ہے۔

کسی بھی ترسیم کے لیے یہی طریقہ لاگو ہوگا اگر آپ کو اسکی مساوات کا پتہ ہے۔

شکل \حوالہء{شکل6.9} میں ایک ترسیم دکھایا گیا ہے جسکی مساوات \(y=f(x)\) یے، یعنی یہ کسی بھی تفاعل کا ترسیم ہے۔ فرض کرین کہ آپ کو نقطہ     \عددی{ p }          پر خط مماس کے ڈھلاؤ کی قیمت معلوم کرنی ہے جسکے محدد \((p,f(p))\) ہیں۔ اس نقطے کو کسی بھی دوسرے نقطے \((p+h,f(p+h))\) سے ملانے والی راگ کے لیے ہم کہہ سکتے ہیں کہ؛ \[\delta x=h, \, \delta y=f(p+h)-f(p)\]
یعنی اسکا ڈھلاؤ ہوگا؛
\[\frac{\delta y}{\delta x}=\frac{f(p+h)-f(p)}{h}\]
 اور اب ذرا \عددی{h} کی قیمت تبدیل کرین تاکہ نقطہ \عددی{ Q } ترسیم میں کسی اور جگہ چلا جاۓ ، تب اگر \عددی{ Q }نقطہ \عددی{ P } کے قریب ہوگا یعنی \عددی{h} صفر کے قریب ہوگا، تو نقطہ \عددی{ P } پر راگ کا ڈھلاؤ اسی نقطے پر بننے والی خط مماس کے ڈھلاؤ کے قریب ترین ہوگا، جبکہ \(x=p\)، حد میں جب \عددی{h} صفر کی طرف بڑھتا ہے تو یہ بیانیہ \(f'(p)\) کی طرف بڑھتا ہے۔

اگر خم، ترسیم \(y=f(x)\) کا نقطہ \((p,f(p))\) پر خط مماس ہے تو ، اسکا ڈھلاؤ \[\lim_{h \to 0} \frac{f(p+h)-f(p)}{h}\] ہوگا۔

اس قیمت کو تفاعل \(f(x)\) کا تفرق بھی کہا جاتا ہے  اور علامت \(f'(p)\) سے ظاہر کیا جاتا ہے۔

اور چونکہ  \عددی{ P } ، \عددی{x} ککی کسی بھی قیمت کے لیے درست ہوگی تو ہم لکھ سکتے ہیں کہ، 

تفاعل \(f(x)\) کا \عددی{x} کی کسی بھی قیمت پر تفرق \(f'(p)\) ہوگا،  \(f'(x)=\lim_{h \to 0} \frac{f(x+h)-f(x)}{h}\)۔


%page 90
%example 6.6.2
 \ابتدا{مثال}
تفاعل \(f(x)=4x-5\) کا تفرق معلوم کریں۔

تفرق کی تعریف کو مد نظر رکھتے ہوۓ،\(f'(x)=\lim_{h \to 0} \frac{f(x+h)-f(x)}{h}\) ۔

کسر کی اوپری سطر کے مطابق؛
\[f(x+h)-f(x)=(4(x+h)-5)-(4x-5)=4x+4h-5-4x+5=4h\]
 اسی لیے؛
\[\frac{f(x+h)-f(x)}{h}=\frac{4h}{h}=4\]
اور پھر حد مین جب \عددی{h} صفر کے قریب تر ہوتا جاتا ہے، 
\[f'(x)=\lim_{h \to 0} \frac{f(x+h)-f(x)}{h}= \lim_{h \to 0} 4=4\]
یقیناً آپ کو اندازہ تھا کہ نتیجہ کیا ہوگا، پہلے ہی سبق میں ہم نے سیکھا تھا کہ \(f(x)=4x-5\) کی ترسیم ایک خط مستقیم ہے جسکا ڈھلاؤ \عددی{ 4} ہے۔ تو یہ کوئ اچنبے کی بات نہیں ہونی چاہئیے کہ مساوات \(f(x)=4x-5\) کا تفرق \عددی{ 4} ہے۔

اسی طرح سے تفاعل \(f(x)=mx+c\) جو کہ ایک سیدھی لکیر ہے کا تفرق \عددی{ m} ہوگا۔ 
 \انتہا{مثال}

 \ابتدا{مثال}
تفاعل \(f(x)=3x^2\) کا تفرق معلوم کریں۔
 اس تفاعل کے لیے \(f(x+h)=3(x+h)^2 =3x^2+6xh+3h^2\)
اسی لیے
\[f(x+h)-f(x)=(3x^2 +6xh+3h^2)-3x^2=6xh+3h^2=h(6x+3h)\]
اور \(\frac{f(x+h)-f(x)}{h}=\frac{h(6x+3h)}{h}=6x+3h\)
اور پھر حد مین جب \عددی{h} صفر کے قریب تر ہوتا جاتا ہے، 
\[f'(x)=\lim_{h \to 0} \frac{f(x+h)-f(x)}{h} =\lim_{h \to 0} (6x+3h)=6x\]
غور کریں کہ تفاعل \(f(x)=x^2\) کا تفرق \(2x\) ہے، اور تفاعل \(f(x)=3x^2\) کا تفرق \(6x\) ہے، جو کہ \(3 \times 2x\) ہے۔ یہ حصہ\حوالہء{حصہ6.5} میں بتاۓ گۓ اصولوں میں سے پہلے کی مثال ہے۔ 

اگر آپ ایک تفاعل کو ایک عدد سے ضرب دیں تو اس تفاعل کا تفرق بھی اسی عدد سے ضرب کھاۓ گا۔
 \انتہا{مثال} 




























%page91
\ابتدا{مثال}\شناخت{ؐمثال6.6.4}  
تفاعل \(f(x)=3x^2+4x-5\) کا تفرق معلوم کریں۔
تفاعل \(f(x)=3x^2+4x-5\) کے لیے ہم کہہ سکتے ہیں کہ؛\[f(x+h)=3(x+h)^2+4(x+h)-5=3x^2+6xh+3h^2+4x+4h-5\]
اس لیے؛ 
\begin{align}
f(x+h)-f(x)&=(3x^2+6xh+3h^2+4x+4h-5)-(3x^2+4x-5)\\
&=3x^2+6xh+3h^2+4x+4h-5-3x^2-4x+5\\
&=6xh+3h^2+4h=h(6x+3h+4)\\
&\frac{f(x+h)-f(x)}{h}=\frac{h(6x+3h+4)}{h}=6x+3h+4
\end{align}
 اور پھر اس حد میں ، جیسے جیسے  \عددی{h} صفر کے قریب جاتا ہے ؛ 
\[f'(x)=\lim_{h \to 0}\frac{f(x+h)-f(x)}{h}=\lim_{h \to 0} (6x+3h+4)=6x+4 \]
\انتہا{مثال} 
مثالیں   \حوالہء{مثال6.6.2} اور   \حوالہء{مثال6.6.3} ایک اور عام اصول کی وضاحت کرتی ہیں۔مثال   \حوالہء{مثال6.6.4} میں جو تفاعل دیا گیا تھا وہ دراصل مثال   \حوالہء{مثال6.6.2} اور مثال   \حوالہء{6.6.3}  میں دہے گۓ تفاعل کا مجموعہ ہے۔ اور اسی نسبت سے   مثال   \حوالہء{مثال6.6.4} میں آنے والی ڈھلاؤ کی قیمت مثال  \حوالہء{مثال6.6.2} اور مثال \حوالہء{6.6.3}   میں آنے والی ڈھلاؤ کی قیمتوں کا مجموعہ ہے۔
 عام قانون کے مطابق ؛
 
\ابتدا{تعریف}
اگر آپ دو تفاعل کو جمع کرتے ہیں ، تو آپ اس نۓ تفاعل کا تفرق معلوم کرنا چاہتے ہیں تو ، آپ کو دونوں تفاعل، جن کا یہ تفاعل مجموعہ ہے، کے تفرق کو جمع کرنا ہوگا۔
\انتہا{تعریف}  

\حصہ{چند مزید تفاعل کے ڈھلاؤ کے کلیے}
چند تفاعل کے لیے حصہ  \حوالہء{حصہ6.6} میں بتایا گیا طریقہ آپکو مشکل الجبرا میں ڈال سکتا ہے، اور یہ آسان ہے کہ آپ ایک دوسری علامت استعمال کریں۔  دونقطوں ، جن میں ایک نقطہ \عددی{p} جو کہ  \عددی{x}-محور پر موجود ہے جبکہ دوسرا نقطہ   \عددی{p+h} (یا   \عددی{x}  اور    \عددی{x+h}) ، کو جوڑنے والی ہم آہنگ لکیر کا ڈھلاؤ معلوم کرنے کے بجاۓ آپ مختلف طرح کے محدد چن سکتے ہیں، جیسے کہ  \(\big( p,f(p) \big)\) یا \(\big( q,f(q) \big)\)
\(\big( q,f(q) \big)\) رکھنے کے لیے نقاط لے سکتے ہیں ،  اس طرح کے \(\delta x=q-p\) اور 

\(\delta y =f(q)-f(p)\) ہے   

تو پھر ڈھلوان یہ \(\frac{\delta y}{\delta x}=\frac{f(q)-f(p)}{q-p}\) ہو گی ۔ 

یہ دیکھنے کے لیے کے یہ کس طرح کام کرتا ہے ، یہاں مثال 3۔6۔6 میں   اس کا استعمال ہوا ہے ۔ 

 \ابتدا{مثال}

\(x=p\) پر تفاعل \(f(x)=3x^2\) کا تفرق معلوم کریں ۔ 

اس تفاعل کے لیے ، \(f(q)-f(p)=3q^2-3p^2=3(q^2-p^2)=3(q-p)(q+p)\) ہے.


%page92
لہذا  \(\frac{\delta y}{\delta x}=\frac{f(q)-f(p)}{q-p}\frac{3(q-p)(q+p)}{q-p}=3(q+p)\) ہو گا۔ 
\انتہا{مثال} 
  
اب اس طریقہ میں\عددی{ q  } نے \( p+h  \) کی جگہ لی ہے ،  چنانچہ آپ ایک حد میں \عددی{ h  } کی قیمت کو \عددی{    0} (صفر) لینے کے بجاۓ ، آپ  \عددی{ q  } کی قیمت کو \عددی{    p   } لے سکتے ہیں۔ 
یہ دیکھنا کافی آسان ہے کہ جب \عددی{ q  }  کی قیمت \عددی{    p   }کی طرف جاتی ہے تو  
\(3(q+p)\) کی قیمت \(3(p+p)=3(2p)=6p\)
 کی طرف جاۓ گی۔


لہذا ، اگر \( f(x)=3x^2  \) تو  \(f'(p)=6p\)ہو گا ۔  چونکہ یہ \عددی{    p   }کی ہر قیمت کے لیے کارآمد ہے ، لہٰذا آپ \(f'(x)=6x\) لکھ سکتے ہیں ۔ 

اس اشارے میں ، تفرق کی تعریف یہ شکل اختیار کرتی ہے:
\(x=p\) پر \(f(x)\) کا تفرق یہ \(f'(p)=\lim_{q \to p}\frac{f(q)-f(p)}{q-p}\) ہے ۔ 
 \(x=p\) پر تفاعل \(f(x)=x^4\) کا تفرق معلوم کریں ۔ 
\(x=p\) پر \(f(p)=p^4\) ہے اور \(x=q\) پر ,\(f(q)=q^4\) ہے ۔ 
  
خط مستقیم جو \((p,p^4)\) اور \((q,q^4)\) کو ملا رہی ہے اس کے لیے \(\delta x=q-p, \delta y= q^4-p^4\) ہے۔ 


دھیان دیں کہ آپ \(\delta y\) کو \(q^2)^2 -(p^2)^2 \) لکھ سکتے ہیں ۔  \[\delta y= (q^2 -p^2)(q^2+p^2)=(q-p)(q+p)(q^2+p^2)\] حاصل کرنے کے لیے،  آپ  دو مربع کے فرق کو دو بار استعمال کر سکتے ہیں  

لہذا، \[\frac{\delta y}{\delta x}=\frac{(q-p)(q+p)(q^2+p^2)}{q-p}=(q+p)(q^2 +p^2)\] ہے ۔

پھر ایک حد میں جب ، \عددی{ q  } کی قیمت \عددی{    p   } ہو  گی تو \[f'(p)=\lim_{q \to p}\frac{f(q)-f(p)}{q-p}=\lim_{q \to p}\big((q+p)(q^2 +p^2)\big)=2p(2p^2)=4p^3\] ہو گا۔ 


 \ابتدا{مثال}  
\(x=p\) پر تفاعل \(f(x)=\sqrt{x}\) کا تفرق معلوم کریں ۔
\(x=p\) پر \(f(p)=\sqrt{p}\) اور \(x=q\) پر \(f(q)=\sqrt{q}\)   ہے۔
خط مستقیم جو \((p,\sqrt{p})\) اور \((q,\sqrt{q})\)  کو ملا رہی ہے  اس کے لیے   \(\delta x=q-p, \, \delta y=\sqrt{q}-\sqrt{p}\) ہے  
دھیان دیں کے آپ \(\delta x\) کو  دو مربع کے فرق کے طور پر اس طرح لکھ سکتے ہیں ۔ \[\delta x=q-p=(\sqrt{q})^2-(\sqrt{p})^2 = (\sqrt{q} - \sqrt{p})(\sqrt{q} + \sqrt{p})\]
,%page93
لہذا \(\frac{\delta y}{\delta x}= \frac{\sqrt{q}-\sqrt{p}}{(\sqrt{q}-\sqrt{p})(\sqrt{q}+\sqrt{p})}=\frac{1}{\sqrt{q}+\sqrt{p}}\) ہے ۔

پھر ایک حد میں جب ، q کی قیمت p ہوگی تو \[f'(p)=\lim_{q \to p}\frac{f(q)-f(p)}{q-p}\] 

\[=\lim_{q \to p}\frac{1}{\sqrt{q}+\sqrt{p}}\]


\[=\frac{1}{\sqrt{q}+\sqrt{p}}=\frac{1}{2\sqrt{p}}=\frac{1}{2}p^{-\frac{1}{2}}\] ہو گا۔

دھیان دیں کہ جب \(p=0\) ہو گا تب یہ کام نہی کرے گا  کیونکہ اس صورت میں \(\frac{\delta y}{\delta x}=\frac{1}{\sqrt{q}}\)  ہو گا ، جس کی کوئی حد نہی ہے چونکہ \(q \to 0\) ہے ۔ آپ شکل 10۔6 میں \(y=\sqrt{x}\) کے ترسیم سے دیکھ سکتے ہیں کہ \(x=0\) پر خط مماس y- محور پر ہے ، جس کی کوئی ڈھلوان نہی ہے۔ 
\انتہا{مثال}

 \ابتدا{مثال}

\(x=p\)  پر تفاعل \(f(x)=\frac{1}{x}\) کا تفرق معلعم کریں ۔ 

\(x=p\) پر \(f(p)=\frac{1}{p}\) ہے اور \(x=q\) پر \(f(q)=\frac{1}{q}\) ہے ۔

جو خط مستقیم \(\big( p,\frac{1}{p} \big)\) اور\(\big( q,\frac{1}{q} \big)\) کو ملا رہی ہے اس کے لیے \(\delta x=q-p, \, \delta y=\frac{1}{q}-\frac{1}{p}=\frac{p-q}{qp}=-\frac{q-p}{qp}\) ہے ۔

اور 

\(\frac{\delta y}{\delta x}=\frac{-\big( \frac{q-p}{qp}\big)}{q-p}=-\frac{1}{qp}\) ہے ۔ 

لہذا ، ایک حد میں جب \عددی{ q  } کی قیمت \عددی{    p   }ہوگی تو \[f'(p)=\lim_{q \to p} \frac{f(q)-f(p)}{q-p}=\lim_{q \to p} \big( -\frac{1}{qp}\big)=-\frac{1}{p^2}=-p^{-2}\] ہو گا۔ 

اگر آپ چکے پاس ، تو آپ اس خط منحنی کی نمائش کریں تب آپ دیکھ سکیں گے کہ اس کی ڈھلوان ہمیشہ منفی کیوں ہوتی ہے ۔ 
\انتہا{مثال}

%excercise 6E

اس مشق میں حصہ \حوالہء{حصہ6.7}   کا طریقہ استعمال کریں ۔
\begin{enumerate}[a.]
\item  
\(x=p\) پر تفاعل \(f(x)=x^3\) کا تفرق معلوم کریں ۔ ( آپ کو یا تو توسیع \((p+h)^3=p^3+3p^2h+3ph^2+h^3\)یا پھر مفرد کی ضرب \((q-p)(q^2+qp+p^2)=q^3-p^3\)  استعمال کرنے کی ضرورت ہوگی )۔
\item 
مساوات \(x=p\) پر تفاعل \(f(x)=x^8\) کا تفرق معلوم کریں ۔ ( فرض کریں \(p+h=q\) ہے اور جتنی بار آپ کر سکتے ہیں دو مربع  کلیہ کے فرق کو   \(q^8-p^8\) 
   پر استعمال کریں )۔
 \end{enumerate}



%page94
\(x=p\) پر تفاعل \(f(x)=\frac{1}{x^2}\)مساوات 
 کا تفرق معلوم کریں ۔

%miscelllenous excercise 6 
\ابتدا{سوال}
 نقطہ\( (2,10)  \) پر  \(y=5x^2-7x+4\) سے خط مماس کی مساوات معلوم کریں  
\انتہا{سوال}
\ابتدا{سوال}
  دیۓ ہوۓ تفاعل \(f(x)=x^3+5x^2-x-4\) کی مدد سے مندرجہ ذیل سوال حل کریں 

\begin{enumerate}[.a]
\item \(f'(-2)\)
\item 
 \عددی{    a   } کی قیمت معلوم کریں اس طرح کہ \(f'(a)=56\) ہے۔
\end{enumerate}
\انتہا{سوال}
\ابتدا{سوال}
 مساوات \(y=x^4-4x^3\) سے عمودی مساوات معلوم کریں ایک ایسے نقطہ پر جہاں \(x=\frac{1}{2}\) ہو
\انتہا{سوال}
\ابتدا{سوال}
  ثابت کریں کہ \(y=\frac{1}{x}\) کی مدد سے نقطہ \(x=p\) پر خط مماس کی مساوات \(p^2y+x=2p\) ہو گی ۔  خط منحنی کے کس نقطہ پر خط مماس کی مساوات \(9y+x+6=0\) ہو گی۔؟
\انتہا{سوال}
\ابتدا{سوال}
 منحنی خط \(y=6 \sqrt{x}\)  سے خط مماس نقطہ \( 4,12\)  پر محور \عددی{ A  }اور\عددی{   B   }ملتے ہیں۔ ثابت کریں کہ \عددی{ AB  }کا فاصلہ \(k \sqrt{13}\) کی شکل میں لکھا جا سکتا ہے ۔ مزید \عددی{   k  } کی قیمت معلوم کریں ۔
\انتہا{سوال}
\ابتدا{سوال}
   خط منحنی \(y=2x^3-5x^2+9x-1\)  سے دو نقاط کے محدد معلوم کریں جہاں خط مماس کی دڈھلوان \عددی{ 13  } ہو۔
\انتہا{سوال}
\ابتدا{سوال}
 نقطہ \( ( 1,8) \) پر \(y=(2x-1)(3x+5)\) سے عمودی مساوات معلوم کریں ۔ اپنا جواب \(ax+by+c=0\) صورت میں دیں جہاں \عددی{ a }، \عددی{ b  } اور\عددی{ c } عدد صحیح ہیں۔
\انتہا{سوال}
\ابتدا{سوال}
 خط منحنی \(y=x^2-3x-4\) \عددی{ P }اور \عددی{ Q  } پر\عددی{ x }-محورکو پار کرتی ہے ـ  \عددی{ P }اور \عددی{ Q  } پر خط منحنی سے خط مماس \عددی{ R  }پر ملتے ہیں ۔ \عددی{ P } اور  \عددی{ Q} پر خط منحنی سے عمودی خط \عددی{ S  } پر ملتے ہیں ۔ \عددی{ RS  } کا فاصلہ معلوم کریں۔
\انتہا{سوال}
\ابتدا{سوال}
 خط منحنی کی مساوات \(y=2x^2-5x+14\) ہے ۔ خط منحنی سے خط عمودی نقطہ \( (1،11)  \) پر خط منحنی سے دوبارہ نقطہ \عددی{ P }پر ملتی ہے ۔ \عددی{ P }کے محدد معلوم کریں ۔
\انتہا{سوال}
\ابتدا{سوال}
  خط منحنی \(y=x^2+k\) کے ایک مخصوص نقطہ پر خط مماس کی مساوات \(y=6x-7\) ہے ۔ مستقل \عددی{ c } کی قیمت معلوم کریں ۔
\انتہا{سوال}
\ابتدا{سوال}
  ثابت کریں کہ خط منحنی  \(y=x^3\) اور \(y=(x+1)(x^2+4)\) میں بالکل ایک نقطہ مشترک ہے ، اور تفریق کا طریقہ استعمال کرتے ہوۓ اسی نقطہ پر ہر ایک خط منحنی کا ڈھلوان معلوم کریں ۔
   \انتہا{سوال}
\ابتدا{سوال}
   خط منحنی \(y=5x^2-12x+1\) کے ایک مخصوص نقطہ پر عمودی مساوات \(x+18y+c=0\) ہے ۔ مستقل \عددی{ c } کی قیمت معلوم کریں ۔
\انتہا{سوال}
\ابتدا{سوال}
 مساوات \(y=x^m\) اور \(y=x^n\) کی ترسیمات نقطہ \( (1,1)  \) ایک دوسرے میں سے گزرتی ہیں ، \عددی{ m  }اور \عددی{ n }کے درمیان تعلق معلوم کریں اگر ہر ایک خط منحنی پر خط مماس نقطہ  \عددی{ P  } پر دوسری خط منحنی کے عمودی ہے۔
\انتہا{سوال}
\ابتدا{سوال}
 نقطہ \(x= \frac{1}{4}\)  پر خط مماس \(y= \sqrt{x}\) اور  \(y= \frac{1}{\sqrt{x}}\) سے نقطہ \عددی{ P  } پر ملتے ہیں۔ \عددی{ P  }کے محدد معلوم کریں ۔
\انتہا{سوال}
\ابتدا{سوال}
 نقطہ \(x=2\) پر خط عمودی \(y=\frac{1}{x^2}\) اور \(y=\frac{1}{x^3}\)  سے  نقطہ \عددی{ Q } پر ملتے ہیں۔\عددی{ Q }کے محدد معلوم کریں ۔
\انتہا{سوال}
