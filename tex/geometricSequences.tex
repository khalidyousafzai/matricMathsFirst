\باب{ہندسی ترتیبات}\شناخت{باب_ہندسی_ترتیبات}
page 114\\
\((a) 1 \: 4 \:  9 \:  16 \;  25 \: ... \) \hspace{10pt} \( (b)\frac{1}{2} \:  \frac{2}{3} \:  \frac{3}{4} \:  \frac{4}{5} \:  \frac{5}{6} ... \) \\ \( (c) 99 \:  97 \:  95 \:  93 \:  91 \: ... \)\hspace{10pt} \((d) 1 \:  1.1 \:  1.21 \:  1.331 \:  1.4641 \: ...\)\\ \( (e) 2 \:  4 \:  8 \: 14 \:  22 \: ... \) \hspace{10pt}\((f) 3 \:  1 \:  4 \:  1 \:  5 \: ...\)\\
\\
کثر اوقات کسی ترتیب کو
\(u_{0}, u_{1}, u_{2}\dotsc\)
یعنی
r=0
سے شروع کرنا آسان رہتا ہے لیکن یہاں اس بات کا خیال کرنا ضروری ہے کے ہم پہلے جزو سے مراد
\(u_{0}\)
لے رہے ہیں یا
\(u_{1}\)
؟
a
اور
b
کے جزو کے لیے ہمیں کلیہ تھے میں کوئی مسئلہ نہیں
a
کے اجرا دراصل
\(1^{2}, 2^{2}, 3^{2}.4^{2}\)
ہیں لہذا اس کا جزو ہوگا
\[u_{r}=r^{2}\]
اسی طرح
b
کے اجرا کو
\(\frac{1}{1+1}, \frac{2}{2+1}, \frac{3}{3+1},\frac{4}{4+1}, \frac{5}{5+1}\)
لکھ جا سکتا ہے یعنی
\[u_{r}=\frac{r}{r+1}\]
اسی طرح
c
اور
d
میں قلیہ موجود ہے لیکن اسے حاصل کرنا آسان نہیں ہے لیکن ہم ہر اگلے جزو کو دیکھ کر بتا سکتے ہیں کہ پچھلا جزو کیسے حاصل کیا جا سکتا ہے . یعنی دو درجے پیچھے جا کہ مثلاً
\[u_{2}=u_{1}-2\]
\[u_{3}=u_{2}-2\]
\[u_{4}=u_{3}-2\]
وغیر . افی ایک کیلے میں ایسے سمویا جاسکتا ہے
\[u_{r+1}=u_{r}-2\]
میں ہر جزو کو 
1.1
سے ضرب دی گئی ہے
\[u_{r+1}=1.1u_{r}\]
به قسمتی سے بہت سی ترتیبات ایسی ہیں جو
\(u_{r+1}=u_{1}-2\)
کے کلہ پر پورا اترتی ہیں مثلاً
10, 8, 6 , 4 , 2
اور
-2, -4, -6, -8, -10 ...\\
\clearpage
صفحه 115 \\
یہ تعریف اس وقت تک مکمل نہیں ہوسکتی جب تک ہم پہلے جزو کے بارے میں معلوم نہ ہو لہذا ترتیب کو تعریف تو مکمل بنانے کے لیے ہمیں
c
اور
d
کو ایسے تلف ہو گا\\
(c)
\(u_{1}=99\)
اور
\(u_{r+1}=u_{r}-2\)\\
(d)
\(u_{1}=1\)
اور
\(u_{r+1}=1.1u_{r}\)\\
اس طرح کی تعریفوں کو استفرادی تعریفیں کہتے ہیں .
ترتیب
c
کی بنیاد جیومیٹری ہے . یہ ترتیب دراصل مختلف دائروں کے ذریعے کسی مسطح کو تقسیم کرنے والے خطوں کو سب سے زیاده تعداد بتائی ہے . ( ا, 2, 3, 4 .... دا ئروں کے ذریعے اپنی اشکال بنانے کی کوشش کیجیے ) .
آپ کے پاس کچھ ایس ترتیب بنے گی۔
\[u_{2}=u_{1}+2, u_{3}=u_{2}+4, u_{4}=u_{3}+6\]
وغیر ان میں ہونے والے اضافے 2 , 4, 6 ... کو
2r
سے ظاہر کیا جاسکتا ہے . لہذا استقراری تعریف یہ ہوگی
\(u_{1}=2\)
اور
\(U_{r+1}=u_{r}+2r\)\\
f
کے بارے میں یم کر سکتے ہیں کہ اگلے تین اعداد
1 6 1
ہونگے (کیونکہ جفت مقامات پہ آنے والے اعداد میں ہر بار 1 درجہ کا اضافہ ہوتا ہے . )
دراصل اس ترتیب کا نقطہ آغاز قدرے مختلف ہے جیسا کہ ہم دیکھ سکتے ہیں کہ اس کے پہلے پانچ اعداد
\(\pi\)
کی نشاندہی کر رہے ہیں . اس کا مطلب یہ ہوا کہ اگلے تین اعداد
9 2 6 
ہونگے \\
یہ مثال بتاتی ہے کہ بھی کسی ترتیب کا یکتا حل صرف چند ابتدائی اجراء کو دیکھ کر نہیں بتایا جا سکتا مثال کے طور پر دی گئی ترتیب کے لیے آٹھو اجزاء معلوم کیجیے .
\[u_{r}=r^{2}+(r-1)(r-2)(r-3)(r-4)(r-5)\]
آپ دیکھیں گے کہ پہلے پانچ اجراء میں ہیں جو کہ ترتیب
a
میں دیے گئے ہیں لیکن اگلے اجراء آپکے اندازے سے برعکس ہونگے \\
ایک ترتیب کو اسی وقت بیان کیا جاسکتا ہے جب ہمارے پاس اس کا کلیہ ہو استقراری تعریف ہو یا واضع اصول \\
مشق
8A\\
ذیل میں دی گئی تعریفیں استعمال کر کر تمام ترتیبات کے پہلے پانچ اجراء لکھے \\
\[(a)u_{1} = 7 ,u_{r+1} =\ u_{r} + 7 \hspace{10pt} (b)u_{1} = 13 , u_{r+1} = u_{r}-5 \]
\[(c) u_{1} = 4, u_{r+1} =\ 3u_{r}    \hspace{10pt} (d) u_{1} = 6 , u_{r+1} = \dfrac{1}{2}u_{r} +3 \]
\[(e) u_{1} = 2,u_{r+1} = 3u_{r}+ 1 \hspace{10pt} (f)\ u_{1} = 1  , u_{r+1} = u_{r}^{2} + 3 \]
ذیل میں دی گئی ترتیبات کے لیے استفزاری تعریفین لکھے \\
\[(a) 2 \: 4 \: 6 \:  8 \:  10 \: ... \hspace{10pt} (b) 11 \: 9 \:  7 \:  5 \:  3 \: ... \]
\[(c)\ 2 \ \ 6 \ \ 10 \ \ 14 \ \ 18  \hspace{46pt} (d) 2 \: 6 \: 18 \: 54 \: 162 \: ...\]
\[(e) \frac{1}{3} \: \frac{1}{9} \:  \frac{1}{27} \:  \frac{1}{81} \: ... \hspace{10pt} (f)\frac{1}{2}a\: \frac{1}{6}a \: \frac{1}{8}a  \: \frac{1}{16}a \: ...\]
\[(g)\ b -2c \ \ b - c \ \  b \ \ b + c \ . \ . \ . \ \hspace{10pt} (h)\ 1 \: -1 \: 1 \: -1 \: 1 \: ...\]
\[(i) \frac{p}{q^{3}} \: \frac{p}{q^{2}} \: \frac{p}{q} \: ... \hspace{10pt} (j) \frac{a^{3}}{b^{2}} \:  \frac{a^{2}}{b} \: a \:  b \: ... \]
\[(k)\ x^{3} \:  5x^{2} \:  25x \: ...\hspace{10pt} (l)  1 \:  1+x \:  (1 + x)^{2} \:  (1 + x)^{3} \]

page 116\\
ر ترتیب کے ابتدائی پانچ اجراء لکھیے اور استقراری تعریف دیں \\
\[(a) u_{r} = 2r + 3  \hspace{5pt} (b) u_{r} = r^{2} \hspace{5pt} (c) u_{r} = \frac{1}{2}r(r  +  1)  \]  
\[(d) u_{r} = \frac{1}{6}r(r+1)(2r+1)  \hspace{5pt} (e) u_{r} = 2 \times 3^{r}  \hspace{5pt} (f) u_{r} = 3 \times 5^{r-1}\]
ر ترتیب کے جزو کے لیے ممکنہ کلیہ لکھے \\
\(a) 9 \: 8 \: 7 \: 6 \: ... \hspace{10pt} (b) 6 \: 18 \: 54 \: 162 \: ... \)
\((c) 4 \: 7 \: 12 \: 19 \: ... \hspace{10pt} (d)\ 4 \: 12 \: 24 \: 40 \: 60 \: ...\)
\[(e)\frac{1}{4} \: \frac{3}{5} \: \frac{5}{6} \: \frac{7}{7} \: ...\hspace{10pt} (f)\ \frac{2}{2} \: \frac{5}{4} \: \frac{10}{8} \: \frac{17}{16} \: ...\]
مثلثی نمبروں کی ترتیب\\
تصویر 8.1 میں مثلثی انداز میں لگائے گئے کاٹے کے نشانات مثلثی اعداد ہیں۔ اگر
\(t_{r}\)
کسی مثلثی نمبر 𝑟 کو ظاہر کر رہا ہو تو ہم بعد میں آنے والی صفوں میں دیکھ سکتے ہیں کہ
\[t_{1}=1\]
اور عمومی طور پر
\[t_{r}=1+2+3+ ... + r\]
یہاں تین نقاط
3
اور
r
کے درمیان آنے والے تمام قدرتی اعداد کو ظاہر کرتے ہیں
تصویر 8.2 میں ایک علامتی مثلثی نمبر 
\(t_{r}\)
کو دکھایا گیا ہے(اسے r=9 کے لئے بنایا گیاہے لیکن ہم r کی کوئی بھی قیمت لے سکتے ہیں۔)
r
کی قیمت معلوم کرنے کا ایک آسان طریقہ یہ ہے کہ اسی طرح نقاط کی ایک شکل بنائی جائے اور اُسے کاٹے کے نشانات کے ساتھ الٹا کر لگا دیا جائے، جیسے تصویر 8.3 میں دکھایا گیا ہے۔ نقاط اور کاٹے کے نشانات ایک مستطیل بنائیں گے، جس کی چوڑائی
r
اور لمبائی 
r+1
ہو گی.
اس میں کُل ملا کر 
(r+1)
چیزیں ہوں گی، جن میں سے آدھے نقاط ہیں اور آدھے کاٹے کے نشانات۔ یعنی کاٹے کے نشانات ہوں گے:
\[t_{r}=\frac{1}{2}r(r+1)\]
اس کا مطلب یہ ہوا کہ:
سے
r
تک تمام قدرتی اعداد کا مجموعہ 
\(\frac{1}{2}r(r+1)\)
ے۔\\
\clearpage
page 117\\
ہم اس دلیل کو الجبرائی شکل میں بھی لکھ سکتے ہیں۔ اگر ہم اوپر سے نیچے کی طرف کاٹے کے نشانات کی گنتی کریں تو ہمیں حاصل ہو گا\\
\[ t_{r} =  1 +  2  +  3  +  ... + (r - 2) + (r - 1) + \: r ,\]
لیکن اگر ہم اوپر سے نیچے کی طرف نقاط کی تعداد گنیں تو ہمیں حاصل ہو گا\\
\[t_{r} = r \: + (r - 1) + (r - 2) + ...  + 3 + 2 + 1.\]
مستطیل میں موجود تمام چیزوں کا گننا ان دونوں مساواتوں کے حاصل جمع کے برابر ہو گا:\\
\[ 2t_{r} = (r + 1) + (r + 1) + (r + 1) + ...  + (r + 1) + (r + 1) + (r + 1) .\]
یعنی ہر 
r
صف کے لئے ایک 
\((r+1)\)
اس کا مطلب یہ ہوا کہ
\[2t_{r}=r(r+1)\]
\[t_{r}=\frac{1}{2}r(r+1)\]
ہم ترتیب 
\(t_{r}\)
کے لئے ایک استقرائی تعریف بھی وضع کر سکتے ہیں۔ تصویر 8.1 میں ہم دیکھ سکتے ہیں کہ کسی بھی اگلے مثلثی نمبر کو حاصل کرنے کے لئے کاٹے کے نشانات کی اگلی صف میں ایک کا اضافہ کر دیا جائے یعنی
\[t_{2}=t_{1}+2, t_{3}=t_{2}+3\]
عمومی طور پر لکھا جا سکتا ہے کہ
\[t_{r+1}=t_{r}+(r+1)\]
اس تعریف کو مکمل کرنے کی غرض سے ہم 
\(t_{1}=1\)
یا
\(t_{0}=0\)
کا انتخاب کر سکتے ہیں.
اگر ہم 
\(t_{0}=0\)
انتخاب کریں تو ہم 
\(t_{1}\)
کو
\(r=0\)
ے معلوم کر سکتے ہیں، جیسے کہ:\\

عدد ضربیہ کی ترتیب\\
اگر ہمیں 
\(t_{r}\)
کی تعریف میں ایک جزو سے دوسراجمع کی بجائے ضرب سے حاصل کریں تو ہمیں عدد ضربیہ کی ترتیب حاصل ہو گی۔\\
\[f_{r+1} = f_{r} \times (r+1)\quad r =0,1,2,3, .... \]
یہاں ایک اہم نقطہ 
\(f_{0}\)
و 0لینا ہے (سوچئے!) بجائے اس کے کہ ہم
\(f=0\)
کو 
1
لیں۔(یہ دیکھنے میں تھوڑا عجیب لگتا ہے لیکن اس کی توجیح اگلے سبق میں واضح ہو گی۔)\\
\[f_{1}\ = f_{o} \times 1 =\ 1 \times 1 =\ 1\ ,\ f_{2} =\ f_{1} \times 2 = 1 \times 2 =\ 2\ , f_{3} =\  f_{2} \times 3 =\ 2 \times 3 = 6 \]
اس طرح 
\(r\ge 1\)
والے کسی بھی عدد کے لئے لکھ سکتے ہیں کہ\\
\[f_{r}\ =\ 1 \times 2 \times 3 \times  ... \times r. \]
یہ اتنی اہم ترتیب ہے کہ اس کی اپنی مخصوص علامت ہے
\(r!\)
جسے
r
ضربیہ پڑھتے ہیں۔\\
r
ضربیہ کی تعریف 
\(0!=1\)
اور
\((r+1)!=r!\times (r+1)\)
سے کی جاتی ہے۔\\
\(r\ge 1\)
کے لئے 
r!
سے مراد 
1
اور
r
کے درمیان آنے والے تمام قدرتی اعداد کا حاصل ضرب ہے۔\\
\clearpage
page 118 \\
بہت سے کیلکولیٹرز میں اس کے لئے ایک مخصوص کلید  ہو تی ہے۔
 n
  کی چھوٹی قیمتوں کے لئے، ہمیں بالکل درست جواب ملتا ہے لیکن
 n=14
    سے بڑے اعداد کے لئے ہمیں صرف ایک تخمینہ ملتا ہے۔\\
    پاسکل کی ترتیب\\
    ضرب کے اصول پہ مبنی ایک اور اہم ترتیب پاسکل کی ترتیب ہے۔  آپ اگلے سبق میں دیکھیں گے کہ یہ ترتیب 
\((x+y)^{n}\)
     جیسی تراکیب کی توسیع میں استعمال ہوتی ہیں۔ ایک علامتی مثال کی استقرائی تعریف یہ ہو گی:\\
     \( p_{0} =1\)
     اور
     \( p_{r+1} = \frac{4 - r}{r + 1}p_{r}\)\\
     یہ تعریف
 r=0,1,2…
       کے لئے مطلوبہ اجزاء دیتی ہے\\
       \[ p_{1} = \frac{4}{1} p_{o} = 4 , \: p_{2} = \frac{3}{2} p_{1} = 6 ,\: p_{3}= \frac{2}{3}p_{2} = 4 \]
\[ p_{4} = \frac{1}{4}p_{3} = 1 \: , p_{5} = \frac{0}{5}p_{4} = 0 , \: p_{6} =\frac{(-1)}{5}p_{5} = 0,\]
       آپ دیکھ سکتے ہیں کہ ایک خاص جگہ آ کے ترتیب میں ایک جزو 0 کے برابر آ جاتا ہے اور اس کی وجہ سے اگلے آنے والے تمام اجزاء بھی 0 کے برابر ہوں گے۔ لہٰذا مکمل ترتیب کچھ یوں بنے گی\\
 \[1 , 4 , 6 , 4 , 1 , 0 , 0 , 0 , 0 , 0 , ....\]
 یہ پا کی ترتیب کی صرف ایک سکل نسل ہے اور اس نسل کی اپنی خاصل علامت
 
 \(\begin{pmatrix} 4 \\ r \end{pmatrix}\)
 ہے
 ۔  مثال کے طور پر
   \(\begin{pmatrix} 4 \\ 0 \end{pmatrix}=1 \begin{pmatrix} 4 \\ 1 \end{pmatrix} =4 \begin{pmatrix} 4 \\ 2\end{pmatrix}=6\)
 وغیرہ۔ پاسکل کی دیگر تراتیب میں 4 کی جگہ مختلف اعداد ہوں گے۔\\

 پاسکل کی ترتیب کی عمومی تعریف یہ ہے:\\
 \[\begin{pmatrix} n \\ 0 \end{pmatrix}=1 \begin{pmatrix} n \\ r +1 \end{pmatrix} = \frac{n-r}{r+1}\begin{pmatrix} n \\ r \end{pmatrix}  r=0 ,1 ,2,3, ... \]
 آپ
 \(n=0,1,2,3\)
 کے لئے پاسکل کی ترتیب خود بنائیے۔\\
 \[ n\ = 0,\: 1,\: 0,\: 0,\: 0,\: 0, ...  \]
\[ n\ = 1,\: 1,\: 1,\: 0,\: 0,\: 0,  ... \]
\[ n\ = 2,\: 1,\: 2,\: 1,\: 0,\: 0, ...  \]
\[ n\ = 3,\: 1,\: 3,\: 3,\: 1,\: 0,  ... \]
 صفر کے اعداد کو شمار کئے بغیر پاسکل کی ترتیب کو پاسکل کی مثلث کہا جاتا ہے۔ اس کا سب سے پہلا تاریخی حوالہ چین میں ملتا ہے لیکن یورپ میں بلیز پاسکل (سترہویں صدی کا ریاضی دان اور نظریۂ احتمال کے بانیان میں سے ایک) کو یورپ میں اس ترتیب کی اشاعت و ترویج کا سبب مانا جاتا ہے۔ اسے عام طور پر مساوی الثاقین مثلث کی صورت میں ظاہر کیا جاتا ہے، جیسے تصویر 8.4 میں دکھایا گیا ہے۔ لیکن تصویر 8.5 میں دکھائی گئی اس ترتیب کی الجبرائی شکل زیادہ عام فہم ہے کہ ہر قطار r کی کسی خاص قیمت کی نشاندہی کرتی ہے۔\\
 \clearpage
 page 119\\
 حیران کُن طور پر ماسوائے 1 کے، تصویر 8.4 میں ہر عدد اپنے سے اوپر والی صف میں موجود دو قریب ترین اعداد کا مجموعہ ہے۔
آپ نے یہ اعداد پہلے بھی دیکھے ہیں۔ فصل 8.1 کی ترتیب d دیکھیے۔\\
مشق 8ب\\
1.
تصویر 8.3 کو بطور مثال سامنے رکھیں\\
کاٹے کے نشانات کی ایک تصویر بنائیے جو r مثلثی نمبر
 \(t_{r}\)
  کو ظاہر کرے۔\\
نقاط کو استعمال کرتے ہوئے ایک اور تصویر بنائیے جو
 \(t_{r-1}\)
  کو ظاہر کرے۔\\
ان دونوں کو تصاویر کو جوڑیں اور ثابت کریں کہ\\
\[ t_{r} + t_{r-1} = r^{2}\]
ذیل میں دیے گئے کلیہ کو استعمال کرتے ہوئے جزو c کو الجبرائی طریقے سے ثابت کریں\\
\[t_{r}= \frac{1}{2}r(r+ 1)\]
2.
\(r\ge1\)
والے تمام اعداد کے لئے rکی صورت میں 
\(t_{r}-t_{r-1}\)
ے لئے ترکیب بنائیں۔\\
جزو 
2(a)
اور سوال 1 
(c)
کو استعمال کرتے ہوئے ثابت کریں کہ\\

\[ 1^{3} , 2^{3} , 3^{3} , ... , n^{3}\]
جزو b کو استعمال کرتے ہوئے تراکیب کو مثلثی اعداد کی صورت میں لکھیں اور ثابت کریں کہ\\
\[ 1^{3} + 2^{3} + 3^{3}\ + ... + n^{3}\ = \frac{1}{4}n^{2}(n + 1)^{2} \]
3.
کیلکولیٹر کو استعمال کئے بغیر مندرجہ ذیل کی قیمتیں معلوم کریں۔\\

\[(a) 7! \hspace{5pt}  (b)\frac{8!}{3!}  \hspace{5pt}  (c)\ \frac{7!}{4! \times 3!} \]
4.
مندرجہ ذیل کو عدد ضربیہ کی صورت میں لکھیں۔\\
\[(a)\ 8\times 7\times 6\times 5 \hspace{5pt} (b)9\times 10\times 11\times 12 \hspace{5pt} (c) n(n-1)(n-2)\]
\[(d)\ n(n^{2} -1) \hspace{5pt} (e) n(n + 1)(n + 2)(n + 3) \hspace{5pt} (f)(n+6)(n+5)(n+4)\]
\[(g) 8\times 7! \hspace{5pt} (h)n\times  (n -1)!\]
5حل کریں۔\\
\[(a) \frac{12!}{11!} \hspace{5pt} (b)23! - 22! \hspace{5pt} (c) \frac{(n+1)!}{n!} \hspace{5pt} (d) (n + 1)! - n! \]
7.ثابت کریں۔\\
\[\frac{(2n)!}{n!}  = 2^{n}(1\times 3\times 5\times ... \times (2n\ -\ 1))\]
فصل 8.4 میں دی گئی استقرائی تعریف کو استعمال کرتے ہوئے مندرجہ ذیل پاسکل کی تراکیب لکھیں۔\\
\[(a) n = 5,\hspace{8ptpt} (b)n = 6, \hspace{8pt} (c)n = 8\]
\clearpage
page 120\\
\( \begin{pmatrix}n \\ r \end{pmatrix}   \)
کی استقرائی تعریف کو استعمال کرتے ہوئے ثابت کریں\\
\[\begin{pmatrix} 9 \\6\end{pmatrix} =\frac{9\times 8\times 7}{1\times 2\times 3}\]
اسی طریقے کو استعمال کرتے ہوئے مندرجہ ذیل کو بھی عدد ضربیہ کی صورت میں لکھیں۔\\
\[(a) \begin{pmatrix}11 \\ 4\end{pmatrix} \hspace{4pt}
(b) \begin{pmatrix}11 \\ 7\end{pmatrix}  \hspace{4pt}
(c) \begin{pmatrix}10 \\ 5 \end{pmatrix} \hspace{4pt}
(d) \begin{pmatrix}12 \\ 3\end{pmatrix} \hspace{4pt} 
(e) \begin{pmatrix}12 \\ 9 \end{pmatrix} \]
9.	سوال 
8 کے جوابات ایک تعمیمی نتیجے کی تجویز دیتے ہیں۔
\(\begin{pmatrix}n \\ r\end{pmatrix} =\frac{n!}{r! \times (n  - r)!}\)
 فرض کریں کہ یہ نتیجہ درست ہے اور ثابت کریں\\
 \[\begin{pmatrix}n \\ r\end{pmatrix} = \begin{pmatrix}n \\ n - r\end{pmatrix}\]
10.	حل کریں\\
\[(a) \begin{pmatrix}6 \\ 3\end{pmatrix} + \begin{pmatrix}6 \\ 4\end{pmatrix} = \begin{pmatrix}7 \\ 4\end{pmatrix} \hspace{8pt} (b)\begin{pmatrix}8 \\ 5\end{pmatrix} + \begin{pmatrix}8 \\ 6\end{pmatrix} =\begin{pmatrix}9 \\ 6 \end{pmatrix}\]
ان جوابات کی بنیاد پر ایک تعمیمی ترکیب لکھیے۔\\
n=2.11
ے لئے پاسک کی ترتیب ہے\\
اس ترکیب کے اجزاء کا مجموعہ 4 کے برابر ہے۔\\
n
کی دیگر قیمتوں کے لئے پاسکل کی تراتیب لکھیں اور اُن کے مجموعے معلوم کریں.\\
حسابی ترتیب\\
حسابی ترتیب یا حسابی تساعد ایک ایسی ترتیب ہے، جس کے اجزاء مستقل قدم کی صورت بڑھتے یا گھٹتے ہیں۔ مثال کے طور پر فصل 8.1 کی ترتیبc دیکھیں۔ حسابی ترتیب کی استقرائی تعریف یہ ہو گی:\\
\[ u_{1} \ = a, \;  u_{r+1}  = u_{r} + d\]
عدد d کو مشترکہ فرق کہا جاتا ہے۔ ترتیب cکا پہلاجزوa=99 جبکہ مشترکہ فرق d= -2 ہے\\
مثال 
8.5.1\\
سیما اگلے دس سالوں کے لئے ایک مخصوص رقم چندے میں دینا چاہتی ہے۔ اُس ے پہلے سال سو روپے اور پھر ہر آنے والے سال میں بیس روپے کا اضافہ کرنے کا فیصلہ کیا ہے۔اس حساب سے وہ اپنے آخری سال میں کتنی رقم چندے میں دے گی؟ اور اُن دس سالوں میں چندے میں دی جانے والی مکمل رقم کتنی ہو گی؟\\
اگرچہ سیما نے رقم 10 بار دینی ہے لیکن رقم میں اضافہ صرف 9 بار ہو گا۔ لہٰذا سیما آخری سال میں\\
دے گی۔\\
\[(100\ +\ 9 \times \ 20) = 280\]
اور چندے میں دی جانے والی مکمل رقم  اگر S ہو تو\\
\[S = 100 + 120 + 140 + ... + 240 + 260 + 280\]
چونکہ یہاں صرف 10 اجزاء ہیں، اس لئے انہیں جمع کرنا آسان ہے لیکن ہم ایک ایسا طریقہ معلوم کریں گے، جس سے ہم کسی بھی حسابی ترتیب کے اعداد کا مجموعہ حاصل کر سکتے ہیں۔ اوپر دیے گئے اعداد کو الٹا کر لکھیے\\
\clearpage
page 121\\
\[S=280+260+240+\dotsc+140+120+100.\]
دونوں مساواتوں کو جمع کیجے\\
\[2S=380+380+380+\dotsc+380+380+380,\]
یہاں عدد 380 دس بار آیا ہے۔ لہٰذا\\
\(2S=380\times10=3800\)
سیما دس سالوں میں چندے میں 1900 روپے دے گی۔\\
ان حسابات کو تصاویر 8.2 اور 8.3 کی مدد سے سمجھایا گیا ہے۔ سیما کی طرف سے دی جانے والی رقم تصویر 8.6 کی پہلی صف میں دیکھی جا سکتی ہے۔ (کاٹے کے ہر نشان کی قیمت 20 روپے ہے۔) تصویر 8.7 میں نقاط کی مدد سے ایک اور نقل تیار کر کے اُسے کاٹے کے نشانات کے ساتھ رکھا گیا ہے۔ اس طرح 10 صفیں ہیں اور ہر صف میں 19 نقاط اور کاٹے کے نشانات ہے۔ ہر صف کی قیمت 380 روپے ہے۔
مثال 8.5.1 میں حسابی ترتیب کے دو مندرجہ ذیل خصائص دیکھے جا سکتے ہیں۔\\
حسابی ترتیب میں صرف متناہی اجزاء ہی ہوتے ہیں۔\\
حسابی اوسط کے تمام اجزاء کا مجموعہ معلوم کرنا کئی حوالوں سے دلچسپی کا باعث ہے۔ اس طرح کی حسابی ترتیب کو حسابی تسلسل بھی کہا جاتا ہے۔\\
مثال 8.5.1 کے تمام اجزاء حسابی ترتیب بناتے ہیں لیکن اگر انہیں جمع کیا جائے\\
\[100+120+140+\dotsc+240+260+280\]
تو یہ حسابی تسلسل کہلائے گا۔\\
عمومی طور پر، حسابی ترتیب \\
\[a,a+d,a+2d,a+3d,\dotsc\]
میں n اجزاء ہوتے ہیں اور پہلے جزو سے n-1 جزو تک تمام اجزاء کا مشترک فرق dہوتا ہے۔ اگر ہم آخری جزو کو l سے ظاہر کریں تو ہم لکھ سکتے ہیں\\
\[l=a+(n-1)d.\]
اس کلیہ کی مدد سے ہم a,l,n,d میں سے کوئی بھی مقدار معلوم کر سکتے ہیں۔\\

فرض کرتے ہیں کہ ان تمام اجزاء کا مجموعہ Sکے برابر ہے۔  سو  a,l,n,d کی صورت میں S کا کلیہ معلوم کیا جا سکتا ہے۔\\
\clearpage
page 122\\
طریقہ نمبر 1
یہ مثال 8.5.1 میں دی گئی دلیل کی تعمیمی صورت ہے۔ تسلسل کو لکھا جا سکتا ہے
\[S=\quad a \quad+(a+d)+(a+2d)+\dotsc+(l-2d)+(l-d)+l.\]
اب اسے الٹا کر لکھیے
\[S=\quad l \quad +(l-d)+(l-2d)+\dotsc+(a+2d)+(a+d)+a.\]
دونوں مساواتوں کو جمع کرتے ہیں۔
\[2S=(a+l)+(a+l)+(a+l)+\dotsc+(a+l)+(a+l)+(a+l),\]
ہم دیکھ سکتے ہیں کہa+l کا جزو n دفعہ آ رہا ہے لہٰذا
\(2S=n(a+l),\),
\(S=\frac{1}{2}n(a+1).\)\\
طریقہ نمبر 2
اس طریقے میں ہم فصل 8.2 میں دیے گئے مثلثی اعداد کے کلیے کو استعمال کریں گے۔ حسابی تسلسل یہ ہے
\[2S=a+(a+d)+(a+2d)+\dotsc+(a+(n-1)d)\]
a اور d والے تمام اجزاء کو علیحدہ علیحدہ جمع کرنے سے
\[S=(a+a+\dotsc+a)+(1+2+3+\dotsc+(n-1))\]
پہلی قوسین میں a دراصل n دفعہ آ رہا ہے۔ دوسری قوسین 1 سے n-1 تک آنے والی تمام قدرتی اعداد کا مجموعہ ہے۔ مثلثی کے اعداد کے کلیہ کو استعمال کرتے ہوئے ہم لکھ سکتے ہیں:
\[t_{n-1}=\frac{1}{2}(n-1)((n-1)+1)=\frac{1}{2}(n-1)n\]
یہاں r برابر ہے n-1 کے۔

\[S=na+\frac{1}{2}(n-1)nd=\frac{1}{2}(2a+(n-1)d)\]
یہ وہی جواب ہے جو ہمیں طریقہ نمبر 1 سے ملا ہے۔
حسابی ترتیب کے تمام نتائج کا خلاصہ یہ ہے۔

ایک حسابی ترتیب کے n اجزاء ہوتے ہیں، جس میں پہلے جزو کو a اور ان کے مشترکہ فرق کو d کہتے ہیں
\(l=a+(n-1)d\)
اور ان کا مجموعہ
\[S=\frac{1}{2}n(a+l)=\frac{1}{2}n(2a+(n-1)d)\]
ہوتا ہے۔
 \clearpage
 page 123\\
 ال 8.5.2
پہلے nطاق قدرتی اعداد کا مجموعہ معلوم کیجے۔
طریقہ نمبر 1
طاق اعداد کی حسابی ترتیب میں پہلا جزو a=1 ہے اور مشترکہ فرق d=2 ہے۔ لہٰذا

\[S=\frac{1}{2}n(2a+(n-1)d)=\frac{1}{2}n(2+(n-1)2)=\frac{1}{2}n(2n)=n^{2}\]
طریقہ نمبر 2
1 سے 2n تک تمام قدرتی نمبر لکھیں۔ اب n جفت اعداد یعنی 2,4,6,…,2n کو مٹا دیں۔ ہمارے پاس صرف n طاق اعداد باقی بچیں گے۔
1 سے 2n تک اعداد کا مجموعہ 
\(t_{r}\)
 ہے۔ یہاں rبرابر ہے 2nکے۔ سو
\[t_{2n}=\frac{1}{2}(2n)(2n+1)=n(2n+1)\]
nجفت اعداد کا مجموعہ ہو گا:
\[2+4+6+\dotsc+2n=2(1+2+3+\dotsc+n)=2t_{n}=n(n+1)\]
پہلے nطاق اعداد کا مجموعہ ہو گا
\[n(2n+1)-n(n+1)=n((2n+1)-(n+1))=n(n)=n^{2}\]
طریقہ نمبر 3
تصویر 8.8 میں ایک مربع بنایا گیا ہے جس میں nصفیں ہیں اور ہر صف میں nکاٹے کے نشانات ہیں۔ (یہ مربع n=7 کے لئے بنایا گیا ہے۔)آپ نقاط کے ذریعے ظاہر کئے گئے ہر Lمیں کاٹے کے نشانات گن سکتے ہیں، لہٰذا
\(n^{2}=1+3+5+\dotsc\)\\
مثال 8.5.3
\\
ایک طالب علم 426 صفحات کی کتاب پڑھتا ہے۔ جیسے جیسے وہ کتاب پڑھتا جاتا ہے، دلچسی کی وجہ سے اُس کی پڑھنے کی رفتار بڑھتی جاتی ہے۔ پہلے دن وہ 19 صفحات پڑھتا ہے اور ہر نئے دن 3 صفحات کا مزید اضافہ کرتا جاتا ہے۔ اُسے کتاب ختم کرنے میں کتنے لگیں گے؟
\(a=19,d=3\).
\(S=426\),
\(S=\frac{1}{2}n(2a+(n-1)d),\)
ہمیں دیا گیا ہے

\(S=\frac{1}{2}n(2a+(n-1)d),\)
\[426=\frac{1}{2}n(38+(n-1)3),\]
\[852=n(3n+35),\]
\[3n^{2}+35n-825=0.\]
دو درجی الجبرائی کلیہ کا اطلاق کرنے سے
\[n=\frac{-35\pm\sqrt{35^{2}-4\times3\times9-(852)}}{2\times3}=\frac{-35\pm107}{6}\]
چونکہ n ہمیشہ مثبت ہو گا۔ لہٰذا
\(n=\frac{-35+107}{6}=\frac{72}{6}=12\)
وہ کتاب کو 12 دن میں کرے گا۔
 \clearpage
 page 124\\
 1.	مندرجہ ذیل میں سے کون سی تراتیب کسی حسابی ترتیب کے پہلے چار اجزاء کو ظاہر کرتی ہیں؟ اُن تراتیب کا مشترکہ فرق بھی لکھیں۔\\
 \[7\quad10\quad13\quad16...\]
\[3\quad5\quad9\quad15...\]
\[1\quad0.1\quad0.01\quad0.001...\]
\[4\quad2\quad0\quad-2...\]
\[2\quad-3\quad4\quad-5...\]
\[p-2q\quad p-q\quad p\quad p+q...\]
\[\frac{1}{2}a\quad \frac{1}{3}a\quad \frac{1}{4}a\quad \frac{1}{5}a...\]
\[x\quad2x\quad3x\quad4x...\]
2.	مندرجہ ذیل حسابی تراتیب کا چھٹا جزو بتائیں اور rجزو کے لئے ترکیب بھی لکھیں۔\\
\[2\quad4\quad6...\]
\[17\quad20\quad23...\]
\[5\quad2\quad-1...\]
\[1.3\quad1.7\quad2.1...\]
\[1\quad1\frac{1}{2}\quad2...\]
\[73\quad67\quad61...\]
\[x\quad x+2\quad x+4...\]
\[1-x\quad1\quad1+x...\]
3.	مندرجہ ذیل حسابی تراتیب میں پہلے تین اور آخری جزو دیے گئے ہیں۔ اجزاء کی تعداد معلوم کریں۔\\
\[4\quad5\quad6\quad...\quad17\]
\[3\quad9\quad15\quad...\quad525\]
\[8\quad2\quad-4\quad...\quad-202\]
\[2\frac{1}{8}\quad3\frac{1}{4}\quad4\frac{3}{8}\quad...\quad13\frac{3}{8}\]
\[3x\quad7x\quad11x\quad...\quad43x\]
\[-3\quad-1\frac{1}{2}\quad0\quad...\quad12\]
\[\frac{1}{6}\quad\frac{1}{3}\quad\frac{1}{2}\quad...\quad2\frac{2}{3}\]
\[1-2x\quad1-x\quad1\quad...\quad1+25x\]
4.	مندرجہ ذیل حسابی تراتیب میں دیے گئے اجزاء کا مجموعہ معلوم کریں۔\\
\[2+5+8+...\]
\(20\)
\[4+11+18+...\]
\(15\)
\[8+5+2+...\]
\(12\)
\[\frac{1}{2}+1+1\frac{1}{2}+...\]
\(58\)
\[7+3+(-1)+...\]
\(25\)
\[1+3+5+...\]
\(999\)
\[a+5a+9a+...\]
\(40\)
\[-3p-6p-9p-...\]
5.	مندرجہ ذیل حسابی تراتیب میں اجزاء کی تعداد معلوم کریں اور اُن کا مجموعہ لکھیں۔\\
\[5+7+9+...+111\]
\[8+12+16...+84\]
\[7+13+19+...+277\]
\[8+5+2+...+(-73)\]
\[-14-10-6-...+94\]
\[157+160+163+...+529\]
\[10+20+30+...+10000\]
\[1.8+1.2+0.6+...+(-34.2)\]
6.	مندرجہ ذیل حسابی تراتیب میں آپ کو دو اجزاء دیے گئے ہیں۔ پہلا جزو اور مشترکہ فرق معلوم کریں۔\\
\(4th\)
\(=15\)
\(9th\)
\(=35\)
\(3rd\)
\(=12\)
\(10th\)
\(=47\)
\(8th\)
\(=3.5\)
\(13th\)
\(=5.0\)
\(5th\)
\(=2\)
\(11th\)
\(=-13\)
\(12th\)
\(=-8\)
\(20th\)
\(=-32\)
\(3rd\)
\(=-3\)
\(7th\)
\(=5\)
\(2nd\)
\(=2x\)
\(11th\)
\(=-7x\)
\(3rd\)
\(=2p+7\)
\(7th\)
\(=4p+19\)
7.	معلوم کریں کہ مندرجہ ذیل حسابی تسلسل میں کتنے اجزاء کا مجموعہ دیے گئے حاصل جمع کے برابر ہو گا۔\\
\(3+7+11+\dotsc\)
\(=820\)
\(8+9+10+\dotsc\)
\(=162\)
\(20+23+26+\dotsc\)
\(=680\)
\(27+23+19+\dotsc\)
\(=-2040\)
\(1.1+1.3+1.5+\dotsc\)
\(=1017.6\)
\(-11-4+3+\dotsc\)
\(=2338\)
\clearpage
page125\\
8.	ایک گلہری اخروٹ اکٹھے کرتی ہے۔ مہینے کے پہلے دن اُسے 5 اخروٹ ملتے ہیں، دوسرے دن 8 اور تیسرے دن 11 اور اس طرح ایک حسابی تساعد چلتا ہے۔
a.	بیسویں دن اُسے کتنے اخروٹ ملیں گے؟
b.	کتنے دنوں کے بعد گلہری کے پاس 1000 سے زیادہ اخروٹ ہوں گے؟
9.	کلثوم نے گاڑی خریدنے کے لئے بغیر سود کے قرضہ لیا۔وہ غیر مساوی اقساط کی صورت میں قرض کی ادائیگی کرتی ہے۔ پہلے مہینے وہ تیس ڈالر ادا کرتی ہے اور پھر ہر آنے والے مہینے میں مزید دو ڈالر کا اضافہ کرتی ہے۔ اُسے 24 اقساط ادا کیں۔
a.	آخری قسط کی رقم معلوم کریں۔
b.	قرضے کی مکمل رقم معلوم کریں۔
10.	
a.	1 سے 100 تک تمام قدرتی اعداد کا مجموعہ معلوم کریں۔
b.	101 سے 200 تک تمام قدرتی اعداد کا مجموعہ معلوم کریں۔
c.	n+1 سے 2n تک قدرتی اعداد کے لئے ترکیب معلوم کریں اور اُس حل کریں۔
11.	ایک ملازم یکم جنوری 2000 ء کو کام کا آغاز کرتا ہے اور اُس کی سالانہ تنخواہ 30000 ڈالر ہے۔ یکم جنوری 2015ء تک اُس کی تنخواہ میں ہر سال 800 ڈالر کا اضافہ ہوتا ہے۔ وہ اسی تنخواہ پر 31 دسمبر2040ء تک کام کرتا ہے۔ اُس نے اپنی مکمل پیشہ وارانہ زندگی میں کتنی رقم کمائی؟
1.	ایک ترتیب کی استقرائی تعریف یہ ہے:
\(\mathcal {U}_{r+1}=3\mathcal{U}_r-1\)
a.
اس ترتیب کے پہلے پانچ اجزاء معلوم کریں اگر
\(c=1\)
\(c=2\)
\(c=0\)
\(c=\frac{1}{2}\)
b.
ثابت کریں کہ c کی ہر قیمت کے لئے حصہ (a)میں دیے گئے اجزاء کو
\(\mathcal{U}_r=\frac{1}{2}+b\times3^r\)
  لکھا جا سکتا ہے۔
c.
ثابت کریں اگر r کی کسی قیمت کے لئے
\(\mathcal{U}_r=\frac{1}{2}+b\times3^r\)
تو
\(\mathcal{U}_{r+1}=\frac{1}{2}+b\times3^{r+1}\)
2.	مندرجہ ذیل ترتیب کے لئے u4 کی قیمت معلوم کریں۔\\
\(\mathcal{U}_1=0\)
\(\mathcal{U}_{r+1}=(2+\mathcal{U}_r)^2\)\\
3.	ایک ترتیب کی تعریف ایسے کی گئی ہے\\
\(\mathcal{U}_{n+1}=\sqrt{({4}-{\mathcal{U}_n})^2}\)\\
یہاں  u1کوئی بھی حقیقی عدد ہے۔\\
a.	اگر u1=1 ہو تو u2,u3,u4 معلوم کریں اور ترتیب کے رویے کی بابت بتائیں۔
b.	اگر u1=6 ہو تو ترتیب کا رویہ کیا ہو گا؟
c.	U1 کی کس قیمت کے لئے ترتیب کے تمام اجزاء ایک دوسرے کے برابر ہوں گے؟
4.	ایک ترتیب کی تعریف ایسے کی گئی ہے
\(\mathcal{U}_{n+1}={\mathcal{U}_n}^2 -1\)\\
یہاں  u1کوئی بھی حقیقی عدد ہے۔
a.	ترتیب کے رویے کی بابت بتائیں اگر u=0, u=1 اور u=2.
b.	اگر u2=u1 ہو تو u1 کی دو ممکنہ قیمتیں معلوم کریں۔
c.	اگر u3=u1 ہو تو ثابت کریں کہ
\({\mathcal{U}_1}^4-2{\mathcal{U}_1}^2-\mathcal{U}_1=0\)
\clearpage
page126
	حسابی تساعد کا rجزو 1+4r کے برابر ہے۔ n کی صورت میں اس تساعد کا مجموعہ معلوم کریں۔
	ایک حسابی تساعد کے پہلے دو اجزاء کا مجموعہ 18 اور پہلے چار اجزاء کا مجموعہ 52 ہے۔ پہلے آٹھ اجزاء کا مجموعہ معلوم کریں۔
	ایک حسابی تساعد کے پہلے 20 اجزاء کا مجموعہ 50 اور اگلے 20 اجزاء کا مجموعہ منفی 50 ہے۔اس تساعد کے پہلے 100 اجزاء کا مجموعہ معلوم کریں۔
	ایک حسابی تساعد کا پہلا جزو a اور مشترک فرق منفی 1 ہے۔ پہلے nاجزاء کا مجموعہ پہلے 3n اجزاء کے مجموعے کے برابر ہے۔ a کو n کی صورت میں لکھیے۔
	مندرجہ ذیل حسابی تساعد کا مجموعہ معلوم کریں۔
اس تساعد کا ہر تیسرا جزو مٹا دیں یعنی
اب باقی اجزاء کا مجموعہ معلوم کریں۔
	ایک حسابی تساعد، جس کا پہلا جزو a اور مشترک فرق d ہے، کا مجموعہ T ہے۔ پہلے 50 طاق اعداد کا مجموعہ 1/2 T-1000 ہے۔ d کی قیمت معلوم کریں۔
	ذیل میں دی گئی ترتیب میں ہر اگلا نمبر پچھلے سے 0.1 درجے بڑا ہے۔
	اس ترتیب میں کُل کتنے اجزاء ہیں؟
	تمام اجزاء کا مجموعہ معلوم کریں۔
	ذیل میں ایک ترتیب دی گئی ہے۔
	U3 کی قیمت معلوم کریں۔
	\(  u_{(n+1)}-u_{n}\)
	   کو n کی صورت میں ظاہر کریں اور حل کریں۔
	اس ترتیب کے اجزاء کا مشترک فرق ایک حسابی ترتیب بناتا ہے۔ اس حسابی تساعد کے لئے پہلا جزو، مشترک فرق اور پہلے 1000 اجزاء کا مجموعہ معلوم کریں۔
	ایک کھلونے بنانے والی کمپنی اپنی پیداواری صلاحیت میں اضافہ کے لئے ہر ہفتہ 8 کھلونے زیادہ بناتی ہے تاوقتیکہ کہ پیداواری صلاحیت 1000 کھلونوں تک بڑھ سکے۔  پہلے ہفتے 280 کھلونے بنائے جاتے ہیں، دوسرے ہفتے 288 اور اسی طرح یہ سلسلہ چلتا رہتا ہے۔ ثابت کریں 91 ویں ہفتہ میں 1000 کھلونے بنائے جائیں گے۔
	سن 1971ء میں تعمیر کردہ ایک مکان 999 سال کے ٹھیکے پر دیا جاتا ہے۔ اس معاہدے میں سالانہ کرایہ بھی شامل ہے۔ ٹھیکے کے پہلے 21 سالوں کے لئے سالانہ کرایہ 28 یورو طے پاتا ہے جسے اگلے 21 سالوں کے لئے 14 یورو سے 42 یورو تک بڑھایا جائے گا اور اُس کے بعد ہر 21 سال کے دورانیے کے اختتام پر 14 یورو کا اضافہ ہو گا۔
	اگر ٹھیکہ 999 سال تک چلتا ہے تو اس عرصے میں 21 سال کے کُل کتنے دورانیے آئیں گے اور بقیہ کتنے سال بچ جائیں گے؟
	21 سال کے تمام دورانیوں میں ادا کیا گیا کُل کرایہ معلوم کریں۔
 
\clearpage
page 127
15.	ایک حسابی تساعد کا پہلا جزو a اور مشترک فرق 10 ہے۔ اس تساعد کے پہلے n اجزاء کا مجموعہ 10000 ہے۔a کو n کی صورت میں ظاہر کریں اور ثابت کریں کہ اس تساعد کا n جزو ہو گا:
\[\frac{10000}{n}+5(n-1)\]
اگر n جزو 500 سے کم ہو تو ثابت کریں
\(n^{2}-101n+2000<0\)
	 اور nکی سب سے بڑی قیمت معلوم کریں۔\\
16.	تین تراتیب کی استقرائی لحاظ سے تعریف کی گئی ہے:\\
\(u_{0}=0\) 
\(u_{r+1}=u_{r}+(2r+1)\)
\(u_{0}=0\)
\(u_{1}=1\)
\(u_{r+1}=2u_{r}-u_{r-1}\)
\(r\ge 1\)
ہر ترتیب کے پہلے چند اجزاء معلوم کریں اور
\(u_{r}\)
 کا کلیہ بتائیں۔ معلوم کریں کہ جو کلیہ آپ نے اخذ کیا ہے، وہ ترتیب کے اجزاء کو درست طور سے معلوم کرتا ہے۔
17.	پاسکل کی ترتیب کو استعمال کرتے ہوئے ایک نئی ترتیب Fn بنائی گئی ہے۔\\
\(
\begin{pmatrix}
0\\
0
\end{pmatrix}\)
\(F_{1}=
\begin{pmatrix}
1\\
0\\
\end{pmatrix}
+
\begin{pmatrix}
0\\
1\\
\end{pmatrix}\)
\(F_{2}=
\begin{pmatrix}
2\\
0
\end{pmatrix}
+
\begin{pmatrix}
1\\
1\\
\end{pmatrix}
+
\begin{pmatrix}
0\\
2\\
\end{pmatrix}\)
\[F_{n}=
\begin{pmatrix}
n\\
0\\

\end{pmatrix}
+
\begin{pmatrix}
n-1\\
1\\
\end{pmatrix}
+\dotsc +
\begin{pmatrix}
1
n-1\\
\end{pmatrix}
+
\begin{pmatrix}
0\\
n
\end{pmatrix}\] 
ثابت کریں کہ تصویر 8.5 میں وتری سطروں کے ساتھ ساتھ دیے گئے اعداد کو جمع کرنے سے
 \(F_{n}\)
  کی ترتیب حاصل کی جا سکتی ہے۔ nکی چھوٹی قیمتوں کے لئے ثابت کریں کہ 	۔ (اسے فیبوناچی ترتیب کہتے ہیں۔ فیبو ناچی سے سن 1200ء کے قریب اٹلی کو عرب دنیا کے الجبرائی طریقوں سے متعارف کروایا تھا۔)
پاسکل کی ترتیب کی خاصیت 
\(
\begin{pmatrix}
n\\
r\\
\end{pmatrix}
+
\begin{pmatrix}
n\\
r+1\\
\end{pmatrix}
=
\begin{pmatrix}
n+1\\
r+1\\
\end{pmatrix}\)
(مشق 8ب کا سوال نمبر 10 دیکھیں۔) استعمال کرتے ہوئے وضاحت کریں:
\(F_{3}+F_{4}=F_{5}\)
\(F_{4}+F_{5}=F{6}\)

